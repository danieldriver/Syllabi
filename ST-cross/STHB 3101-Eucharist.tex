% Copyright (c) 2019 by David Deane and Daniel R. Driver.
% !TEX encoding = UTF-8 Unicode
% !TEX TS-program = XeLaTeX

\documentclass[titlepage]{article}

% This document presumes a file structure and set of inputs that are
% available at: git@github.com:danieldriver/syllabi.git

\newcommand\policy{../policy}
\newcommand\incl{../includes}
\ProvidesFile{variables.tex}[2018/05/24 v2.1 -- Syllabus variables]

\usepackage{xspace} % make manual spaces (like \mycmd\ ) unnecessary
\usepackage{xifthen} % provides \isempty test

% variables for internal use
\newcommand\prof{}
\newcommand\pdegree{}
\newcommand\pphone{}
\newcommand\pemail{}
\newcommand\poffice{}
\newcommand\phours{}
%
\newcommand\ccode{}
\newcommand\ctitle{}
\newcommand\cseries{}
\newcommand\cversion{}
\newcommand\csemester{}
\newcommand\cmeetson{}
\newcommand\cmeetsat{}
\newcommand\cmeetsin{}
\newcommand\cwebsite{}
\newcommand\cdescrip{}
\newcommand\cprereqs{}
\newcommand\edobject{}

% in case of fully online courses - https://tex.stackexchange.com/a/5896
\newif\ifonline
\newcommand\Int[2]{\ifonline#1\else#2\fi}

% commands for setting variables in the preamble
\newcommand\professor[2][PhD]{
  \renewcommand\pdegree{#1\xspace}
  \renewcommand\prof{#2\xspace}}
\newcommand\phone[1]{
  \renewcommand\pphone{\addfontfeatures{Numbers=Monospaced}#1\xspace}}
\newcommand\email[1]{
  \renewcommand\pemail{\href{mailto:#1}{#1}\xspace}}
\newcommand\officehours[2][Library, Room 5-North]{
  \renewcommand\poffice{#1\xspace}
  \renewcommand\phours{#2\xspace}}
%
\newcommand\coursecode[2][1.0]{
  \renewcommand\cversion{#1\Int{-i}{}\xspace}
  \renewcommand\ccode{#2\Int{(Int)}{}\xspace}}
\newcommand\coursetitle[2][]{
  \ifthenelse{\isempty{#1}}%
    {}% do nothing if #1 is empty, else:
    {\renewcommand\cseries{#1\\[1ex]}}
  \renewcommand\ctitle{#2\xspace}}
\newcommand\semester[1]{
  \renewcommand\csemester{#1\xspace}}
\newcommand\meets[3]{
  \newcommand\AM{\textsc{am}}
  \newcommand\PM{\textsc{pm}}
  \renewcommand\cmeetson{#1\xspace}
  \renewcommand\cmeetsat{\Int{From 9:00 \AM}{#2}\xspace}
  \renewcommand\cmeetsin{\Int{\href{https://smu.brightspace.com/d2l/login}{Brightspace}}{#3}\xspace}}
\newcommand\website[1]{
  \renewcommand\cwebsite{\href{http://#1}{#1}\xspace}}
\newcommand\cdescription[2][RM 1000 or GTRS 6000; and BF 1001]{
  \renewcommand\cprereqs{#1}
  \renewcommand\cdescrip{#2\par}}
\newcommand\objectives[1]{
  \renewcommand\edobject{#1\par}}


%\onlinetrue % \Int{true}{false}
\coursecode[1.0]{ST/HB 3101}
\coursetitle[Systematic Theology/Hebrew Bible]{The Eucharist in\\[.8ex] Ecumenical Perspective}

% Taught as AST in:
%   - Winter 2019

\professor[PhDs]{D. Deane \& D. Driver}
\phone{902-222-0930 (Driver)}
\email{ddeane@astheology.ns.ca}
\officehours[660 Building, then Library?]{By Appointment}

\semester{Winter Term 2019}
\meets{\Int{Thursdays}{Mondays}}% \meets{on}{at}{in}
      {1:00--3:30 \PM}
      {King's Room}
\website{danieldriver.com}
\cdescription[RM 1000 or GTRS 6000; and BF 1001 and TF 1000]{% copy from the current Academic Calendar; [] for prereqs
	It is a painful irony that the Eucharist, once the form and
	instrument of Christian unity, has become a point at which Christian
	division and brokenness is most visible. Why is this so? Can the
	Eucharist once again serve as an engine and sign of Christian unity?
	And, if so, how? This course provides a space in which students can
	ask these questions and begin to shape responses. It does so by
	offering an account of the problem’s history and scope, and a set of
	biblical and theological resources for the engagement of Eucharist
	theology in an ecumenical setting.

	The course has three distinct sections. First, we will explore the
	Eucharist through the lens of scripture. Second, we will offer an
	engagement with some key voices in Christian tradition. Finally, in
	the third section of the course, we will facilitate an ecumenical
	conversation about the role of the Eucharist in the life of the
	Church. This conversation will be informed by the scriptural and
	traditional lenses we have engaged in sections one and two.
}% end of course description
\objectives{% review Bloom's taxonomy: http://www.celt.iastate.edu/teaching/effective-teaching-practices/revised-blooms-taxonomy
            % or try this new one at Rice: http://cte.rice.edu/blogarchive/2017/2/23/taxonomy

	This course aims to do two things. (1) To offer an overview of the
	ecumenical core of Christian eucharistic theology. (2) To introduce
	and discuss some of the different perspectives on the Eucharist that
	shape ecumenical relations today.

\subsection{Course Themes}

	Inconveniently for us, early Christians did not write with us in
	mind. Almost all early Christian texts are written for insiders by
	insiders. There are no welcome pamphlets with detailed descriptions
	of their practices. If there were, we would have a much better sense
	of how they understood the bread they broke and the cup they drank.
	Instead, all we have are tantalising snippets – most of which you’ll
	read – from the early Church where they refer to their eucharistic
	practices. But how are we to understand these snippets? They make
	reference to biblical foundations, and so, before you read them,
	you’ll get acquainted with some important Jewish and Christian
	sources. \emph{But from these scant sources you’ll be invited, in
	dialog with each other and the instructors, to develop your
	understanding of the Eucharist}. What is it? A memorial meal? A
	symbolic re-enactment of the last supper? Golgotha continued as the
	hand of God reached out from the Triune life to weave us into union
	with God and each other? Does it do anything to you? If so, under
	what conditions is its potency greatest? Why do Roman Catholics ask
	non-Roman Catholics to refrain from taking communion at Roman
	Catholic Masses? Is this not ecumenical elitism – a refusal of the
	very unity the Eucharist symbolises (or instantiates)? We will ask
	all of these questions in dialog with a subtheme of this course –
	\emph{the past and future of eucharistic sharing at AST}. AST was at
	the forefront of post Vatican II Christian ecumenism and eucharistic
	sharing was once practiced, thought the school now observes a
	eucharistic fast. What needs to be in place for this sharing to
	resume? Is the Eucharist an instrument for bringing about Christian
	unity? Is it a sacramental consummation of ontological unity – the
	sign of Christian oneness? Or, is it an act of friendship between
	diverse and different strands of Christianity?

	In addition to exploring the Eucharist \emph{per se}, and the
	Eucharist in ecumenical perspective, there is also an opportunity
	for each student to investigate particular aspects of eucharistic
	and/or ecumenical theology via the two main papers you will be
	writing in fulfillment of course requirements. Touch base with the
	instructors when something grabs your interest. We will recommend
	readings as you follow your ideas where they lead.

\subsection{Learning Outcomes}

Students completing this course will know:

\begin{enumerate}
\item The Old and New Testament foundations from which Christians have traditionally developed their theology of the Eucharist.
\item The primary references for Christian approaches to the Eucharist in the first 300 years of the Church.
\item The nuanced differences in eucharistic theology that become explicit in the late medieval and reformation eras.
\item How these late medieval a reformation era distinctions inform differences in denominational eucharistic theology today.
\item The history of ecumenical eucharistic theology since Vatican II.
\end{enumerate}

Students completing this course will have developed:

\begin{enumerate}
\item A personal theology of the Eucharist in dialog with scripture and tradition.
\item An understanding of how the Eucharist is understood within the three founding denominations at AST.
\item A capacity to correlate liturgical practices in the settings they lead and the theological ideas this course explores.
\item Nuanced, well informed perspectives on the challenges to and possibilities for eucharistic sharing in our 21st century context.
\end{enumerate}

}% end of learning objectives

\ProvidesFile{preamble.tex}[2013/09/06 v1.0 -- Syllabus preamble]

% basic typography
\usepackage{fontspec}
\setmainfont[Ligatures=TeX]{Meta Serif Pro}
\setsansfont[Ligatures=TeX]{Meta Pro}
\newfontfamily\Heb{Meta Hebrew}
\setmonofont[Scale=MatchLowercase]{Menlo}
\usepackage{sectsty}
\allsectionsfont{\sffamily}
\frenchspacing
\setlength{\emergencystretch}{3em} % prevent overfull lines

% custom font size and leading
\renewcommand\tiny{\fontsize{6}{9}\selectfont}
\renewcommand\scriptsize{\fontsize{7}{10}\selectfont}
\renewcommand\footnotesize{\fontsize{8}{11}\selectfont}
\renewcommand\small{\fontsize{8.5}{11.5}\selectfont}
\renewcommand\normalsize{\fontsize{9}{12}\selectfont}% base size
\renewcommand\large{\fontsize{11}{14}\selectfont}
\renewcommand\Large{\fontsize{13}{16}\selectfont}
\renewcommand\LARGE{\fontsize{16}{19}\selectfont}% "course syllabus \\ semester" benefits from more lead
\renewcommand\huge{\fontsize{19}{21}\selectfont}
\renewcommand\Huge{\fontsize{24}{26}\selectfont}

% layout packages: page, logo, tables
\usepackage[scale={0.6,0.8},
            xetex]{geometry}
\usepackage{graphicx}
\usepackage{array}     % allow insertions of column styling with >{}
\usepackage{booktabs}  % elegant horizontal rules in tables
\usepackage{marginfix} % protect positioning of margin table in policy/grades

% custom macros for a session count in the schedule of readings
\newcounter{session}
\newcounter{columns}
\newcounter{courseunit}
\newcommand\setcolumncount[2][0]{ % optionally set count to other than 0,
  \setcounter{session}{#1}        % e.g. to -1, or to a standing count
  \setcounter{columns}{#2}}
\newcommand\sessioncount{\stepcounter{session}\arabic{session}}
\newcommand\sessionskip[1]{\multicolumn{1}{@{}r@{ }}{#1}}
\newcommand\unit[1]{\multicolumn{\thecolumns}{c}{%
  \scshape\stepcounter{courseunit}\roman{courseunit}. \MakeLowercase{#1}}}
\newcommand\noclass[1]{\multicolumn{1}{@{}l}{\itshape No Class: #1}}

% color to match Tyndale's branding
\usepackage[usenames]{xcolor}
% predefined: black, white, red, green, blue, cyan, magenta, yellow
\definecolor{TyndaleURLs}{HTML}{0062A0} % links on tyndale.ca
\definecolor{TyndaleBlue}{cmyk}{1,1,0,.32}
\definecolor{TyndaleGold}{cmyk}{0,.27,1,0}
\definecolor{TyndaleRed}{cmyk}{0,1,.99,.04}
\definecolor{TyndaleBlack}{cmyk}{0,0,0,1}
\definecolor{TyndaleGreen}{cmyk}{.45,0,1,.24}
\definecolor{TyndaleOrange}{cmyk}{0,.79,1,0}
\definecolor{TyndaleAqua}{cmyk}{.47,0,.24,0}
\definecolor{TyndaleYellow}{cmyk}{.03,.03,.35,0}

% metadata (assumes a host of definitions are made in the main file)
\usepackage[setpagesize=false,     % leave this to geometry
            hyperfootnotes=false,  % fragile and distracting
            xetex]{hyperref}
\hypersetup{breaklinks=true,       % allow link text to break across lines
            colorlinks=true,       % colorlinks resets pdfborder to 0 0 0
            urlcolor=TyndaleURLs,  % for external links
            linkcolor=TyndaleRed,  % for normal internal links
            citecolor=TyndaleGold, % for bibliographical citations in text
            pdfauthor={\prof},
            pdftitle={\ccode: \ctitle},
            pdfsubject={Tyndale UC, \csemester},
            pdfcreator={github.com/danieldriver/syllabus}}
\urlstyle{same}                    % don't use monospace font for urls

% custom footlines
\usepackage{fancyhdr}
\pagestyle{fancy} % turn it on
\fancyhf{}        % reset everything
\renewcommand{\headrulewidth}{0pt} % remove header line as well
\lfoot{\sffamily\scshape\footnotesize\MakeLowercase{\ctitle, v\cversion}}
\rfoot{\sffamily\scshape\footnotesize\MakeLowercase{\prof\quad\thepage}}

% gratuitous with custom title page, but useful as a fallback
\title{\ccode: \ctitle}
\author{\professor}
\date{\semester}


\begin{document}
\ProvidesFile{title.tex}[2013/09/06 v1.0 -- Syllabus title page]

\begin{titlepage}
  \begin{center}

    \LARGE\sffamily % set title elements in a large sans serif

    \begin{minipage}{\textwidth}
      \parbox[t]{0.5\textwidth}{
        \mbox{}\\[-13pt] % dummy line to align parboxes
        \includegraphics[width=0.5\textwidth]{.syllabus/includes/TyndaleUC}}
      \hfill
      \parbox[t]{0.4\textwidth}{
        \raggedleft Course Syllabus\\
        \csemester}
    \end{minipage}

    \vfill

    {\textsc{\MakeLowercase\ccode}\\[1ex]
      \bfseries\cseries\Huge\ctitle}

    \vfill

    \normalsize\rmfamily % switch back to body type

    \begin{tabular}{>{\bfseries}rl>{\bfseries}rl}
      \toprule
      Instructor & \prof, \pdegree & Course  & Version \cversion \\
      \midrule
      Phone      & \pphone         & Meets   & \cmeetson         \\
      Email      & \pemail         & Time    & \cmeetsat         \\
      Office     & \poffice        & Room    & \cmeetsin         \\
      Hours      & \phours         & Website & \cwebsite         \\
      \bottomrule
    \end{tabular}

    \vfill

    \begin{description}\small
      \item[Commuter Hotline]
        Class cancellations due to inclement weather or illness will
        be announced on the commuter hotline at \texttt{416.226.6620
        x2187}. Alternately, weather cancellation information is posted
        at \href{http://tyndale.ca/weather}{tyndale.ca/weather}.
      \item[MyTyndale.ca]
        This course may have materials stored on its website, such as
        handouts or readings that may be needed in order to complete
        assignments. Students are responsible for checking these course
        pages on a regular basis. Here, too, students are able to view
        their grades throughout the semester. For more information see
        Section~\ref{mytyndale}, below.
      \item[Mail]
        Students are responsible for information communicated through
        their campus mailboxes and student e-mail accounts. A mailbox
        directory hangs beside the mailboxes. For more information
        contact the Registrar's office.
    \end{description}

  \end{center}

  \section{Course Description}
  \label{description}

  \emph{From the Academic Calendar:} \cdescrip

\end{titlepage}
\setcounter{page}{2} % count the title page as page 1


  \section{Learning Objectives}
  \label{objectives}

  \edobject

\section{Required Texts \& Materials}
\label{texts}

The following texts are required. Students are strongly encouraged to
purchase their own copies. Library copies that are not reference works
have been placed on a 2-hour reserve.

\begingroup
\renewcommand{\section}[2]{}% temporarily remove the section heading
\begin{thebibliography}{9}% use the longest item in the bibliography

	\bibitem{Hunsinger} George Hunsinger.
	\emph{The Eucharist and Ecumenism: Let Us Keep the Feast}. Current Issues in Theology.
	Cambridge: Cambridge University Press, 2012. ISBN 978-0521719179.

	\bibitem{Pitre} Brant Pitre.
	\emph{Jesus and the Jewish Roots of the Eucharist: Unlocking the Secrets of the Last Supper}.
	New York: Doubleday, 2011. ISBN 978-0385531849.

%

%Bouyer
%: Louis Bouyer. *Eucharist: Theology and Spirituality of the Eucharistic Prayer.* Translated by Charles Underhill Quinn. Notre Dame, IN: University of Notre Dame Press, 1989.
%: Order it in [Canada](https://amzn.to/2T2gPuS) or the [USA](https://amzn.to/2RSXmLX). ISBN 978-0268004989.
%
%Schmemann
%: Alexander Schmemann. *The Eucharist: Sacrament of the Kingdom.* Crestwood, NY: St. Vladimir’s Seminary Press, 1988.
%: Order it in [Canada](https://amzn.to/2Fj5Pqk) or the [USA](https://amzn.to/2PTTI74). ISBN 978-0881410181.


\end{thebibliography}
\endgroup

Students will also want to have a good, modern translation of the Bible,
perhaps in the form of a study Bible. The \emph{New Oxford Annotated
Bible with Apocrypha} (NRSV), 5th ed., ed. M.\,D. Coogan (Oxford: Oxford
University Press, 2018) is a solid choice. For the Hebrew Bible,
\emph{The Jewish Study Bible: Second Edition} (NJPS), ed. Adele Berlin
and Marc Zvi Brettler (Oxford: Oxford University Press, 2014) is a fine
alternative. Reference copies of both are available in the AST Library
and through \href{http://ezproxy.astheology.ns.ca:2048/login?url=http://www.oxfordbiblicalstudies.com/}{Oxford Biblical Studies Online}.

\section{Supplementary Texts}
\label{supplementary}

Supplementary readings will be recommended throughout the semester and
either placed on reserve or made available online. Give this material
good effort and attention.

Also, the following works are worth owning and consulting.
\cite{Schmemann} offers an important Orthodox voice. \cite{rlgs}
contains sound advice on core skills like reading religious texts,
writing essays and reviews, revising essays, and making oral
presentations.

\begingroup
\renewcommand{\section}[2]{}% temporarily remove the section heading
\begin{thebibliography}{Making Sense}% use the longest item in the bibliography

	\bibitem[Schmemann]{Schmemann} Alexander Schmemann.
	\emph{The Eucharist: Sacrament of the Kingdom}.
	Crestwood, NY: St. Vladimir’s Seminary Press, 1988. ISBN 978-0881410181.

	\bibitem[Making Sense]{rlgs} Northey, Margot, Bradford A. Anderson, and Joel N. Lohr.
	\emph{Making Sense in Religious Studies: A Student's Guide to Research and Writing}.
	3rd ed. Toronto: Oxford University Press, 2019. ISBN 978-0199026838.

	\bibitem[SBL2]{sbl2} Collins, Billie Jean, et al.
	\emph{The SBL Handbook of Style}.
	2nd ed. Atlanta: SBL Press, 2014. ISBN 978-1589839649.
	See the online \href{https://www.sbl-site.org/assets/pdfs/pubs/SBLHSsupp2015-02.pdf}{Student Supplement}.

\end{thebibliography}
\endgroup


\section{Course Outline}
\label{outline}

We will adhere to the schedule in \autoref{schedule} as closely as
possible, though the professors reserve the right to adjust it to suit
the needs of the class.

\setcolumncount{4}% set up \sessioncount, \unit{}, \noclass{}, and \reminder{memo}{date} macros
\begin{table}[htb]% set to `p' to put the schedule on its own page
  \centering
  \begin{tabular}{>{\sessioncount.}r@{ }lp{5.5cm}r}% make sure the column config agrees with \setcolumncount
	\toprule
	\sessionskip{\textbf{\S}.}&\textbf{Topic}&\textbf{Reading}&\textbf{Date}\\
	\midrule

	\unit{Biblical Foundations} \\

		& Introduction and Syllabus & \href{https://www.cardus.ca/comment/article/trading-brunch-for-the-eucharist/}{Sarah Dahl, “Trading Brunch”} &  7 Jan. \\
		& Jewish Roots 1            & Pitre, Chs. 1–4; Exodus 12, 16            & 14 Jan. \\
		& Jewish Roots 2            & Pitre, Chs. 4–8; Exodus 24--25            & 21 Jan. \\
		& New Testament Foundations & Mark 14, Matthew 26, Luke 13 \& 22, John 6 \& 13, Acts 2 \& 20, 1 Cor 10--11;
									  \emph{Optional supplement: Levering 2005} & 28 Jan. \\ [2.7em]

	\unit{Church Tradition} \\

		& Early Church Before Nicaea & “Christianity Before Nicaea,” feat. the Didache, Ignatius of Antioch, Justin Martyr, Tertullian, Origen, Clement, and Cyprian
		                                                          &  4 Feb. \\
		& Early Church After Nicaea  & “4th and 5th Centuries,” feat. Cyril of Jerusalem, Hillary of Poitiers, Gregory of Nyssa, and John Chrysostom
		                                                          & 11 Feb. \\
	\reminder{The first paper is \textbf{due} by the end of the sixth week of class}{15 Feb.} \\
	\noclass{AST Term Break}                                      & 18 Feb. \\

		& Aquinas and Transubstantiation & \emph{Summa Theologica} III, Q 73–83  & 25 Feb. \\
		& Cranmer and “True Presence”    & Peter Brooks 1992      &  4 Mar. \\[1ex]

	\unit{Ecumenical Lenses} \\

		& Ecumenical Perspectives 1   & Hunsinger, pp. 1–92       & 11 Mar. \\
		& Ecumenical Perspectives 2   & Hunsinger, pp. 93–186     & 18 Mar. \\
		& Ecumenical Perspectives 3   & Hunsinger, pp. 187–244    & 25 Mar. \\
		& AST and Eucharistic Sharing & AST Papers and Documents  &  1 Apr. \\
	\reminder{The second short paper is \textbf{due} by the end of the last day of class}{1 Apr.} \\ [1ex]

	\reminder{End of Term: Final marks are due for all courses}{10 Apr.} \\

	\bottomrule
  \end{tabular}
  \caption{Schedule of Readings}
  \label{schedule}
\end{table}

See the AST website for a list of other \href{http://www.astheology.ns.ca/students/academic-dates.html}{important dates}.

\section{Evaluation}
\label{evaluation}

Assigned readings must be completed before each class because they form
a basis for activity in class. After each class you will prepare a short
reflection based on (1) the material you read for class and (2) the
lecture/conversation which took place in class. Thus the typical pattern
will be: reading before class, lecture and discussion in class, written
reflection after class.

The grade structure for \ccode has the following elements.

\begin{enumerate}

	\item Engagement with the \textbf{reading} will be evaluated mostly
	on the basis of completion of weekly reflections. Superior control
	of the material in class discussion may weigh in your favour in the
	final assessment.

	\item The \textbf{weekly reflections} are intended to facilitate a
	one-on-one exchange between you and the teaching team. Submissions
	should normally be 1–2 pages, but can be as longer (like a good book
	review) or shorter (like a brief, well-composed email) as necessary.
	When you write, spend 75\% of your reflection speaking about the
	material you engaged. What did the reading say and what did you hear
	in class? Then, spend 25\% offering your response to it. What made
	sense? What didn’t? What points was the writer or the instructor
	trying to make that weren’t accessible to you? Every week you will
	receive a response from a member of the teaching team.

	\item Two \textbf{formal papers} will facilitate your engagement
	with the key questions from the course. One paper could be more
	biblical in focus, and one more theological, but you are free to
	develop each as you see fit. The papers provide space for you to
	pursue a topic introduced by the course but not explored with enough
	depth to satisfy your interest. Each should be about 3,000 words in
	length. The first is due at the end of the sixth week of class; the
	second, at the end of the last day of class.

\end{enumerate}

The breakdown for the semester's total work is shown in
\autoref{grade-dist}.

\begin{table}[htbp]
  \centering
  {\lining
  \begin{tabular}{lr}
    \toprule
    Weekly Reflections & 40\% \\
    First Paper        & 30\% \\
    Second Paper       & 30\% \\
    \bottomrule
  \end{tabular}}
  \caption{Distribution of Grades}
  \label{grade-dist}
\end{table}

%AST's \href{http://www.astheology.ns.ca/students/resources.html}{Academic
%Calendar} provides guidelines and detailed criteria for academic
%assessment. Marks are assigned by letter grade using these benchmarks.

\ProvidesFile{grades.tex}[2016/09/03 v2.0 -- Course policy]

\subsection{Grading System at AST}
\label{grades}

AST's \href{http://www.astheology.ns.ca/webfiles/AST_2016Calendar_web(A5)-06APR2016.pdf}{Academic
Calendar} provides guidelines and detailed criteria for academic
assessment. Marks are assigned by letter grade using the benchmarks in
\autoref{grade-syst}.

\begin{table}[htbp]
  \centering
  {\lining
  \begin{tabular}{lll}
    \toprule
%    Letter      & Percent & Assessment        \\
%	\midrule
    A+          & 94--100    & Exceptional    \\
    A           & 87--93     & Outstanding    \\
    A\char"2212 & 80--86     & Excellent      \\ [1ex]
    B+          & 77--79     & Good           \\
    B           & 73--76     & Acceptable     \\
    B\char"2212 & 70--72     & Marginal       \\ [1ex]
    C           & 60--69     & Unsatisfactory \\
    F           & 0--59      & Failure        \\
    FP          & 0          & Failure due to Plagiarism \\
    \bottomrule
  \end{tabular}}
  \caption{Summary of Grading System}
  \label{grade-syst}
\end{table}

% More detailed grading criteria from pp. 61--62 of `16.0406-I2-AST Academic Calendar.pdf'
%
%\begin{description}
%  \item[A+ (94-100) ‘Exceptional’]
%    A superior performance with consistent evidence of a comprehensive,
%    incisive grasp of all aspects of the subject matter; a very wide
%    knowledge base; insightful critical evaluation and analysis of the
%    material; an exceptional capacity for original, creative, and/or
%    logical thinking; an exceptional ability to organize, analyse,
%    synthesize, and to express thoughts fluently.
%  \item[A (87-93) ‘Outstanding’]
%    A comprehensive grasp of the subject matter, outstanding evidence of
%    original thought; sound critical evaluation of the material; an
%    excellent ability to organize, analyse, synthesize and to express
%    thoughts; mastery of an extensive knowledge base.
%  \item[A- (80-86) ‘Excellent’]
%    All the qualities of a B-level performance and an excellent capacity
%    for original, creative, and/ or logical thinking; excellent ability
%    to organize, analyse, synthesize, and integrate ideas; broad
%    knowledge base in the subject matter.
%  \item[B+ (77-79) ‘Good’]
%    A good performance with substantial knowledge of the subject matter;
%    a very good understanding of the relevant issues; familiarity with
%    relevant literature and techniques; good ability to organize,
%    analyse, and examine the material in a constructive and critical
%    manner.
%  \item[B (73-76) ‘Acceptable’]
%    A generally adequate performance with a good knowledge of the
%    subject matter; a fair understanding of relevant issues; some
%    ability to work with relevant literature and techniques; some
%    ability to develop solutions to difficult problems related to the
%    subject material.
%  \item[B- (70-72) ‘Marginally Acceptable’]
%    Some familiarity with the subject material; some understanding.
%    Satisfactory understanding of relevant issues; attempts to solve
%    moderately difficult problems related to the subject material in a
%    critical and analytical manner are only partially successful.
%  \item[C (60-69) ‘Unsatisfactory’]
%    A C grade indicates unsatisfactory academic performance. At the
%    discretion of the instructor, supplemental work may be negotiated to
%    upgrade the mark to a B range. A student may carry two C grades
%    without penalty in all courses except Foundations Courses,
%    Supervised Field Education, Supervised Ministry Practicum and the
%    Graduate Project. In these courses, a minimum grade of B- is
%    required to graduate. A student who receives a C in a Foundation
%    course must repeat the course to achieve a B- or better, and cannot
%    use the C grade to meet prerequisite requirements for advanced
%    courses. If the student repeats one of these courses and receives a
%    B- or better, the previous C grade remains on the transcript and can
%    be counted toward the total of unsatisfactory grades that may lead
%    to academic dismissal. Credit will be given only once for any
%    course. (See Policy on Unsatisfactory Academic Performance in the
%    AST Student Handbook.)
%  \item[F (0-59) ‘Failure’]
%    Student has not grasped subject matter; does not understand issues
%    involved; cannot work with relevant literature. (See Policy on
%    Unsatisfactory Academic Performance in the AST Student Handbook.)
%  \item[P ‘Pass’]
%    Credit awarded, but no mark assigned.
%  \item[FP ‘Failure due to Plagiarism’]
%    A student will receive this grade only after proven incident(s) of
%    plagiarism in a course.
%\end{description}
%\ProvidesFile{other.tex}[2022/06/08 v2.9.1 -- Course policy]

\section{Other Course Policy}
\label{policy}

Late work will not be accepted, except in genuinely extenuating
circumstances. Students must submit something before the deadline if
they wish to receive credit. Unless I state otherwise, assignments are
to be uploaded by 11:59 \PM\ (Atlantic) on the date indicated.

Essay submissions must be typewritten and double-spaced. They should be
free from error. In this course they should follow SBL Style (see
\cite{sbl2} in \autoref{supplementary}, above). As a reminder, AST
upholds an Inclusive Language Policy. Please use gender-inclusive
language when referring to human beings. Our traditions have different
norms for speech about God; you are of course free to follow and explore
those traditions when referring to God.


Plagiarism is the
\href{http://www.eerdmans.com/Pages/Item/59043/Commentary-Statement.aspx}{failure}
to \href{https://www.theguardian.com/world/2013/feb/09/german-education-minister-quits-phd-plagiarism}{attribute}
(by means of footnotes when writing or aloud when speaking) any ideas,
phrases, sentences, materials, syntheses, et cetera, that another author
has composed and that you have borrowed for your own work. Plagiarism is
unethical. Academic penalties for plagiarism at AST are serious, and may
include failure of the course or even suspension of further studies.
Unintentional plagiarism is considered plagiarism. AST's Plagiarism
Policy is found under that heading in the Academic
Calendar.

Students should request permission to record a class or lecture. If
permission is granted, or if recordings are provided (as in the case of
an online or hybrid course), I stipulate that all recordings be for
personal use only. They may not be shared or distributed.

If you have needs that require modifications to any aspect of this
course, please consult with the instructor as soon as possible. Any
documentation regarding disabilities that you wish to divulge to AST
should be provided to the Registrar’s Office, where it will be kept in a
confidential file.

Finally, I encourage the conscientious use of laptops, tablets, and
other technology in my classes. In classroom settings, realize that, as
\href{http://dx.doi.org/10.1016/j.compedu.2012.10.003}{cognitive
psychologists have demonstrated}, ``laptop multitasking hinders
classroom learning for both users and nearby peers.'' Do your part to
foster an environment for dialogue by honouring the presence of your
classmates. In online and hybrid settings, consider both the physical
environment in which you choose to work and the virtual environment that
you help create through your participation in various forums. Let your
engagement in this course be marked by rigour and charity alike.


\section{Other Course Policy}
\label{policy}

Late work will not be accepted, except in genuinely extenuating
circumstances. Students must submit something before the deadline if
they wish to receive credit. Unless we state otherwise, written
assignments are to be uploaded by 11:59 \PM\ on the date indicated.

Formal paper submissions must be typewritten, double-spaced, and formatted as
PDFs. They should be free from error. In this course they should follow
\href{http://www.chicagomanualofstyle.org/home.html}{\emph{The Chicago
Manual of Style}} or its derivative, SBL Style (see \cite{sbl2} in
\autoref{supplementary}, above). As a reminder, AST also upholds an
Inclusive Language Policy.

\href{http://www.eerdmans.com/Pages/Item/59043/Commentary-Statement.aspx}{Plagiarism},
if \href{https://www.theguardian.com/world/2013/feb/09/german-education-minister-quits-phd-plagiarism}{detected},
will result in failure of the course.

Students should request permission to record a class or lecture. If
permission is granted, or if recordings are provided (as in the case of
an online course), we stipulate that all recordings be for personal use
only. They may not be shared or distributed.

If you have abilities or disabilities that require modifications to the
assessment process or other aspects of this course, please advise the
course instructors as soon as possible.

Finally, we encourage the conscientious use of laptops, tablets, and
other technology in class. Please realize that, as
\href{http://dx.doi.org/10.1016/j.compedu.2012.10.003}{cognitive
psychologists have demonstrated}, ``laptop multitasking hinders
classroom learning for both users and nearby peers.'' Do your part to
foster an environment of open dialogue by honouring the presence of your
classmates. In online and hybrid settings, consider both the physical
environment in which you choose to work and the virtual environment that
you help create through your participation in various forums. Let your
engagement in this course be marked by rigour and charity alike.

\end{document}

