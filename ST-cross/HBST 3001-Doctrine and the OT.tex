% Copyright (c) 2022 by Daniel R. Driver.
% !TEX encoding = UTF-8 Unicode
% !TEX TS-program = XeLaTeX

\documentclass[titlepage]{article}

% This document presumes a file structure and set of inputs that are
% available at: git@github.com:danieldriver/syllabi.git

\newcommand\policy{../policy}
\newcommand\incl{../includes}
\ProvidesFile{variables.tex}[2018/05/24 v2.1 -- Syllabus variables]

\usepackage{xspace} % make manual spaces (like \mycmd\ ) unnecessary
\usepackage{xifthen} % provides \isempty test

% variables for internal use
\newcommand\prof{}
\newcommand\pdegree{}
\newcommand\pphone{}
\newcommand\pemail{}
\newcommand\poffice{}
\newcommand\phours{}
%
\newcommand\ccode{}
\newcommand\ctitle{}
\newcommand\cseries{}
\newcommand\cversion{}
\newcommand\csemester{}
\newcommand\cmeetson{}
\newcommand\cmeetsat{}
\newcommand\cmeetsin{}
\newcommand\cwebsite{}
\newcommand\cdescrip{}
\newcommand\cprereqs{}
\newcommand\edobject{}

% in case of fully online courses - https://tex.stackexchange.com/a/5896
\newif\ifonline
\newcommand\Int[2]{\ifonline#1\else#2\fi}

% commands for setting variables in the preamble
\newcommand\professor[2][PhD]{
  \renewcommand\pdegree{#1\xspace}
  \renewcommand\prof{#2\xspace}}
\newcommand\phone[1]{
  \renewcommand\pphone{\addfontfeatures{Numbers=Monospaced}#1\xspace}}
\newcommand\email[1]{
  \renewcommand\pemail{\href{mailto:#1}{#1}\xspace}}
\newcommand\officehours[2][Library, Room 5-North]{
  \renewcommand\poffice{#1\xspace}
  \renewcommand\phours{#2\xspace}}
%
\newcommand\coursecode[2][1.0]{
  \renewcommand\cversion{#1\Int{-i}{}\xspace}
  \renewcommand\ccode{#2\Int{(Int)}{}\xspace}}
\newcommand\coursetitle[2][]{
  \ifthenelse{\isempty{#1}}%
    {}% do nothing if #1 is empty, else:
    {\renewcommand\cseries{#1\\[1ex]}}
  \renewcommand\ctitle{#2\xspace}}
\newcommand\semester[1]{
  \renewcommand\csemester{#1\xspace}}
\newcommand\meets[3]{
  \newcommand\AM{\textsc{am}}
  \newcommand\PM{\textsc{pm}}
  \renewcommand\cmeetson{#1\xspace}
  \renewcommand\cmeetsat{\Int{From 9:00 \AM}{#2}\xspace}
  \renewcommand\cmeetsin{\Int{\href{https://smu.brightspace.com/d2l/login}{Brightspace}}{#3}\xspace}}
\newcommand\website[1]{
  \renewcommand\cwebsite{\href{http://#1}{#1}\xspace}}
\newcommand\cdescription[2][RM 1000 or GTRS 6000; and BF 1001]{
  \renewcommand\cprereqs{#1}
  \renewcommand\cdescrip{#2\par}}
\newcommand\objectives[1]{
  \renewcommand\edobject{#1\par}}


\coursecode[2.0]{HB/ST 3001}
\coursetitle[Hebrew Bible/Systematic Theology]{Christian Doctrine and\\[.8ex] the Old Testament}

% Taught as RLGS 4483 (a remote antecedent) at Tyndale UC in:
%   - Winter 2009 ("Biblical Theological Themes: The Old Testament as Christian Scripture")
% Taught as HB/ST 3001 (largely a new course) at AST in:
%   - Summer 2017 (Session 3: July 17-21 and 24-28)
%   - Winter 2022

\professor{Daniel R. Driver}
\phone{902-425-7051}
\email{ddriver@astheology.ns.ca}
\officehours[Microsoft Teams]{Tuesdays, 1:00--3:00 \PM}

\semester{Winter Term 2022}
\meets{Tuesdays}% \meets{on}{at}{in}
      {4:00--6:30 \PM\ Atlantic}
      {Teams/Classroom 3}
\website{danieldriver.com}
\cdescription{% copy from the current Academic Calendar
	What is the relationship of scripture and theology? Some answers to
	this question start with the Bible, while others begin with
	Christian doctrine, while still others call for a dialectical
	movement between canon and creed. However it is approached, the
	question presents a number of challenges.

	To illustrate the problem in general, and to outline some possible
	solutions, this course highlights a variety of current approaches to
	the Old Testament as the scripture of the Church. Students will
	encounter work by an array of biblical and theological specialists
	who transgress the conventional limits of their disciplines. They
	will study the views of Catholics, Protestants, Jews, and those who
	disavow religious affiliation. Because the Hebrew Bible is Jewish
	scripture in the first instance, students will consider why some
	Jews are not interested in biblical theology, and what it means for
	the Church to adhere to the Scriptures of Israel. Specific doctrines
	under consideration include: creation, sin, election, incarnation
	(the body of God), divine impassibility, Christology, Mariology, law
	and gospel, death and resurrection, and the afterlife.
}% end of course description
\objectives{% review Bloom's taxonomy: http://www.celt.iastate.edu/teaching/effective-teaching-practices/revised-blooms-taxonomy
% or try this one at Rice: http://cte.rice.edu/blogarchive/2017/2/23/taxonomy
	By the end of this course students should develop knowledge of some
	ways different churches -- including Roman Catholic, Anglican, and
	United -- approach the Bible in general, and the Old Testament in
	particular, as scripture. They should become familiar with some
	leading narratives about the relationship of Bible and Theology, and
	with different theories about how to frame knowledge of God (in
	terms apophatic and kataphatic, for example). They should be able to
	analyze and evaluate doctrinal claims by biblical scholars and
	theologians who use the OT as a core theological resource. Students
	should also develop curiosity about the role of Hebrew scripture in
	communities beyond the church, including in the academy and the
	synagogue. Their creativity and confidence in expressing their own
	ideas about the Scriptures of Israel should grow, even as they learn
	to do so with appropriate humility and charity. Finally, students
	should experience a process of discovery in dialogue with biblical
	traditions ancient and strange as they pursue fresh ways ways of
	thinking and speaking about the God of Israel.
}% end of learning objectives

\ProvidesFile{preamble.tex}[2013/09/06 v1.0 -- Syllabus preamble]

% basic typography
\usepackage{fontspec}
\setmainfont[Ligatures=TeX]{Meta Serif Pro}
\setsansfont[Ligatures=TeX]{Meta Pro}
\newfontfamily\Heb{Meta Hebrew}
\setmonofont[Scale=MatchLowercase]{Menlo}
\usepackage{sectsty}
\allsectionsfont{\sffamily}
\frenchspacing
\setlength{\emergencystretch}{3em} % prevent overfull lines

% custom font size and leading
\renewcommand\tiny{\fontsize{6}{9}\selectfont}
\renewcommand\scriptsize{\fontsize{7}{10}\selectfont}
\renewcommand\footnotesize{\fontsize{8}{11}\selectfont}
\renewcommand\small{\fontsize{8.5}{11.5}\selectfont}
\renewcommand\normalsize{\fontsize{9}{12}\selectfont}% base size
\renewcommand\large{\fontsize{11}{14}\selectfont}
\renewcommand\Large{\fontsize{13}{16}\selectfont}
\renewcommand\LARGE{\fontsize{16}{19}\selectfont}% "course syllabus \\ semester" benefits from more lead
\renewcommand\huge{\fontsize{19}{21}\selectfont}
\renewcommand\Huge{\fontsize{24}{26}\selectfont}

% layout packages: page, logo, tables
\usepackage[scale={0.6,0.8},
            xetex]{geometry}
\usepackage{graphicx}
\usepackage{array}     % allow insertions of column styling with >{}
\usepackage{booktabs}  % elegant horizontal rules in tables
\usepackage{marginfix} % protect positioning of margin table in policy/grades

% custom macros for a session count in the schedule of readings
\newcounter{session}
\newcounter{columns}
\newcounter{courseunit}
\newcommand\setcolumncount[2][0]{ % optionally set count to other than 0,
  \setcounter{session}{#1}        % e.g. to -1, or to a standing count
  \setcounter{columns}{#2}}
\newcommand\sessioncount{\stepcounter{session}\arabic{session}}
\newcommand\sessionskip[1]{\multicolumn{1}{@{}r@{ }}{#1}}
\newcommand\unit[1]{\multicolumn{\thecolumns}{c}{%
  \scshape\stepcounter{courseunit}\roman{courseunit}. \MakeLowercase{#1}}}
\newcommand\noclass[1]{\multicolumn{1}{@{}l}{\itshape No Class: #1}}

% color to match Tyndale's branding
\usepackage[usenames]{xcolor}
% predefined: black, white, red, green, blue, cyan, magenta, yellow
\definecolor{TyndaleURLs}{HTML}{0062A0} % links on tyndale.ca
\definecolor{TyndaleBlue}{cmyk}{1,1,0,.32}
\definecolor{TyndaleGold}{cmyk}{0,.27,1,0}
\definecolor{TyndaleRed}{cmyk}{0,1,.99,.04}
\definecolor{TyndaleBlack}{cmyk}{0,0,0,1}
\definecolor{TyndaleGreen}{cmyk}{.45,0,1,.24}
\definecolor{TyndaleOrange}{cmyk}{0,.79,1,0}
\definecolor{TyndaleAqua}{cmyk}{.47,0,.24,0}
\definecolor{TyndaleYellow}{cmyk}{.03,.03,.35,0}

% metadata (assumes a host of definitions are made in the main file)
\usepackage[setpagesize=false,     % leave this to geometry
            hyperfootnotes=false,  % fragile and distracting
            xetex]{hyperref}
\hypersetup{breaklinks=true,       % allow link text to break across lines
            colorlinks=true,       % colorlinks resets pdfborder to 0 0 0
            urlcolor=TyndaleURLs,  % for external links
            linkcolor=TyndaleRed,  % for normal internal links
            citecolor=TyndaleGold, % for bibliographical citations in text
            pdfauthor={\prof},
            pdftitle={\ccode: \ctitle},
            pdfsubject={Tyndale UC, \csemester},
            pdfcreator={github.com/danieldriver/syllabus}}
\urlstyle{same}                    % don't use monospace font for urls

% custom footlines
\usepackage{fancyhdr}
\pagestyle{fancy} % turn it on
\fancyhf{}        % reset everything
\renewcommand{\headrulewidth}{0pt} % remove header line as well
\lfoot{\sffamily\scshape\footnotesize\MakeLowercase{\ctitle, v\cversion}}
\rfoot{\sffamily\scshape\footnotesize\MakeLowercase{\prof\quad\thepage}}

% gratuitous with custom title page, but useful as a fallback
\title{\ccode: \ctitle}
\author{\professor}
\date{\semester}


\begin{document}
\ProvidesFile{title.tex}[2013/09/06 v1.0 -- Syllabus title page]

\begin{titlepage}
  \begin{center}

    \LARGE\sffamily % set title elements in a large sans serif

    \begin{minipage}{\textwidth}
      \parbox[t]{0.5\textwidth}{
        \mbox{}\\[-13pt] % dummy line to align parboxes
        \includegraphics[width=0.5\textwidth]{.syllabus/includes/TyndaleUC}}
      \hfill
      \parbox[t]{0.4\textwidth}{
        \raggedleft Course Syllabus\\
        \csemester}
    \end{minipage}

    \vfill

    {\textsc{\MakeLowercase\ccode}\\[1ex]
      \bfseries\cseries\Huge\ctitle}

    \vfill

    \normalsize\rmfamily % switch back to body type

    \begin{tabular}{>{\bfseries}rl>{\bfseries}rl}
      \toprule
      Instructor & \prof, \pdegree & Course  & Version \cversion \\
      \midrule
      Phone      & \pphone         & Meets   & \cmeetson         \\
      Email      & \pemail         & Time    & \cmeetsat         \\
      Office     & \poffice        & Room    & \cmeetsin         \\
      Hours      & \phours         & Website & \cwebsite         \\
      \bottomrule
    \end{tabular}

    \vfill

    \begin{description}\small
      \item[Commuter Hotline]
        Class cancellations due to inclement weather or illness will
        be announced on the commuter hotline at \texttt{416.226.6620
        x2187}. Alternately, weather cancellation information is posted
        at \href{http://tyndale.ca/weather}{tyndale.ca/weather}.
      \item[MyTyndale.ca]
        This course may have materials stored on its website, such as
        handouts or readings that may be needed in order to complete
        assignments. Students are responsible for checking these course
        pages on a regular basis. Here, too, students are able to view
        their grades throughout the semester. For more information see
        Section~\ref{mytyndale}, below.
      \item[Mail]
        Students are responsible for information communicated through
        their campus mailboxes and student e-mail accounts. A mailbox
        directory hangs beside the mailboxes. For more information
        contact the Registrar's office.
    \end{description}

  \end{center}

  \section{Course Description}
  \label{description}

  \emph{From the Academic Calendar:} \cdescrip

\end{titlepage}
\setcounter{page}{2} % count the title page as page 1


\section{Learning Objectives}
\label{objectives}
\edobject

\section{Required Texts \& Materials}
\label{texts}

The following texts are required. Library copies may be placed on reserve, but students are strongly encouraged to purchase their own copies. Visit the professor's website for links to
\href{https://danieldriver.com/courses/hbst-3001/}{order the assigned editions}.

\begingroup
\renewcommand{\section}[2]{}% temporarily remove the section heading
\begin{thebibliography}{Anderson}% use the longest item in the bibliography

	\bibitem[Anderson]{anderson} Gary A. Anderson.
	\emph{Christian Doctrine and the Old Testament: Theology in the Service of Biblical Exegesis}.
	Grand Rapids: Baker Academic, 2017.
	\textsc{isbn} 978-0801098253.

	\bibitem[Davis]{davis} Ellen F. Davis.
	\emph{Opening Israel’s Scriptures}.
	Oxford: Oxford University Press, 2019.
	\textsc{isbn} 978-0190948948.

	\bibitem[NRSV]{nrsv} M.\,D. Coogan, ed.
	\emph{New Oxford Annotated Bible with Apocrypha: NRSV}. 5th ed.
	Oxford: Oxford University Press, 2018.
	\textsc{isbn} 978-0190276072.

%	Assigned in Summer 2017 - students struggled with it, though
%
%	\bibitem[MacDonald]{macdonald} Neil B. MacDonald.
%	\emph{Metaphysics and the God of Israel: Systematic Theology of the Old and New Testaments}.
%	Grand Rapids: Baker Academic, 2006.
%	\textsc{isbn} 978-0801032431.

%	\bibitem[tk]{tk} tk.
%	\emph{tk}.
%	tk: tk, 20tk.
%	\textsc{isbn} 978-tk.

\end{thebibliography}
\endgroup

An good alternative study Bible is the NJPS: Adele Berlin and Marc Zvi
Brettler, eds., \emph{The Jewish Study Bible: Second Edition} (Oxford:
Oxford University Press, 2014). Reference copies of both the NRSV and
NJPS are available in the AST library. Electronic versions are also
available through
\href{http://ezproxy.astheology.ns.ca:2048/login?url=http://www.oxfordbiblicalstudies.com/}{Oxford Biblical Studies Online} (accessible with your AST student card barcode number and password).

\section{Supplementary Texts}
\label{supplementary}

Supplementary readings will be recommended throughout the semester and
either placed on reserve or made available through the course website.
Give this material good effort and attention. As a foundation for this
course, students are asked to explore relevant church statements about
Christian scripture, particularly the Old Testament. Resources from the
founding churches of AST include the following.

\begin{enumerate}

\item RCC: Roman Catholic Church

	\begin{enumerate}
	\item \href{}{\emph{Divino Afflante Spiritu}, the encyclical from Pope Pius XII [\href{http://www.bc.edu/schools/stm/crossroads/resources/deathofjesus/intro/the_catholic_approachtothebible.html}{overview}] (1943)}
	\item \href{}{\emph{Dei Verbum}, Vatican II's Dogmatic Constitution on Divine Revelation (1965)}
	\item PBC = Pontifical Biblical Commission: \emph{L'interprétation de la Bible dans l'Église}, \href{http://www.catholic-resources.org/ChurchDocs/PBC_Interp.htm}{The Interpretation of the Bible in the Church} (April 15, 1993)
	\item PBC: \emph{Le peuple juif et ses Saintes Écritures dans la Bible chrétienne}, \href{http://www.vatican.va/roman_curia/congregations/cfaith/pcb_documents/rc_con_cfaith_doc_20020212_popolo-ebraico_en.html}{The Jewish People and Their Sacred Scriptures in the Christian Bible} (May 24, 2001)
	\end{enumerate}

\item ACC: Anglican Church of Canada (a \href{https://www.anglicancommunion.org/structures/member-churches.aspx}{member of the Anglican Communion})

	\begin{enumerate}
	\item Articles VI–VII of \href{https://www.churchofengland.org/prayer-worship/worship/book-of-common-prayer/articles-of-religion.aspx}{The Thirty-nine Articles of Religion} (1562)
	\item \href{http://anglicansonline.org/basics/Chicago_Lambeth.html}{The Chicago-Lambeth Quadrilateral} (1886, 1888)
	\item Paragraphs 52–62 of \href{http://www.anglicancommunion.org/resources/document-library.aspx?author=The+Windsor+Process&language=English}{The Windsor Report} [\href{http://www.anglicancommunion.org/media/68225/windsor2004full.pdf}{PDF}] (18 October 2004)
	\item \href{http://www.anglicancommunion.org/media/98131/Final-Report-for-the-web.pdf}{Bible in the Life of the Church (BILC) Project Report} (2012)
	\end{enumerate}

\item UCC: United Church of Canada

	\begin{enumerate}
	\item The UCC requires its ministers to be in ``\href{http://www.united-church.ca/community-faith/welcome-united-church-canada/what-we-believe}{essential agreement}'' with four doctrinal statements that are ``\href{http://www.united-church.ca/community-faith/welcome-united-church-canada/faith-statements}{subordinate to the primacy of scripture}'' (\href{https://commons.united-church.ca/Documents/What%20We%20Believe%20and%20Why/Theology%20and%20Mission%20of%20the%20Church/Our%20Words%20of%20Faith.pdf#search=study}{GC 41 2012 - 005}):
		\begin{enumerate}
			\item\href{http://www.united-church.ca/community-faith/welcome-united-church-canada/twenty-articles-doctrine-1925}{Twenty Articles of Doctrine} (1925)
			\item\href{http://www.united-church.ca/community-faith/welcome-united-church-canada/statement-faith-1940}{A Statement of Faith} (1940)
			\item\href{http://www.united-church.ca/community-faith/welcome-united-church-canada/new-creed}{A New Creed} (1968)
			\item\href{http://www.united-church.ca/community-faith/welcome-united-church-canada/song-faith}{A Song of Faith} (2006)
		\end{enumerate}
	\item \href{https://ecumenism.net/archive/docu/1992_ucc_authority_interpretation_scripture.pdf}{\emph{The Authority and Interpretation of Scripture: A Statement of the United Church of Canada}} (Toronto: UC Publishing House, 1992) [BS 480 U65 1992 -- \textbf{on reserve}]
	\item See Robert C. Fennell's series “How Does the United Church Interpret the Bible?” in \href{http://touchstonecanada.ca}{\emph{Touchstone}}, a Canadian journal of theology and heritage:
		\begin{enumerate}
			\item\href{http://touchstonecanada.ca/wp-content/uploads/2013/08/Article2_2008-051.pdf}{Part I, 1904–1940s: Tradition and Resistance} (26.2 [2008]: 13--24)
			\item\href{http://touchstonecanada.ca/wp-content/uploads/2013/08/3rdArticle_2008-09.pdf}{Part II, 1950s–1990s: Tradition and Liberation} (26.3 [2008]: 31--42)
			\item\href{http://touchstonecanada.ca/wp-content/uploads/2013/08/Touchstone_May_2011.pdf}{Part III: A Song of Faith} (29.2 [2011]: 21--29)
		\end{enumerate}

		\item From an OT scholar, see also: Gerald T. Sheppard, \emph{The Future of the Bible: Beyond Liberalism and Literalism} (Toronto: United Church Publishing House, 1990).

	\end{enumerate}

\end{enumerate}

Also, the following reference works are worth owning and consulting.
\cite{rlgs} in particular contains sound advice on core skills like
reading religious texts, writing essays and reviews, revising essays,
making oral presentations, and learning languages.

\begingroup
\renewcommand{\section}[2]{}% temporarily remove the section heading
\begin{thebibliography}{Making Sense}% use the longest item in the bibliography

	\bibitem[Making Sense]{rlgs} Northey, Margot, Bradford A. Anderson, and Joel N. Lohr.
	\emph{Making Sense in Religious Studies: A Student's Guide to Research and Writing}.
	3rd ed. Toronto: Oxford University Press, 2019. ISBN 978-0199026838.

	\bibitem[SBL2]{sbl2} Collins, Billie Jean, et al.
	\emph{The SBL Handbook of Style}.
	2nd ed. Atlanta: SBL Press, 2014. ISBN 978-1589839649.
	Designed to augment \href{http://www.chicagomanualofstyle.org/home.html}{\emph{Chicago Style}}
	(the standard at AST), there is also a free
	\href{https://www.sbl-site.org/assets/pdfs/pubs/SBLHSsupp2015-02.pdf}{Student Supplement for SBL2}.

\end{thebibliography}
\endgroup


\section{Course Outline}
\label{outline}

We will adhere to the schedule in \autoref{schedule} as closely as
possible, though the professor reserves the right to adjust it to suit
the needs of the class.

\setcolumncount{6}% set up \sessioncount, \unit{}, \noclass{}, and \reminder{memo}{date} macros
\begin{table}[htbp]% set to `p' to put the schedule on its own page
  \centering
  \begin{tabular}{>{\sessioncount.}r@{ }lcccr}% make sure the column config agrees with \setcolumncount
	\toprule
	\sessionskip{\textbf{\S}.}&\textbf{Topic}&\textbf{Anderson}&\textbf{Davis}&\textbf{OT Scripture}&\textbf{Date}\\
	\midrule

	\unit{Theology \emph{\&} Hermeneutics} \\
		& Introductions       & Intro & Intro &             & 13 Jan. \\
		& Apophaticism        & 1     & 3     & Lev 8--10   & 20 Jan. \\
		& Impassibility       & 2     & 10    & Jonah 1--4  & 27 Jan. \\
	\unit{Origins} \\
		& Creation            & 3     & 1     & Gen 1--3    &  3 Feb. \\
		& Sin                 & 4     & 2     & Exod 32--34 & 10 Feb. \\
		& Election            & 5     & 4--5  & Gen 50      & 17 Feb. \\% Anderson 88n24: Does Joseph forgive? Are the brothers fully reconciled?
	\noclass{Reading Week}                                  & 24 Feb. \\
	\unit{Christology} \\
		& Christology 1       & 6     & 9    & Exod 40      &  3 Mar. \\
		& Mariology           & 7     & 8    & 1 Sam 1--2   & 10 Mar. \\
		& Christology 2       & 8     & 11   & Tobit 1--4   & 17 Mar. \\
	\unit{Sanctification} \\
		& Merits              & 9     & 15   & Prov 19      & 24 Mar. \\
		& Purgation           & 10    & 17   & Dan 4        & 31 Mar. \\
		& Conclusions         &       & 14   & Ps 19        &  7 Apr. \\
	\reminder{End of Term: Final marks due for all courses}{13 Apr.} \\

	\bottomrule
  \end{tabular}
  \caption{Schedule of Readings}
  \label{schedule}
\end{table}


%	\reminder{Articles of Faith \textbf{due} at the start of class four}{} \\
%	\reminder{Paper is \textbf{due} at the start of class ten}{}  \\



See the AST website for a list of other \href{http://www.astheology.ns.ca/students/academic-dates.html}{important dates}.

\section{Evaluation}
\label{evaluation}

\ProvidesFile{grades.tex}[2022/01/11 v1.0 -- Course policy]

% Tutorial-style assessment, adapted from Evangeline Kozitza Dean's
% policy at Oxford, which she said was adapted from others in turn

Given the small class size this term, I have adopted a tutorial format
for \ccode.

\begin{enumerate}

	\item Each week you will write a 1000–1200 word essay responding to
	a prompt.

	\item Essays should interact thoughtfully with both primary and
	secondary sources, and should be \textbf{thesis-driven}. Your essay
	must argue a thesis, and not simply describe what you have learned
	or what others have argued. At the tutorial, you should know what
	you think about the question, and should be able to summarize the
	argument of your essay in one sentence.

	\item Reading preparation should always give priority of attention
	to the assigned primary texts. Secondary literature will be
	important in lending guidance to the essay, but scholarly opinions
	should always be assessed on the basis of how helpful they are in
	explaining the primary text.

	\item By email or Teams, please send me your essay as a Microsoft
	Word document by 4pm the day before class. Essays received on time
	will receive written feedback. Late essays are not guaranteed
	written feedback. If you do not provide an essay at all, class may
	be cancelled. Essays will not be accepted after class (with rare
	exceptions).

	\item Written feedback will be returned by email or on Teams
	following discussion in the tutorial. Marks will not be assigned to
	the essays submitted on a weekly basis. Instead, affirmation and
	redirection will be provided as needed.

	\item Select two of your best essays to revise and resubmit for a
	formal mark. Both essays must be submitted before the last day of
	class. You can opt to submit them early.

\end{enumerate}

The breakdown for the semester's total work is shown in
\autoref{grade-dist}.

\begin{table}[htbp]
  \centering
  {\lining
  \begin{tabular}{lr}
    \toprule
    Completion of All Essays      & 70\% \\
    Assessment of Selected Essays & 30\% \\
    \bottomrule
  \end{tabular}}
  \caption{Distribution of Grades}
  \label{grade-dist}
\end{table}


% Evaluation scheme from Summer 2017 intensive
%
%\begin{enumerate}% 200*7+250*8 = 3400 words, plus the annotated articles
%
%	\item Before each class, complete the assigned reading and write a
%	200--250 word (one page) \textbf{reading response}. Use the exercise
%	to elaborate, reflect on, test, challenge, or extend an idea from
%	the reading that you find noteworthy. Type your responses, write in
%	complete sentences, and combine those sentences into one or two
%	healthy paragraphs. Due at the start of \textbf{each class}, unless
%	it is the first class or another graded assignment is due. Students
%	who submit fewer than 6 responses cannot earn an “A” for the course.
%
%	\item On one sheet of paper, annotate a list of between 7 and 12
%	essential \textbf{articles of faith} that Christians associate with
%	Old Testament scripture. For guidance on what counts as “essential”
%	you may consult various catechisms (Catholic, Lutheran), creeds
%	(Apostles', Nicene, New), and other official documents (Articles,
%	Encyclicals, Statements). In your annotations, cite OT texts that
%	seem to speak to each of the doctrines you select, whether for or
%	against or both. Due at the start of \textbf{class four}.
%
%	\item A \textbf{final paper} allows you to explore one Christian
%	Doctrine of your choice in terms of its relationship to Old
%	Testament scripture. The paper must have a clear thesis and be at
%	least 2,000 words (eight pages) long. It should interact with the
%	assigned reading and cite at least four secondary sources in total.
%	Due at the start of the \textbf{last class}.
%
%\end{enumerate}
%
%Note that this course is an intensive. Its format is highly condensed.
%In a two week period students are expected to attend 30 hours of class
%and to spend a further 50 hours \emph{or more} in independent study.
%Students are therefore encouraged not to have other major commitments
%during the course session so that they can have the best chance of
%success.
%
%The breakdown for the course's total work is shown in
%\autoref{grade-dist}.
%
%\begin{table}[htbp]
%  \centering
%  {\lining
%  \begin{tabular}{lr}
%    \toprule
%    Reading Responses  & 35\% \\
%    Annotated Articles & 20\% \\
%    Final Paper        & 45\% \\
%    \bottomrule
%  \end{tabular}}
%  \caption{Distribution of Grades}
%  \label{grade-dist}
%\end{table}

\ProvidesFile{grades.tex}[2016/09/03 v2.0 -- Course policy]

\subsection{Grading System at AST}
\label{grades}

AST's \href{http://www.astheology.ns.ca/webfiles/AST_2016Calendar_web(A5)-06APR2016.pdf}{Academic
Calendar} provides guidelines and detailed criteria for academic
assessment. Marks are assigned by letter grade using the benchmarks in
\autoref{grade-syst}.

\begin{table}[htbp]
  \centering
  {\lining
  \begin{tabular}{lll}
    \toprule
%    Letter      & Percent & Assessment        \\
%	\midrule
    A+          & 94--100    & Exceptional    \\
    A           & 87--93     & Outstanding    \\
    A\char"2212 & 80--86     & Excellent      \\ [1ex]
    B+          & 77--79     & Good           \\
    B           & 73--76     & Acceptable     \\
    B\char"2212 & 70--72     & Marginal       \\ [1ex]
    C           & 60--69     & Unsatisfactory \\
    F           & 0--59      & Failure        \\
    FP          & 0          & Failure due to Plagiarism \\
    \bottomrule
  \end{tabular}}
  \caption{Summary of Grading System}
  \label{grade-syst}
\end{table}

% More detailed grading criteria from pp. 61--62 of `16.0406-I2-AST Academic Calendar.pdf'
%
%\begin{description}
%  \item[A+ (94-100) ‘Exceptional’]
%    A superior performance with consistent evidence of a comprehensive,
%    incisive grasp of all aspects of the subject matter; a very wide
%    knowledge base; insightful critical evaluation and analysis of the
%    material; an exceptional capacity for original, creative, and/or
%    logical thinking; an exceptional ability to organize, analyse,
%    synthesize, and to express thoughts fluently.
%  \item[A (87-93) ‘Outstanding’]
%    A comprehensive grasp of the subject matter, outstanding evidence of
%    original thought; sound critical evaluation of the material; an
%    excellent ability to organize, analyse, synthesize and to express
%    thoughts; mastery of an extensive knowledge base.
%  \item[A- (80-86) ‘Excellent’]
%    All the qualities of a B-level performance and an excellent capacity
%    for original, creative, and/ or logical thinking; excellent ability
%    to organize, analyse, synthesize, and integrate ideas; broad
%    knowledge base in the subject matter.
%  \item[B+ (77-79) ‘Good’]
%    A good performance with substantial knowledge of the subject matter;
%    a very good understanding of the relevant issues; familiarity with
%    relevant literature and techniques; good ability to organize,
%    analyse, and examine the material in a constructive and critical
%    manner.
%  \item[B (73-76) ‘Acceptable’]
%    A generally adequate performance with a good knowledge of the
%    subject matter; a fair understanding of relevant issues; some
%    ability to work with relevant literature and techniques; some
%    ability to develop solutions to difficult problems related to the
%    subject material.
%  \item[B- (70-72) ‘Marginally Acceptable’]
%    Some familiarity with the subject material; some understanding.
%    Satisfactory understanding of relevant issues; attempts to solve
%    moderately difficult problems related to the subject material in a
%    critical and analytical manner are only partially successful.
%  \item[C (60-69) ‘Unsatisfactory’]
%    A C grade indicates unsatisfactory academic performance. At the
%    discretion of the instructor, supplemental work may be negotiated to
%    upgrade the mark to a B range. A student may carry two C grades
%    without penalty in all courses except Foundations Courses,
%    Supervised Field Education, Supervised Ministry Practicum and the
%    Graduate Project. In these courses, a minimum grade of B- is
%    required to graduate. A student who receives a C in a Foundation
%    course must repeat the course to achieve a B- or better, and cannot
%    use the C grade to meet prerequisite requirements for advanced
%    courses. If the student repeats one of these courses and receives a
%    B- or better, the previous C grade remains on the transcript and can
%    be counted toward the total of unsatisfactory grades that may lead
%    to academic dismissal. Credit will be given only once for any
%    course. (See Policy on Unsatisfactory Academic Performance in the
%    AST Student Handbook.)
%  \item[F (0-59) ‘Failure’]
%    Student has not grasped subject matter; does not understand issues
%    involved; cannot work with relevant literature. (See Policy on
%    Unsatisfactory Academic Performance in the AST Student Handbook.)
%  \item[P ‘Pass’]
%    Credit awarded, but no mark assigned.
%  \item[FP ‘Failure due to Plagiarism’]
%    A student will receive this grade only after proven incident(s) of
%    plagiarism in a course.
%\end{description}
\ProvidesFile{other.tex}[2022/06/08 v2.9.1 -- Course policy]

\section{Other Course Policy}
\label{policy}

Late work will not be accepted, except in genuinely extenuating
circumstances. Students must submit something before the deadline if
they wish to receive credit. Unless I state otherwise, assignments are
to be uploaded by 11:59 \PM\ (Atlantic) on the date indicated.

Essay submissions must be typewritten and double-spaced. They should be
free from error. In this course they should follow SBL Style (see
\cite{sbl2} in \autoref{supplementary}, above). As a reminder, AST
upholds an Inclusive Language Policy. Please use gender-inclusive
language when referring to human beings. Our traditions have different
norms for speech about God; you are of course free to follow and explore
those traditions when referring to God.


Plagiarism is the
\href{http://www.eerdmans.com/Pages/Item/59043/Commentary-Statement.aspx}{failure}
to \href{https://www.theguardian.com/world/2013/feb/09/german-education-minister-quits-phd-plagiarism}{attribute}
(by means of footnotes when writing or aloud when speaking) any ideas,
phrases, sentences, materials, syntheses, et cetera, that another author
has composed and that you have borrowed for your own work. Plagiarism is
unethical. Academic penalties for plagiarism at AST are serious, and may
include failure of the course or even suspension of further studies.
Unintentional plagiarism is considered plagiarism. AST's Plagiarism
Policy is found under that heading in the Academic
Calendar.

Students should request permission to record a class or lecture. If
permission is granted, or if recordings are provided (as in the case of
an online or hybrid course), I stipulate that all recordings be for
personal use only. They may not be shared or distributed.

If you have needs that require modifications to any aspect of this
course, please consult with the instructor as soon as possible. Any
documentation regarding disabilities that you wish to divulge to AST
should be provided to the Registrar’s Office, where it will be kept in a
confidential file.

Finally, I encourage the conscientious use of laptops, tablets, and
other technology in my classes. In classroom settings, realize that, as
\href{http://dx.doi.org/10.1016/j.compedu.2012.10.003}{cognitive
psychologists have demonstrated}, ``laptop multitasking hinders
classroom learning for both users and nearby peers.'' Do your part to
foster an environment for dialogue by honouring the presence of your
classmates. In online and hybrid settings, consider both the physical
environment in which you choose to work and the virtual environment that
you help create through your participation in various forums. Let your
engagement in this course be marked by rigour and charity alike.


\end{document}
