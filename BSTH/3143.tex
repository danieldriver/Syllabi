% !TEX TS-program = XeLaTeX
% !TEX encoding = UTF-8 Unicode
% Copyright (c) 2014 by Daniel R. Driver. All rights reserved.

\documentclass[titlepage]{article}

\newcommand\policy{../policy}
\newcommand\incl{../includes}
\ProvidesFile{variables.tex}[2018/05/24 v2.1 -- Syllabus variables]

\usepackage{xspace} % make manual spaces (like \mycmd\ ) unnecessary
\usepackage{xifthen} % provides \isempty test

% variables for internal use
\newcommand\prof{}
\newcommand\pdegree{}
\newcommand\pphone{}
\newcommand\pemail{}
\newcommand\poffice{}
\newcommand\phours{}
%
\newcommand\ccode{}
\newcommand\ctitle{}
\newcommand\cseries{}
\newcommand\cversion{}
\newcommand\csemester{}
\newcommand\cmeetson{}
\newcommand\cmeetsat{}
\newcommand\cmeetsin{}
\newcommand\cwebsite{}
\newcommand\cdescrip{}
\newcommand\cprereqs{}
\newcommand\edobject{}

% in case of fully online courses - https://tex.stackexchange.com/a/5896
\newif\ifonline
\newcommand\Int[2]{\ifonline#1\else#2\fi}

% commands for setting variables in the preamble
\newcommand\professor[2][PhD]{
  \renewcommand\pdegree{#1\xspace}
  \renewcommand\prof{#2\xspace}}
\newcommand\phone[1]{
  \renewcommand\pphone{\addfontfeatures{Numbers=Monospaced}#1\xspace}}
\newcommand\email[1]{
  \renewcommand\pemail{\href{mailto:#1}{#1}\xspace}}
\newcommand\officehours[2][Library, Room 5-North]{
  \renewcommand\poffice{#1\xspace}
  \renewcommand\phours{#2\xspace}}
%
\newcommand\coursecode[2][1.0]{
  \renewcommand\cversion{#1\Int{-i}{}\xspace}
  \renewcommand\ccode{#2\Int{(Int)}{}\xspace}}
\newcommand\coursetitle[2][]{
  \ifthenelse{\isempty{#1}}%
    {}% do nothing if #1 is empty, else:
    {\renewcommand\cseries{#1\\[1ex]}}
  \renewcommand\ctitle{#2\xspace}}
\newcommand\semester[1]{
  \renewcommand\csemester{#1\xspace}}
\newcommand\meets[3]{
  \newcommand\AM{\textsc{am}}
  \newcommand\PM{\textsc{pm}}
  \renewcommand\cmeetson{#1\xspace}
  \renewcommand\cmeetsat{\Int{From 9:00 \AM}{#2}\xspace}
  \renewcommand\cmeetsin{\Int{\href{https://smu.brightspace.com/d2l/login}{Brightspace}}{#3}\xspace}}
\newcommand\website[1]{
  \renewcommand\cwebsite{\href{http://#1}{#1}\xspace}}
\newcommand\cdescription[2][RM 1000 or GTRS 6000; and BF 1001]{
  \renewcommand\cprereqs{#1}
  \renewcommand\cdescrip{#2\par}}
\newcommand\objectives[1]{
  \renewcommand\edobject{#1\par}}


\coursecode[3.0-WIP]{BSTH 3143}
\coursetitle{Psalms in Interpretation}

% Previously taught in: Fall 2008, Winter 2011 as RLGS 3263 (Text &
% Interpretation): The Psalms in the Christian Tradition

\professor{Daniel R. Driver}
\phone{416.226.6620 x2201}
\email{ddriver@tyndale.ca}
\officehours{Mon. \& Thur., 1:30--3:30 \PM}

\semester{Fall 2014}
\meets{Mondays \& Thursdays}
      {11:15 \AM--12:35 \PM}
      {Ballyconnor 2083}
\website{classes.tyndale.ca}
\cdescription{% copy from tyndale.ca/registrar/calendar
  Examines the hymns of the people of Israel with regard to their
  theology and literary types of the Psalms. The arrangement of the
  Psalms and the history of reception of specific Psalms will form a
  significant portion of the course. Prerequisites: \textsc{bsth} 101,
  102, 201.
}

\ProvidesFile{preamble.tex}[2013/09/06 v1.0 -- Syllabus preamble]

% basic typography
\usepackage{fontspec}
\setmainfont[Ligatures=TeX]{Meta Serif Pro}
\setsansfont[Ligatures=TeX]{Meta Pro}
\newfontfamily\Heb{Meta Hebrew}
\setmonofont[Scale=MatchLowercase]{Menlo}
\usepackage{sectsty}
\allsectionsfont{\sffamily}
\frenchspacing
\setlength{\emergencystretch}{3em} % prevent overfull lines

% custom font size and leading
\renewcommand\tiny{\fontsize{6}{9}\selectfont}
\renewcommand\scriptsize{\fontsize{7}{10}\selectfont}
\renewcommand\footnotesize{\fontsize{8}{11}\selectfont}
\renewcommand\small{\fontsize{8.5}{11.5}\selectfont}
\renewcommand\normalsize{\fontsize{9}{12}\selectfont}% base size
\renewcommand\large{\fontsize{11}{14}\selectfont}
\renewcommand\Large{\fontsize{13}{16}\selectfont}
\renewcommand\LARGE{\fontsize{16}{19}\selectfont}% "course syllabus \\ semester" benefits from more lead
\renewcommand\huge{\fontsize{19}{21}\selectfont}
\renewcommand\Huge{\fontsize{24}{26}\selectfont}

% layout packages: page, logo, tables
\usepackage[scale={0.6,0.8},
            xetex]{geometry}
\usepackage{graphicx}
\usepackage{array}     % allow insertions of column styling with >{}
\usepackage{booktabs}  % elegant horizontal rules in tables
\usepackage{marginfix} % protect positioning of margin table in policy/grades

% custom macros for a session count in the schedule of readings
\newcounter{session}
\newcounter{columns}
\newcounter{courseunit}
\newcommand\setcolumncount[2][0]{ % optionally set count to other than 0,
  \setcounter{session}{#1}        % e.g. to -1, or to a standing count
  \setcounter{columns}{#2}}
\newcommand\sessioncount{\stepcounter{session}\arabic{session}}
\newcommand\sessionskip[1]{\multicolumn{1}{@{}r@{ }}{#1}}
\newcommand\unit[1]{\multicolumn{\thecolumns}{c}{%
  \scshape\stepcounter{courseunit}\roman{courseunit}. \MakeLowercase{#1}}}
\newcommand\noclass[1]{\multicolumn{1}{@{}l}{\itshape No Class: #1}}

% color to match Tyndale's branding
\usepackage[usenames]{xcolor}
% predefined: black, white, red, green, blue, cyan, magenta, yellow
\definecolor{TyndaleURLs}{HTML}{0062A0} % links on tyndale.ca
\definecolor{TyndaleBlue}{cmyk}{1,1,0,.32}
\definecolor{TyndaleGold}{cmyk}{0,.27,1,0}
\definecolor{TyndaleRed}{cmyk}{0,1,.99,.04}
\definecolor{TyndaleBlack}{cmyk}{0,0,0,1}
\definecolor{TyndaleGreen}{cmyk}{.45,0,1,.24}
\definecolor{TyndaleOrange}{cmyk}{0,.79,1,0}
\definecolor{TyndaleAqua}{cmyk}{.47,0,.24,0}
\definecolor{TyndaleYellow}{cmyk}{.03,.03,.35,0}

% metadata (assumes a host of definitions are made in the main file)
\usepackage[setpagesize=false,     % leave this to geometry
            hyperfootnotes=false,  % fragile and distracting
            xetex]{hyperref}
\hypersetup{breaklinks=true,       % allow link text to break across lines
            colorlinks=true,       % colorlinks resets pdfborder to 0 0 0
            urlcolor=TyndaleURLs,  % for external links
            linkcolor=TyndaleRed,  % for normal internal links
            citecolor=TyndaleGold, % for bibliographical citations in text
            pdfauthor={\prof},
            pdftitle={\ccode: \ctitle},
            pdfsubject={Tyndale UC, \csemester},
            pdfcreator={github.com/danieldriver/syllabus}}
\urlstyle{same}                    % don't use monospace font for urls

% custom footlines
\usepackage{fancyhdr}
\pagestyle{fancy} % turn it on
\fancyhf{}        % reset everything
\renewcommand{\headrulewidth}{0pt} % remove header line as well
\lfoot{\sffamily\scshape\footnotesize\MakeLowercase{\ctitle, v\cversion}}
\rfoot{\sffamily\scshape\footnotesize\MakeLowercase{\prof\quad\thepage}}

% gratuitous with custom title page, but useful as a fallback
\title{\ccode: \ctitle}
\author{\professor}
\date{\semester}


\begin{document}
\ProvidesFile{title.tex}[2013/09/06 v1.0 -- Syllabus title page]

\begin{titlepage}
  \begin{center}

    \LARGE\sffamily % set title elements in a large sans serif

    \begin{minipage}{\textwidth}
      \parbox[t]{0.5\textwidth}{
        \mbox{}\\[-13pt] % dummy line to align parboxes
        \includegraphics[width=0.5\textwidth]{.syllabus/includes/TyndaleUC}}
      \hfill
      \parbox[t]{0.4\textwidth}{
        \raggedleft Course Syllabus\\
        \csemester}
    \end{minipage}

    \vfill

    {\textsc{\MakeLowercase\ccode}\\[1ex]
      \bfseries\cseries\Huge\ctitle}

    \vfill

    \normalsize\rmfamily % switch back to body type

    \begin{tabular}{>{\bfseries}rl>{\bfseries}rl}
      \toprule
      Instructor & \prof, \pdegree & Course  & Version \cversion \\
      \midrule
      Phone      & \pphone         & Meets   & \cmeetson         \\
      Email      & \pemail         & Time    & \cmeetsat         \\
      Office     & \poffice        & Room    & \cmeetsin         \\
      Hours      & \phours         & Website & \cwebsite         \\
      \bottomrule
    \end{tabular}

    \vfill

    \begin{description}\small
      \item[Commuter Hotline]
        Class cancellations due to inclement weather or illness will
        be announced on the commuter hotline at \texttt{416.226.6620
        x2187}. Alternately, weather cancellation information is posted
        at \href{http://tyndale.ca/weather}{tyndale.ca/weather}.
      \item[MyTyndale.ca]
        This course may have materials stored on its website, such as
        handouts or readings that may be needed in order to complete
        assignments. Students are responsible for checking these course
        pages on a regular basis. Here, too, students are able to view
        their grades throughout the semester. For more information see
        Section~\ref{mytyndale}, below.
      \item[Mail]
        Students are responsible for information communicated through
        their campus mailboxes and student e-mail accounts. A mailbox
        directory hangs beside the mailboxes. For more information
        contact the Registrar's office.
    \end{description}

  \end{center}

  \section{Course Description}
  \label{description}

  \emph{From the Academic Calendar:} \cdescrip

\end{titlepage}
\setcounter{page}{2} % count the title page as page 1


\section{Learning Outcomes}
\label{outcomes}

% A Model of Learning Objectives
% Adapted from http://www.celt.iastate.edu/teaching/RevisedBlooms1.html
%
% A learning objective should contain a *subject* (a verb related to
% cognitive process) and an *object* (a noun identifying the knowledge
% or skill sought). Statements should be precise and measurable, e.g.,
% "By the end of this course, students will be able to <verb> <noun>."
%
% Anderson and Krathwohl name four *classes of knowledge* that students
% can be expected to acquire. Terms move from the concrete to the
% abstract:
%
%   1. factual           (basic elements, details, terminology)
%   2. conceptual        (interrelationship of basic elements)
%   3. procedural        (skills, methods, techniques, criteria)
%   4. metacognitive     (strategy, context and conditions, self-knowledge)
%
% Metacognitive knowledge is a special case. It concerns one's own
% "cognition and about oneself in relation to various subject matters"
% (Anderson and Krathwohl, 2001, p.44).
%
% A&K also identify 19 *cognitive processes*, arranged in six categories
% (a 2001 revision of Bloom's 1957 taxonomy) that progress from lower
% order to higher order thinking skills (from LOTS to HOTS):
%
%   1. remember
%      * recognizing     (identifying)
%      * recalling       (retrieving)
%   2. understand
%      * interpreting    (clarifying, paraphrasing, representing, translating)
%      * exemplifying    (illustrating, instantiating)
%      * classifying     (categorizing, subsuming)
%      * summarizing     (abstracting, generalizing)
%      * inferring       (concluding, extrapolating, interpolating, predicting)
%      * comparing       (contrasting, mapping, matching)
%      * explaining      (constructing models)
%   3. apply
%      * executing       (carrying out)
%      * implementing    (using)
%   4. analyze
%      * differentiating (discriminating, distinguishing, focusing, selecting)
%      * organizing      (finding coherence, integrating, outlining, parsing, structuring)
%      * attributing     (deconstructing)
%   5. evaluate
%      * checking        (coordinating, detecting, monitoring, testing)
%      * critiquing      (judging)
%   6. create
%      * generating      (hypothesizing)
%      * planning        (designing)
%      * producing       (constructing)
%
% If you reference a List of Measurable Verbs Used to Assess Learning
% Outcomes, remember that A&K reverse Bloom's 6th and 5th categories.

The Psalms of the Hebrew Bible/Old Testament have always played an
important part in the life of the church and synagogue. A major goal of
this course is therefore to gain some historical perspective on how the
Psalter has fed and informed Jewish and Christian faith in various
periods: in biblical times, Rabbinic and Patristic periods, the Middle
Ages, the Reformation, the Enlightenment, and finally the early and late
modern periods. Informed by recent research into the shape and shaping
of the Psalter we will also consider the arrangement of this great
anthology.

The course meets twice a week, which facilitates two main foci. Mondays
are dedicated to hermeneutical and historical factors involved in
reading the Psalter in light of long reception. Thursdays allow us to
put theory into practice by reading particular psalms in the context of
their reception. By the end of the course students should be able to:
interpret at least four distinct periods in the reception of the
Psalter; analyze key differences between ancient and modern
interpreters, and a variety of Jewish and Christian interpreters;
produce a sustained discussion of a psalm that interacts with its text
and interpretation; recite at least ten verses of a psalm; respond
creatively to a psalm in a way that demonstrates awareness of its impact.

\section{Required Texts \& Materials}
\label{texts}

All required texts are available for purchase in the Tyndale Bookstore.

\begingroup
\renewcommand{\section}[2]{}% temporarily remove the section heading
\begin{thebibliography}{Gillingham}% use the longest item in the bibliography
  \bibitem[Gillingham]{gillingham}
    Susan Gillingham. \emph{Psalms Through the Centuries: Volume One}.
    \href{http://bbibcomm.net/}{Blackwell Bible Commentaries}. Oxford:
    Blackwell, 2008. Repr., Chichester: Wiley-Blackwell, 2012.
  \bibitem[Wenham]{wenham}
    Gordon J. Wenham. \emph{Psalms as Torah: Reading Biblical Song
    Ethically}. Studies in Theological Interpretation. Grand Rapids:
    Baker Academic, 2012.
\end{thebibliography}
\endgroup

\section{Supplementary Texts}
\label{supplementary}

The following articles or book chapters represent a range of scholarly
and theological interest in the Psalms. They will either be available
for download through the course website, or kept on reserve in the
library. Students are not strictly required to read the supplementary
material; then again, students who choose not to read it should not
expect to get an A for the course.

\begin{enumerate}
 \item Katharine Dell, “\href{http://ezproxy.mytyndale.ca:2048/login?url=http://www.oxfordhandbooks.com/view/10.1093/oxfordhb/9780199204540.001.0001/oxfordhb-9780199204540-e-4}{Psalms},” pages 37--51 in M. Lieb et al., eds., \emph{The Oxford Handbook of the Reception History of the Bible} (Oxford: Oxford University Press, 2011).
 \item James Kugel, “The Psalms of David,” pages 458–473 in \emph{How to Read the Bible: A Guide to Scripture, Then and Now} (New York/Toronto: Free Press, 2007).
 \item Rolf Rendtorff, “The Psalms of David: David in the Psalms,” pages 53–64 in P. Flint and P.~Miller, eds., \emph{The Book of Psalms} (Leiden: Brill, 2005).
 \item Hermann Gunkel, “The Religion of the Psalms,” pages 134–167 in \emph{Water for a Thirsty Land} (Minneapolis: Fortress, 2001). German original published in 1922.
 \item Hans-Joachim Kraus, “The Psalms in the New Testament,” pages 177–203 in \emph{Theology of the Psalms} (Minneapolis: Augsburg, 1986). German original published in 1979.
 \item Brevard Childs, “Psalm 8 in the Context of the Christian Canon,” pages 151–163 in \emph{Biblical Theology in Crisis} (Philadelphia: Westminster, 1970).
 \item John O’Keefe, “‘A Letter that Killeth’: Toward a Reassessment of Antiochene Exegesis, or Diodore, Theodore and Theodoret on the Psalms,” \emph{Journal of Early Christian Studies} 8/1 (2000): 83–104.
 \item Jonathan Magonet, “Through Rabbinic Eyes: Psalm 23,” pages 47–61 in \emph{A Rabbi Reads the Psalms} (2nd ed.; London: SCM, 2004).
 \item Gerald Wilson, “Shaping the Psalter: A Consideration of Editorial Linkage in the Book of Psalms,” pages 72–82 in J.\,C. McCann, ed., \emph{The Shape and Shaping of the Psalter} (JSOTSup 159; Sheffield: Sheffield Academic Press, 1993).
 \item J. Clinton McCann, “Books I–III and the Editorial Purpose of the Hebrew Psalter,” pages 93–107 in J.\,C. McCann, ed., \emph{The Shape and Shaping of the Psalter} (JSOTSup 159; Sheffield: Sheffield Academic Press, 1993).
 \item Gerald Sheppard, “Theology and the Book of Psalms,” \emph{Interpretation} 46 (1992): 143–155.
 \item Erhard Gerstenberger, “Theologies in the Book of Psalms,” pages 603–625 in P. Flint and P. Miller, eds., \emph{The Book of Psalms} (Leiden: Brill, 2005).
\end{enumerate}

\section{Evaluation}
\label{evaluation}

\subsection{Grade Structure for \ccode}
\label{structure}

% Please consider the following structure in your course. Evaluation
% should be clearly linked to Learning Outcomes.
%   - Some form of meaningful evaluation should occur during the first 4
%     weeks of the course; **all students should (and first-year
%     students must) receive an indication of their performance in the
%     first half of the course.**
%   - **Plan your assignments so that each can be returned to students
%     before the next is due.**
%   - Most courses should have a rigorous final exam of 2--3 hours
%     duration. Some upper-year courses may not require a final,
%     depending on pedagogical design and assessment strategy. **Final
%     exams should be appropriately demanding in both breadth and depth
%     to assess accurately the achievement of learning outcomes.**
%   - **No assignments should be due in the last week of classes; final
%     examinations will be scheduled only by the Registrar’s Office.**

\begin{enumerate}
 \item Each week of class will typically begin with a \textbf{reading
   quiz}. These are designed to ensure that you have read the assigned
   material carefully. Quizzes given in class may not be made up in the
   case of absence, though in special cases they may be taken in advance.
 \item Students are to \textbf{memorize 10 verses of a psalm} and recite
   it to me privately, in my office, before the end of tenth week of
   class. Any psalm is allowed unless you have memorized it before
   (Psalm 8 is therefore excluded for many).
 \item Students must \textbf{creatively represent one psalm} in the class
   immediately following that psalm’s workshop (e.g., Psalm 1 at the
   start of Week 3). You can and should use visuals, audio---or whatever
   you can find or create yourself. Be inventive! The only conditions
   are (a) that you explain / demonstrate a connection with the text
   studied and (b) that your presentation last between 5 and 10 minutes.
 \item The last day of each week (Thursdays unless the only class is on
   Monday) will be devoted to a \textbf{seminar-style discussion} of
   individual psalms. Each student will be assigned a commentator for a
   period of 3 weeks, and they are responsible to report on that
   commentator and bring their insight to bear on the discussion. How
   should you prepare?
	\begin{enumerate}
	 \item Read the seminar text closely and carefully. Read it several
       times. Look for distinctive language, themes and images, and stay
       alert to intertextual connections between the main passage and
       other biblical literature. Jot down your observations.
	 \item Consult your designated commentary. It should be read
       carefully, but after you have got a sense for the psalm on its
       own. One good approach might be to read the psalm twice, make
       some preliminary notes, and then to read your commentator before
       filling out the notes.
	 \item Notes will not necessarily be collected, although I may ask to
       see them if there is any doubt about the adequacy of your
       preparation. Remember that you will be called upon to represent
       your unique commentator’s views to the class.
	\end{enumerate}
 \item In lieu of a final exam students are to \textbf{compose an essay}
   on one of the psalms we workshop. Consult all the perspectives
   brought to bear in the seminar on the psalm of your choice, and
   incorporate as many as are expedient for your discussion. The aim is
   to dig deeper into one psalm. This will be facilitated in part by the
   preliminary researches of your classmates. You will need to read more
   widely on your own, however; the seminar is only a starting point.
   You are also expected to develop a strong thesis, and to interact
   significantly with the secondary reading. The final paper will be due
   at the start of the exam period (TBA).

\end{enumerate}

The breakdown for the semester's total work is as follows:

\begin{table}[htbp]
  \centering
  \begin{tabular}{lr}
    \toprule
    Reading Quizzes         & 30\% \\
    Psalm Memorization      & 10\% \\
    Creative Representation & 10\% \\
    Seminar Contributions   & 20\% \\
    Final Paper             & 30\% \\
    \bottomrule
  \end{tabular}
  \caption{Distribution of Grades}
  \label{distribution}
\end{table}

\ProvidesFile{grades.tex}[2016/09/03 v2.0 -- Course policy]

\subsection{Grading System at AST}
\label{grades}

AST's \href{http://www.astheology.ns.ca/webfiles/AST_2016Calendar_web(A5)-06APR2016.pdf}{Academic
Calendar} provides guidelines and detailed criteria for academic
assessment. Marks are assigned by letter grade using the benchmarks in
\autoref{grade-syst}.

\begin{table}[htbp]
  \centering
  {\lining
  \begin{tabular}{lll}
    \toprule
%    Letter      & Percent & Assessment        \\
%	\midrule
    A+          & 94--100    & Exceptional    \\
    A           & 87--93     & Outstanding    \\
    A\char"2212 & 80--86     & Excellent      \\ [1ex]
    B+          & 77--79     & Good           \\
    B           & 73--76     & Acceptable     \\
    B\char"2212 & 70--72     & Marginal       \\ [1ex]
    C           & 60--69     & Unsatisfactory \\
    F           & 0--59      & Failure        \\
    FP          & 0          & Failure due to Plagiarism \\
    \bottomrule
  \end{tabular}}
  \caption{Summary of Grading System}
  \label{grade-syst}
\end{table}

% More detailed grading criteria from pp. 61--62 of `16.0406-I2-AST Academic Calendar.pdf'
%
%\begin{description}
%  \item[A+ (94-100) ‘Exceptional’]
%    A superior performance with consistent evidence of a comprehensive,
%    incisive grasp of all aspects of the subject matter; a very wide
%    knowledge base; insightful critical evaluation and analysis of the
%    material; an exceptional capacity for original, creative, and/or
%    logical thinking; an exceptional ability to organize, analyse,
%    synthesize, and to express thoughts fluently.
%  \item[A (87-93) ‘Outstanding’]
%    A comprehensive grasp of the subject matter, outstanding evidence of
%    original thought; sound critical evaluation of the material; an
%    excellent ability to organize, analyse, synthesize and to express
%    thoughts; mastery of an extensive knowledge base.
%  \item[A- (80-86) ‘Excellent’]
%    All the qualities of a B-level performance and an excellent capacity
%    for original, creative, and/ or logical thinking; excellent ability
%    to organize, analyse, synthesize, and integrate ideas; broad
%    knowledge base in the subject matter.
%  \item[B+ (77-79) ‘Good’]
%    A good performance with substantial knowledge of the subject matter;
%    a very good understanding of the relevant issues; familiarity with
%    relevant literature and techniques; good ability to organize,
%    analyse, and examine the material in a constructive and critical
%    manner.
%  \item[B (73-76) ‘Acceptable’]
%    A generally adequate performance with a good knowledge of the
%    subject matter; a fair understanding of relevant issues; some
%    ability to work with relevant literature and techniques; some
%    ability to develop solutions to difficult problems related to the
%    subject material.
%  \item[B- (70-72) ‘Marginally Acceptable’]
%    Some familiarity with the subject material; some understanding.
%    Satisfactory understanding of relevant issues; attempts to solve
%    moderately difficult problems related to the subject material in a
%    critical and analytical manner are only partially successful.
%  \item[C (60-69) ‘Unsatisfactory’]
%    A C grade indicates unsatisfactory academic performance. At the
%    discretion of the instructor, supplemental work may be negotiated to
%    upgrade the mark to a B range. A student may carry two C grades
%    without penalty in all courses except Foundations Courses,
%    Supervised Field Education, Supervised Ministry Practicum and the
%    Graduate Project. In these courses, a minimum grade of B- is
%    required to graduate. A student who receives a C in a Foundation
%    course must repeat the course to achieve a B- or better, and cannot
%    use the C grade to meet prerequisite requirements for advanced
%    courses. If the student repeats one of these courses and receives a
%    B- or better, the previous C grade remains on the transcript and can
%    be counted toward the total of unsatisfactory grades that may lead
%    to academic dismissal. Credit will be given only once for any
%    course. (See Policy on Unsatisfactory Academic Performance in the
%    AST Student Handbook.)
%  \item[F (0-59) ‘Failure’]
%    Student has not grasped subject matter; does not understand issues
%    involved; cannot work with relevant literature. (See Policy on
%    Unsatisfactory Academic Performance in the AST Student Handbook.)
%  \item[P ‘Pass’]
%    Credit awarded, but no mark assigned.
%  \item[FP ‘Failure due to Plagiarism’]
%    A student will receive this grade only after proven incident(s) of
%    plagiarism in a course.
%\end{description}

\section{Policy on Assignments \& Exams}
\label{policy}

\ProvidesFile{academic_calendar.tex}[2015/09/10 v1.1 -- Course policy]

\newcommand{\AC}{Academic Calendar}
\newcommand{\SecAC}{the \AC}% Starting in 2015, the AC no longer has sections!

All policy in Sections~\ref{policy}, \ref{expectations} and \ref{support} of
this syllabus applies to this course in addition to policy in the current
\href{http://www.tyndale.ca/registrar/calendar}{\AC}. In some cases the
syllabus underscores the general policy, while in other cases it supersedes it.

For all matters not covered in this syllabus, refer to \SecAC, ``University
College Academic Policies, Procedures, and Notices.'' Students are strongly
encouraged to read this document carefully at least once in their career at
Tyndale, and to review it every year they matriculate.

\ProvidesFile{assignments.tex}[2015/07/28 v1.1.1 -- Course policy]

\subsection{Assignments}
\label{assignments}

This is a university course. All papers and other writing assignments
should therefore be written at the university level. Submissions must be
typewritten and double-spaced, should be free from error, and in this course
should follow the \emph{SBL Handbook of Style} (refer to the free, online
\href{https://www.sbl-site.org/wp-content/uploads/2025/04/SBLHSsupp2015-02.pdf}{SBLHS
Student Supplement}.)

If you ever struggle with composition---anything from the relatively simple
matters of spelling, grammar and proper citation to deeper-level issues of
tone, structure and argument---then please make use of the Writing Centre (see
Section~\ref{centre}). Experienced writers know that drafts and peer
feedback are integral to the writing process. Inexperienced writers are often
unaware that their surface-level errors create credibility problems with their
readers. When you \href{http://theoatmeal.com/comics/misspelling}{misspell
common words}, fail to know \href{http://theoatmeal.com/comics/apostrophe}{how
to use an apostrophe}, or do not bother to cite your sources correctly, why
should your readers trust you with more important matters like the facts and
ideas under discussion?

\subsubsection{Deadlines}
\label{deadlines}

Assignments \emph{must} be submitted on time. Even if the work is rough or
incomplete, you must turn in something by the due date to receive any credit
whatsoever. Unless I specify differently in class, papers and take-home exams
are due by 11:59 \textsc{pm} on the due date. All other work is due at the
start of the day's class.

Note that, because no late work is accepted in this class, there is no scale
of penalty for unexcused late assignments. If a truly extraordinary event
keeps you from doing your best work, then let me know so that we can make
special arrangements. I am guided by the \AC\ in what counts as extenuation.
``Extensions are not granted for what best could be described as `poor time
management' or `over-involvement' in an extracurricular activity.''

\subsubsection{Submission as PDFs}
\label{submission}

Papers and some other assignments in this course are to be submitted
electronically through the course pages (Section~\ref{lms}). To preserve
formatting, formal writing assignments must be uploaded in Portable Document
Format. There are many ways of creating PDFs; it is your responsibility to
know how to do so on the computer platform you use, and to generate and submit
your PDFs on time.

\subsubsection{Backup}
\label{backup}

In the event of the loss of assignments post-submission---electronic systems
fail, and my office has flooded before---students are required to keep backup
copies of all assignments submitted.

Learning how to secure and preserve your work is a peculiar challenge of the
digital age. Plan on the crash of your hard drive, and the theft of your
laptop (the first is inevitable, the second quite probable). If you do not
have a backup strategy, I recommend that you start with a free account on
\href{http://db.tt/U7eP1vs}{dropbox.com}.

%\ProvidesFile{exams.tex}[2013/08/19 v1.0 -- Course policy]

\subsection{Examinations}
\label{exams}

My examination policy follows that outlined in \SecAC, part of which is
summarized below for emphasis.

\begin{enumerate}
  \item
    Midterm exams will be held as scheduled by the instructor. If you miss
    the exam for a legitimate reason, you must write the exam within the same
    number of days that you were absent from school (possibly the next day).
  \item
    Final examinations will take place during the exam period as scheduled
    by the Registrar. Students are responsible for noting the date, time and
    location of their final exam in this class. Students are also responsible
    for familiarizing themselves with the Registrar's examination policies.
  \item
    Special rules apply when midterms and finals are held, including one that
    prohibits students from leaving their seats during the final fifteen minutes
    of the exam period. See the \AC\ for full details.
%    The following rules apply to every final examination:
%    \begin{enumerate}
%      \item
%        No student is permitted to take into the examination room any materials
%        relating to the examination subject, including Bibles.
%      \item
%        No student may leave the room without permission from the proctor.
%      \item
%        No student may leave his or her seat during the final fifteen minutes.
%      \item
%        Students must not linger in the halls outside the examination rooms
%        while examinations are being written.
%      \item
%        No student will be permitted to write beyond the allotted time without
%        special permission of the Registrar (see Section~\ref{accessibility}).
%    \end{enumerate}
  \item
    Provisions exist for students who are justifiably unable to write the final
    exam at the scheduled time. See the \AC\ for details, and make arrangements
    through the Office of the Registrar.
  \item
    Normally, a final exam can only be reschedule in two circumstances:
    (a) a documented illness, or (b) a conflict with another
    exam (two at the same time, or three within 24 hours).
    \href{http://www.tyndale.ca/registrar/final-exam-schedule-and-policies}
    {Apply to the Registrar} in either case.
\end{enumerate}


\section{Student Expectations \& Guidelines}
\label{expectations}

\ProvidesFile{academic_integrity.tex}[2013/08/19 v1.0 -- Course policy]

\subsection{Academic Integrity}
\label{integrity}

Integrity in academic work is required of all students. Academic dishonesty
is any breach of this integrity. It includes such practices as cheating (the
use of unauthorized material on tests and examinations), submitting the same
work for different classes without permission of the instructors, using false
information in an assignment (including false references to secondary sources),
improper or unacknowledged collaboration with other students, and plagiarism.

Tyndale takes seriously its responsibility to uphold academic integrity,
and to apply consequences for academic dishonesty. Consult \SecAC\ for more
information on the school's policy and its application to your work in this
course.

\ProvidesFile{attendance.tex}[2013/08/22 v1.0 -- Course policy]

\subsection{Attendance}
\label{attendance}

``Faithful attendance at classes is an important indicator of student maturity
and involvement'' (\AC). Remember, too, that you are responsible for everything
that happens in every class. Your best policy is to attend and engage. Please
do not ask me to repeat for your benefit anything I have said in a class you
have missed.

Keeping a record of attendance is mandatory for faculty at Tyndale (in contrast
to many other colleges and universities). The University College publishes
guidelines for how attendance should bear on your final evaluation in a course,
and I adhere to them. Note that four lates equals one absence.

What should you do if you miss an undue number of classes? First, arrange for
a classmate to brief you on the material missed, or get my permission for a
classmate to make a recording for you (see Section~\ref{recording}). Second,
notify the Dean of Students in person or by phone. If illness is the cause you
will need to submit a doctor's certificate upon return. The Dean of Students
will notify your professors of the reason for the absence and suggest that they
take this into consideration when assigning grades.

\ProvidesFile{technology.tex}[2014/09/02 v1.1 -- Course policy]

\subsection{Technology}
\label{technology}

Technological innovation has brought students and educators a number of
powerful new tools, and I encourage you to use them as you research, write, and
collaborate. Some of these tools also call for disciplined use and management.

\subsubsection{Email}
\label{email}

Email can be a chore, and you may prefer other channels of communication. As a
matter of policy, however, students must use their myTyndale accounts for all
course-related email correspondence. During term time you should check your
school account at least once a day (optional on weekends). I myself aim to
check my school email at the beginning and end of each workday. At other times
my email client is often closed. I will try to answer your messages within 24
hours, though you should not expect replies on weekends.

Keep your messages topical and brief. I would vastly prefer to conduct any
conversations of substance in person, or else over the phone. Please note and
make use of my office hours. If these hours do not suit your schedule, I would
happily receive an email from you requesting an alternate meeting time.

% Note the prof who adopted an anti-email policy! https://www.insidehighered.com/news/2014/08/27/sake-student-faculty-interaction-professor-bans-student-email?utm_source=slate&utm_medium=referral&utm_term=partner

\subsubsection{MyTyndale.ca / classes.tyndale.ca}
\label{mytyndale}

Tyndale's course pages are an efficient means of distributing articles, notes,
slides, and other course-related materials. This is also where instructors log
attendance and upload grades for assignments. Students are therefore required to
check the site for updates about their classes as well as for any materials
needed for lectures and assignments.

My own use of this platform varies from semester to semester, and from
course to course. At times I may ask you to use the forums, quiz module, or
other parts of the system. At a minimum I will use the site as a repository
for course materials, and as a destination for your submission of PDFs
(Section~\ref{submission}).

\subsubsection{Laptops and Other Devices}
\label{laptops}

Use of laptops is forbidden in my classroom, except to facilitate presentation.
I implement this policy because of the cognitive costs of multitasking, with the
aim of giving you and your peers the best chance of success. I also hope to
foster a culture of keen intellectual engagement.

As \href{http://dx.doi.org/10.1016/j.compedu.2012.10.003}{cognitive
psychologists at McMaster and York Universities demonstrated in 2013}, ``laptop
multitasking hinders classroom learning for both users and nearby peers.'' There
is little new in their finding that the allegedly multitasking student does less
well in class (11\% worse on the quiz in their experiment). This effect has been
shown many times. Rather, their novel discovery is that classmates
\emph{without} laptops who sat with a \emph{view} of another student's screen
did worse than the students who had a computer (17\% worse than those with no
laptop in sight).

Prohibiting laptops is not the only possible response to these findings.
However, there is evidence that \href{http://on.wsj.com/pjtJaK}{writing by hand}
brings a number of cognitive benefits, and a
\href{http://dx.doi.org/10.1177/0956797614524581}{2014 Princeton University
study} ``found that students who took notes on laptops performed worse on
conceptual questions than students who took notes longhand.'' If you are a heavy
laptop user then consider this an opportunity to experiment with different
technologies in the classroom.

As for the myriad networked devices that many of us carry, it's a simple matter
of professionalism to keep these things silent and out of sight. E-readers and
tablets are permitted \emph{only if they are used to display the assigned
reading}. If this is how you choose to read, let me invite you to put the
machine in airplane mode while class is in session.

\subsubsection{Recording of Classes}
\label{recording}

Students must request permission from the professor of any class that they
would like to record. Where permission is granted, students are expected to
supply their own equipment. In general I prefer \emph{not} to have my classes
recorded, and I am not at all friendly to being recorded without my knowledge.
In cases where I grant permission, I stipulate that the recordings must be for
personal use only. They should not be shared with other students, even with
students in the same section, and they absolutely must not be posted online or
otherwise distributed.

If a student is not able to attend a lecture and would like to have it recorded,
it is the responsibility of the student first to obtain the professor's
permission, and then to find another student to record the lecture. I will not
make the recording for you.

\ProvidesFile{support.tex}[2014/08/06 v1.1 -- Course policy]

\section{Student Support}
\label{support}

\subsection{Tyndale Writing Centre}
\label{writing_centre}

Through a combination of tutorials, workshops and resources, Tyndale’s Writing
Centre offers a comprehensive program of writing support to Tyndale students,
including individual 30-minute tutoring sessions. Students may bring essays that
have been graded (and, at least for my classes, essays that have not yet been
submitted for a grade) and will receive detailed suggestions for improving their
writing. This service, at no charge to students, is available by appointment.

Professors may recommend that a student go to the Writing Centre for help:
students are strongly encouraged to follow such recommendations. The Academic
Standards Committee may require an undergraduate student who is experiencing
difficulty in his or her academic program to go to the Writing Centre for
assistance and support. Many top students also elect to go.

\subsection{Tyndale UC Tutoring Program}
\label{tutoring}

Tyndale University College is committed to helping its students achieve academic
success. For this purpose, students in need of academic assistance may request
peer tutoring, free of charge, in each academic department. This includes
students on academic probation, students who have received failing grades in a
course or courses, or students who have been referred for tutoring by their
instructor.

For more information on scheduling tutoring appointments, or for those
interested in becoming peer tutors, students may contact the Office of the
Senior Vice President Academic or their respective University College department
chairs.

\subsection{Accommodation}
\label{accommodation}

Students with documented disabilities may be granted special accommodation for
exams, and in some cases for other assignments. It is even possible to get
permission to use a laptop in class (Section~\ref{laptops}), although I will
need to be convinced of the use case. It is up to the student to contact the
Dean of Students as early as possible in the semester---not later than the
second week---and to document the need. The Dean of Students will then advise
each of the student's professors of the accommodations that may be required.
Please note that special arrangements for assignments need to be made with me
well in advance of assignment due dates (Section~\ref{deadlines}). Timely
requests shall not unreasonably be denied.


\section{Course Outline}
\label{outline}

Readings should be completed before the first class of the week in which
they are assigned. Research into seminar texts and any supplementary
readings that I circulate (see Section~\ref{supplementary}) should be
completed before the last class of their week. We will adhere to the
schedule as closely as possible, though I reserve the right to adjust it
to suit the needs of the class.

\newcommand\Yhwh{\textsc{Yhwh}}
\newcommand\rarr{\char"2192\hspace*{0.5pt}}

\setcolumncount{5} % set up \sessioncount, \unit{} and \noclass{} macros
\begin{table}[htb]% add p to put the schedule on its own page
  \centering
  \begin{tabular}{>{\sessioncount.}r@{ }lllr}
    \toprule
    \sessionskip{\textbf{Wk}.}
          &\textbf{Seminar Text}
                         &\textbf{Gillingham}
                                        &\textbf{Wenham} &\textbf{Date}\\
    \midrule
          & Psalm 150    & Introduction & Introduction   & 4 Sep.      \\
          & Psalm 1      & pp. 5--27    & Ch. 1          & 8, 11 Sep.  \\
          & Psalm 2      & pp. 28--55   & Ch. 2          & 15, 18 Sep. \\
          & Psalm 19     & pp. 55--76   & Ch. 3          & 22, 25 Sep. \\
          & Psalm 23     & pp. 77--104  & Ch. 4          & 29 Sep., 2 Oct. \\
          & Psalm 24     & pp. 104--130 & Ch. 5          & 6, 9 Oct.   \\
    \noclass{Thanksgiving}                               & 13 Oct.     \\
          & Psalm 51     & pp. 131--162 &                & 16 Oct.     \\
          & Psalm 73     & pp. 163--191 &                & 20 Oct.     \\
    \noclass{Reading Days}                               & 23--24 Oct. \\
          & Psalm 88     & pp. 192--215 & Ch. 6          & 27, 30 Oct. \\
          & Psalm 90     & pp. 215--241 & Ch. 7          & 3, 6 Nov.   \\
          & Psalm 110    & pp. 242--266 & Ch. 8          & 10, 13 Nov. \\
          & Psalm 116    & pp. 266--290 & Ch. 9          & 17, 20 Nov. \\
          & Psalm 137    & pp. 290--312 & Ch. 10         & 24, 27 Nov. \\
          & Psalm 145    &              &                & 1 Dec.      \\
    \noclass{Reading Days}                               & 2--3 Dec.   \\
    \bottomrule
  \end{tabular}
  \caption{Schedule of Readings}
  \label{schedule}
\end{table}

See the \href{http://www.tyndale.ca/registrar/important-dates}{%
Registrar's website} for a list of other important dates. Generally, the
last day to add or drop a class without penalty is the end of the second
week of class.

\section{Course Bibliography}
\label{bibliography}

For the weekly Psalms seminar, students should focus on a commentator
from the list below and give a short (5 minute) overview of the
interpretation of the Psalm in question. Be sure to work with four
different commentators over the semester, from at least two different
major periods. (For those with the language skills, registering notes on
one of the Hebrew or Greek can be one of the four options.) The aim is
to gain as much insight into the message of the Psalm by close attention
to text, translation and commentary.

\begin{enumerate}

 \item Ancient Versions
  \begin{enumerate}

	\item John R. Kohlenberger, III, ed., \emph{The Comparative Psalter: Hebrew-Greek-English} (Oxford: Oxford University Press, 2007) [Ref BS 1419.K645]. Very useful for our purposes, but not to be mistaken as a critical edition or a full account of ancient versions.

	\item The Septuagint/Old Greek (LXX/OG) is available in \href{http://ccat.sas.upenn.edu/ioscs/editions.html}{two main editions}, each with major and minor editions. A. Rahlfs produced the Göttingen Septuagint's \emph{editio maior} of \emph{Psalmi cum Odis} in 1931 (3rd ed., 1979). The Rahlfs-Hanhart \emph{editio minor} of the entire OG Bible is available in our library's reference section [Ref BS741.R3 2006].

	\item Vulgate/Jerome: Biblia Sacra Latina [Ref BS 75 1800 and \href{https://www.biblegateway.com/versions/?action=getVersionInfo&vid=4}{online}].

	\item Targum: \emph{The Aramaic Bible, Vol. 16: The Targum of the Psalms} (trans. David M. Stec; Collegeville, Minn: Liturgical Press, 2004) [Ref BS 709 .2 .A72 v.16 2004]. Also, \href{http://targum.info/targumic-texts/targum-psalms/}{Edward Cook's translation is available online}.

  \end{enumerate}
 \item Early Church / Synagogue
  \begin{enumerate}

	\item John Chrysostom, \emph{Commentary on the Psalms} (trans. R.\,C. Hill; 2 vols.; Brookline, Mass.: Holy Cross Orthodox Press, 1998) [BS 1430.3.J63 1998].

	\item Augustine, \emph{Expositions of the Psalms} (trans. M. Boulding; 6 vols.; Hyde Park, N.Y.: New City Press, 2000--2004) [BR 65.A5 E53 1990 pt.III/15--20]. Older English translations of \emph{Enarrationes in Psalmos} appear in the Ancient Christian Writers series (trans. S. Hebgin, F. Corrigan; Westminster, Md., Newman Press, 1960--) and in \href{http://www.ccel.org/ccel/schaff/npnf108.toc.html}{P. Schaff's NPNF translation, first printed 1847--57 and now online}. For the Latin original see Corpus Scriptorum Ecclesiasticorum Latinorum (CSEL), vols. 93--95.

	\item Diodore of Tarsus, \emph{Commentary on Psalms 1--51} (trans. R.\,C. Hill; Atlanta: SBL, 2005) [\href{http://tyndale.worldcat.org/oclc/191953419}{various e-books}].

	\item Theodoret of Cyrus, \emph{Commentary on the Psalms} (trans. R.\,C. Hill; 2 vols; Washington, DC: Catholic University of America Press, 2000--2001) [BS 1430.T54 and \href{http://tyndale.worldcat.org/oclc/41649596}{various e-books}].

	\item Theodore of Mopsuestia, \emph{Commentary on Psalms 1--81} (trans. R.\,C. Hill; Atlanta: SBL, 2006) [BS 1430.53.T4813 2006 and an \href{http://ezproxy.mytyndale.ca:2048/login?url=http://hdl.handle.net/2027/heb.07760.0001.001}{e-book}].

	\item Cassiodorus, \emph{Explanation of the Psalms} (trans. P.G. Walsh; New York: Paulist Press, 1990--1991) [BR 60.A3 v.51--53 1990].

	\item Midrash Tehillim / \emph{Midrash on the Psalms, Translated from the Hebrew and Aramaic} (trans. William G. Braude; 2 vols.; New Haven: Yale University Press, 1959) [BM 517.T52 E5 1959].

  \end{enumerate}
 \item Medieval
  \begin{enumerate}

	\item Aquinas, \emph{Postilla super Psalmos / Commentary on the Psalms} (1272--1273). A Latin / English parallel edition for (most of) Psalms 1--54, a work in progress ed. by Stephen Loughlin, is \href{http://www4.desales.edu/~philtheo/loughlin/ATP/}{online as part of the Aquinas Translation Project}. See also: Thomas Ryan, \emph{Thomas Aquinas as Reader of the Psalms} (Notre Dame: University of Notre Dame Press, 2000) [BX 2350.65.R93 2000].

	\item \emph{Rashi’s Commentary on Psalms} (trans. Mayer I. Gruber; Leiden: Brill, 2004) [BS 1429.R3713 2007]. % See also: Hermann Hailperin, \emph{Rashi and the Christian Scholars} (Pittsburgh: University of Pittsburgh Press, 1963).

  \end{enumerate}
 \item Reformation Era
  \begin{enumerate}

	\item Desiderius Erasmus (1466--1536): \emph{Expositions of the Psalms} (Collected Works of Erasmus, Vol. 63--65; Toronto: University of Toronto Press, 1997, 2005, 2010) [\href{http://ezproxy.mytyndale.ca:2048/login?url=http://site.ebrary.com/lib/tyndale/docDetail.action?docID=10226282}{CWE 63: Pss 1, 2, 3, 4}; \href{http://ezproxy.mytyndale.ca:2048/login?url=http://site.ebrary.com/lib/tyndale/docDetail.action?docID=10218753}{CWE 64: Pss 88, 22, 28, 33}; CWE 65: Pss 38, 83, 14].

	\item Martin Luther (1483--1546): \emph{Werke}, 35 = \emph{Luther’s Works}, 10--14 [BR 330.E5P4 1955]. See also: J.S. Preus, \emph{From Shadow to Promise: Old Testament Interpretation from Augustine to the Young Luther} (Cambridge, Mass.: Belknap, 1969) [BR 333.5.B5 P7 1969].

	\item John Calvin (1509--1564): \emph{Commentary on the Psalms} (1557--, ET 1839--). All of Calvin's commentaries are available online, in English translation, at the \href{}{Christian Classics Ethereal Library (CCEL)}. Psalms: \href{http://www.ccel.org/ccel/calvin/calcom08.html}{1--35}, \href{http://www.ccel.org/ccel/calvin/calcom09.html}{36--66}, \href{http://www.ccel.org/ccel/calvin/calcom10.html}{67--92}, \href{http://www.ccel.org/ccel/calvin/calcom11.html}{93--119}, \href{http://www.ccel.org/ccel/calvin/calcom12.html}{119--150} [see also BS 1429.C34 1840 and Ref BS 485.C245 1979 vols. 4–6].

  \end{enumerate}
 \item Early Modern / Critical
  \begin{enumerate}

	\item E.\,W. Hengstenberg, \emph{Commentary on the Psalms} (trans. P. Fairbairn and J. Thomson; 3 vols.; Edinburgh: T\&T Clark, 1851--1858)
	[BS 1430.H46 and \href{http://tyndale.worldcat.org/oclc/21399344}{various e-books}].

	\item Franz Delitzsch, \emph{Biblical Commentary on the Psalms} (trans. Franics Bolton; 3 vols; Grand Rapids: Eerdmans, 1952)
	[BS 1430.D44 1952 and \href{http://tyndale.worldcat.org/oclc/677380663}{various e-books}].

	\item C.\,H. Spurgeon, \emph{The Treasury of David: An Expository and Devotional Commentary on the Psalms} (Nashville: T. Nelson, 1984)
	[BS 1430.S6 1984, 1976, 1940 and \href{http://www.spurgeon.org/treasury/treasury.htm}{online}].

	\item C.\,A. Briggs, \emph{A Critical and Exegetical Commentary on the Book of Psalms} (2 vols.; Edinburgh: T\&T Clark, 1906--1907)
	[Ref BS 491.I6 v.19a–19b, BS 1430 .B7 and \href{https://archive.org/details/criticalexegetic02briguoft}{online}].

	\item Hermann Gunkel, \emph{Die Psalmen} (Göttingen: Vandenhoeck \& Ruprecht, 1929). Any German readers? I can lend out my copy. See also: \emph{The Psalms: A Form-critical Introduction} (Philadelphia: Fortress, 1967) [BS 1430.2.G7813 1967].

  \end{enumerate}
 \item Modern
  \begin{enumerate}

	\item H.-J. Kraus, \emph{Psalms: A Continental Commentary} (trans. H.C. Oswald; 2 vols; Minneapolis: Fortress Press, 1988--1993) [BS 1430.3 .K725].

	\item A.\,A. Anderson, \emph{The Book of Psalms} (2 vols.; Grand Rapids: Eerdmans, 1981) [BS 1430.3.A63 1972; Ref BS 491.2.N48 v19 1981].

	\item M.\,J. Dahood, \emph{Psalms I, II, and III} (3 vols.; Garden City, NY: Doubleday, 1956--1970) [Ref BS 491 .2 .A52 v.19a 1965; … v.19b 1968; … v.19c 1970]. Stacks also.

	\item Word Commentary Series: P. Craigie, \emph{Psalms 1--50} (Waco, TX: Word Books, 1983) [Ref BS 491 .2 .W67 v.19 1983]; M. Tate, \emph{Psalms 51--100} (Dallas, TX: Word Books, 1990) [Ref BS 491 .2 .W67 v.20 1990]; L. Allen, \emph{Psalms 101--50} (rev. ed.; Nashville: Thomas Nelson, 2002) [Ref BS 491 .2 .W67 v.21 2002]. Stacks also.

	\item F.-L. Hossfeld and E. Zenger, \emph{Psalms 2: A Commentary on Psalms 51--100} (Hermeneia; Minneapolis: Fortress, 2005) [Ref BS 491.2.H47 v.19b 2005] and idem, \emph{Psalms 3: A Commentary on Psalms 101--150} (Hermeneia; Minneapolis: Fortress Press, 2011) [Ref BS 491.2.H47 v.19c 2011].

	\item Craig Broyles, \emph{Psalms} (New International Biblical Commentary 11; Peabody, Mass.: Hendrickson, 1999) [Ref BS 491.2.N47 v.11 1999 and BS 1430.3.B75 1999].

	\item You may nominate a commentator of your own, subject to my approval.

  \end{enumerate}
\end{enumerate}

\noindent Other Notable Literature

\begin{itemize}

	\item Ancient Christian Commentary on Scripture, OT Vols. 7--8:
	  Craig A. Blaising and Carmen Hardin, eds., \emph{Psalms 1--50} (Downers Grove: InterVarsity Press, 2008) [Ref BS 491.2.A523 v.7 2008] and
	  Quentin F. Wesselschmidt and Thomas C. Oden, eds., \emph{Psalms 51--150} (Downers Grove: InterVarsity Press, 2007) [Ref BS 491.2.A523 v.8 2007].

	\item H. Attridge and M. Fassler, eds., \emph{Psalms in Community: Jewish and Christian Textual, Liturgical, and Artistic Traditions} (Leiden: Brill, 2004) [BS 1435.P74 2003].

	\item Willaim Holladay, \emph{The Psalms through Three Thousand Years: Prayerbook of a Cloud of Witnesses} (Minneapolis: Fortress, 1993) [BS 1430.3.H65 1993].

	\item Susan Gillingham, \emph{Psalms Through the Centuries, Vol. 1} (Oxford: Blackwell, 2008) offers an \href{http://www.blackwellpublishing.com/pdf/9780631218555.pdf}{expanded online bibliography (PDF)}. See also \href{http://www.bbibcomm.net/}{www.bbibcomm.net}. Volume 2 has not yet appeared, but see her \emph{A Journey of Two Psalms: The Reception of Psalms 1 and 2 in Jewish and Christian Tradition} (Oxford: Oxford University Press, 2013) [BS 1430.52.G547 2013] and \emph{Jewish and Christian Approaches to the Psalms: Conflict and Convergence} (Oxford: Oxford University Press, 2013) [BS 1430.52.J49 2013].

	\item Sæbø, HB/OT: For help with specific periods and commentators in biblical reception history see Magne Sæbø et al., eds., \emph{Hebrew Bible, Old Testament: The History of its Interpretation} (3 vols.; Göttingen : Vandenhoeck \& Ruprecht, 1996--2015).

\end{itemize}

\end{document}
