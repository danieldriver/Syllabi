% Copyright (c) 2015 by Daniel R. Driver.
% !TEX encoding = UTF-8 Unicode
% !TEX TS-program = XeLaTeX

\documentclass[titlepage]{article}

% This document presumes a file structure and set of inputs that are
% available at: git@github.com:danieldriver/syllabi.git

\newcommand\policy{../policy}
\newcommand\incl{../includes}
\ProvidesFile{variables.tex}[2018/05/24 v2.1 -- Syllabus variables]

\usepackage{xspace} % make manual spaces (like \mycmd\ ) unnecessary
\usepackage{xifthen} % provides \isempty test

% variables for internal use
\newcommand\prof{}
\newcommand\pdegree{}
\newcommand\pphone{}
\newcommand\pemail{}
\newcommand\poffice{}
\newcommand\phours{}
%
\newcommand\ccode{}
\newcommand\ctitle{}
\newcommand\cseries{}
\newcommand\cversion{}
\newcommand\csemester{}
\newcommand\cmeetson{}
\newcommand\cmeetsat{}
\newcommand\cmeetsin{}
\newcommand\cwebsite{}
\newcommand\cdescrip{}
\newcommand\cprereqs{}
\newcommand\edobject{}

% in case of fully online courses - https://tex.stackexchange.com/a/5896
\newif\ifonline
\newcommand\Int[2]{\ifonline#1\else#2\fi}

% commands for setting variables in the preamble
\newcommand\professor[2][PhD]{
  \renewcommand\pdegree{#1\xspace}
  \renewcommand\prof{#2\xspace}}
\newcommand\phone[1]{
  \renewcommand\pphone{\addfontfeatures{Numbers=Monospaced}#1\xspace}}
\newcommand\email[1]{
  \renewcommand\pemail{\href{mailto:#1}{#1}\xspace}}
\newcommand\officehours[2][Library, Room 5-North]{
  \renewcommand\poffice{#1\xspace}
  \renewcommand\phours{#2\xspace}}
%
\newcommand\coursecode[2][1.0]{
  \renewcommand\cversion{#1\Int{-i}{}\xspace}
  \renewcommand\ccode{#2\Int{(Int)}{}\xspace}}
\newcommand\coursetitle[2][]{
  \ifthenelse{\isempty{#1}}%
    {}% do nothing if #1 is empty, else:
    {\renewcommand\cseries{#1\\[1ex]}}
  \renewcommand\ctitle{#2\xspace}}
\newcommand\semester[1]{
  \renewcommand\csemester{#1\xspace}}
\newcommand\meets[3]{
  \newcommand\AM{\textsc{am}}
  \newcommand\PM{\textsc{pm}}
  \renewcommand\cmeetson{#1\xspace}
  \renewcommand\cmeetsat{\Int{From 9:00 \AM}{#2}\xspace}
  \renewcommand\cmeetsin{\Int{\href{https://smu.brightspace.com/d2l/login}{Brightspace}}{#3}\xspace}}
\newcommand\website[1]{
  \renewcommand\cwebsite{\href{http://#1}{#1}\xspace}}
\newcommand\cdescription[2][RM 1000 or GTRS 6000; and BF 1001]{
  \renewcommand\cprereqs{#1}
  \renewcommand\cdescrip{#2\par}}
\newcommand\objectives[1]{
  \renewcommand\edobject{#1\par}}


\coursecode[8.5]{BSTH 1013}
\coursetitle{Old Testament Scripture}

% Previously taught as RLGS 1013 in:
%   - Fall 2008
%   - Winter 2009
%   - Fall 2009
%   - Winter 2010
%   - Fall 2010
%   - Winter 2011 (Fall 2011 was covered by Joel Lohr in a course release)
%   - Winter 2012
%   - Fall 2012
%   - Winter 2013
% Previously taught as BSTH 1013 in:
%   - Fall 2013
%   - Winter 2014
%   - Fall 2014
%   - Winter 2015
%   - Fall 2015

\professor{Daniel R. Driver}
\phone{416-226-6620 ext. 2201}
\email{ddriver@tyndale.ca}
\officehours{Mon. \& Thur., 2:30--3:30 \PM}

\semester{Winter 2016}
\meets{Tue. \& Thur.}% \meets{on}{at}{in}
      {5:15--6:35 \PM}
      {Bayview E320}
\website{classes.tyndale.ca}
\cdescription{% copy from tyndale.ca/registrar/calendar
	Survey of the redemptive story in the three major divisions of the
	Old Testament (the Law/Torah, the Prophets, and the Writings),
	including an orientation to the historical backgrounds, religious
	context, literary forms, apocryphal dimensions, prophetic elements,
	matters of canon, text, interpretation, and critical issues.

	{\itshape Note that this class is both a general studies requirement
	and a prerequisite for advanced courses in Biblical Studies \&
	Theology.}
}% end of course description
\objectives{% recall Bloom's taxonomy: http://www.celt.iastate.edu/teaching/RevisedBlooms1.html

	The learning outcomes for this course are in part a function of each
	student's ability to meet specific objectives. By the end of the
	course students should be able to:
		name major Old Testament people and events;
		identify representative biblical quotes by book and chapter;
		give key dates for Israel's history and summarize the succession of superpowers in the Ancient Near Eastern political theatre from the time of Egypt to Greece;
		locate a few important biblical sites on a map;
		classify prophetic literature relative to the exile;
		recognize genres of biblical literature and cite examples from the reading;
		understand the general shape of the Masoretic Text tradition and differentiate it from other canonical orders;
%		list the Ten Commandments in order;
%		defend a decision to pronounce or circumlocute the divine name;
		report on parallel and divergent material across the Law and the Prophets;
		articulate multiple rationales for sabbath observance;
		memorize and recite Psalm 8.

	Students should also be able to identify settings in which the
	scriptures of Israel are read (notably the synagogue, church, and
	academy), employ terminology appropriate to these reading
	communities, recognize where their own biographies and commitments
	place them in relation to the Hebrew Bible/Old Testament and its
	uses, and monitor and test their individual attitudes and
	assumptions. They should be able to extend their awareness of the
	Bible's reading communities to the Bible's long history of
	reception. Finally, students should be able to infer what Jesus
	might have meant in speaking of ``the law of Moses, the prophets,
	and the psalms'' (Luke 24:44), and begin to hear claims about New
	Testament fulfilment of scripture in light of the unique voice that
	the Old Testament retains along side of the New in Christian
	scripture.

}% end of learning objectives

\ProvidesFile{preamble.tex}[2013/09/06 v1.0 -- Syllabus preamble]

% basic typography
\usepackage{fontspec}
\setmainfont[Ligatures=TeX]{Meta Serif Pro}
\setsansfont[Ligatures=TeX]{Meta Pro}
\newfontfamily\Heb{Meta Hebrew}
\setmonofont[Scale=MatchLowercase]{Menlo}
\usepackage{sectsty}
\allsectionsfont{\sffamily}
\frenchspacing
\setlength{\emergencystretch}{3em} % prevent overfull lines

% custom font size and leading
\renewcommand\tiny{\fontsize{6}{9}\selectfont}
\renewcommand\scriptsize{\fontsize{7}{10}\selectfont}
\renewcommand\footnotesize{\fontsize{8}{11}\selectfont}
\renewcommand\small{\fontsize{8.5}{11.5}\selectfont}
\renewcommand\normalsize{\fontsize{9}{12}\selectfont}% base size
\renewcommand\large{\fontsize{11}{14}\selectfont}
\renewcommand\Large{\fontsize{13}{16}\selectfont}
\renewcommand\LARGE{\fontsize{16}{19}\selectfont}% "course syllabus \\ semester" benefits from more lead
\renewcommand\huge{\fontsize{19}{21}\selectfont}
\renewcommand\Huge{\fontsize{24}{26}\selectfont}

% layout packages: page, logo, tables
\usepackage[scale={0.6,0.8},
            xetex]{geometry}
\usepackage{graphicx}
\usepackage{array}     % allow insertions of column styling with >{}
\usepackage{booktabs}  % elegant horizontal rules in tables
\usepackage{marginfix} % protect positioning of margin table in policy/grades

% custom macros for a session count in the schedule of readings
\newcounter{session}
\newcounter{columns}
\newcounter{courseunit}
\newcommand\setcolumncount[2][0]{ % optionally set count to other than 0,
  \setcounter{session}{#1}        % e.g. to -1, or to a standing count
  \setcounter{columns}{#2}}
\newcommand\sessioncount{\stepcounter{session}\arabic{session}}
\newcommand\sessionskip[1]{\multicolumn{1}{@{}r@{ }}{#1}}
\newcommand\unit[1]{\multicolumn{\thecolumns}{c}{%
  \scshape\stepcounter{courseunit}\roman{courseunit}. \MakeLowercase{#1}}}
\newcommand\noclass[1]{\multicolumn{1}{@{}l}{\itshape No Class: #1}}

% color to match Tyndale's branding
\usepackage[usenames]{xcolor}
% predefined: black, white, red, green, blue, cyan, magenta, yellow
\definecolor{TyndaleURLs}{HTML}{0062A0} % links on tyndale.ca
\definecolor{TyndaleBlue}{cmyk}{1,1,0,.32}
\definecolor{TyndaleGold}{cmyk}{0,.27,1,0}
\definecolor{TyndaleRed}{cmyk}{0,1,.99,.04}
\definecolor{TyndaleBlack}{cmyk}{0,0,0,1}
\definecolor{TyndaleGreen}{cmyk}{.45,0,1,.24}
\definecolor{TyndaleOrange}{cmyk}{0,.79,1,0}
\definecolor{TyndaleAqua}{cmyk}{.47,0,.24,0}
\definecolor{TyndaleYellow}{cmyk}{.03,.03,.35,0}

% metadata (assumes a host of definitions are made in the main file)
\usepackage[setpagesize=false,     % leave this to geometry
            hyperfootnotes=false,  % fragile and distracting
            xetex]{hyperref}
\hypersetup{breaklinks=true,       % allow link text to break across lines
            colorlinks=true,       % colorlinks resets pdfborder to 0 0 0
            urlcolor=TyndaleURLs,  % for external links
            linkcolor=TyndaleRed,  % for normal internal links
            citecolor=TyndaleGold, % for bibliographical citations in text
            pdfauthor={\prof},
            pdftitle={\ccode: \ctitle},
            pdfsubject={Tyndale UC, \csemester},
            pdfcreator={github.com/danieldriver/syllabus}}
\urlstyle{same}                    % don't use monospace font for urls

% custom footlines
\usepackage{fancyhdr}
\pagestyle{fancy} % turn it on
\fancyhf{}        % reset everything
\renewcommand{\headrulewidth}{0pt} % remove header line as well
\lfoot{\sffamily\scshape\footnotesize\MakeLowercase{\ctitle, v\cversion}}
\rfoot{\sffamily\scshape\footnotesize\MakeLowercase{\prof\quad\thepage}}

% gratuitous with custom title page, but useful as a fallback
\title{\ccode: \ctitle}
\author{\professor}
\date{\semester}


\begin{document}
\ProvidesFile{title.tex}[2013/09/06 v1.0 -- Syllabus title page]

\begin{titlepage}
  \begin{center}

    \LARGE\sffamily % set title elements in a large sans serif

    \begin{minipage}{\textwidth}
      \parbox[t]{0.5\textwidth}{
        \mbox{}\\[-13pt] % dummy line to align parboxes
        \includegraphics[width=0.5\textwidth]{.syllabus/includes/TyndaleUC}}
      \hfill
      \parbox[t]{0.4\textwidth}{
        \raggedleft Course Syllabus\\
        \csemester}
    \end{minipage}

    \vfill

    {\textsc{\MakeLowercase\ccode}\\[1ex]
      \bfseries\cseries\Huge\ctitle}

    \vfill

    \normalsize\rmfamily % switch back to body type

    \begin{tabular}{>{\bfseries}rl>{\bfseries}rl}
      \toprule
      Instructor & \prof, \pdegree & Course  & Version \cversion \\
      \midrule
      Phone      & \pphone         & Meets   & \cmeetson         \\
      Email      & \pemail         & Time    & \cmeetsat         \\
      Office     & \poffice        & Room    & \cmeetsin         \\
      Hours      & \phours         & Website & \cwebsite         \\
      \bottomrule
    \end{tabular}

    \vfill

    \begin{description}\small
      \item[Commuter Hotline]
        Class cancellations due to inclement weather or illness will
        be announced on the commuter hotline at \texttt{416.226.6620
        x2187}. Alternately, weather cancellation information is posted
        at \href{http://tyndale.ca/weather}{tyndale.ca/weather}.
      \item[MyTyndale.ca]
        This course may have materials stored on its website, such as
        handouts or readings that may be needed in order to complete
        assignments. Students are responsible for checking these course
        pages on a regular basis. Here, too, students are able to view
        their grades throughout the semester. For more information see
        Section~\ref{mytyndale}, below.
      \item[Mail]
        Students are responsible for information communicated through
        their campus mailboxes and student e-mail accounts. A mailbox
        directory hangs beside the mailboxes. For more information
        contact the Registrar's office.
    \end{description}

  \end{center}

  \section{Course Description}
  \label{description}

  \emph{From the Academic Calendar:} \cdescrip

\end{titlepage}
\setcounter{page}{2} % count the title page as page 1


\section{Required Texts \& Materials}
\label{texts}

All required textbooks are available in the Tyndale Bookstore. The
coursepack is available in class, directly from Dr. Driver.

\begingroup
\renewcommand{\section}[2]{}% temporarily remove the section heading
\begin{thebibliography}{Heschel}% use the longest item in the bibliography

	\bibitem[NJPS]{njps}
	Adele Berlin and Marc Zvi Brettler, eds.
    \emph{The Jewish Study Bible: Second Edition}.
    Oxford / New York: Oxford University Press, 2014.
    ISBN 978-0199978465.
%	Many serviceable translations of the Bible exist, including the NRSV
%	and ESV. The NJPS allows students in this course to better
%	appreciate the relationship of the first part of Christian Scripture
%	to the Scriptures of Israel (also called the Hebrew Bible), which is
%	one of the course's fundamental objectives. In addition, this
%	revised study Bible contains an outstanding set of introductory
%	notes and essays in a highly readable, single-column format. It is
%	required.

%	\bibitem[Heschel]{heschel} Abraham Joshua Heschel. \emph{The
%	Sabbath: Its Meaning for Modern Man}. New York: Farrar, Straus and
%	Giroux, 1951 (repr. 2005). Read this classic example of
%	20\textsuperscript{th} century Jewish spirituality on your own, no
%	later than the fourth week of class. Review it in preparation for
%	the class on which it is assigned.

	\bibitem[OTSIC]{otsic} Daniel R. Driver. \emph{\ctitle\ Introductory
	Coursepack}. Toronto, \csemester. In response to the requests of
	many students I have compiled a pack of notes for my lectures. I
	will circulate the relevant pages before each class. It behooves you
	to gather, annotate, and study this material carefully.

\end{thebibliography}
\endgroup

\section{Supplementary Texts}
\label{supplementary}

Supplementary readings are listed in the bibliographies that follow each
page of notes in the coursepack. Excerpts from this literature,
ordinarily an article or a book chapter per week, will either be made
available for download through the course website, or else be designated
from the introductory essays included in the NJPS Study Bible.

Students are not strictly required to read this material; then again,
students who choose not to read it should not expect to get an A for the
course.
%Note that the required readings for this survey emphasize a selection of
%primary texts from the Old Testament. Reading in the vast body of
%secondary literature on the Bible is largely left for upper-division
%courses.
Diligent students, and all prospective majors, should make a special
point of reviewing the designated supplementary texts.

\section{Course Outline}
\label{outline}

Readings should be completed before the start of the class for which
they are assigned. We will adhere to the schedule in
Table~\ref{schedule} (below) as closely as possible, though I reserve
the right to adjust it to suit the needs of the class.

%Readings should be completed before the start of the class for which
%they are assigned. Primary readings (from scripture) and required
%secondary texts (Heschel and Moberly) are compulsory. Any supplementary
%readings that I circulate (per Section~\ref{supplementary}) will be
%keyed to the session number (listed under \textbf{\S} in
%Table~\ref{schedule}), and should ideally be read before class as well.
%We will adhere to the schedule as closely as possible, though I reserve
%the right to adjust it to suit the needs of the class.

\setcolumncount{4}% set up \sessioncount, \unit{}, \noclass{}, and \reminder{memo}{date} macros
\begin{table}[p]% add p to put the schedule on its own page
  \centering
  \begin{tabular}{>{\sessioncount.}r@{ }llr}% make sure the column config agrees with \setcolumncount
	\toprule
	\sessionskip{\textbf{\S}.}&\textbf{Primary Reading}&\textbf{Secondary Reading}&\textbf{Date}\\
	\midrule

	\unit{Beginnings} \\

		& Psalms 8, 74, 104               & Syllabus         &  5 Jan.        \\
		& Genesis 1--3                    & P. Enns          &  7 Jan.        \\ [1ex]

		& Genesis 4--25                   &                  & 12 Jan.        \\
		& Genesis 26--50                  & J. Levenson      & 14 Jan.        \\ [1ex]

		& Exodus 1--6, 32--34             &                  & 19 Jan.        \\
		& Exodus 7--18                    & C. Mathews McGinnis  & 21 Jan.    \\ [1ex]

	\unit{Legal Traditions} \\

	\reminder{\textbf{Recitation of Psalm 8} (during Dr. Driver's office hours)}{\emph{from} 26 Jan.} \\
		& Exodus 19--24                   & D. Patrick       & 26 Jan.        \\
		& Exodus 20, Deuteronomy 5        &                  & 28 Jan.        \\ [1ex]

    	& Deuteronomy 1--11               &                  &  2 Feb.        \\
		& Deuteronomy 12--26              & B. Levinson      &  4 Feb.        \\ [1ex]

		& Deuteronomy 27--34              &                  &  9 Feb.        \\
		& \textbf{Midterm Exam} (given in class) &           & 11 Feb.        \\ [1ex]

	\noclass{Reading Week}                                   & 15--19 Feb.    \\ [1ex]

	\unit{Former Prophets} \\

		& Joshua 1--13:7, 22--24          & D. Earl          & 23 Feb.        \\
		& Judges 1--16, 19--21            &                  & 25 Feb.        \\ [1ex]

		& 1 Samuel 1--15                  &                  &  1 Mar.        \\
		& 1 Samuel 16--31, 2 Samuel 1--12 & M. Sweeney       &  3 Mar.        \\ [1ex]

		& 1 Kings 1--14                   &                  &  8 Mar.        \\
		& 2 Kings 17--25                  & O. Lipschits     & 10 Mar.        \\ [1ex]

	\unit{Latter Prophets} \\

		& Isaiah 1--12, 36--39            &                  & 15 Mar.        \\
		& Isaiah 40-54                    & B. Sommer        & 17 Mar.        \\ [1ex]

        & Hosea, Joel, Amos               &                  & 22 Mar.        \\
		& Jonah, Micah, Nahum             & E. Ben Zvi       & 24 Mar.        \\ [1ex]

	\unit{Wisdom \& Praise} \\

		& Ecclesiastes 1--12              &                  & 29 Mar.        \\
		& Psalms 1--2, 72--73, 89--90, 104--106 & M. Cameron & 31 Mar.        \\ [1ex]

	\noclass{Reading Day}                                    &  5 Apr.        \\ [1ex]
	\reminder{\textbf{Final Exam} (as schedule by the Registrar)}{6--13 Apr.} \\
	\bottomrule
  \end{tabular}
  \caption{Schedule of Readings}
  \label{schedule}
\end{table}

See the Registrar's website for a list of other
\href{http://www.tyndale.ca/registrar/important-dates}{important dates}.
The last day to add a class, or to drop one without penalty, is
ordinarily the end of the term's second week.

\section{Evaluation}
\label{evaluation}

\subsection{Grade Structure for \ccode}
\label{structure}

\begin{enumerate}

	\item At various points throughout the semester I will announce
	\textbf{reading quizzes}. They are designed to ensure that you have
	read the assigned material carefully, and may be administered online
	(outside of class) or on paper (in class). Online quizzes are open
	book, but they are timed and limited to a single attempt. Quizzes
	given in class are closed book, and they may not be made up in the
	case of absence.

	\item Students are to \textbf{memorize Psalm 8} in the King James
	Version (KJV) and recite it to me privately, in my office, in the
	fourth or fifth week of class. An online sign-up sheet will be
	posted to the class website in week three.

	\item A \textbf{midterm exam} will be given in class as scheduled in
	the course outline (Section~\ref{outline}). At the instructor's
	option, it may include a take-home writing component.

	\item A comprehensive \textbf{final exam} will be held during the
	exam period as schedule by the Registrar (see Section~\ref{exams}).
	It is designed to assess your attainment of the objectives in
	Section~\ref{objectives}. Test your readiness for the final by
	asking yourself how well you can perform the tasks outlined in those
	paragraphs.

\end{enumerate}

The breakdown for the semester's total work is shown in Table~\ref{distribution} (below).

\begin{table}[htbp]
  \centering
  {\lining
  \begin{tabular}{lr}
    \toprule
    Reading Quizzes & 30\% \\
    Memorization    & 10\% \\
    Midterm Exam    & 30\% \\
    Final Exam      & 30\% \\
    \bottomrule
  \end{tabular}}
  \caption{Distribution of Grades}
  \label{distribution}
\end{table}

\ProvidesFile{grades.tex}[2016/09/03 v2.0 -- Course policy]

\subsection{Grading System at AST}
\label{grades}

AST's \href{http://www.astheology.ns.ca/webfiles/AST_2016Calendar_web(A5)-06APR2016.pdf}{Academic
Calendar} provides guidelines and detailed criteria for academic
assessment. Marks are assigned by letter grade using the benchmarks in
\autoref{grade-syst}.

\begin{table}[htbp]
  \centering
  {\lining
  \begin{tabular}{lll}
    \toprule
%    Letter      & Percent & Assessment        \\
%	\midrule
    A+          & 94--100    & Exceptional    \\
    A           & 87--93     & Outstanding    \\
    A\char"2212 & 80--86     & Excellent      \\ [1ex]
    B+          & 77--79     & Good           \\
    B           & 73--76     & Acceptable     \\
    B\char"2212 & 70--72     & Marginal       \\ [1ex]
    C           & 60--69     & Unsatisfactory \\
    F           & 0--59      & Failure        \\
    FP          & 0          & Failure due to Plagiarism \\
    \bottomrule
  \end{tabular}}
  \caption{Summary of Grading System}
  \label{grade-syst}
\end{table}

% More detailed grading criteria from pp. 61--62 of `16.0406-I2-AST Academic Calendar.pdf'
%
%\begin{description}
%  \item[A+ (94-100) ‘Exceptional’]
%    A superior performance with consistent evidence of a comprehensive,
%    incisive grasp of all aspects of the subject matter; a very wide
%    knowledge base; insightful critical evaluation and analysis of the
%    material; an exceptional capacity for original, creative, and/or
%    logical thinking; an exceptional ability to organize, analyse,
%    synthesize, and to express thoughts fluently.
%  \item[A (87-93) ‘Outstanding’]
%    A comprehensive grasp of the subject matter, outstanding evidence of
%    original thought; sound critical evaluation of the material; an
%    excellent ability to organize, analyse, synthesize and to express
%    thoughts; mastery of an extensive knowledge base.
%  \item[A- (80-86) ‘Excellent’]
%    All the qualities of a B-level performance and an excellent capacity
%    for original, creative, and/ or logical thinking; excellent ability
%    to organize, analyse, synthesize, and integrate ideas; broad
%    knowledge base in the subject matter.
%  \item[B+ (77-79) ‘Good’]
%    A good performance with substantial knowledge of the subject matter;
%    a very good understanding of the relevant issues; familiarity with
%    relevant literature and techniques; good ability to organize,
%    analyse, and examine the material in a constructive and critical
%    manner.
%  \item[B (73-76) ‘Acceptable’]
%    A generally adequate performance with a good knowledge of the
%    subject matter; a fair understanding of relevant issues; some
%    ability to work with relevant literature and techniques; some
%    ability to develop solutions to difficult problems related to the
%    subject material.
%  \item[B- (70-72) ‘Marginally Acceptable’]
%    Some familiarity with the subject material; some understanding.
%    Satisfactory understanding of relevant issues; attempts to solve
%    moderately difficult problems related to the subject material in a
%    critical and analytical manner are only partially successful.
%  \item[C (60-69) ‘Unsatisfactory’]
%    A C grade indicates unsatisfactory academic performance. At the
%    discretion of the instructor, supplemental work may be negotiated to
%    upgrade the mark to a B range. A student may carry two C grades
%    without penalty in all courses except Foundations Courses,
%    Supervised Field Education, Supervised Ministry Practicum and the
%    Graduate Project. In these courses, a minimum grade of B- is
%    required to graduate. A student who receives a C in a Foundation
%    course must repeat the course to achieve a B- or better, and cannot
%    use the C grade to meet prerequisite requirements for advanced
%    courses. If the student repeats one of these courses and receives a
%    B- or better, the previous C grade remains on the transcript and can
%    be counted toward the total of unsatisfactory grades that may lead
%    to academic dismissal. Credit will be given only once for any
%    course. (See Policy on Unsatisfactory Academic Performance in the
%    AST Student Handbook.)
%  \item[F (0-59) ‘Failure’]
%    Student has not grasped subject matter; does not understand issues
%    involved; cannot work with relevant literature. (See Policy on
%    Unsatisfactory Academic Performance in the AST Student Handbook.)
%  \item[P ‘Pass’]
%    Credit awarded, but no mark assigned.
%  \item[FP ‘Failure due to Plagiarism’]
%    A student will receive this grade only after proven incident(s) of
%    plagiarism in a course.
%\end{description}

\section{Policy on Assignments \& Exams}
\label{policy}

\ProvidesFile{academic_calendar.tex}[2015/09/10 v1.1 -- Course policy]

\newcommand{\AC}{Academic Calendar}
\newcommand{\SecAC}{the \AC}% Starting in 2015, the AC no longer has sections!

All policy in Sections~\ref{policy}, \ref{expectations} and \ref{support} of
this syllabus applies to this course in addition to policy in the current
\href{http://www.tyndale.ca/registrar/calendar}{\AC}. In some cases the
syllabus underscores the general policy, while in other cases it supersedes it.

For all matters not covered in this syllabus, refer to \SecAC, ``University
College Academic Policies, Procedures, and Notices.'' Students are strongly
encouraged to read this document carefully at least once in their career at
Tyndale, and to review it every year they matriculate.

\ProvidesFile{assignments.tex}[2015/07/28 v1.1.1 -- Course policy]

\subsection{Assignments}
\label{assignments}

This is a university course. All papers and other writing assignments
should therefore be written at the university level. Submissions must be
typewritten and double-spaced, should be free from error, and in this course
should follow the \emph{SBL Handbook of Style} (refer to the free, online
\href{https://www.sbl-site.org/wp-content/uploads/2025/04/SBLHSsupp2015-02.pdf}{SBLHS
Student Supplement}.)

If you ever struggle with composition---anything from the relatively simple
matters of spelling, grammar and proper citation to deeper-level issues of
tone, structure and argument---then please make use of the Writing Centre (see
Section~\ref{centre}). Experienced writers know that drafts and peer
feedback are integral to the writing process. Inexperienced writers are often
unaware that their surface-level errors create credibility problems with their
readers. When you \href{http://theoatmeal.com/comics/misspelling}{misspell
common words}, fail to know \href{http://theoatmeal.com/comics/apostrophe}{how
to use an apostrophe}, or do not bother to cite your sources correctly, why
should your readers trust you with more important matters like the facts and
ideas under discussion?

\subsubsection{Deadlines}
\label{deadlines}

Assignments \emph{must} be submitted on time. Even if the work is rough or
incomplete, you must turn in something by the due date to receive any credit
whatsoever. Unless I specify differently in class, papers and take-home exams
are due by 11:59 \textsc{pm} on the due date. All other work is due at the
start of the day's class.

Note that, because no late work is accepted in this class, there is no scale
of penalty for unexcused late assignments. If a truly extraordinary event
keeps you from doing your best work, then let me know so that we can make
special arrangements. I am guided by the \AC\ in what counts as extenuation.
``Extensions are not granted for what best could be described as `poor time
management' or `over-involvement' in an extracurricular activity.''

\subsubsection{Submission as PDFs}
\label{submission}

Papers and some other assignments in this course are to be submitted
electronically through the course pages (Section~\ref{lms}). To preserve
formatting, formal writing assignments must be uploaded in Portable Document
Format. There are many ways of creating PDFs; it is your responsibility to
know how to do so on the computer platform you use, and to generate and submit
your PDFs on time.

\subsubsection{Backup}
\label{backup}

In the event of the loss of assignments post-submission---electronic systems
fail, and my office has flooded before---students are required to keep backup
copies of all assignments submitted.

Learning how to secure and preserve your work is a peculiar challenge of the
digital age. Plan on the crash of your hard drive, and the theft of your
laptop (the first is inevitable, the second quite probable). If you do not
have a backup strategy, I recommend that you start with a free account on
\href{http://db.tt/U7eP1vs}{dropbox.com}.

\ProvidesFile{exams.tex}[2013/08/19 v1.0 -- Course policy]

\subsection{Examinations}
\label{exams}

My examination policy follows that outlined in \SecAC, part of which is
summarized below for emphasis.

\begin{enumerate}
  \item
    Midterm exams will be held as scheduled by the instructor. If you miss
    the exam for a legitimate reason, you must write the exam within the same
    number of days that you were absent from school (possibly the next day).
  \item
    Final examinations will take place during the exam period as scheduled
    by the Registrar. Students are responsible for noting the date, time and
    location of their final exam in this class. Students are also responsible
    for familiarizing themselves with the Registrar's examination policies.
  \item
    Special rules apply when midterms and finals are held, including one that
    prohibits students from leaving their seats during the final fifteen minutes
    of the exam period. See the \AC\ for full details.
%    The following rules apply to every final examination:
%    \begin{enumerate}
%      \item
%        No student is permitted to take into the examination room any materials
%        relating to the examination subject, including Bibles.
%      \item
%        No student may leave the room without permission from the proctor.
%      \item
%        No student may leave his or her seat during the final fifteen minutes.
%      \item
%        Students must not linger in the halls outside the examination rooms
%        while examinations are being written.
%      \item
%        No student will be permitted to write beyond the allotted time without
%        special permission of the Registrar (see Section~\ref{accessibility}).
%    \end{enumerate}
  \item
    Provisions exist for students who are justifiably unable to write the final
    exam at the scheduled time. See the \AC\ for details, and make arrangements
    through the Office of the Registrar.
  \item
    Normally, a final exam can only be reschedule in two circumstances:
    (a) a documented illness, or (b) a conflict with another
    exam (two at the same time, or three within 24 hours).
    \href{http://www.tyndale.ca/registrar/final-exam-schedule-and-policies}
    {Apply to the Registrar} in either case.
\end{enumerate}


\section{Student Expectations \& Guidelines}
\label{expectations}

\ProvidesFile{academic_integrity.tex}[2013/08/19 v1.0 -- Course policy]

\subsection{Academic Integrity}
\label{integrity}

Integrity in academic work is required of all students. Academic dishonesty
is any breach of this integrity. It includes such practices as cheating (the
use of unauthorized material on tests and examinations), submitting the same
work for different classes without permission of the instructors, using false
information in an assignment (including false references to secondary sources),
improper or unacknowledged collaboration with other students, and plagiarism.

Tyndale takes seriously its responsibility to uphold academic integrity,
and to apply consequences for academic dishonesty. Consult \SecAC\ for more
information on the school's policy and its application to your work in this
course.

\ProvidesFile{attendance.tex}[2013/08/22 v1.0 -- Course policy]

\subsection{Attendance}
\label{attendance}

``Faithful attendance at classes is an important indicator of student maturity
and involvement'' (\AC). Remember, too, that you are responsible for everything
that happens in every class. Your best policy is to attend and engage. Please
do not ask me to repeat for your benefit anything I have said in a class you
have missed.

Keeping a record of attendance is mandatory for faculty at Tyndale (in contrast
to many other colleges and universities). The University College publishes
guidelines for how attendance should bear on your final evaluation in a course,
and I adhere to them. Note that four lates equals one absence.

What should you do if you miss an undue number of classes? First, arrange for
a classmate to brief you on the material missed, or get my permission for a
classmate to make a recording for you (see Section~\ref{recording}). Second,
notify the Dean of Students in person or by phone. If illness is the cause you
will need to submit a doctor's certificate upon return. The Dean of Students
will notify your professors of the reason for the absence and suggest that they
take this into consideration when assigning grades.

\ProvidesFile{technology.tex}[2014/09/02 v1.1 -- Course policy]

\subsection{Technology}
\label{technology}

Technological innovation has brought students and educators a number of
powerful new tools, and I encourage you to use them as you research, write, and
collaborate. Some of these tools also call for disciplined use and management.

\subsubsection{Email}
\label{email}

Email can be a chore, and you may prefer other channels of communication. As a
matter of policy, however, students must use their myTyndale accounts for all
course-related email correspondence. During term time you should check your
school account at least once a day (optional on weekends). I myself aim to
check my school email at the beginning and end of each workday. At other times
my email client is often closed. I will try to answer your messages within 24
hours, though you should not expect replies on weekends.

Keep your messages topical and brief. I would vastly prefer to conduct any
conversations of substance in person, or else over the phone. Please note and
make use of my office hours. If these hours do not suit your schedule, I would
happily receive an email from you requesting an alternate meeting time.

% Note the prof who adopted an anti-email policy! https://www.insidehighered.com/news/2014/08/27/sake-student-faculty-interaction-professor-bans-student-email?utm_source=slate&utm_medium=referral&utm_term=partner

\subsubsection{MyTyndale.ca / classes.tyndale.ca}
\label{mytyndale}

Tyndale's course pages are an efficient means of distributing articles, notes,
slides, and other course-related materials. This is also where instructors log
attendance and upload grades for assignments. Students are therefore required to
check the site for updates about their classes as well as for any materials
needed for lectures and assignments.

My own use of this platform varies from semester to semester, and from
course to course. At times I may ask you to use the forums, quiz module, or
other parts of the system. At a minimum I will use the site as a repository
for course materials, and as a destination for your submission of PDFs
(Section~\ref{submission}).

\subsubsection{Laptops and Other Devices}
\label{laptops}

Use of laptops is forbidden in my classroom, except to facilitate presentation.
I implement this policy because of the cognitive costs of multitasking, with the
aim of giving you and your peers the best chance of success. I also hope to
foster a culture of keen intellectual engagement.

As \href{http://dx.doi.org/10.1016/j.compedu.2012.10.003}{cognitive
psychologists at McMaster and York Universities demonstrated in 2013}, ``laptop
multitasking hinders classroom learning for both users and nearby peers.'' There
is little new in their finding that the allegedly multitasking student does less
well in class (11\% worse on the quiz in their experiment). This effect has been
shown many times. Rather, their novel discovery is that classmates
\emph{without} laptops who sat with a \emph{view} of another student's screen
did worse than the students who had a computer (17\% worse than those with no
laptop in sight).

Prohibiting laptops is not the only possible response to these findings.
However, there is evidence that \href{http://on.wsj.com/pjtJaK}{writing by hand}
brings a number of cognitive benefits, and a
\href{http://dx.doi.org/10.1177/0956797614524581}{2014 Princeton University
study} ``found that students who took notes on laptops performed worse on
conceptual questions than students who took notes longhand.'' If you are a heavy
laptop user then consider this an opportunity to experiment with different
technologies in the classroom.

As for the myriad networked devices that many of us carry, it's a simple matter
of professionalism to keep these things silent and out of sight. E-readers and
tablets are permitted \emph{only if they are used to display the assigned
reading}. If this is how you choose to read, let me invite you to put the
machine in airplane mode while class is in session.

\subsubsection{Recording of Classes}
\label{recording}

Students must request permission from the professor of any class that they
would like to record. Where permission is granted, students are expected to
supply their own equipment. In general I prefer \emph{not} to have my classes
recorded, and I am not at all friendly to being recorded without my knowledge.
In cases where I grant permission, I stipulate that the recordings must be for
personal use only. They should not be shared with other students, even with
students in the same section, and they absolutely must not be posted online or
otherwise distributed.

If a student is not able to attend a lecture and would like to have it recorded,
it is the responsibility of the student first to obtain the professor's
permission, and then to find another student to record the lecture. I will not
make the recording for you.

\ProvidesFile{support.tex}[2014/08/06 v1.1 -- Course policy]

\section{Student Support}
\label{support}

\subsection{Tyndale Writing Centre}
\label{writing_centre}

Through a combination of tutorials, workshops and resources, Tyndale’s Writing
Centre offers a comprehensive program of writing support to Tyndale students,
including individual 30-minute tutoring sessions. Students may bring essays that
have been graded (and, at least for my classes, essays that have not yet been
submitted for a grade) and will receive detailed suggestions for improving their
writing. This service, at no charge to students, is available by appointment.

Professors may recommend that a student go to the Writing Centre for help:
students are strongly encouraged to follow such recommendations. The Academic
Standards Committee may require an undergraduate student who is experiencing
difficulty in his or her academic program to go to the Writing Centre for
assistance and support. Many top students also elect to go.

\subsection{Tyndale UC Tutoring Program}
\label{tutoring}

Tyndale University College is committed to helping its students achieve academic
success. For this purpose, students in need of academic assistance may request
peer tutoring, free of charge, in each academic department. This includes
students on academic probation, students who have received failing grades in a
course or courses, or students who have been referred for tutoring by their
instructor.

For more information on scheduling tutoring appointments, or for those
interested in becoming peer tutors, students may contact the Office of the
Senior Vice President Academic or their respective University College department
chairs.

\subsection{Accommodation}
\label{accommodation}

Students with documented disabilities may be granted special accommodation for
exams, and in some cases for other assignments. It is even possible to get
permission to use a laptop in class (Section~\ref{laptops}), although I will
need to be convinced of the use case. It is up to the student to contact the
Dean of Students as early as possible in the semester---not later than the
second week---and to document the need. The Dean of Students will then advise
each of the student's professors of the accommodations that may be required.
Please note that special arrangements for assignments need to be made with me
well in advance of assignment due dates (Section~\ref{deadlines}). Timely
requests shall not unreasonably be denied.


\section{Course Bibliography}
\label{bibliography}

An introductory bibliography is provided at the end of each page of
notes in the coursepack. Pursue this material as your interest dictates,
or if you wish to check my handling of any of the material I present.
Pay special attention to items flagged as supplementary reading
(Section~\ref{supplementary}).

\end{document}
