% Copyright (c) 2023 by Daniel R. Driver.
% !TEX encoding = UTF-8 Unicode
% !TEX TS-program = XeLaTeX

\documentclass[titlepage]{article}

% This document presumes a file structure and set of inputs that are
% available at: git@github.com:danieldriver/syllabi.git

\newcommand\policy{../policy}
\newcommand\incl{../includes}
\ProvidesFile{variables.tex}[2018/05/24 v2.1 -- Syllabus variables]

\usepackage{xspace} % make manual spaces (like \mycmd\ ) unnecessary
\usepackage{xifthen} % provides \isempty test

% variables for internal use
\newcommand\prof{}
\newcommand\pdegree{}
\newcommand\pphone{}
\newcommand\pemail{}
\newcommand\poffice{}
\newcommand\phours{}
%
\newcommand\ccode{}
\newcommand\ctitle{}
\newcommand\cseries{}
\newcommand\cversion{}
\newcommand\csemester{}
\newcommand\cmeetson{}
\newcommand\cmeetsat{}
\newcommand\cmeetsin{}
\newcommand\cwebsite{}
\newcommand\cdescrip{}
\newcommand\cprereqs{}
\newcommand\edobject{}

% in case of fully online courses - https://tex.stackexchange.com/a/5896
\newif\ifonline
\newcommand\Int[2]{\ifonline#1\else#2\fi}

% commands for setting variables in the preamble
\newcommand\professor[2][PhD]{
  \renewcommand\pdegree{#1\xspace}
  \renewcommand\prof{#2\xspace}}
\newcommand\phone[1]{
  \renewcommand\pphone{\addfontfeatures{Numbers=Monospaced}#1\xspace}}
\newcommand\email[1]{
  \renewcommand\pemail{\href{mailto:#1}{#1}\xspace}}
\newcommand\officehours[2][Library, Room 5-North]{
  \renewcommand\poffice{#1\xspace}
  \renewcommand\phours{#2\xspace}}
%
\newcommand\coursecode[2][1.0]{
  \renewcommand\cversion{#1\Int{-i}{}\xspace}
  \renewcommand\ccode{#2\Int{(Int)}{}\xspace}}
\newcommand\coursetitle[2][]{
  \ifthenelse{\isempty{#1}}%
    {}% do nothing if #1 is empty, else:
    {\renewcommand\cseries{#1\\[1ex]}}
  \renewcommand\ctitle{#2\xspace}}
\newcommand\semester[1]{
  \renewcommand\csemester{#1\xspace}}
\newcommand\meets[3]{
  \newcommand\AM{\textsc{am}}
  \newcommand\PM{\textsc{pm}}
  \renewcommand\cmeetson{#1\xspace}
  \renewcommand\cmeetsat{\Int{From 9:00 \AM}{#2}\xspace}
  \renewcommand\cmeetsin{\Int{\href{https://smu.brightspace.com/d2l/login}{Brightspace}}{#3}\xspace}}
\newcommand\website[1]{
  \renewcommand\cwebsite{\href{http://#1}{#1}\xspace}}
\newcommand\cdescription[2][RM 1000 or GTRS 6000; and BF 1001]{
  \renewcommand\cprereqs{#1}
  \renewcommand\cdescrip{#2\par}}
\newcommand\objectives[1]{
  \renewcommand\edobject{#1\par}}


%\onlinetrue % \Int{true}{false}
\coursecode[4.0.1]{BF 1001}
\coursetitle[Biblical Foundations]{Hebrew Bible/Old Testament}

% Taught as RLGS 1013 in:
%   - Fall 2008
%   - Winter 2009
%   - Fall 2009
%   - Winter 2010
%   - Fall 2010
%   - Winter 2011 (Fall 2011 was covered by Joel Lohr in a course release)
%   - Winter 2012
%   - Fall 2012
%   - Winter 2013
% Taught as BSTH 1013 in:
%   - Fall 2013
%   - Winter 2014
%   - Fall 2014
%   - Winter 2015
%   - Fall 2015
%   - Winter 2016
% Taught as BF 1001 in:
%   - Fall 2016
%   - Fall 2017
%   - Fall 2017(Int - online section run parallel to campus course)
%   - Fall 2018
%   - Winter 2020(Int) - Nate Wall helped as TA
%   - Fall 2020 - not scheduled Int, but driven online by Covid-19
%   - Fall 2021(Int)
%   - Fall 2022(Int) - run in parallel as DTS 2000: Intro to the Old Testament
%   - Fall 2023 - Version 4.0; back to campus; only text: NJPS Study Bible

\professor{Daniel R. Driver}
\phone{902-425-7051}
\email{daniel.driver@astheology.ns.ca}
\officehours{Thursdays, 2:00--4:00 \PM}

\semester{Fall 2023}
\meets{Thursdays}% \meets{on}{at}{in} - on could be: \Int{Mondays}{Thursdays}
      {6:00--8:30 \PM}% if online, it prints "From 9:00 AM"
      {Flahiff Room}% if online, it prints "Brightspace" with link
\website{danieldriver.com}
\cdescription[none. The course is required for HB courses at or above the 3000-level. It is open to Continuing Education participants]{% copy from the current Academic Calendar; [] for prereqs
	The First Testament of Christian Scripture is a fundamental part of
	Christian tradition and durable rule of faith and practice. Students in
	this course will be introduced to historical and literary data important
	for understanding the origins of the Hebrew Bible/Old Testament and its
	subsequent function as scripture in a variety of contexts. The outlook
	will be broadly ecumenical, with case studies that exemplify Jewish and
	Christian interpretation.

	The broad sweep of biblical tradition will be presented through a survey
	of representative books from the Law, the Prophets, and the Writings. To
	help foster an ecumenical outlook, primary readings from the Bible will
	be supplemented by illustrative commentary from various biblical
	interpreters. Students will develop their exegetical skills by studying
	these examples, and so learn to appreciate the diverse canonical,
	cultural, hermeneutical, historical, literary, and theological elements
	involved in the art of biblical interpretation.
}% end of course description
\objectives{% recall Bloom's taxonomy: http://www.celt.iastate.edu/teaching/RevisedBlooms1.html
	By the end of the course students should be able to:
		name major Old Testament people and events;
		locate a few important biblical sites on a map;
		give key dates for Israel's history and summarize the succession of superpowers in the ancient Near Eastern political theatre from Egypt to Greece;
		recognize and cite examples of most genres of biblical literature;
		defend a decision to pronounce or circumlocute the Divine Name;
		understand the general shape of the Masoretic Text tradition and differentiate it from other canonical orders;
		report on parallel and divergent material across the Law and the Prophets;
		classify and begin to evaluate a variety of ancient and modern approaches to the Bible.

	Students should also be able to identify settings in which the
	Scriptures of Israel are read (notably the synagogue, church, and
	academy), employ terminology appropriate to these communities, recognize
	where their own biographies place them in relation to the Hebrew
	Bible/Old Testament and its uses, and monitor and test their individual
	attitudes and assumptions. They should be able to extend their awareness
	of the Bible's contemporary readers to the Bible's long history of
	reception. Finally, students should begin to infer what Jesus meant in
	speaking of “the law of Moses, the prophets, and the psalms” (Luke
	24:44), and so learn to hear claims about New Testament fulfilment of
	scripture in light of the unique voice that the Old Testament retains
	along side of the New in Christian Scripture.
}% end of learning objectives

\ProvidesFile{preamble.tex}[2013/09/06 v1.0 -- Syllabus preamble]

% basic typography
\usepackage{fontspec}
\setmainfont[Ligatures=TeX]{Meta Serif Pro}
\setsansfont[Ligatures=TeX]{Meta Pro}
\newfontfamily\Heb{Meta Hebrew}
\setmonofont[Scale=MatchLowercase]{Menlo}
\usepackage{sectsty}
\allsectionsfont{\sffamily}
\frenchspacing
\setlength{\emergencystretch}{3em} % prevent overfull lines

% custom font size and leading
\renewcommand\tiny{\fontsize{6}{9}\selectfont}
\renewcommand\scriptsize{\fontsize{7}{10}\selectfont}
\renewcommand\footnotesize{\fontsize{8}{11}\selectfont}
\renewcommand\small{\fontsize{8.5}{11.5}\selectfont}
\renewcommand\normalsize{\fontsize{9}{12}\selectfont}% base size
\renewcommand\large{\fontsize{11}{14}\selectfont}
\renewcommand\Large{\fontsize{13}{16}\selectfont}
\renewcommand\LARGE{\fontsize{16}{19}\selectfont}% "course syllabus \\ semester" benefits from more lead
\renewcommand\huge{\fontsize{19}{21}\selectfont}
\renewcommand\Huge{\fontsize{24}{26}\selectfont}

% layout packages: page, logo, tables
\usepackage[scale={0.6,0.8},
            xetex]{geometry}
\usepackage{graphicx}
\usepackage{array}     % allow insertions of column styling with >{}
\usepackage{booktabs}  % elegant horizontal rules in tables
\usepackage{marginfix} % protect positioning of margin table in policy/grades

% custom macros for a session count in the schedule of readings
\newcounter{session}
\newcounter{columns}
\newcounter{courseunit}
\newcommand\setcolumncount[2][0]{ % optionally set count to other than 0,
  \setcounter{session}{#1}        % e.g. to -1, or to a standing count
  \setcounter{columns}{#2}}
\newcommand\sessioncount{\stepcounter{session}\arabic{session}}
\newcommand\sessionskip[1]{\multicolumn{1}{@{}r@{ }}{#1}}
\newcommand\unit[1]{\multicolumn{\thecolumns}{c}{%
  \scshape\stepcounter{courseunit}\roman{courseunit}. \MakeLowercase{#1}}}
\newcommand\noclass[1]{\multicolumn{1}{@{}l}{\itshape No Class: #1}}

% color to match Tyndale's branding
\usepackage[usenames]{xcolor}
% predefined: black, white, red, green, blue, cyan, magenta, yellow
\definecolor{TyndaleURLs}{HTML}{0062A0} % links on tyndale.ca
\definecolor{TyndaleBlue}{cmyk}{1,1,0,.32}
\definecolor{TyndaleGold}{cmyk}{0,.27,1,0}
\definecolor{TyndaleRed}{cmyk}{0,1,.99,.04}
\definecolor{TyndaleBlack}{cmyk}{0,0,0,1}
\definecolor{TyndaleGreen}{cmyk}{.45,0,1,.24}
\definecolor{TyndaleOrange}{cmyk}{0,.79,1,0}
\definecolor{TyndaleAqua}{cmyk}{.47,0,.24,0}
\definecolor{TyndaleYellow}{cmyk}{.03,.03,.35,0}

% metadata (assumes a host of definitions are made in the main file)
\usepackage[setpagesize=false,     % leave this to geometry
            hyperfootnotes=false,  % fragile and distracting
            xetex]{hyperref}
\hypersetup{breaklinks=true,       % allow link text to break across lines
            colorlinks=true,       % colorlinks resets pdfborder to 0 0 0
            urlcolor=TyndaleURLs,  % for external links
            linkcolor=TyndaleRed,  % for normal internal links
            citecolor=TyndaleGold, % for bibliographical citations in text
            pdfauthor={\prof},
            pdftitle={\ccode: \ctitle},
            pdfsubject={Tyndale UC, \csemester},
            pdfcreator={github.com/danieldriver/syllabus}}
\urlstyle{same}                    % don't use monospace font for urls

% custom footlines
\usepackage{fancyhdr}
\pagestyle{fancy} % turn it on
\fancyhf{}        % reset everything
\renewcommand{\headrulewidth}{0pt} % remove header line as well
\lfoot{\sffamily\scshape\footnotesize\MakeLowercase{\ctitle, v\cversion}}
\rfoot{\sffamily\scshape\footnotesize\MakeLowercase{\prof\quad\thepage}}

% gratuitous with custom title page, but useful as a fallback
\title{\ccode: \ctitle}
\author{\professor}
\date{\semester}


\begin{document}
\ProvidesFile{title.tex}[2013/09/06 v1.0 -- Syllabus title page]

\begin{titlepage}
  \begin{center}

    \LARGE\sffamily % set title elements in a large sans serif

    \begin{minipage}{\textwidth}
      \parbox[t]{0.5\textwidth}{
        \mbox{}\\[-13pt] % dummy line to align parboxes
        \includegraphics[width=0.5\textwidth]{.syllabus/includes/TyndaleUC}}
      \hfill
      \parbox[t]{0.4\textwidth}{
        \raggedleft Course Syllabus\\
        \csemester}
    \end{minipage}

    \vfill

    {\textsc{\MakeLowercase\ccode}\\[1ex]
      \bfseries\cseries\Huge\ctitle}

    \vfill

    \normalsize\rmfamily % switch back to body type

    \begin{tabular}{>{\bfseries}rl>{\bfseries}rl}
      \toprule
      Instructor & \prof, \pdegree & Course  & Version \cversion \\
      \midrule
      Phone      & \pphone         & Meets   & \cmeetson         \\
      Email      & \pemail         & Time    & \cmeetsat         \\
      Office     & \poffice        & Room    & \cmeetsin         \\
      Hours      & \phours         & Website & \cwebsite         \\
      \bottomrule
    \end{tabular}

    \vfill

    \begin{description}\small
      \item[Commuter Hotline]
        Class cancellations due to inclement weather or illness will
        be announced on the commuter hotline at \texttt{416.226.6620
        x2187}. Alternately, weather cancellation information is posted
        at \href{http://tyndale.ca/weather}{tyndale.ca/weather}.
      \item[MyTyndale.ca]
        This course may have materials stored on its website, such as
        handouts or readings that may be needed in order to complete
        assignments. Students are responsible for checking these course
        pages on a regular basis. Here, too, students are able to view
        their grades throughout the semester. For more information see
        Section~\ref{mytyndale}, below.
      \item[Mail]
        Students are responsible for information communicated through
        their campus mailboxes and student e-mail accounts. A mailbox
        directory hangs beside the mailboxes. For more information
        contact the Registrar's office.
    \end{description}

  \end{center}

  \section{Course Description}
  \label{description}

  \emph{From the Academic Calendar:} \cdescrip

\end{titlepage}
\setcounter{page}{2} % count the title page as page 1


  \section{Learning Objectives}
  \label{objectives}

  \edobject

\section{Required Texts \& Materials}
\label{texts}

The following text is required. Students are strongly encouraged to
purchase their own copy. A non-circulating copy is held in the AST
Library: Reference BS 895 J4 2014.

\begingroup
\renewcommand{\section}[2]{}% temporarily remove the section heading
\begin{thebibliography}{NJPS}% use the longest item in the bibliography

	\bibitem[NJPS]{njps} Berlin, Adele, and Marc Zvi Brettler, eds.
    \emph{The Jewish Study Bible: Jewish Publication Society Tanakh Translation}.
    2nd ed. Oxford: Oxford University Press, 2014.
    % ISBN 978-0199978465.

\end{thebibliography}
\endgroup

Optionally, for an ecumenical study Bible with a gender-inclusive
translation of the full Christian Bible, see Michael D. Coogan, ed.,
\emph{The New Oxford Annotated Bible: New Revised Standard Version with
the Apocrypha}, 5th ed. (Oxford: Oxford University Press, 2018).
% ISBN 978-0190276072.

\section{Supplementary Texts}
\label{supplementary}

Supplementary readings will be recommended throughout the semester and
either placed on reserve or made available through the course website.
See the bibliography below, in \autoref{bib}. Please give this material
good effort and attention.

Introductions to the Hebrew Bible/Old Testament abound. I have not
required one for this class, though I have assigned a number of them in
the past. Students may find one or more the following introductory and
companion volumes useful.

\begingroup
\renewcommand{\section}[2]{}% temporarily remove the section heading
\begin{thebibliography}{99}% use the longest item in the bibliography

	\bibitem{jb16} Barton, John, ed.
	\emph{The Hebrew Bible: A Critical Companion}.
	Princeton: Princeton University Press, 2016. 

	\bibitem{bb05} Birch, Bruce, Walter Brueggemann, Terence E. Fretheim, and David L. Petersen.
	\emph{A Theological Introduction to the Old Testament}.
	2nd ed. Nashville: Abingdon, 2005.
	AST Library: BS 1192.5 T43 2005.

	\bibitem{wbtl21} Brueggemann, Walter, and Tod Linafelt.
	\emph{An Introduction to the Old Testament: The Canon and Christian Imagination}.
	3rd ed. Louisville: Westminster John Knox, 2021.
	AST Library (2nd ed.): BS 1140.3 B78 2012.

	\bibitem{scms16} Chapman, Stephen B., and Marvin A. Sweeney, eds.
	\emph{The Cambridge Companion to the Hebrew Bible/Old Testament}.
	New York: Cambridge University Press, 2016.
	AST Library: BS 1140.3 C35 2016.

	\bibitem{jc04} Collins, John J.
	\emph{Introduction to the Hebrew Bible}.
	Minneapolis: Fortress, 2004.
	AST Library: BS 1140.3 C65 2004.

	\bibitem{mc10} Coogan, Michael D.
	\emph{The Old Testament: A Historical and Literary Introduction to the Hebrew Scriptures}.
	3rd ed. New York: Oxford University Press, 2014.
	AST Library: BS 1197 C56 2014.

	\bibitem{pdjr05} Davies, Philip R., and John Rogerson.
	\emph{The Old Testament World}.
	2nd ed. Louisville: Westminster John Knox, 2005.
	AST Library: BS 1171.3 D38 2005.

	\bibitem{ed19} Davis, Ellen F.   
	\emph{Opening Israel's Scriptures}.
	New York: Oxford University Press, 2019.
	AST Library: \href{https://doi.org/10.1093/oso/9780190260545.001.0001}{online}.

	\bibitem{ahjw09} Hill, Andrew E., and John H. Walton.
	\emph{A Survey of the Old Testament}.
	3rd ed. Grand Rapids: Zondervan, 2009.

	\bibitem{hbfb} Kaminsky, Joel S., and Joel N. Lohr.
	\emph{The Hebrew Bible for Beginners: A Jewish and Christian Introduction}.
	Nashville: Abingdon Press, 2015.
	AST Library: BS 1171.3 K39 2015 and \href{https://search.ebscohost.com/login.aspx?direct=true&AuthType=cookie,ip,shib&db=nlebk&AN=969753&site=ehost-live&scope=site&custid=s5315951}{online}.

	\bibitem{jk07} Kugel, James L. 
	\emph{How To Read the Bible: A Guide to Scripture, Then and Now}.
	New York: Free, 2007.

	\bibitem{hprb10} Page, Hugh R., Jr., and Randall C. Bailey, eds.
	\emph{The Africana Bible: Reading Israel’s Scriptures from Africa and the African Diaspora}.
	Minneapolis: Fortress, 2010.
	AST Library: BS 1171.3 A37 2010.

	\bibitem{rr86} Rendtorff, Rolf.
	\emph{The Old Testament: An Introduction}. Translated by John Bowden.
	Philadelphia: Fortress, 1986.
	AST Library: BS 1140.2 R3913 1986.

	\bibitem{ks12} Schmid, Konrad.
	\emph{The Old Testament: A Literary History}. Translated by Linda Maloney.
	Minneapolis: Fortress, 2012.
	AST Library: BS 1174.3 S3613 2012.
	
\end{thebibliography}
\endgroup

Also, the following basic works are also worth consulting and even owning.
\cite{rlgs} in particular contains sound advice on core skills like
reading religious texts, writing essays and reviews, revising essays,
making oral presentations, and learning languages.

\begingroup
\renewcommand{\section}[2]{}% temporarily remove the section heading
\begin{thebibliography}{Making}% use the longest item in the bibliography

	\bibitem[Making]{rlgs} Northey, Margot, Bradford A. Anderson, and Joel N. Lohr.
	\emph{Making Sense in Religious Studies: A Student's Guide to Research and Writing}.
	3rd ed. Toronto: Oxford University Press, 2019. AST Library: Reference BL 41 N67 2019.

	\bibitem[SBL2]{sbl2} Collins, Billie Jean, et al.
	\emph{The SBL Handbook of Style}.
	2nd ed. Atlanta: SBL Press, 2014.
	Designed to augment \href{https://proxy.openathens.net/login?qurl=https%3A%2F%2Fwww.chicagomanualofstyle.org%2Fbook%2Fed17%2Ffrontmatter%2Ftoc.html}{\emph{Chicago Style}}
	(the standard at AST), there is also a free
	\href{https://www.sbl-site.org/assets/pdfs/pubs/SBLHSsupp2015-02.pdf}{Student Supplement for SBL2}. AST Library: Reference PN 147 S26 2014.

\end{thebibliography}
\endgroup


\section{Course Outline}
\label{outline}

We will adhere to the schedule in \autoref{schedule} as closely as
possible, though the professor reserves the right to adjust it to suit
the needs of the class.

\setcolumncount{4}% set up \sessioncount, \unit{}, \noclass{}, and \reminder{memo}{date} macros
\begin{table}[htbp]% set to `p' to put the schedule on its own page
  \centering
  \begin{tabular}{>{\sessioncount.}r@{ }llr}% make sure the column config agrees with \setcolumncount
	\toprule
	\sessionskip{\textbf{\S}.}&\textbf{Primary reading (chs)}&\textbf{Supplementary reading}&\textbf{Due}\\
	\midrule

	\unit{Law / Torah / Pentateuch} \\

		& Genesis 1--11        & Davis \cite{ed03}; Seitz \cite{cs96} &  7 Sep. \\
		& Genesis 12--50       & Levenson                 \cite{jl12} & 14 Sep. \\
		& Exodus 1--18         & McGinnis                 \cite{cm12} & 21 Sep. \\
		& Exodus 19--40        & Sommer                   \cite{bs15} & 28 Sep. \\
	    & Leviticus (all if no chs) & Anderson            \cite{ga17} &  5 Oct. \\
	\noclass{Nicholson Lectures by E. Radner @ 7pm}                   & 12 Oct. \\
	\reminder{A short paper is \textbf{due} before the \textbf{sixth week} of class}{17 Oct.} \\% Have moved btw wks 5 and 6 depending on how holidays (eg, Thanksgiving Monday) impact the schedule
		& Deuteronomy          & Heschel \cite{ah51}; Moberly \cite{wm13} & 19 Oct. \\ [1ex]

	\unit{Prophets: Former \textit{\&} Latter} \\

	\noclass{Reading Week from Tuesday to Friday}                     & 24--27 Oct. \\
		& Joshua; Judges       & Trible                   \cite{pt84} &  2 Nov. \\
		& 1 \& 2 Samuel        & Chapman                  \cite{sc16} &  9 Nov. \\
		& 1 \& 2 Kings         & Alter                    \cite{ra81} & 16 Nov. \\
		& Isaiah 1--12, 36--40 & Irenaeus \cite{il97}; Davis \cite{ed14} & 23 Nov. \\
		& Hosea--Jonah         & Nogalski         \cite{jn07a, jn07b} & 30 Nov. \\ [1ex]

	\unit{Writings: Wisdom \textit{\&} Praise} \\

	\reminder{A short paper is \textbf{due} before the \textbf{twelfth week} of class}{5 Dec.} \\% NB Moved back from wk 11 to wk 10 because of the nearness of the last day of class in F22. Perhaps move back to wk 11 in future.
		& Psalms 1--8; Ecclesiastes & Childs \cite{bc69}; Fox \cite{mf04} & 7 Dec. \\
%		& Ecclesiastes         & Dell \cite{kd13} or  &  7 Dec. \\ [1ex]

	\reminder{End of Term: Final marks are due for all courses}{14 Dec.} \\

	\bottomrule
  \end{tabular}
  \caption{Schedule of Readings}
  \label{schedule}
\end{table}

See the AST website for a list of other \href{https://www.astheology.ns.ca/students/index.html}{important dates}.

\section{Evaluation}
\label{evaluation}

The grade structure for \ccode has the following elements.

\begin{enumerate}

	\item \textbf{Notes and quotes} will be solicited from students at the
	start of each class. These are to be drawn from the primary reading
	assigned for each week. What do you note about the material on this
	occasion? What words, phrases, or verses stand out to you from the
	\cite{njps} translation? Pay attention to things you notice from a
	survey of large sections of the Bible. Stay alert, too, to details that
	strike you from this English edition of the Hebrew (Masoretic) text.
	Make notes as you read. Finally, be prepared to share highlights with
	the class (three minutes maximum). Participation is required.

%	\item \textbf{Writing prompts} will be given throughout the
%	semester. They are low-risk writing exercises designed to encourage
%	careful reading of the assigned material, and to help explore its
%	significance. Individual posts are not marked, though points may be
%	deducted from your final grade if the exercise is not taken
%	seriously. Prompt responses submitted online should be around 250
%	words long, and should not exceed 300 words. Try to post the in the
%	morning of class days, after you do the reading but before you
%	listen to the lecture. You are invited (but not strictly required)
%	to respond to and interact with classmates on the forums through the
%	rest of the day.

	\item Two \textbf{short papers} will facilitate student engagement
	with the art of biblical interpretation. One is keyed to the
	exposition of a model, the other to a short biblical text. Each
	should be about 2,500 words in length. Either paper (your choice) is
	due before the sixth week of class; the other one is due before the
	twelfth week.
	% Deadlines have sometimes been wks 6 and 11 or wks 5 and 10 depending
	% on scheduling issues. Make sure that what is stated here matches the
	% schedule above.

	\begin{enumerate}

		\item A \textbf{review essay} invites student reflection on a model
		work of biblical interpretation. Approved options are listed as
		supplementary readings in \autoref{schedule}. Alternatively, in
		consultation with the professor, you may select another
		supplementary article from \autoref{bib} or propose an entirely
		different source.
		
		Note that a review is not the same thing as a report. Devote the
		first half of the paper to a summary the interpretation or
		argument under review, and be mindful that traditional
		interpretation requires special attention to context.
		(Irenaeus's treatment of Isaiah is quite unlike modern academic
		commentary.) Devote the second half of the paper to critical
		analysis and evaluation of your chosen example. Be fair, but do
		not fail to take a position. The paper needs to develop a
		\textbf{thesis}. See me and \cite[chs 3, 5–7, 11]{rlgs} for
		guidance.

		\item An \textbf{exegetical essay} provides an opportunity for
		direct work with the biblical text. The first task is to identify an
		appropriate text. Select a suitably short passage from the HB/OT.
		Then, conduct an analysis and explication of it. Interact with at
		least three sources and commentators. Advance a \textbf{thesis} that
		relates to the text itself. See me and \cite[chs 3, 5, 8, 11]{rlgs}
		for guidance.

	\end{enumerate}

%\Int{
%	\item One week, instead of answering a writing prompt, each student
%	will make an \textbf{online presentation} on one of the twelve
%	supplementary articles. Be creative! It may take the form of a
%	podcast or online video, and should be sharable by URL (try
%	SoundCloud, YouTube, or Vimeo). The presentation should last 8–10
%	minutes (12 min at most). It must begin with a brief (3–5~min)
%	summary of the article. The balance of the time should be spent
%	helping other readers elaborate, reflect on, test, challenge, or
%	extend the main ideas presented there. Post your link to the forums
%	by the morning so that you can help shape the forum discussions on
%	the same article.
%}{
%	\item Each student will lead a \textbf{seminar discussion} of one
%	chapter from the HB/OT, freely chosen from a week's primary reading.
%	The discussion should last about 25~minutes. It needs to begin with
%	a brief (5~min) \textbf{outline} of the chapter. Outlines must be
%	original, not borrowed from commentators. They should be circulated
%	as handouts and explained orally. The balance of the time should be
%	spent helping the class discover, elaborate, reflect on, test, or
%	otherwise explore the meaning of the passage. Critical to success
%	here is the ability to ask good questions of the Bible.
%}

	\item \textbf{Seminar discussions}: Graduate students will lead
	\textbf{two} seminar discussions over the course of the semester;
	undergraduate (BTh) students will lead \textbf{one}. Handouts may be
	developed and circulated, but they are not required as part of the mark.
	

	\begin{enumerate}

		\item Each student in the course must lead a discussion of one
		chapter from the primary reading, to occur in the same week in which
		it is assigned. Any chapter may be selected. Discussion should last
		twenty minutes. It should begin with a brief overview (five minutes
		maximum) of the chapter and key issues raised by it. The balance of
		the time should be spent helping the class discover, elaborate,
		reflect on, test, or otherwise explore the meaning of the passage.
		Vital to success here is the ability to ask good questions of the
		Bible.
		
		\item Each graduate student must lead a second discussion on a
		supplementary reading. If the reading selection deviates from what
		is set in the syllabus, consult with the professor to make sure the
		alternate reading is circulated at least one week in advance. These
		seminars, too, should occur in the same week in which the material
		is assigned and should last twenty minutes. The allocation of time
		is different, however. Approximately fifteen minutes should be given
		to summary and analysis, with just five minutes at the end for
		questions and class discussion. The goal here is for advanced degree
		students to report on the results of critical inquiries into various
		noteworthy examples of biblical interpretation.
		
		\item Students are encouraged to align their essay writing with
		their seminar topics. For example, it is perfectly acceptable to
		lead a discussion on the biblical chapter in which the short passage
		for your exegetical essay appears. It is also acceptable to let
		class discussion spur revisions to your paper, of course with the
		understanding that you credit others appropriately for their input.

	\end{enumerate}

\end{enumerate}

The breakdown for the semester's total work is shown in
\autoref{grade-dist}.

\begin{table}[htbp]
  \centering
  {\lining
  \begin{tabular}{lr}
    \toprule
    Notes \& Quotes     & 20\% \\
    Review Essay        & 30\% \\
    Exegetical Essay    & 30\% \\
    Seminar Discussions & 20\% \\
%    \Int{Online Presentation}{Seminar Discussion} & 20\% \\
    \bottomrule
  \end{tabular}}
  \caption{Distribution of Grades}
  \label{grade-dist}
\end{table}

%AST's \href{https://www.astheology.ns.ca/students/resources.html}{Academic
%Calendar} provides guidelines and detailed criteria for academic
%assessment. Marks are assigned by letter grade using these benchmarks.

\ProvidesFile{grades.tex}[2016/09/03 v2.0 -- Course policy]

\subsection{Grading System at AST}
\label{grades}

AST's \href{http://www.astheology.ns.ca/webfiles/AST_2016Calendar_web(A5)-06APR2016.pdf}{Academic
Calendar} provides guidelines and detailed criteria for academic
assessment. Marks are assigned by letter grade using the benchmarks in
\autoref{grade-syst}.

\begin{table}[htbp]
  \centering
  {\lining
  \begin{tabular}{lll}
    \toprule
%    Letter      & Percent & Assessment        \\
%	\midrule
    A+          & 94--100    & Exceptional    \\
    A           & 87--93     & Outstanding    \\
    A\char"2212 & 80--86     & Excellent      \\ [1ex]
    B+          & 77--79     & Good           \\
    B           & 73--76     & Acceptable     \\
    B\char"2212 & 70--72     & Marginal       \\ [1ex]
    C           & 60--69     & Unsatisfactory \\
    F           & 0--59      & Failure        \\
    FP          & 0          & Failure due to Plagiarism \\
    \bottomrule
  \end{tabular}}
  \caption{Summary of Grading System}
  \label{grade-syst}
\end{table}

% More detailed grading criteria from pp. 61--62 of `16.0406-I2-AST Academic Calendar.pdf'
%
%\begin{description}
%  \item[A+ (94-100) ‘Exceptional’]
%    A superior performance with consistent evidence of a comprehensive,
%    incisive grasp of all aspects of the subject matter; a very wide
%    knowledge base; insightful critical evaluation and analysis of the
%    material; an exceptional capacity for original, creative, and/or
%    logical thinking; an exceptional ability to organize, analyse,
%    synthesize, and to express thoughts fluently.
%  \item[A (87-93) ‘Outstanding’]
%    A comprehensive grasp of the subject matter, outstanding evidence of
%    original thought; sound critical evaluation of the material; an
%    excellent ability to organize, analyse, synthesize and to express
%    thoughts; mastery of an extensive knowledge base.
%  \item[A- (80-86) ‘Excellent’]
%    All the qualities of a B-level performance and an excellent capacity
%    for original, creative, and/ or logical thinking; excellent ability
%    to organize, analyse, synthesize, and integrate ideas; broad
%    knowledge base in the subject matter.
%  \item[B+ (77-79) ‘Good’]
%    A good performance with substantial knowledge of the subject matter;
%    a very good understanding of the relevant issues; familiarity with
%    relevant literature and techniques; good ability to organize,
%    analyse, and examine the material in a constructive and critical
%    manner.
%  \item[B (73-76) ‘Acceptable’]
%    A generally adequate performance with a good knowledge of the
%    subject matter; a fair understanding of relevant issues; some
%    ability to work with relevant literature and techniques; some
%    ability to develop solutions to difficult problems related to the
%    subject material.
%  \item[B- (70-72) ‘Marginally Acceptable’]
%    Some familiarity with the subject material; some understanding.
%    Satisfactory understanding of relevant issues; attempts to solve
%    moderately difficult problems related to the subject material in a
%    critical and analytical manner are only partially successful.
%  \item[C (60-69) ‘Unsatisfactory’]
%    A C grade indicates unsatisfactory academic performance. At the
%    discretion of the instructor, supplemental work may be negotiated to
%    upgrade the mark to a B range. A student may carry two C grades
%    without penalty in all courses except Foundations Courses,
%    Supervised Field Education, Supervised Ministry Practicum and the
%    Graduate Project. In these courses, a minimum grade of B- is
%    required to graduate. A student who receives a C in a Foundation
%    course must repeat the course to achieve a B- or better, and cannot
%    use the C grade to meet prerequisite requirements for advanced
%    courses. If the student repeats one of these courses and receives a
%    B- or better, the previous C grade remains on the transcript and can
%    be counted toward the total of unsatisfactory grades that may lead
%    to academic dismissal. Credit will be given only once for any
%    course. (See Policy on Unsatisfactory Academic Performance in the
%    AST Student Handbook.)
%  \item[F (0-59) ‘Failure’]
%    Student has not grasped subject matter; does not understand issues
%    involved; cannot work with relevant literature. (See Policy on
%    Unsatisfactory Academic Performance in the AST Student Handbook.)
%  \item[P ‘Pass’]
%    Credit awarded, but no mark assigned.
%  \item[FP ‘Failure due to Plagiarism’]
%    A student will receive this grade only after proven incident(s) of
%    plagiarism in a course.
%\end{description}
\ProvidesFile{other.tex}[2022/06/08 v2.9.1 -- Course policy]

\section{Other Course Policy}
\label{policy}

Late work will not be accepted, except in genuinely extenuating
circumstances. Students must submit something before the deadline if
they wish to receive credit. Unless I state otherwise, assignments are
to be uploaded by 11:59 \PM\ (Atlantic) on the date indicated.

Essay submissions must be typewritten and double-spaced. They should be
free from error. In this course they should follow SBL Style (see
\cite{sbl2} in \autoref{supplementary}, above). As a reminder, AST
upholds an Inclusive Language Policy. Please use gender-inclusive
language when referring to human beings. Our traditions have different
norms for speech about God; you are of course free to follow and explore
those traditions when referring to God.


Plagiarism is the
\href{http://www.eerdmans.com/Pages/Item/59043/Commentary-Statement.aspx}{failure}
to \href{https://www.theguardian.com/world/2013/feb/09/german-education-minister-quits-phd-plagiarism}{attribute}
(by means of footnotes when writing or aloud when speaking) any ideas,
phrases, sentences, materials, syntheses, et cetera, that another author
has composed and that you have borrowed for your own work. Plagiarism is
unethical. Academic penalties for plagiarism at AST are serious, and may
include failure of the course or even suspension of further studies.
Unintentional plagiarism is considered plagiarism. AST's Plagiarism
Policy is found under that heading in the Academic
Calendar.

Students should request permission to record a class or lecture. If
permission is granted, or if recordings are provided (as in the case of
an online or hybrid course), I stipulate that all recordings be for
personal use only. They may not be shared or distributed.

If you have needs that require modifications to any aspect of this
course, please consult with the instructor as soon as possible. Any
documentation regarding disabilities that you wish to divulge to AST
should be provided to the Registrar’s Office, where it will be kept in a
confidential file.

Finally, I encourage the conscientious use of laptops, tablets, and
other technology in my classes. In classroom settings, realize that, as
\href{http://dx.doi.org/10.1016/j.compedu.2012.10.003}{cognitive
psychologists have demonstrated}, ``laptop multitasking hinders
classroom learning for both users and nearby peers.'' Do your part to
foster an environment for dialogue by honouring the presence of your
classmates. In online and hybrid settings, consider both the physical
environment in which you choose to work and the virtual environment that
you help create through your participation in various forums. Let your
engagement in this course be marked by rigour and charity alike.


\section{Further Bibliography}
\label{bib}

Literature on the Bible is vast. The works listed here have been
selected for clarity, insight, and theological alertness. Some will be
distributed as supplementary texts, as per \autoref{supplementary}.

\begingroup
\renewcommand{\section}[2]{}% temporarily remove the section heading
\begin{thebibliography}{99}
\makeatletter%https://tex.stackexchange.com/a/114434
\addtocounter{\@listctr}{14}
\makeatother

\bibitem{ra81} Alter, Robert. “The Techniques of Repetition.” Pages 88–113 in \emph{The Art of Biblical Narrative}. New York: Basic Books, 1981.

\bibitem{ga01} Anderson, Gary A. “Biblical Origins and the Fall.” Pages 197–210 in \emph{The Genesis of Perfection: Adam and Eve in Jewish and Christian Imagination}. Louisville: Westminster John Knox, 2001.

\bibitem{ga17} Anderson, Gary A. “Apophatic Theology: The Transcendence of God and the Story of Nadab and Abihu.” Pages 3–22 in \emph{Christian Doctrine and the Old Testament: Theology in the Service of Biblical Exegesis}. Grand Rapids: Baker Academic, 2017.

\bibitem{bz09} Ben Zvi, Ehud and James D. Nogalski. \emph{Two Sides of a Coin: Juxtaposing Views on Interpreting the Book of the Twelve / the Twelve Prophetic Books}. Analecta Gorgiana 201. Piscataway, NJ: Gorgias, 2009.

\bibitem{sc16} Chapman, Stephen B. “1 Samuel 1–12.” Pages 71–119 in \emph{1 Samuel as Christian Scripture: A Theological Commentary}. Grand Rapids: Eerdmans, 2016.

\bibitem{bc79} Childs, Brevard S. \emph{Introduction to the Old Testament as Scripture}. Philadelphia: Fortress, 1979.

\bibitem{bc85} Childs, Brevard S. \emph{Old Testament Theology in a Canonical Context}. Philadelphia: Fortress, 1985.

%\bibitem{bc92} Childs, Brevard S. \emph{Biblical Theology of the Old and New Testaments: Theological Reflection on the Christian Bible}. Minneapolis: Fortress, 1992.

\bibitem{bc04} Childs, Brevard S. \emph{The Struggle to Understand Isaiah as Christian Scripture}. Grand Rapids: Eerdmans, 2004.

\bibitem{bc69} Childs, Brevard S. “Psalm 8 in the Context of the Christian Canon.” Pages 85--93 in \emph{Canon as Rule and Guide: Collected Essays}. Edited by Daniel R. Driver. Tübingen: Mohr Siebeck, 2023. Essay first published in 1969.

\bibitem{ed03} Davis, Ellen F. “Teaching the Bible Confessionally in the Church.” Pages 9–26 in \emph{The Art of Reading Scripture}. Edited by Ellen F. Davis and Richard B. Hays. Grand Rapids: Eerdmans, 2003.

\bibitem{ed14} Davis, Ellen F. \emph{Biblical Prophecy: Perspectives for Christian Theology, Discipleship, and Ministry}. Louisville: Westminster John Knox, 2014.

\bibitem{ndw14} deClaissé-Walford, Nancy L. “The Meta-Narrative of the Psalter.” Pages 363–76 in \emph{The Oxford Handbook of the Psalms}. Edited by William P. Brown. Oxford: Oxford University Press, 2014.

\bibitem{kd13} Dell, Katherine J. “Ecclesiastes as Wisdom: Consulting Early Interpreters.” Pages 9–36 in \emph{Interpreting Ecclesiastes: Readers Old and New}. Winona Lake, IN: Eisenbrauns, 2013.

\bibitem{mf04} Fox, Michael V. \emph{Ecclesiastes: The Traditional Hebrew Text with the New JPS Translation}. The JPS Bible Commentary. Philadelphia: Jewish Publication Society, 2004.

\bibitem{ah51} Heschel, Abraham Joshua. \emph{The Sabbath: Its Meaning for Modern Man}. New York: Farrar, Straus and Giroux, 1951.

\bibitem{il97} Irenaeus of Lyons. \emph{On the Apostolic Preaching}. Translated by John Behr. Popular Patristics Series 17. St Vladimir’s Seminary Press, 1997.

\bibitem{bj13} Janowski, Bernd. \emph{Arguing with God: A Theological Anthropology of the Psalms}. Louisville: Westminster John Knox, 2013.

\bibitem{jl85} Levenson, Jon D. \emph{Sinai and Zion: An Entry into the Jewish Bible}. Minneapolis: Winston, 1985.

\bibitem{jl12} Levenson, Jon D. “The Test.” Pages 66–112 in \emph{Inheriting Abraham: The Legacy of the Patriarch in Judaism, Christianity, and Islam}. Princeton: Princeton University Press, 2012.

\bibitem{nm09} MacDonald, Nathan. “Israel and the Old Testament Story in Irenaeus’s Presentation of the Rule of Faith.” \emph{Journal of Theological Interpretation} 3.2 (2009): 281–98.

\bibitem{cm12} McGinnis, Claire Mathews. “The Hardening of Pharaoh’s Heart in Christian and Jewish Interpretation.” \emph{Journal of Theological Interpretation} 6.1 (2012): 43–64.

\bibitem{wm92} Moberly, R. W. L. \emph{The Old Testament of the Old Testament: Patriarchal Narratives and Mosaic Yahwism}. Minneapolis: Fortress, 1992.

\bibitem{wm13} Moberly, R. W. L. “A Love Supreme.” Pages 7–40 in \emph{Old Testament Theology: Reading the Hebrew Bible as Christian Scripture}. Grand Rapids: Baker Academic, 2013.

\bibitem{jn07a} Nogalski, James D. “Reading the Book of the Twelve Theologically.” \emph{Interpretation} 61.2 (2007): 115–22.

\bibitem{jn07b} Nogalski, James D. “Recurring Themes in the Book of the Twelve: Creating Points of Contact for a Theological Reading.” \emph{Interpretation} 61.2 (2007): 125–36.

\bibitem{db89} Patrick, Dale. “Studying Biblical Law as a Humanities.” \emph{Semeia} 45 (1989): 27–47.

\bibitem{fr11} Rutledge, Fleming. \emph{And God Spoke to Abraham: Preaching From the Old Testament}. Grand Rapids: Eerdmans, 2011.

\bibitem{ks19} Schmid, Konrad. \emph{A Historical Theology of the Hebrew Bible}. Translated by Peter Altmann. Grand Rapids: Eerdmans, 2019. ISBN 978-0802876935.

\bibitem{cs96} Seitz, Christopher R. “Old Testament or Hebrew Bible? Some Theological Considerations.” \emph{Pro Ecclesia} 5.3 (1996): 292–303.

\bibitem{cs18} Seitz, Christopher R. “‘Can We Read This Book?’ Reader Response-ability.” Pages 51–68 in \emph{The Elder Testament: Canon, Theology, Trinity}. Waco, TX: Baylor University Press, 2018.

\bibitem{gs92} Sheppard, Gerald T. “Theology and the Book of Psalms.” \emph{Interpretation} 46.2 (1992): 143–55.

\bibitem{bs15} Sommer, Benjamin D. “What Happened at Sinai? Maximalist and Minimalist Approaches.” Pages 27–98 in \emph{Revelation and Authority: Sinai in Jewish Scripture and Tradition}. New Haven: Yale University Press, 2015.

\bibitem{pt84} Trible, Phyllis. “An Unnamed Woman: The Extravagance of Violence.” Pages 65–91 in \emph{Texts of Terror: Literary-Feminist Readings of Biblical Narratives}. Overtures to Biblical Theology. Philadelphia: Fortress, 1984.

\end{thebibliography}
\endgroup

For additional literature, I highly recommend exploring \href{https://go.openathens.net/redirector/astheology.ns.ca?url=https://www.oxfordbibliographies.com/obo/page/biblical-studies}{Oxford Bibliographies: Biblical Studies (Full Text)}.
You can access the database automatically while on campus or remotely with your
OpenAthens credentials. Numerous articles by subject-area specialists appear
under such headings as: Ancient Near East; Bible; Early Christianity;
Greco-Roman World; Hebrew Bible; New Testament; Rabbinic Judaism; Second Temple
Judaism.

\end{document}
