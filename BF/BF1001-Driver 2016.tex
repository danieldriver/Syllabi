% Copyright (c) 2016 by Daniel R. Driver.
% !TEX encoding = UTF-8 Unicode
% !TEX TS-program = XeLaTeX

\documentclass[titlepage]{article}

% This document presumes a file structure and set of inputs that are
% available at: git@github.com:danieldriver/syllabi.git

\newcommand\policy{../policy}
\newcommand\incl{../includes}
\ProvidesFile{variables.tex}[2018/05/24 v2.1 -- Syllabus variables]

\usepackage{xspace} % make manual spaces (like \mycmd\ ) unnecessary
\usepackage{xifthen} % provides \isempty test

% variables for internal use
\newcommand\prof{}
\newcommand\pdegree{}
\newcommand\pphone{}
\newcommand\pemail{}
\newcommand\poffice{}
\newcommand\phours{}
%
\newcommand\ccode{}
\newcommand\ctitle{}
\newcommand\cseries{}
\newcommand\cversion{}
\newcommand\csemester{}
\newcommand\cmeetson{}
\newcommand\cmeetsat{}
\newcommand\cmeetsin{}
\newcommand\cwebsite{}
\newcommand\cdescrip{}
\newcommand\cprereqs{}
\newcommand\edobject{}

% in case of fully online courses - https://tex.stackexchange.com/a/5896
\newif\ifonline
\newcommand\Int[2]{\ifonline#1\else#2\fi}

% commands for setting variables in the preamble
\newcommand\professor[2][PhD]{
  \renewcommand\pdegree{#1\xspace}
  \renewcommand\prof{#2\xspace}}
\newcommand\phone[1]{
  \renewcommand\pphone{\addfontfeatures{Numbers=Monospaced}#1\xspace}}
\newcommand\email[1]{
  \renewcommand\pemail{\href{mailto:#1}{#1}\xspace}}
\newcommand\officehours[2][Library, Room 5-North]{
  \renewcommand\poffice{#1\xspace}
  \renewcommand\phours{#2\xspace}}
%
\newcommand\coursecode[2][1.0]{
  \renewcommand\cversion{#1\Int{-i}{}\xspace}
  \renewcommand\ccode{#2\Int{(Int)}{}\xspace}}
\newcommand\coursetitle[2][]{
  \ifthenelse{\isempty{#1}}%
    {}% do nothing if #1 is empty, else:
    {\renewcommand\cseries{#1\\[1ex]}}
  \renewcommand\ctitle{#2\xspace}}
\newcommand\semester[1]{
  \renewcommand\csemester{#1\xspace}}
\newcommand\meets[3]{
  \newcommand\AM{\textsc{am}}
  \newcommand\PM{\textsc{pm}}
  \renewcommand\cmeetson{#1\xspace}
  \renewcommand\cmeetsat{\Int{From 9:00 \AM}{#2}\xspace}
  \renewcommand\cmeetsin{\Int{\href{https://smu.brightspace.com/d2l/login}{Brightspace}}{#3}\xspace}}
\newcommand\website[1]{
  \renewcommand\cwebsite{\href{http://#1}{#1}\xspace}}
\newcommand\cdescription[2][RM 1000 or GTRS 6000; and BF 1001]{
  \renewcommand\cprereqs{#1}
  \renewcommand\cdescrip{#2\par}}
\newcommand\objectives[1]{
  \renewcommand\edobject{#1\par}}


\coursecode[1.0]{BF 1001}
\coursetitle[Biblical Foundations]{Hebrew Bible/Old Testament}

% Taught as RLGS 1013 in:
%   - Fall 2008
%   - Winter 2009
%   - Fall 2009
%   - Winter 2010
%   - Fall 2010
%   - Winter 2011 (Fall 2011 was covered by Joel Lohr in a course release)
%   - Winter 2012
%   - Fall 2012
%   - Winter 2013
% Taught as BSTH 1013 in:
%   - Fall 2013
%   - Winter 2014
%   - Fall 2014
%   - Winter 2015
%   - Fall 2015
%   - Winter 2016
% Taught as BF 1001 in:
%   - Fall 2016

\professor{Daniel R. Driver}
\phone{902-425-7051}
\email{ddriver@astheology.ns.ca}
\officehours{Thursdays, 2:00--4:00 \PM}

\semester{Fall Term 2016}
\meets{Thursdays}% \meets{on}{at}{in}
      {9:00--11:30 \AM}
      {Three (3)}
\website{astheology.ns.ca}
\cdescription{% copy from the current Academic Calendar
	This course explores the First Testament of the Bible as a
	foundational element of Christian religious heritage. Students will
	be introduced to a progression of historical and literary data
	important to understanding the Old Testament in its originating
	cultural contexts and to considering its resonances in later
	cultural contexts, including our own.

	The broad sweep of biblical tradition and related historical
	considerations will be presented in large part through guided study
	of the course textbook. Each week we will also give close attention
	in class to a particular textual instance, allowing students to
	develop their exegetical and interpretive skills in a process that
	moves towards integration of historical, literary and theological
	elements.

	In addition, we will explore a range of practices—“things you can do
	with the Bible”—thus encountering the Hebrew Bible in the context
	of applications that have constituted much of its experienced
	liveliness within faith communities over millennia. The Bible’s
	historical and contemporary contributions to personal and communal
	spiritual formation and practice come into focus here, and students
	are given an opportunity to consider how the Bible might figure in
	formative ways within their practice of ministry.
}% end of course description
\objectives{% recall Bloom's taxonomy: http://www.celt.iastate.edu/teaching/RevisedBlooms1.html

	By the end of the course students should be able to:
		name major Old Testament people and events;
		give key dates for Israel's history and summarize the succession of superpowers in the Ancient Near Eastern political theatre from the time of Egypt to Greece;
		locate a few important biblical sites on a map;
		classify prophetic literature relative to the exile;
		recognize genres of biblical literature and cite examples from the reading;
		understand the general shape of the Masoretic Text tradition and differentiate it from other canonical orders;
		defend a decision to pronounce or circumlocute the divine name;
		report on parallel and divergent material across the Law and the Prophets;
		articulate multiple rationales for sabbath observance;
		memorize and recite ten verses of a psalm.

	Students should also be able to identify settings in which the
	scriptures of Israel are read (notably the synagogue, church, and
	academy), employ terminology appropriate to these communities,
	recognize where their own biographies place them in relation to the
	Hebrew Bible/Old Testament and its uses, and monitor and test their
	individual attitudes and assumptions. They should be able to extend
	their awareness of the Bible's contemporary readers to the Bible's
	long history of reception. Finally, students should be able to infer
	what Jesus meant in speaking of ``the law of Moses, the prophets,
	and the psalms'' (Luke 24:44), and so begin to hear claims about New
	Testament fulfilment of scripture in light of the unique voice that
	the Old Testament retains along side of the New in Christian
	scripture.
}% end of learning objectives

\ProvidesFile{preamble.tex}[2013/09/06 v1.0 -- Syllabus preamble]

% basic typography
\usepackage{fontspec}
\setmainfont[Ligatures=TeX]{Meta Serif Pro}
\setsansfont[Ligatures=TeX]{Meta Pro}
\newfontfamily\Heb{Meta Hebrew}
\setmonofont[Scale=MatchLowercase]{Menlo}
\usepackage{sectsty}
\allsectionsfont{\sffamily}
\frenchspacing
\setlength{\emergencystretch}{3em} % prevent overfull lines

% custom font size and leading
\renewcommand\tiny{\fontsize{6}{9}\selectfont}
\renewcommand\scriptsize{\fontsize{7}{10}\selectfont}
\renewcommand\footnotesize{\fontsize{8}{11}\selectfont}
\renewcommand\small{\fontsize{8.5}{11.5}\selectfont}
\renewcommand\normalsize{\fontsize{9}{12}\selectfont}% base size
\renewcommand\large{\fontsize{11}{14}\selectfont}
\renewcommand\Large{\fontsize{13}{16}\selectfont}
\renewcommand\LARGE{\fontsize{16}{19}\selectfont}% "course syllabus \\ semester" benefits from more lead
\renewcommand\huge{\fontsize{19}{21}\selectfont}
\renewcommand\Huge{\fontsize{24}{26}\selectfont}

% layout packages: page, logo, tables
\usepackage[scale={0.6,0.8},
            xetex]{geometry}
\usepackage{graphicx}
\usepackage{array}     % allow insertions of column styling with >{}
\usepackage{booktabs}  % elegant horizontal rules in tables
\usepackage{marginfix} % protect positioning of margin table in policy/grades

% custom macros for a session count in the schedule of readings
\newcounter{session}
\newcounter{columns}
\newcounter{courseunit}
\newcommand\setcolumncount[2][0]{ % optionally set count to other than 0,
  \setcounter{session}{#1}        % e.g. to -1, or to a standing count
  \setcounter{columns}{#2}}
\newcommand\sessioncount{\stepcounter{session}\arabic{session}}
\newcommand\sessionskip[1]{\multicolumn{1}{@{}r@{ }}{#1}}
\newcommand\unit[1]{\multicolumn{\thecolumns}{c}{%
  \scshape\stepcounter{courseunit}\roman{courseunit}. \MakeLowercase{#1}}}
\newcommand\noclass[1]{\multicolumn{1}{@{}l}{\itshape No Class: #1}}

% color to match Tyndale's branding
\usepackage[usenames]{xcolor}
% predefined: black, white, red, green, blue, cyan, magenta, yellow
\definecolor{TyndaleURLs}{HTML}{0062A0} % links on tyndale.ca
\definecolor{TyndaleBlue}{cmyk}{1,1,0,.32}
\definecolor{TyndaleGold}{cmyk}{0,.27,1,0}
\definecolor{TyndaleRed}{cmyk}{0,1,.99,.04}
\definecolor{TyndaleBlack}{cmyk}{0,0,0,1}
\definecolor{TyndaleGreen}{cmyk}{.45,0,1,.24}
\definecolor{TyndaleOrange}{cmyk}{0,.79,1,0}
\definecolor{TyndaleAqua}{cmyk}{.47,0,.24,0}
\definecolor{TyndaleYellow}{cmyk}{.03,.03,.35,0}

% metadata (assumes a host of definitions are made in the main file)
\usepackage[setpagesize=false,     % leave this to geometry
            hyperfootnotes=false,  % fragile and distracting
            xetex]{hyperref}
\hypersetup{breaklinks=true,       % allow link text to break across lines
            colorlinks=true,       % colorlinks resets pdfborder to 0 0 0
            urlcolor=TyndaleURLs,  % for external links
            linkcolor=TyndaleRed,  % for normal internal links
            citecolor=TyndaleGold, % for bibliographical citations in text
            pdfauthor={\prof},
            pdftitle={\ccode: \ctitle},
            pdfsubject={Tyndale UC, \csemester},
            pdfcreator={github.com/danieldriver/syllabus}}
\urlstyle{same}                    % don't use monospace font for urls

% custom footlines
\usepackage{fancyhdr}
\pagestyle{fancy} % turn it on
\fancyhf{}        % reset everything
\renewcommand{\headrulewidth}{0pt} % remove header line as well
\lfoot{\sffamily\scshape\footnotesize\MakeLowercase{\ctitle, v\cversion}}
\rfoot{\sffamily\scshape\footnotesize\MakeLowercase{\prof\quad\thepage}}

% gratuitous with custom title page, but useful as a fallback
\title{\ccode: \ctitle}
\author{\professor}
\date{\semester}


\begin{document}
\ProvidesFile{title.tex}[2013/09/06 v1.0 -- Syllabus title page]

\begin{titlepage}
  \begin{center}

    \LARGE\sffamily % set title elements in a large sans serif

    \begin{minipage}{\textwidth}
      \parbox[t]{0.5\textwidth}{
        \mbox{}\\[-13pt] % dummy line to align parboxes
        \includegraphics[width=0.5\textwidth]{.syllabus/includes/TyndaleUC}}
      \hfill
      \parbox[t]{0.4\textwidth}{
        \raggedleft Course Syllabus\\
        \csemester}
    \end{minipage}

    \vfill

    {\textsc{\MakeLowercase\ccode}\\[1ex]
      \bfseries\cseries\Huge\ctitle}

    \vfill

    \normalsize\rmfamily % switch back to body type

    \begin{tabular}{>{\bfseries}rl>{\bfseries}rl}
      \toprule
      Instructor & \prof, \pdegree & Course  & Version \cversion \\
      \midrule
      Phone      & \pphone         & Meets   & \cmeetson         \\
      Email      & \pemail         & Time    & \cmeetsat         \\
      Office     & \poffice        & Room    & \cmeetsin         \\
      Hours      & \phours         & Website & \cwebsite         \\
      \bottomrule
    \end{tabular}

    \vfill

    \begin{description}\small
      \item[Commuter Hotline]
        Class cancellations due to inclement weather or illness will
        be announced on the commuter hotline at \texttt{416.226.6620
        x2187}. Alternately, weather cancellation information is posted
        at \href{http://tyndale.ca/weather}{tyndale.ca/weather}.
      \item[MyTyndale.ca]
        This course may have materials stored on its website, such as
        handouts or readings that may be needed in order to complete
        assignments. Students are responsible for checking these course
        pages on a regular basis. Here, too, students are able to view
        their grades throughout the semester. For more information see
        Section~\ref{mytyndale}, below.
      \item[Mail]
        Students are responsible for information communicated through
        their campus mailboxes and student e-mail accounts. A mailbox
        directory hangs beside the mailboxes. For more information
        contact the Registrar's office.
    \end{description}

  \end{center}

  \section{Course Description}
  \label{description}

  \emph{From the Academic Calendar:} \cdescrip

\end{titlepage}
\setcounter{page}{2} % count the title page as page 1


  \section{Learning Objectives}
  \label{objectives}

  \edobject

\section{Required Texts \& Materials}
\label{texts}

The following texts are required. Students are strongly encouraged to
purchase their own copies. Library copies have been placed on a 2-hour
reserve.

\begingroup
\renewcommand{\section}[2]{}% temporarily remove the section heading
\begin{thebibliography}{Irenaeus}% use the longest item in the bibliography

	\bibitem[NRSV]{nrsv} M.\,D. Coogan, ed.
    \emph{New Oxford Annotated Bible with Apocrypha: NRSV}. 4th ed.
    Oxford / New York: Oxford University Press, 2010.
    ISBN 978-0195289602.

%	\bibitem[NJPS]{njps}
%	Adele Berlin and Marc Zvi Brettler, eds.
%	\emph{The Jewish Study Bible: Second Edition}.
%	Oxford / New York: Oxford University Press, 2014.
%	ISBN 978-0199978465.

	\bibitem[HBFB]{hbfb} J. Kaminsky and J. Lohr.
	\emph{The Hebrew Bible for Beginners: A Jewish and Christian Introduction}.
	Nashville: Abingdon Press, 2015.
	ISBN 978-1426775635.

	\bibitem[Heschel]{heschel} Abraham Heschel.
	\emph{The Sabbath: Its Meaning for Modern Man}.
	New York: Farrar, Straus and Giroux, 1951 (repr. 2005).
	ISBN 978-0374529758.

	\bibitem[Irenaeus]{irenaeus} St Irenaeus of Lyons.
	\emph{On the Apostolic Preaching}.
	Trans. John Behr.
	Crestwood, NY: St Vladimir’s Seminary Press, 1997.
	ISBN 978-0881411744.

%	\bibitem[OTIC]{otic} Daniel R. Driver. \emph{\ctitle\ Introductory
%	Coursepack}. Halifax, \csemester. In response to the requests of
%	many students I have compiled a pack of notes for my lectures. I
%	will circulate the relevant pages before each class. It behooves you
%	to gather, annotate, and study this material carefully.

\end{thebibliography}
\endgroup

An acceptable alternative study Bible is the NJPS: Adele Berlin and Marc
Zvi Brettler, eds., \emph{The Jewish Study Bible: Second Edition}
(Oxford / New York: Oxford University Press, 2014). A reference copy is
available in the library, and it is well worth consulting. However, the
instructor will cite the NRSV when writing quizzes and exams.

\section{Supplementary Texts}
\label{supplementary}

Supplementary readings will be recommended throughout the semester.
Excerpts from this literature, ordinarily an article or a book chapter
per week, will either be placed on reserve or made available for
download through the course website.

Students are not strictly required to read this additional material;
then again, students who choose not to read it should not expect to earn
an ``A'' for the course.

\section{Course Outline}
\label{outline}

We will adhere to the schedule in \autoref{schedule} as closely as
possible, though the instructor reserves the right to adjust it to suit
the needs of the class.

\newcommand\HBFB[1]{\cite[pp.~#1]{hbfb}}

\setcolumncount{5}% set up \sessioncount, \unit{}, \noclass{}, and \reminder{memo}{date} macros
\begin{table}[htbp]% set to `p' to put the schedule on its own page
  \centering
  \begin{tabular}{>{\sessioncount.}r@{ }lllr}% make sure the column config agrees with \setcolumncount
	\toprule
	\sessionskip{\textbf{\S}.}&\textbf{Primary (total chapters)}&\textbf{Secondary}&\textbf{Supplementary}&\textbf{Date}\\
	\midrule

	\unit{Law / Torah / Pentateuch} \\

		& Genesis 1--11        & \HBFB{1--26}    & C. Seitz             &  8 Sep. \\
		& Genesis 12--50       & \HBFB{27--64}   & P. Enns              & 15 Sep. \\
		& Exodus 1--18         & \HBFB{65--75}   & C. Mathews McGinnis  & 22 Sep. \\
		& Exodus 19--40        & \cite[all]{heschel} & W. Moberly       & 29 Sep. \\
		& Deuteronomy (34)     & \HBFB{77--99}   & J. Levenson          &  6 Oct. \\ [1ex]

	\unit{Prophets: Former \textit{\&} Latter} \\

	\reminder{First paper is \textbf{due} at the start of class six}{}            \\
		& Joshua, Judges (45)  & \HBFB{103--121} & P. Trible            & 13 Oct. \\
		& 1 \& 2 Samuel (55)   & \HBFB{123--143} & R. Alter             & 20 Oct. \\
		& 1 \& 2 Kings (47)    & \cite[all]{irenaeus} & N. MacDonald    & 27 Oct. \\ [1ex]

		& Isaiah 1--12, 36--66 & \HBFB{145--168} & B. Sommer            &  3 Nov. \\
	\noclass{Term Break (Tuesday to Thursday)}                          & 10 Nov. \\
		& Hosea--Nahum (41)  & \HBFB{169--184}   & J. Nogalski          & 17 Nov. \\ [1ex]

	\unit{Writings: Wisdom \textit{\&} Praise} \\

	\reminder{Second paper is \textbf{due} at the start of class eleven}{}        \\
		& Psalms 1--19, 72--74, 89--106 & \HBFB{187--202} & G. Sheppard & 24 Nov. \\
		& Ecclesiastes (12)    & \HBFB{203--246} & M. Fox, K. Dell      &  1 Dec. \\ [1ex]

	\reminder{End of Term: Final marks are due for all courses}{12 Dec.} \\

	\bottomrule
  \end{tabular}
  \caption{Schedule of Readings}
  \label{schedule}
\end{table}

See the AST website for a list of other \href{http://www.astheology.ns.ca/students/academic-dates.html}{important dates}.

\section{Evaluation}
\label{evaluation}

\subsection{Grade Structure for \ccode}
\label{structure}

\begin{enumerate}

	\item I will announce \textbf{reading quizzes} throughout the
	semester, as often as once a week. They are designed to ensure that
	you have read the assigned material carefully. Quizzes are given in
	class, and they may not be made up in the case of absence.

	\item Students are to \textbf{memorize ten verses from the Psalms}
	in the King James Version and recite it to me privately, in my
	office, by the last day of class. You may select ten verses from one
	psalm, one verse from ten psalms, or any combination in between.

	\item Two \textbf{short papers} will facilitate student reflection
	on the two extended works of biblical interpretation, by
	\cite{heschel} and \cite{irenaeus}. Each should be 4--5,000 words
	long. They are due at the start of classes six and eleven,
	respectively.

	\begin{enumerate}

		\item The \textbf{first} paper will articulate and evaluate
		multiple rationales for sabbath observance by interacting with
		biblical and post-biblical traditions. The latter must include,
		but need not be limited to, Heschel's \emph{The Sabath}.

		\item The \textbf{second} paper will explore Irenaeus' use of
		the Old Testament as Christian scripture by: succinctly
		summarizing \emph{On the Apostolic Preaching}, selecting a
		characteristic example of biblical interpretation in that work,
		and then developing and defending a coherent thesis about that
		instance of interpretation.

	\end{enumerate}

	\item Each student will \textbf{lead a seminar} on one of the
	supplementary articles. The discussion, which should last between 35
	and 45 minutes, should begin with a brief (5 min) summary of the
	article. The balance of the time should be spent helping the class
	elaborate, reflect on, test, challenge, or extend the main ideas
	presented there.

\end{enumerate}

The breakdown for the semester's total work is shown in
\autoref{grade-dist}.

\begin{table}[htbp]
  \centering
  {\lining
  \begin{tabular}{lr}
    \toprule
    Reading Quizzes & 25\% \\
    Memorization    & 10\% \\
    First Paper     & 25\% \\
    Second Paper    & 25\% \\
    Seminar         & 15\% \\
    \bottomrule
  \end{tabular}}
  \caption{Distribution of Grades}
  \label{grade-dist}
\end{table}

\ProvidesFile{grades.tex}[2016/09/03 v2.0 -- Course policy]

\subsection{Grading System at AST}
\label{grades}

AST's \href{http://www.astheology.ns.ca/webfiles/AST_2016Calendar_web(A5)-06APR2016.pdf}{Academic
Calendar} provides guidelines and detailed criteria for academic
assessment. Marks are assigned by letter grade using the benchmarks in
\autoref{grade-syst}.

\begin{table}[htbp]
  \centering
  {\lining
  \begin{tabular}{lll}
    \toprule
%    Letter      & Percent & Assessment        \\
%	\midrule
    A+          & 94--100    & Exceptional    \\
    A           & 87--93     & Outstanding    \\
    A\char"2212 & 80--86     & Excellent      \\ [1ex]
    B+          & 77--79     & Good           \\
    B           & 73--76     & Acceptable     \\
    B\char"2212 & 70--72     & Marginal       \\ [1ex]
    C           & 60--69     & Unsatisfactory \\
    F           & 0--59      & Failure        \\
    FP          & 0          & Failure due to Plagiarism \\
    \bottomrule
  \end{tabular}}
  \caption{Summary of Grading System}
  \label{grade-syst}
\end{table}

% More detailed grading criteria from pp. 61--62 of `16.0406-I2-AST Academic Calendar.pdf'
%
%\begin{description}
%  \item[A+ (94-100) ‘Exceptional’]
%    A superior performance with consistent evidence of a comprehensive,
%    incisive grasp of all aspects of the subject matter; a very wide
%    knowledge base; insightful critical evaluation and analysis of the
%    material; an exceptional capacity for original, creative, and/or
%    logical thinking; an exceptional ability to organize, analyse,
%    synthesize, and to express thoughts fluently.
%  \item[A (87-93) ‘Outstanding’]
%    A comprehensive grasp of the subject matter, outstanding evidence of
%    original thought; sound critical evaluation of the material; an
%    excellent ability to organize, analyse, synthesize and to express
%    thoughts; mastery of an extensive knowledge base.
%  \item[A- (80-86) ‘Excellent’]
%    All the qualities of a B-level performance and an excellent capacity
%    for original, creative, and/ or logical thinking; excellent ability
%    to organize, analyse, synthesize, and integrate ideas; broad
%    knowledge base in the subject matter.
%  \item[B+ (77-79) ‘Good’]
%    A good performance with substantial knowledge of the subject matter;
%    a very good understanding of the relevant issues; familiarity with
%    relevant literature and techniques; good ability to organize,
%    analyse, and examine the material in a constructive and critical
%    manner.
%  \item[B (73-76) ‘Acceptable’]
%    A generally adequate performance with a good knowledge of the
%    subject matter; a fair understanding of relevant issues; some
%    ability to work with relevant literature and techniques; some
%    ability to develop solutions to difficult problems related to the
%    subject material.
%  \item[B- (70-72) ‘Marginally Acceptable’]
%    Some familiarity with the subject material; some understanding.
%    Satisfactory understanding of relevant issues; attempts to solve
%    moderately difficult problems related to the subject material in a
%    critical and analytical manner are only partially successful.
%  \item[C (60-69) ‘Unsatisfactory’]
%    A C grade indicates unsatisfactory academic performance. At the
%    discretion of the instructor, supplemental work may be negotiated to
%    upgrade the mark to a B range. A student may carry two C grades
%    without penalty in all courses except Foundations Courses,
%    Supervised Field Education, Supervised Ministry Practicum and the
%    Graduate Project. In these courses, a minimum grade of B- is
%    required to graduate. A student who receives a C in a Foundation
%    course must repeat the course to achieve a B- or better, and cannot
%    use the C grade to meet prerequisite requirements for advanced
%    courses. If the student repeats one of these courses and receives a
%    B- or better, the previous C grade remains on the transcript and can
%    be counted toward the total of unsatisfactory grades that may lead
%    to academic dismissal. Credit will be given only once for any
%    course. (See Policy on Unsatisfactory Academic Performance in the
%    AST Student Handbook.)
%  \item[F (0-59) ‘Failure’]
%    Student has not grasped subject matter; does not understand issues
%    involved; cannot work with relevant literature. (See Policy on
%    Unsatisfactory Academic Performance in the AST Student Handbook.)
%  \item[P ‘Pass’]
%    Credit awarded, but no mark assigned.
%  \item[FP ‘Failure due to Plagiarism’]
%    A student will receive this grade only after proven incident(s) of
%    plagiarism in a course.
%\end{description}
\ProvidesFile{other.tex}[2022/06/08 v2.9.1 -- Course policy]

\section{Other Course Policy}
\label{policy}

Late work will not be accepted, except in genuinely extenuating
circumstances. Students must submit something before the deadline if
they wish to receive credit. Unless I state otherwise, assignments are
to be uploaded by 11:59 \PM\ (Atlantic) on the date indicated.

Essay submissions must be typewritten and double-spaced. They should be
free from error. In this course they should follow SBL Style (see
\cite{sbl2} in \autoref{supplementary}, above). As a reminder, AST
upholds an Inclusive Language Policy. Please use gender-inclusive
language when referring to human beings. Our traditions have different
norms for speech about God; you are of course free to follow and explore
those traditions when referring to God.


Plagiarism is the
\href{http://www.eerdmans.com/Pages/Item/59043/Commentary-Statement.aspx}{failure}
to \href{https://www.theguardian.com/world/2013/feb/09/german-education-minister-quits-phd-plagiarism}{attribute}
(by means of footnotes when writing or aloud when speaking) any ideas,
phrases, sentences, materials, syntheses, et cetera, that another author
has composed and that you have borrowed for your own work. Plagiarism is
unethical. Academic penalties for plagiarism at AST are serious, and may
include failure of the course or even suspension of further studies.
Unintentional plagiarism is considered plagiarism. AST's Plagiarism
Policy is found under that heading in the Academic
Calendar.

Students should request permission to record a class or lecture. If
permission is granted, or if recordings are provided (as in the case of
an online or hybrid course), I stipulate that all recordings be for
personal use only. They may not be shared or distributed.

If you have needs that require modifications to any aspect of this
course, please consult with the instructor as soon as possible. Any
documentation regarding disabilities that you wish to divulge to AST
should be provided to the Registrar’s Office, where it will be kept in a
confidential file.

Finally, I encourage the conscientious use of laptops, tablets, and
other technology in my classes. In classroom settings, realize that, as
\href{http://dx.doi.org/10.1016/j.compedu.2012.10.003}{cognitive
psychologists have demonstrated}, ``laptop multitasking hinders
classroom learning for both users and nearby peers.'' Do your part to
foster an environment for dialogue by honouring the presence of your
classmates. In online and hybrid settings, consider both the physical
environment in which you choose to work and the virtual environment that
you help create through your participation in various forums. Let your
engagement in this course be marked by rigour and charity alike.


\end{document}
