% Copyright (c) 2024 by Daniel R. Driver.
% !TEX encoding = UTF-8 Unicode
% !TEX TS-program = XeLaTeX

\documentclass[titlepage]{article}

% This document presumes a file structure and set of inputs that are
% available at: git@github.com:danieldriver/syllabi.git

\newcommand\policy{../policy}
\newcommand\incl{../includes}
\ProvidesFile{variables.tex}[2018/05/24 v2.1 -- Syllabus variables]

\usepackage{xspace} % make manual spaces (like \mycmd\ ) unnecessary
\usepackage{xifthen} % provides \isempty test

% variables for internal use
\newcommand\prof{}
\newcommand\pdegree{}
\newcommand\pphone{}
\newcommand\pemail{}
\newcommand\poffice{}
\newcommand\phours{}
%
\newcommand\ccode{}
\newcommand\ctitle{}
\newcommand\cseries{}
\newcommand\cversion{}
\newcommand\csemester{}
\newcommand\cmeetson{}
\newcommand\cmeetsat{}
\newcommand\cmeetsin{}
\newcommand\cwebsite{}
\newcommand\cdescrip{}
\newcommand\cprereqs{}
\newcommand\edobject{}

% in case of fully online courses - https://tex.stackexchange.com/a/5896
\newif\ifonline
\newcommand\Int[2]{\ifonline#1\else#2\fi}

% commands for setting variables in the preamble
\newcommand\professor[2][PhD]{
  \renewcommand\pdegree{#1\xspace}
  \renewcommand\prof{#2\xspace}}
\newcommand\phone[1]{
  \renewcommand\pphone{\addfontfeatures{Numbers=Monospaced}#1\xspace}}
\newcommand\email[1]{
  \renewcommand\pemail{\href{mailto:#1}{#1}\xspace}}
\newcommand\officehours[2][Library, Room 5-North]{
  \renewcommand\poffice{#1\xspace}
  \renewcommand\phours{#2\xspace}}
%
\newcommand\coursecode[2][1.0]{
  \renewcommand\cversion{#1\Int{-i}{}\xspace}
  \renewcommand\ccode{#2\Int{(Int)}{}\xspace}}
\newcommand\coursetitle[2][]{
  \ifthenelse{\isempty{#1}}%
    {}% do nothing if #1 is empty, else:
    {\renewcommand\cseries{#1\\[1ex]}}
  \renewcommand\ctitle{#2\xspace}}
\newcommand\semester[1]{
  \renewcommand\csemester{#1\xspace}}
\newcommand\meets[3]{
  \newcommand\AM{\textsc{am}}
  \newcommand\PM{\textsc{pm}}
  \renewcommand\cmeetson{#1\xspace}
  \renewcommand\cmeetsat{\Int{From 9:00 \AM}{#2}\xspace}
  \renewcommand\cmeetsin{\Int{\href{https://smu.brightspace.com/d2l/login}{Brightspace}}{#3}\xspace}}
\newcommand\website[1]{
  \renewcommand\cwebsite{\href{http://#1}{#1}\xspace}}
\newcommand\cdescription[2][RM 1000 or GTRS 6000; and BF 1001]{
  \renewcommand\cprereqs{#1}
  \renewcommand\cdescrip{#2\par}}
\newcommand\objectives[1]{
  \renewcommand\edobject{#1\par}}


\onlinetrue % \Int{true}{false}
%\UGtrue % \UG{true}{false} terms for undergraduate students
\coursecode[2.0]{DTS 2000}
\coursetitle[Diploma in Theological Studies]{Introduction to the Old Testament}

% Taught in parallel with BF 1001 in:
%   - Fall 2022 - Taken over in late Aug after Hugh Farquhar dropped
%   - Fall 2024 - Redeveloped as 1st iteration at request of Dean SMG

\professor{Daniel R. Driver}
\phone{902-425-7051}
\email{daniel.driver@astheology.ns.ca}
\officehours{Mondays, 1:00--3:00 \PM}

\semester{Fall 2024}
\meets{\Int{Mondays}{Thursdays}}% \meets{on}{at}{in}
      {1:00--3:30 \PM}% if online, it prints "From 9:00 AM"
      {Classroom 3}% if online, it prints "Brightspace" with link
\website{danieldriver.com}
\cdescription[none. This diploma program is a two-year program intended for anyone who wants an introduction
to biblical and theological studies. To be admitted to this program, you must have a high school diploma or its equivalent. Some post-secondary (university or college) education is desirable but not required]{% copy from the current Academic Calendar; [] for prereqs
	\emph{From the Academic Calendar}: DTS 2000 -- Introduction to the First
	Testament/Hebrew Scriptures. This unit uses critical methods of
	biblical study to familiarize you with the Pentateuch and the
	Prophetic Literature of the Hebrew Scriptures (Old Testament).
}% end of course description
\objectives{% recall Bloom's taxonomy: http://www.celt.iastate.edu/teaching/RevisedBlooms1.html
	The First Testament of Christian Scripture is a fundamental part of
	Christian tradition and durable rule of faith and practice. Students
	in this course will be introduced to historical and literary data
	important for understanding the origins of the Hebrew Bible/Old
	Testament and its subsequent function as scripture in a variety of
	contexts. The outlook will be broadly ecumenical, with case studies
	that exemplify Jewish and Christian interpretation.

	The broad sweep of biblical tradition will be presented through a
	survey of representative books from the Law, the Prophets, and the
	Writings. To help foster an ecumenical outlook, primary readings
	from the Bible will be supplemented by illustrative commentary from
	various biblical interpreters. Students will develop an appreciation
	for the diverse canonical, cultural, hermeneutical, historical,
	literary, and theological elements involved in the art of biblical
	interpretation.

	By the end of the course students should be able to identify
	settings in which the Scriptures of Israel are read (notably the
	synagogue, church, and academy), employ terminology appropriate to
	these communities, recognize where their own biographies place them
	in relation to the Hebrew Bible/Old Testament and its uses, and
	monitor and test their individual attitudes and assumptions. They
	should be able to extend their awareness of the Bible's contemporary
	readers to the Bible's long history of reception. Finally, students
	should begin to infer what Jesus meant in speaking of “the law of
	Moses, the prophets, and the psalms” (Luke 24:44), and so learn to
	hear claims about New Testament fulfilment of scripture in light of
	the unique voice that the Old Testament retains along side of the
	New in Christian Scripture.
}% end of learning objectives

\ProvidesFile{preamble.tex}[2013/09/06 v1.0 -- Syllabus preamble]

% basic typography
\usepackage{fontspec}
\setmainfont[Ligatures=TeX]{Meta Serif Pro}
\setsansfont[Ligatures=TeX]{Meta Pro}
\newfontfamily\Heb{Meta Hebrew}
\setmonofont[Scale=MatchLowercase]{Menlo}
\usepackage{sectsty}
\allsectionsfont{\sffamily}
\frenchspacing
\setlength{\emergencystretch}{3em} % prevent overfull lines

% custom font size and leading
\renewcommand\tiny{\fontsize{6}{9}\selectfont}
\renewcommand\scriptsize{\fontsize{7}{10}\selectfont}
\renewcommand\footnotesize{\fontsize{8}{11}\selectfont}
\renewcommand\small{\fontsize{8.5}{11.5}\selectfont}
\renewcommand\normalsize{\fontsize{9}{12}\selectfont}% base size
\renewcommand\large{\fontsize{11}{14}\selectfont}
\renewcommand\Large{\fontsize{13}{16}\selectfont}
\renewcommand\LARGE{\fontsize{16}{19}\selectfont}% "course syllabus \\ semester" benefits from more lead
\renewcommand\huge{\fontsize{19}{21}\selectfont}
\renewcommand\Huge{\fontsize{24}{26}\selectfont}

% layout packages: page, logo, tables
\usepackage[scale={0.6,0.8},
            xetex]{geometry}
\usepackage{graphicx}
\usepackage{array}     % allow insertions of column styling with >{}
\usepackage{booktabs}  % elegant horizontal rules in tables
\usepackage{marginfix} % protect positioning of margin table in policy/grades

% custom macros for a session count in the schedule of readings
\newcounter{session}
\newcounter{columns}
\newcounter{courseunit}
\newcommand\setcolumncount[2][0]{ % optionally set count to other than 0,
  \setcounter{session}{#1}        % e.g. to -1, or to a standing count
  \setcounter{columns}{#2}}
\newcommand\sessioncount{\stepcounter{session}\arabic{session}}
\newcommand\sessionskip[1]{\multicolumn{1}{@{}r@{ }}{#1}}
\newcommand\unit[1]{\multicolumn{\thecolumns}{c}{%
  \scshape\stepcounter{courseunit}\roman{courseunit}. \MakeLowercase{#1}}}
\newcommand\noclass[1]{\multicolumn{1}{@{}l}{\itshape No Class: #1}}

% color to match Tyndale's branding
\usepackage[usenames]{xcolor}
% predefined: black, white, red, green, blue, cyan, magenta, yellow
\definecolor{TyndaleURLs}{HTML}{0062A0} % links on tyndale.ca
\definecolor{TyndaleBlue}{cmyk}{1,1,0,.32}
\definecolor{TyndaleGold}{cmyk}{0,.27,1,0}
\definecolor{TyndaleRed}{cmyk}{0,1,.99,.04}
\definecolor{TyndaleBlack}{cmyk}{0,0,0,1}
\definecolor{TyndaleGreen}{cmyk}{.45,0,1,.24}
\definecolor{TyndaleOrange}{cmyk}{0,.79,1,0}
\definecolor{TyndaleAqua}{cmyk}{.47,0,.24,0}
\definecolor{TyndaleYellow}{cmyk}{.03,.03,.35,0}

% metadata (assumes a host of definitions are made in the main file)
\usepackage[setpagesize=false,     % leave this to geometry
            hyperfootnotes=false,  % fragile and distracting
            xetex]{hyperref}
\hypersetup{breaklinks=true,       % allow link text to break across lines
            colorlinks=true,       % colorlinks resets pdfborder to 0 0 0
            urlcolor=TyndaleURLs,  % for external links
            linkcolor=TyndaleRed,  % for normal internal links
            citecolor=TyndaleGold, % for bibliographical citations in text
            pdfauthor={\prof},
            pdftitle={\ccode: \ctitle},
            pdfsubject={Tyndale UC, \csemester},
            pdfcreator={github.com/danieldriver/syllabus}}
\urlstyle{same}                    % don't use monospace font for urls

% custom footlines
\usepackage{fancyhdr}
\pagestyle{fancy} % turn it on
\fancyhf{}        % reset everything
\renewcommand{\headrulewidth}{0pt} % remove header line as well
\lfoot{\sffamily\scshape\footnotesize\MakeLowercase{\ctitle, v\cversion}}
\rfoot{\sffamily\scshape\footnotesize\MakeLowercase{\prof\quad\thepage}}

% gratuitous with custom title page, but useful as a fallback
\title{\ccode: \ctitle}
\author{\professor}
\date{\semester}


\begin{document}
\ProvidesFile{title.tex}[2013/09/06 v1.0 -- Syllabus title page]

\begin{titlepage}
  \begin{center}

    \LARGE\sffamily % set title elements in a large sans serif

    \begin{minipage}{\textwidth}
      \parbox[t]{0.5\textwidth}{
        \mbox{}\\[-13pt] % dummy line to align parboxes
        \includegraphics[width=0.5\textwidth]{.syllabus/includes/TyndaleUC}}
      \hfill
      \parbox[t]{0.4\textwidth}{
        \raggedleft Course Syllabus\\
        \csemester}
    \end{minipage}

    \vfill

    {\textsc{\MakeLowercase\ccode}\\[1ex]
      \bfseries\cseries\Huge\ctitle}

    \vfill

    \normalsize\rmfamily % switch back to body type

    \begin{tabular}{>{\bfseries}rl>{\bfseries}rl}
      \toprule
      Instructor & \prof, \pdegree & Course  & Version \cversion \\
      \midrule
      Phone      & \pphone         & Meets   & \cmeetson         \\
      Email      & \pemail         & Time    & \cmeetsat         \\
      Office     & \poffice        & Room    & \cmeetsin         \\
      Hours      & \phours         & Website & \cwebsite         \\
      \bottomrule
    \end{tabular}

    \vfill

    \begin{description}\small
      \item[Commuter Hotline]
        Class cancellations due to inclement weather or illness will
        be announced on the commuter hotline at \texttt{416.226.6620
        x2187}. Alternately, weather cancellation information is posted
        at \href{http://tyndale.ca/weather}{tyndale.ca/weather}.
      \item[MyTyndale.ca]
        This course may have materials stored on its website, such as
        handouts or readings that may be needed in order to complete
        assignments. Students are responsible for checking these course
        pages on a regular basis. Here, too, students are able to view
        their grades throughout the semester. For more information see
        Section~\ref{mytyndale}, below.
      \item[Mail]
        Students are responsible for information communicated through
        their campus mailboxes and student e-mail accounts. A mailbox
        directory hangs beside the mailboxes. For more information
        contact the Registrar's office.
    \end{description}

  \end{center}

  \section{Course Description}
  \label{description}

  \emph{From the Academic Calendar:} \cdescrip

\end{titlepage}
\setcounter{page}{2} % count the title page as page 1


  \section{Learning Objectives}
  \label{objectives}

  \edobject

\section{Required Texts \& Materials}
\label{texts}

The following \textbf{primary text} is required. All students should have their own copies.

\begingroup
\renewcommand{\section}[2]{}% temporarily remove the section heading
\begin{thebibliography}{NJPS}% use the longest item in the bibliography

	\bibitem[NJPS]{njps} Berlin, Adele, and Marc Zvi Brettler, eds.
    \emph{The Jewish Study Bible: Jewish Publication Society Tanakh Translation}.
    2nd ed. Oxford: Oxford University Press, 2014.
    % ISBN 978-0199978465.
     A non-circulating reference copy is held in the AST Library: Ref BS 895.J4 2014.

\end{thebibliography}
\endgroup

Alternatively, for an ecumenical study Bible with a gender-inclusive
translation of the full Christian Bible, see Michael D. Coogan, ed.,
\emph{The New Oxford Annotated Bible: New Revised Standard Version with
the Apocrypha}, 5th ed. (Oxford: Oxford University Press, 2018).
% ISBN 978-0190276072.

\section{Supplementary Texts}
\label{supplementary}

Supplementary readings will be recommended throughout the semester and
either placed on reserve or made available through the course website.
Please give this material good effort and attention. Some of this
material is listed in the bibliography below, in \autoref{bib}.

The following \textbf{secondary text} is highly recommended but not required.

\begingroup
\renewcommand{\section}[2]{}% temporarily remove the section heading
\begin{thebibliography}{Davis}% use the longest item in the bibliography

	\bibitem[Davis]{ed19} Davis, Ellen F.
	\emph{Opening Israel's Scriptures}.
	New York: Oxford University Press, 2019.
	An AST Library e-book (single user license) is \href{https://doi.org/10.1093/oso/9780190260545.001.0001}{available online}.

\end{thebibliography}
\endgroup

Also, the professor especially recommends two titles as profound but
accessible, pocket-sized exemplars of Jewish and Christian interpretive
traditions. Both titles will be referenced in the course. They repay
careful reading and are worth adding to your personal libraries.


\begingroup
\renewcommand{\section}[2]{}% temporarily remove the section heading
\begin{thebibliography}{Irenaeus}% use the longest item in the bibliography

	\bibitem[Heschel]{Heschel} Heschel, Abraham Joshua.
	\emph{The Sabbath: Its Meaning for Modern Man}.
	New York: Farrar, Straus and Giroux, 1951.

	\bibitem[Irenaeus]{Irenaeus} Irenaeus of Lyons.
	\emph{On the Apostolic Preaching}. Translated by John Behr. Popular Patristics Series 17.
	St Vladimir’s Seminary Press, 1997.


\end{thebibliography}
\endgroup


\section{Course Outline}
\label{outline}

We will adhere to the schedule in \autoref{schedule} as closely as
possible, though the professor reserves the right to adjust it to suit
the needs of the class.

\setcolumncount{5}% set up \sessioncount, \unit{}, \noclass{}, and \reminder{memo}{date} macros
\begin{table}[htbp]% set to `p' to put the schedule on its own page
  \centering
  \begin{tabular}{>{\sessioncount.}r@{ }lllr}% make sure the column config agrees with \setcolumncount
	\toprule
	\sessionskip{\textbf{\S}.}&\textbf{Primary reading}&\textbf{Focal texts}&\textbf{Secondary}&\textbf{From}\\
	\midrule

	\unit{Law / Torah / Pentateuch} \\

		& Gen 1--11              & Gen 1; 9           & Syllabus            &  9 Sep. \\
		& Gen 12--36             & Gen 12; 22         & \cite[ch. 1]{ed19}  & 16 Sep. \\
		& Exod 1--18             & Exod 3; 16         & \cite[ch. 2a]{ed19} & 23 Sep. \\
		& Exod 19--34            & Exod 20--21; 34    & \cite[ch. 2b]{ed19} & 30 Sep. \\
	    & Lev 1--10; 25--26      & Lev 10; 26         & \cite[ch. 3]{ed19}  &  7 Oct. \\
	\noclass{Thanksgiving (but resume Mon of Reading Week)}                 & 14 Oct. \\
%	\noclass{Reading Week from Tuesday to Friday}                           & 22--25 Oct. \\
%	\reminder{A short paper is \textbf{due} before the \textbf{sixth week} of class}{17 Oct.} \\% Have moved btw wks 5 and 6 depending on how holidays (eg, Thanksgiving Monday) impact the schedule
		& Deut 1--16; 34         & Deut 6; 16         & \cite[ch. 5]{ed19}  & 21 Oct. \\ [1ex]

	\unit{Prophets: Former \textit{\&} Latter} \\

		& Josh 1--12; Judg 1--5; 19--21 & Josh 1; Judg 19 & \cite[ch. 6--7]{ed19} & 28 Oct. \\
		& Ruth 1--4; 1 Sam 1--31 & 1 Sam 1--2; 8      & \cite[ch. 8]{ed19}  &  4 Nov. \\
	\noclass{Remembrance Day}                                               & 11 Nov. \\
		& 2 Sam 1--1 Kgs 12      & 2 Sam 7; 1 Kgs 11  & \cite[ch. 9]{ed19}  & 18 Nov. \\
		& Hosea 1--Micah 7       & Joel 2; Jonah 4    & \cite[ch. 10]{ed19} & 25 Nov. \\ [1ex]

	\unit{Writings: Wisdom \textit{\&} Praise} \\

%	\reminder{A short paper is \textbf{due} before the \textbf{twelfth week} of class}{5 Dec.} \\% NB Moved back from wk 11 to wk 10 because of the nearness of the last day of class in F22. Perhaps move back to wk 11 in future.
		& Pss 1--8; 51--60; 89--90 & Pss 1--2; 51; 90 & \cite[ch. 14]{ed19} &  2 Dec. \\
		& Prov 1--9; Eccl 1--12   & Prov 8; Eccl 1; 7 & \cite[ch. 15]{ed19} &  9 Dec. \\

	\reminder{End of Term: Final marks are due for all courses}{13 Dec.} \\

	\bottomrule
  \end{tabular}
  \caption{Schedule of Readings}
  \label{schedule}
\end{table}

See the AST website for a list of other \href{https://www.astheology.ns.ca/students/index.html}{important dates}.

\section{Evaluation}
\label{evaluation}

The grade structure for \ccode has the following elements.

\begin{enumerate}

	\item \textbf{Notes and quotes} will be solicited from students in
	each week of class. These are to be drawn from the primary reading
	\cite{njps}. What do you note about the material on this occasion?
	What words, phrases, or verses stand out to you? Pay attention to
	things you notice from a survey of large sections of the Bible. Stay
	alert, too, to details that strike you from this translation. Mark
	up your Bible. Make notes as you read.

	\begin{enumerate}

		\item As the course is asynchronous, such notes and quotes are
		to be posted weekly to the class forums. Each week opens on the
		Mondays listed by date in the course outline
		(\autoref{schedule}). You are invited (but not strictly
		required) to respond to and interact with classmates on the
		forums through the week. Posts should be submitted by Thursday
		at the latest so that any interaction can happen within the work
		week. Aim to make posts around 250 words long; 200 words is
		adequate, if you have little to say, but in no case should they
		exceed 300 words.

		\item Topical prompts may be suggested by the professor based on
		the lectures or other course material. You are welcome to use
		these prompts to focus your notes and quotes in the forum
		discussions. You are also free to ignore the prompts and to
		pursue your own questions and observations.

	\end{enumerate}

	\item Two \textbf{reflection papers} give students an opportunity to
	consider their relationship to the Scriptures of Israel and to
	reflect on its development over the course. Each of these papers is
	to be about 800 words long, including any footnotes. The first paper
	is due before the third week of class; the second paper is due by
	the final day of class.

	\begin{enumerate}

		\item Before the third week, answer the following question:
		\emph{What does my tradition tell me about the Scriptures of
		Israel?} To start, identify a tradition that broadly informs
		your thinking about the Tanakh/OT/HB. Our traditions are often
		multiple, but for the purposes of this assignment, select one
		that is, for you, especially relevant, dominant, or in need of
		investigation. Then, summarize what your tradition says. Focus
		on official statements, where available. You may review informal
		attitudes, too, but keep to things that are identifiable with a
		tradition (not too personal, in other words). Students
		associated with AST's founding parties should consider sources
		like: the Catechism of the Catholic Church; the Thirty-nine
		Articles of Religion; the four subordinate standards of the
		UCCan. Students with other religious affiliation, or with none,
		should consider what pertains to them. (Anglo-American celebrity
		atheists embody a tradition, too, for example.)

		\item Before the final class, answer the following question:
		\emph{What would I like to say to my tradition about the
		Scriptures of Israel?} Revisit your first reflection paper and
		the tradition you identified at the start of the term. Then, as
		you think about what you might want to affirm, critique, or
		reform in that tradition, reflect on what we have studied. Based
		on what you have learned, what is something you would say about
		the Tanakh/OT/HB to others today who are shaped by your
		tradition? If desired, you may craft this paper as an open
		letter to a peer (someone like you before this course) or an
		authority (an offical leader or public figure). In your
		discussion, incorporate at least two good examples from the
		Bible. Whether you make points broad or narrow, be sure to cite
		chapter and verse.

	\end{enumerate}

	\item Two \textbf{exegetical papers} facilitate direct work with the
	biblical text as students practice the art of biblical
	interpretation. Each of these papers should focus on one narrow
	passage and be about 1,500 words long, including any footnotes. The
	professor encourages, but does not require, the selection of shorter
	passages from within the focal chapters listed in the course outline
	(\autoref{schedule}). The first paper is due before the sixth week;
	the second paper is due before the eleventh week.

	\begin{enumerate}

		\item Before the sixth week, compose and submit an exegetical
		analysis of any suitably short passage from the \textbf{Law}
		(Torah).

		\item Before the eleventh week, compose and submit an
		exegetical analysis of any suitably short passage from the
		\textbf{Prophets} (Former or Latter).

		\item Each paper must advance a \textbf{thesis} related to the
		selected text. This thesis should be clearly identifiable as a
		single sentence in an obvious location. Each paper should
		interact with at least three (3) sources or commentators, not
		counting lectures (which need not be cited).
		
		\item See my \href{https://danieldriver.com/assets/pdf/handouts/Guidelines_for_Papers.pdf}{“Guidelines for Papers and Other Written Assignments” (PDF)}.

	\end{enumerate}

\end{enumerate}

The breakdown for the semester's total work is shown in
\autoref{grade-dist}.

\begin{table}[htbp]
  \centering
  {\lining
  \begin{tabular}{lr}
    \toprule
    Notes \& Quotes     & 20\% \\
    Reflection Paper 1  & 15\% \\
    Reflection Paper 2  & 15\% \\
    Exegetical Essay 1  & 25\% \\
    Exegetical Essay 2  & 25\% \\
    \bottomrule
  \end{tabular}}
  \caption{Distribution of Grades}
  \label{grade-dist}
\end{table}

Per AST's \href{https://www.astheology.ns.ca/students/resources.html}{Academic Calendar},
you are expected to complete all assigned readings and written
assignments, participate in all group discussions and practice sessions,
and engage in cooperative peer and/or faculty evaluation. All DTS
courses are graded Pass/Fail.

%\ProvidesFile{grades.tex}[2016/09/03 v2.0 -- Course policy]

\subsection{Grading System at AST}
\label{grades}

AST's \href{http://www.astheology.ns.ca/webfiles/AST_2016Calendar_web(A5)-06APR2016.pdf}{Academic
Calendar} provides guidelines and detailed criteria for academic
assessment. Marks are assigned by letter grade using the benchmarks in
\autoref{grade-syst}.

\begin{table}[htbp]
  \centering
  {\lining
  \begin{tabular}{lll}
    \toprule
%    Letter      & Percent & Assessment        \\
%	\midrule
    A+          & 94--100    & Exceptional    \\
    A           & 87--93     & Outstanding    \\
    A\char"2212 & 80--86     & Excellent      \\ [1ex]
    B+          & 77--79     & Good           \\
    B           & 73--76     & Acceptable     \\
    B\char"2212 & 70--72     & Marginal       \\ [1ex]
    C           & 60--69     & Unsatisfactory \\
    F           & 0--59      & Failure        \\
    FP          & 0          & Failure due to Plagiarism \\
    \bottomrule
  \end{tabular}}
  \caption{Summary of Grading System}
  \label{grade-syst}
\end{table}

% More detailed grading criteria from pp. 61--62 of `16.0406-I2-AST Academic Calendar.pdf'
%
%\begin{description}
%  \item[A+ (94-100) ‘Exceptional’]
%    A superior performance with consistent evidence of a comprehensive,
%    incisive grasp of all aspects of the subject matter; a very wide
%    knowledge base; insightful critical evaluation and analysis of the
%    material; an exceptional capacity for original, creative, and/or
%    logical thinking; an exceptional ability to organize, analyse,
%    synthesize, and to express thoughts fluently.
%  \item[A (87-93) ‘Outstanding’]
%    A comprehensive grasp of the subject matter, outstanding evidence of
%    original thought; sound critical evaluation of the material; an
%    excellent ability to organize, analyse, synthesize and to express
%    thoughts; mastery of an extensive knowledge base.
%  \item[A- (80-86) ‘Excellent’]
%    All the qualities of a B-level performance and an excellent capacity
%    for original, creative, and/ or logical thinking; excellent ability
%    to organize, analyse, synthesize, and integrate ideas; broad
%    knowledge base in the subject matter.
%  \item[B+ (77-79) ‘Good’]
%    A good performance with substantial knowledge of the subject matter;
%    a very good understanding of the relevant issues; familiarity with
%    relevant literature and techniques; good ability to organize,
%    analyse, and examine the material in a constructive and critical
%    manner.
%  \item[B (73-76) ‘Acceptable’]
%    A generally adequate performance with a good knowledge of the
%    subject matter; a fair understanding of relevant issues; some
%    ability to work with relevant literature and techniques; some
%    ability to develop solutions to difficult problems related to the
%    subject material.
%  \item[B- (70-72) ‘Marginally Acceptable’]
%    Some familiarity with the subject material; some understanding.
%    Satisfactory understanding of relevant issues; attempts to solve
%    moderately difficult problems related to the subject material in a
%    critical and analytical manner are only partially successful.
%  \item[C (60-69) ‘Unsatisfactory’]
%    A C grade indicates unsatisfactory academic performance. At the
%    discretion of the instructor, supplemental work may be negotiated to
%    upgrade the mark to a B range. A student may carry two C grades
%    without penalty in all courses except Foundations Courses,
%    Supervised Field Education, Supervised Ministry Practicum and the
%    Graduate Project. In these courses, a minimum grade of B- is
%    required to graduate. A student who receives a C in a Foundation
%    course must repeat the course to achieve a B- or better, and cannot
%    use the C grade to meet prerequisite requirements for advanced
%    courses. If the student repeats one of these courses and receives a
%    B- or better, the previous C grade remains on the transcript and can
%    be counted toward the total of unsatisfactory grades that may lead
%    to academic dismissal. Credit will be given only once for any
%    course. (See Policy on Unsatisfactory Academic Performance in the
%    AST Student Handbook.)
%  \item[F (0-59) ‘Failure’]
%    Student has not grasped subject matter; does not understand issues
%    involved; cannot work with relevant literature. (See Policy on
%    Unsatisfactory Academic Performance in the AST Student Handbook.)
%  \item[P ‘Pass’]
%    Credit awarded, but no mark assigned.
%  \item[FP ‘Failure due to Plagiarism’]
%    A student will receive this grade only after proven incident(s) of
%    plagiarism in a course.
%\end{description}
\ProvidesFile{other.tex}[2022/06/08 v2.9.1 -- Course policy]

\section{Other Course Policy}
\label{policy}

Late work will not be accepted, except in genuinely extenuating
circumstances. Students must submit something before the deadline if
they wish to receive credit. Unless I state otherwise, assignments are
to be uploaded by 11:59 \PM\ (Atlantic) on the date indicated.

Essay submissions must be typewritten and double-spaced. They should be
free from error. In this course they should follow SBL Style (see
\cite{sbl2} in \autoref{supplementary}, above). As a reminder, AST
upholds an Inclusive Language Policy. Please use gender-inclusive
language when referring to human beings. Our traditions have different
norms for speech about God; you are of course free to follow and explore
those traditions when referring to God.


Plagiarism is the
\href{http://www.eerdmans.com/Pages/Item/59043/Commentary-Statement.aspx}{failure}
to \href{https://www.theguardian.com/world/2013/feb/09/german-education-minister-quits-phd-plagiarism}{attribute}
(by means of footnotes when writing or aloud when speaking) any ideas,
phrases, sentences, materials, syntheses, et cetera, that another author
has composed and that you have borrowed for your own work. Plagiarism is
unethical. Academic penalties for plagiarism at AST are serious, and may
include failure of the course or even suspension of further studies.
Unintentional plagiarism is considered plagiarism. AST's Plagiarism
Policy is found under that heading in the Academic
Calendar.

Students should request permission to record a class or lecture. If
permission is granted, or if recordings are provided (as in the case of
an online or hybrid course), I stipulate that all recordings be for
personal use only. They may not be shared or distributed.

If you have needs that require modifications to any aspect of this
course, please consult with the instructor as soon as possible. Any
documentation regarding disabilities that you wish to divulge to AST
should be provided to the Registrar’s Office, where it will be kept in a
confidential file.

Finally, I encourage the conscientious use of laptops, tablets, and
other technology in my classes. In classroom settings, realize that, as
\href{http://dx.doi.org/10.1016/j.compedu.2012.10.003}{cognitive
psychologists have demonstrated}, ``laptop multitasking hinders
classroom learning for both users and nearby peers.'' Do your part to
foster an environment for dialogue by honouring the presence of your
classmates. In online and hybrid settings, consider both the physical
environment in which you choose to work and the virtual environment that
you help create through your participation in various forums. Let your
engagement in this course be marked by rigour and charity alike.


\section{Further Bibliography}
\label{bib}

Introductions to the Hebrew Bible/Old Testament abound. Students looking
for a second or third opinion on basic matters may find one or more the
following volumes useful. They arise from a variety of contexts and, in
terms of orientation, have different kinds of utility. The one by
Kaminsky and Lohr \cite{hbfb} is notable for its brevity and interfaith
outlook.

\begingroup
\renewcommand{\section}[2]{}% temporarily remove the section heading
\begin{thebibliography}{99}% use the longest item in the bibliography

	\bibitem{jb16} Barton, John, ed.
	\emph{The Hebrew Bible: A Critical Companion}.
	Princeton: Princeton University Press, 2016.

	\bibitem{bb05} Birch, Bruce, Walter Brueggemann, Terence E. Fretheim, and David L. Petersen.
	\emph{A Theological Introduction to the Old Testament}.
	2nd ed. Nashville: Abingdon, 2005.
	AST Library: BS 1192.5 T43 2005.

	\bibitem{wbtl21} Brueggemann, Walter, and Tod Linafelt.
	\emph{An Introduction to the Old Testament: The Canon and Christian Imagination}.
	3rd ed. Louisville: Westminster John Knox, 2021.
	AST Library (2nd ed.): BS 1140.3 B78 2012.

	\bibitem{scms16} Chapman, Stephen B., and Marvin A. Sweeney, eds.
	\emph{The Cambridge Companion to the Hebrew Bible/Old Testament}.
	New York: Cambridge University Press, 2016.
	AST Library: BS 1140.3 C35 2016.

	\bibitem{jc04} Collins, John J.
	\emph{Introduction to the Hebrew Bible}.
	Minneapolis: Fortress, 2004.
	AST Library: BS 1140.3 C65 2004.

	\bibitem{mc10} Coogan, Michael D.
	\emph{The Old Testament: A Historical and Literary Introduction to the Hebrew Scriptures}.
	3rd ed. New York: Oxford University Press, 2014.
	AST Library: BS 1197 C56 2014.

	\bibitem{pdjr05} Davies, Philip R., and John Rogerson.
	\emph{The Old Testament World}.
	2nd ed. Louisville: Westminster John Knox, 2005.
	AST Library: BS 1171.3 D38 2005.

	\bibitem{ahjw09} Hill, Andrew E., and John H. Walton.
	\emph{A Survey of the Old Testament}.
	3rd ed. Grand Rapids: Zondervan, 2009.

	\bibitem{jc23} Jacobson, Rolf A., and Michael J. Chan.
	\emph{Introducing the Old Testament: A Historical, Literary, and Theological Survey}.
	Grand Rapids: Baker Academic, 2023.

	\bibitem{hbfb} Kaminsky, Joel S., and Joel N. Lohr.
	\emph{The Hebrew Bible for Beginners: A Jewish and Christian Introduction}.
	Nashville: Abingdon Press, 2015.
	AST Library: BS 1171.3 K39 2015 and \href{https://search.ebscohost.com/login.aspx?direct=true&AuthType=cookie,ip,shib&db=nlebk&AN=969753&site=ehost-live&scope=site&custid=s5315951}{online}.

	\bibitem{jk07} Kugel, James L.
	\emph{How To Read the Bible: A Guide to Scripture, Then and Now}.
	New York: Free, 2007.

	\bibitem{hprb10} Page, Hugh R., Jr., and Randall C. Bailey, eds.
	\emph{The Africana Bible: Reading Israel’s Scriptures from Africa and the African Diaspora}.
	Minneapolis: Fortress, 2010.
	AST Library: BS 1171.3 A37 2010.

	\bibitem{rr86} Rendtorff, Rolf.
	\emph{The Old Testament: An Introduction}. Translated by John Bowden.
	Philadelphia: Fortress, 1986.
	AST Library: BS 1140.2 R3913 1986.

	\bibitem{ks12} Schmid, Konrad.
	\emph{The Old Testament: A Literary History}. Translated by Linda Maloney.
	Minneapolis: Fortress, 2012.
	AST Library: BS 1174.3 S3613 2012.

\end{thebibliography}
\endgroup


Further literature on the Bible is vast. The works listed here have been
selected for clarity, insight, and theological alertness. Some will be
supplementary texts, as per \autoref{supplementary}.

\begingroup
\renewcommand{\section}[2]{}% temporarily remove the section heading
\begin{thebibliography}{99}
\makeatletter%https://tex.stackexchange.com/a/114434
\addtocounter{\@listctr}{14}
\makeatother

\bibitem{ra81} Alter, Robert. “The Techniques of Repetition.” Pages 88–113 in \emph{The Art of Biblical Narrative}. New York: Basic Books, 1981.

\bibitem{ga01} Anderson, Gary A. “Biblical Origins and the Fall.” Pages 197–210 in \emph{The Genesis of Perfection: Adam and Eve in Jewish and Christian Imagination}. Louisville: Westminster John Knox, 2001.

\bibitem{ga17} Anderson, Gary A. “Apophatic Theology: The Transcendence of God and the Story of Nadab and Abihu.” Pages 3–22 in \emph{Christian Doctrine and the Old Testament: Theology in the Service of Biblical Exegesis}. Grand Rapids: Baker Academic, 2017.

\bibitem{bz09} Ben Zvi, Ehud and James D. Nogalski. \emph{Two Sides of a Coin: Juxtaposing Views on Interpreting the Book of the Twelve / the Twelve Prophetic Books}. Analecta Gorgiana 201. Piscataway, NJ: Gorgias, 2009.

\bibitem{sc16} Chapman, Stephen B. “1 Samuel 1–12.” Pages 71–119 in \emph{1 Samuel as Christian Scripture: A Theological Commentary}. Grand Rapids: Eerdmans, 2016.

\bibitem{bc69} Childs, Brevard S. “Psalm 8 in the Context of the Christian Canon.” Pages 85--93 in \emph{Canon as Rule and Guide: Collected Essays}. Edited by Daniel R. Driver. FAT 1.174. Tübingen: Mohr Siebeck, 2023.

\bibitem{bc77} Childs, Brevard S. “The \emph{Sensus Literalis} of Scripture: An Ancient and Modern Problem.” Pages 169--80 in \emph{Canon as Rule and Guide: Collected Essays}. Edited by Daniel R. Driver. FAT 1.174. Tübingen: Mohr Siebeck, 2023.

\bibitem{bc79} Childs, Brevard S. \emph{Introduction to the Old Testament as Scripture}. Philadelphia: Fortress, 1979.

%\bibitem{bc85} Childs, Brevard S. \emph{Old Testament Theology in a Canonical Context}. Philadelphia: Fortress, 1985.

%\bibitem{bc92} Childs, Brevard S. \emph{Biblical Theology of the Old and New Testaments: Theological Reflection on the Christian Bible}. Minneapolis: Fortress, 1992.

\bibitem{bc04} Childs, Brevard S. \emph{The Struggle to Understand Isaiah as Christian Scripture}. Grand Rapids: Eerdmans, 2004.

\bibitem{ed03} Davis, Ellen F. “Teaching the Bible Confessionally in the Church.” Pages 9–26 in \emph{The Art of Reading Scripture}. Edited by Ellen F. Davis and Richard B. Hays. Grand Rapids: Eerdmans, 2003.

\bibitem{ed14} Davis, Ellen F. \emph{Biblical Prophecy: Perspectives for Christian Theology, Discipleship, and Ministry}. Louisville: Westminster John Knox, 2014.

\bibitem{ndw14} deClaissé-Walford, Nancy L. “The Meta-Narrative of the Psalter.” Pages 363–76 in \emph{The Oxford Handbook of the Psalms}. Edited by William P. Brown. Oxford: Oxford University Press, 2014.

\bibitem{kd13} Dell, Katherine J. “Ecclesiastes as Wisdom: Consulting Early Interpreters.” Pages 9–36 in \emph{Interpreting Ecclesiastes: Readers Old and New}. Winona Lake, IN: Eisenbrauns, 2013.

\bibitem{mf04} Fox, Michael V. \emph{Ecclesiastes: The Traditional Hebrew Text with the New JPS Translation}. The JPS Bible Commentary. Philadelphia: Jewish Publication Society, 2004.

\bibitem{bj13} Janowski, Bernd. \emph{Arguing with God: A Theological Anthropology of the Psalms}. Louisville: Westminster John Knox, 2013.

\bibitem{jl85} Levenson, Jon D. \emph{Sinai and Zion: An Entry into the Jewish Bible}. Minneapolis: Winston, 1985.

\bibitem{jl12} Levenson, Jon D. “The Test.” Pages 66–112 in \emph{Inheriting Abraham: The Legacy of the Patriarch in Judaism, Christianity, and Islam}. Princeton: Princeton University Press, 2012.

\bibitem{nm09} MacDonald, Nathan. “Israel and the Old Testament Story in Irenaeus’s Presentation of the Rule of Faith.” \emph{Journal of Theological Interpretation} 3.2 (2009): 281–98.

\bibitem{cm12} McGinnis, Claire Mathews. “The Hardening of Pharaoh’s Heart in Christian and Jewish Interpretation.” \emph{Journal of Theological Interpretation} 6.1 (2012): 43–64.

\bibitem{wm92} Moberly, R. W. L. \emph{The Old Testament of the Old Testament: Patriarchal Narratives and Mosaic Yahwism}. Minneapolis: Fortress, 1992.

\bibitem{wm13} Moberly, R. W. L. “A Love Supreme.” Pages 7–40 in \emph{Old Testament Theology: Reading the Hebrew Bible as Christian Scripture}. Grand Rapids: Baker Academic, 2013.

\bibitem{jn07a} Nogalski, James D. “Reading the Book of the Twelve Theologically.” \emph{Interpretation} 61.2 (2007): 115–22.

\bibitem{jn07b} Nogalski, James D. “Recurring Themes in the Book of the Twelve: Creating Points of Contact for a Theological Reading.” \emph{Interpretation} 61.2 (2007): 125–36.

\bibitem{db89} Patrick, Dale. “Studying Biblical Law as a Humanities.” \emph{Semeia} 45 (1989): 27–47.

\bibitem{ip14} Provan, Ian. \emph{Seriously Dangerous Religion: What the Old Testament Really Says and Why It Matters}. Waco: Baylor University Press, 2014.

\bibitem{fr11} Rutledge, Fleming. \emph{And God Spoke to Abraham: Preaching From the Old Testament}. Grand Rapids: Eerdmans, 2011.

\bibitem{ks19} Schmid, Konrad. \emph{A Historical Theology of the Hebrew Bible}. Translated by Peter Altmann. Grand Rapids: Eerdmans, 2019. ISBN 978-0802876935.

\bibitem{cs96} Seitz, Christopher R. “Old Testament or Hebrew Bible? Some Theological Considerations.” \emph{Pro Ecclesia} 5.3 (1996): 292–303.

\bibitem{cs99} Seitz, Christopher R. “The Call of Moses and the ‘Revelation’ of the Divine Name: Source-Critical Logic and Its Legacy.” Pages 145--61 in Christopher R. Seitz and Kathryn Greene-McCreight, eds. \emph{Theological Exegesis: Essays in Honor of Brevard S. Childs}. Grand Rapids: Eerdmans, 1999.

%\bibitem{cs18} Seitz, Christopher R. “‘Can We Read This Book?’ Reader Response-ability.” Pages 51–68 in \emph{The Elder Testament: Canon, Theology, Trinity}. Waco, TX: Baylor University Press, 2018.

\bibitem{gs92} Sheppard, Gerald T. “Theology and the Book of Psalms.” \emph{Interpretation} 46.2 (1992): 143–55.

\bibitem{bs15} Sommer, Benjamin D. “What Happened at Sinai? Maximalist and Minimalist Approaches.” Pages 27–98 in \emph{Revelation and Authority: Sinai in Jewish Scripture and Tradition}. New Haven: Yale University Press, 2015.

\bibitem{pt84} Trible, Phyllis. “An Unnamed Woman: The Extravagance of Violence.” Pages 65–91 in \emph{Texts of Terror: Literary-Feminist Readings of Biblical Narratives}. Overtures to Biblical Theology. Philadelphia: Fortress, 1984.

\end{thebibliography}
\endgroup

For additional literature, I recommend exploring \href{https://go.openathens.net/redirector/astheology.ns.ca?url=https://www.oxfordbibliographies.com/obo/page/biblical-studies}{Oxford Bibliographies: Biblical Studies (Full Text)}.
You can access the database automatically while on campus or remotely
with OpenAthens credentials. Numerous articles by subject-area
specialists appear under such headings as: Ancient Near East; Bible;
Early Christianity; Greco-Roman World; Hebrew Bible; New Testament;
Rabbinic Judaism; Second Temple Judaism.

\end{document}
