\ProvidesFile{grades.tex}[2013/08/19 v1.0 -- Course policy]

\subsection{Grading System}
\label{grades}

Tyndale University College provides these benchmarks for summative assessment.
I may furnish more detailed rubrics for particular assignments.

\mparshift{0.5\baselineskip}
\marginpar{
\footnotesize\addfontfeatures{Numbers=Monospaced}
\begin{tabular}[t]{@{}>{\scshape}c@{\hspace{1em}}c@{\hspace{1em}}c@{}}
  \toprule
  +           & 90--100 & 4.00 \\
  a           & 85--89  & 4.00 \\
  \char"2212  & 80--84  & 3.70 \\ [1ex]
  +           & 77--79  & 3.30 \\
  b           & 73--76  & 3.00 \\
  \char"2212  & 70--72  & 2.70 \\ [1ex]
  +           & 67--69  & 2.30 \\
  c           & 63--66  & 2.00 \\
  \char"2212  & 60--62  & 1.70 \\ [1ex]
  +           & 57--59  & 1.30 \\
  d           & 53--56  & 1.00 \\
  \char"2212  & 50--52  & 0.70 \\ [1ex]
  f           & 0--49   & 0.00 \\
  \bottomrule
\end{tabular}}

\begin{description}
  \item[A, B -- Excellent, Good]
    These grades are earned only when evidence indicates that the student
    has consistently maintained above average progress in the subject.
    Sufficient evidence may involve such qualities as creativity, originality,
    thoroughness, responsibility and consistency.
  \item[C -- Satisfactory]
    This grade means that the student has fulfilled the requirements of the
    subject to the satisfaction of the instructor. These requirements include
    the understanding of subject matter, adequacy and promptness in the
    preparation of assignments and participation in the work of the class.
  \item[D -- Poor]
    This grade indicates that the accuracy and content of work submitted meets
    only the minimal standards of the instructor. Consistent performance at
    this level is considered inadequate for graduation.
  \item[F -- Failing]
    Work submitted is inadequate. Attitude, performance and attendance are
    considered insufficient for a passing grade.
\end{description}
