% !TEX TS-program = XeLaTeX
% !TEX encoding = UTF-8 Unicode

\documentclass[12pt]{article}
% basic typography
\usepackage{fontspec,realscripts}
\defaultfontfeatures{Mapping=tex-text,Numbers=OldStyle}
\setmainfont[BoldFont={Kepler Std Semibold}]{Kepler Std Light}
\setsansfont{DIN 1451 Std Mittelschrift}
\setmonofont[Scale=MatchLowercase]{Menlo}
\newfontfamily{\sbl}[Script=Hebrew]{SBLHebrew}
\usepackage{microtype}
\frenchspacing
% add logo and set in DIN
\usepackage{titling,graphicx}%\usepackage{showframe}
%\setlength{\droptitle}{-46pt}
\renewcommand{\maketitlehooka}{\sffamily}
\pretitle{\begin{center}\LARGE
          \includegraphics[width=0.4\textwidth]{tuc_vblack.png}
          \par\vskip 1.5em}
\posttitle{\par\end{center}\vskip 0.5em}
% set section headers in DIN
\usepackage{titlesec}
\titleformat*{\section}{\Large\sffamily}
\titleformat*{\subsection}{\large\sffamily}
\titleformat*{\subsubsection}{\normalsize\sffamily}
\titleformat*{\paragraph}{\normalsize\sffamily}
\titleformat*{\subparagraph}{\normalsize\sffamily}
\renewcommand{\descriptionlabel}[1]{\hspace{\labelsep}\textsf{#1}}
% provide for tables and set caption in DIN
\usepackage{ctable}
\usepackage{float} % provides the H option for float placement
\usepackage[font=sf,labelformat=empty]{caption}% set captions in DIN
\renewcommand{\tmark}[1][a]{\hbox{\normalfont\textit{\textsuperscript#1}}}
% realscripts somehow makes \textsuperscript not work as ctable expects

% Redefine labelwidth for lists; otherwise, the enumerate package will cause
% markers to extend beyond the left margin.
\makeatletter\AtBeginDocument{%
  \renewcommand{\@listi}
    {\setlength{\labelwidth}{4em}}
}\makeatother
\usepackage{enumerate}

\usepackage[setpagesize=false, % page size defined by xetex
            unicode=false, % unicode breaks when used with xetex // alt: unicode,pdfencoding=auto
            xetex]{hyperref}
\hypersetup{breaklinks=true,
            bookmarks=true,
            pdfauthor={Daniel R. Driver},
            pdftitle={HEBR 3013: Readings in Biblical Hebrew I},
            pdfsubject={Tyndale UC Course Syllabus},
            pdfkeywords={biblical studies} {theology},
            colorlinks=true,
            urlcolor=blue,
            linkcolor=magenta,
            pdfborder={0 0 0}}
%\setlength{\parindent}{0pt} % alternate setting of paragraphs
%\setlength{\parskip}{6pt plus 2pt minus 1pt}
\setlength{\emergencystretch}{3em} % prevent overfull lines
\setcounter{secnumdepth}{1} % set to 0 to remove

\title{HEBR 3013: Readings in Biblical Hebrew I}
\author{Daniel R. Driver, PhD\thanks{
  Ballyconnor office 1107, phone 416.226.6620 x2201,
  email \href{mailto:ddriver@tyndale.ca}{\texttt{ddriver@tyndale.ca}}.}}
\date{Fall 2012 Course Syllabus (\texttt{2.0})\thanks{Meets in Ballyconnor 2084, Tues \& Thur, 1:00--2:20 \textsc{pm}.
  Version date: 9 October 2012.}}

\begin{document}
\maketitle
\thispagestyle{empty}

\begin{description}
\item[Commuter Hotline]
416.226.6620 x2187 --- Class cancellations due to inclement weather or
illness will be announced/posted on the commuter hotline.
\item[MyTyndale.ca]
This course may have \textsc{lms} pages that are meant to enhance your
learning experience and that you may need to access in order to complete
some assignments. It is where news items, notices and marks may be
posted. Log in often with your mytyndale username and password.
\item[Mailboxes]
Every student is responsible for information communicated through the
student mailboxes. A mailbox directory is posted beside the mailboxes.
For more information contact the Registrar's office.
\end{description}

\section{Course Description}

\emph{From the Academic Calendar}: Provides the student with a broad
exposure to an extensive amount of Biblical Hebrew prose, facilitating
an environment where the student's reading ability is enhanced. In
addition to providing a platform of advanced morphology and syntax, also
provides a preliminary introduction to matters of textual criticism and
exposure to the Septuagint and Qumran materials. Prerequisites:
\textsc{hebr} 201, 202; \textsc{rlgs} 101, 102, 201. Exclusion:
\textsc{hebr} 401.

\section{Course Objectives}

The basic goal of this course is to become proficient in reading
biblical Hebrew. We will build on what students have done in previous
study, aiming to bridge whatever gaps remain between rudimentary Hebrew
and the demands that confront serious readers of the Hebrew Bible.
Students should move beyond rote memorization and toward the mature use
of language skills. Of course the need for memorization does not
disappear: the steady acquisition of vocabulary will continue, as will
the recitation of select passages of scripture. We will work together to
make our common labor as enjoyable as possible, encouraging and
motivating one another in a variety of ways.

\section{Required Texts}

\begin{enumerate}[1.]
\item
  \emph{Biblia Hebraica Stuttgartensia}. Paperback/Student ed.
  Stuttgart: Deutsche Bibelgesellschaft, 1997.
\item
  Brown, F., S. Driver and C. Briggs, \emph{Brown-Driver-Briggs Hebrew
  and English Lexicon} (Peabody, Mass: Hendrickson, 2004; repr. of 1906
  ed.).
\item
  Buth, Randal. \emph{Living Biblical Hebrew: Selected Readings with 500
  Friends (Gimel)}. Jerusalem: Biblical Language Center, 2006. Order the
  book and CD {[}ISBN 978-9657352045{]} from
  \url{http://www.biblicallanguagecenter.com/books-products/biblical-hebrew/}.
\item
  Cook, John A. and Robert D. Holmstedt. \emph{Biblical Hebrew: A
  Student Grammar}. Creative Commons BY-NC 3.0 textbook released for
  personal or classroom use. Draft, 2007; revised 2009, 2010. Get it
  online at \url{http://individual.utoronto.ca/holmstedt/Textbook.html}.
\end{enumerate}

\section{Supplementary Texts}

\begin{enumerate}[1.]
\item
  Sáenz-Badillos, Angel. \emph{A History of the Hebrew Language}.
  Cambridge: Cambridge University Press, 1996.
\item
  Scott, W.R. and H.P. Scanlin. \emph{A Simplified Guide to
  BHS: Critical Apparatus, Masora, Accents, Unusual Letters \& Other
  Markings}. Richland Hills:
  BIBAL, 2007.
\item
  Tov, Emanuel. \emph{Textual Criticism of the Hebrew Bible}.
  3\textsuperscript{rd} edition, revised and expanded. Minneapolis:
  Fortress, 2012.
\item
  Zvi, Ehud Ben, Ms.~Maxine Hancock and Mr.~Richard A. Beinert.
  \emph{Readings in Biblical Hebrew: An Intermediate Textbook}. Yale
  Language Series. New Haven: Yale University Press, 1993.
\end{enumerate}

\section{Course Evaluation}

\begin{enumerate}[1.]
\item
  The backbone of the course is reading and translation. Biblical texts
  are set for each week. Students are to provide their own translations,
  to be collected and refined for submission at the end of term.
\item
  Ten weekly quizzes make sure that you have a rudimentary vocabulary
  (500 friends) in hand by the end of the first semester.
\item
  At some point in the first semester you will need to schedule office
  hours to recite a dozen (12) verses of Hebrew. You can select any
  verses you like, though I recommend choosing one of the texts we study
  together. The memorization is due on the last day of class.
\item
  The final exam will be held during exam week, as scheduled by the
  Registrar.
\item
  The semester's work will be weighted as follows, though the instructor
  reserves the right to adjust the balance as necessary:
\end{enumerate}

\ctable[pos = H, center, botcap]{lr}
{% notes
}
{% rows
\FL
Translation Set & 25\%
\\
Vocabulary Quizzes & 25\%
\\
Memorization & 25\%
\\
Final Exam & 25\%
\LL
}

Finally, students are responsible to keep a backup print copy of all
assignments.

\section{Course Outline}

See the table at the end of this syllabus.

\ctable[caption = {Course Outline and Schedule of Readings},
pos = B, center, topcap]{llr}
{% notes
\tnote[a]{The final exam will be held during exam week as scheduled by
  the Registrar.}
}
{% rows
\FL
\emph{Cook \& Holmstedt} & \emph{Buth \& your 500 friends} & \emph{Date}
\ML
Introductions &  & September 11
\\
Review &  & September 13
\\\noalign{\medskip}
Lesson 2 &  & September 18
\\
Lessons 3 \& 4 & Quiz 1: Friends through \sbl א & September 20
\\\noalign{\medskip}
Lessons 5 \& 6 &  & September 25
\\
Lessons 7 \& 8 & Quiz 2: Friends through \sbl ד & September 27
\\\noalign{\medskip}
Lessons 9 \& 10 & Genesis 22 & October 02
\\
\emph{No class: DRD away} & No quiz; Friends through \sbl ה & October 04
\\\noalign{\medskip}
Lessons 11 \& 12 & Genesis 22 plus accents & October 09
\\
Lessons 13 \& 14 & Quiz 3: Friends through \sbl ח & October 11
\\\noalign{\medskip}
Lessons 15 \& 16 & Genesis 1--2:3 & October 16
\\
Lessons 17 \& 18 & Quiz 4: Friends through \sbl כ & October 18
\\\noalign{\medskip}
\emph{No class: reading day} &  & October 23
\\
Lessons 19 \& 20 & Quiz 5: Friends through \sbl מ & October 25
\\\noalign{\medskip}
Lesson 21 & Exodus 19 & October 30
\\
Lesson 22 & Quiz 6: Friends through \sbl נ & November 01
\\\noalign{\medskip}
Lesson 23 & Exodus 20 & November 06
\\
Lesson 24 & Quiz 7: Friends through \sbl ע & November 08
\\\noalign{\medskip}
Lesson 25 & Ruth 1 & November 13
\\
Lesson 26 & Quiz 8: Friends through \sbl ק & November 15
\\\noalign{\medskip}
Lesson 27 & Ruth 2 & November 20
\\
Lesson 28 & Quiz 9: Friends through \sbl ת & November 22
\\\noalign{\medskip}
Lesson 29 & Ruth 3 & November 27
\\
Lesson 30 & Quiz 10: All 500 friends & November 29
\\\noalign{\medskip}
Review & Ruth 4 & December 04
\\
\emph{No class: reading day} & Translation sets \textbf{due} & December
06
\\\noalign{\medskip}
Final Exam &  & TBA\tmark[a]
\LL
}

\section{Academic Integrity}

Integrity in academic work is required of all our students. Academic
dishonesty is any breach of this integrity, and includes such practices
as cheating (the use of unauthorized material on tests and
examinations), submitting the same work for different classes without
permission of the instructors, using false information (including false
references to secondary sources) in an assignment, improper or
unacknowledged collaboration with other students, and plagiarism.

Tyndale University College \& Seminary takes seriously its
responsibility to uphold academic integrity, and to apply consequences
for academic dishonesty. Students are advised to consult the
\href{http://www.tyndale.ca/registrar/calendar}{Academic Calendar} for
more information on this policy and its application to their work in
this course.

\section{Bibliography}

In \textsc{hebr} 201 and 202 last year I assigned a text by Kittel,
Hoffer and Wright. Ben Zvi makes reference to it, so it may be useful
for reference in \textsc{hebr} 302.

\begin{itemize}
\item
  Kittel, Bonnie, Victoria Hoffer and Rebecca Wright, \emph{Biblical
  Hebrew, 2nd Ed.: Text and Workbook} (New Haven: Yale UP, 2004).
  {[}ISBN 978-0300098624{]}
\end{itemize}

A useful index to standard texts (cheaper than the major software
packages):

\begin{itemize}
\item
  Einspahr, Bruce, \emph{Index to Brown, Driver \& Briggs Hebrew
  Lexicon} (Chicago: Moody, 1976). {[}Ref PJ 4833.B6 E35 1976{]}
\end{itemize}

Other major dictionaries of classical Hebrew:

\begin{itemize}
\item
  Köhler, Ludwig, and Walter Baumgartner, \emph{The Hebrew and Aramaic
  lexicon of the Old Testament} (Leiden: Brill, 1994--2000). {[}Ref PJ
  4833.K61813{]}
\item
  Clines, David J. A., ed., \emph{The Dictionary of Classical Hebrew}
  (Sheffield: Sheffield Academic Press, 1993--2011). {[}Ref PJ
  4833.D53{]}
\end{itemize}

Other Hebrew grammars (only a selection, as a glance at the right shelf
in our library will tell you):

\begin{itemize}
\item
  Seow, C. L., \emph{A Grammar for Biblical Hebrew} (Nashville:
  Abingdon, 1995). I may reference this grammar on occasion. {[}PJ
  4567.S424 1995{]}
\item
  Weingreen, J., \emph{A Practical Grammar for Classical Hebrew}
  (Oxford: Clarendon, 1959). {[}Two copies: PJ 4567.W4 1939{]}
\item
  Martin, J. D., \emph{Davidson's Introductory Hebrew Grammar} (27th
  edition; Edinburgh: T\&T Clark, 1993). This is the grammar I was
  assigned in St Andrews, Scotland. Our library has copies of some older
  editions, including the 26th. {[}PJ 4567.D3 1966{]}
\item
  Kautsch, E., \emph{Gesenius' Hebrew Grammar} (revised in accordance
  with the 28th German edition by A. E. Crowley; Oxford: Clarendon,
  1910). This is still the standard reference grammar in English. {[}Ref
  PJ 4567.G46 1910{]}
\end{itemize}

There are many other resources out there, including some Biblical Hebrew
vocabulary cards (paper and now electronic). In my experience, making
your own cards and other study resources is the best way to cement your
learning.

\end{document}