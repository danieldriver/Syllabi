% Copyright (c) 2018 by Daniel R. Driver.
% !TEX encoding = UTF-8 Unicode
% !TEX TS-program = XeLaTeX

\documentclass[titlepage]{article}

% This document presumes a file structure and set of inputs that are
% available at: git@github.com:danieldriver/syllabi.git

\newcommand\policy{../policy}
\newcommand\incl{../includes}
\ProvidesFile{variables.tex}[2018/05/24 v2.1 -- Syllabus variables]

\usepackage{xspace} % make manual spaces (like \mycmd\ ) unnecessary
\usepackage{xifthen} % provides \isempty test

% variables for internal use
\newcommand\prof{}
\newcommand\pdegree{}
\newcommand\pphone{}
\newcommand\pemail{}
\newcommand\poffice{}
\newcommand\phours{}
%
\newcommand\ccode{}
\newcommand\ctitle{}
\newcommand\cseries{}
\newcommand\cversion{}
\newcommand\csemester{}
\newcommand\cmeetson{}
\newcommand\cmeetsat{}
\newcommand\cmeetsin{}
\newcommand\cwebsite{}
\newcommand\cdescrip{}
\newcommand\cprereqs{}
\newcommand\edobject{}

% in case of fully online courses - https://tex.stackexchange.com/a/5896
\newif\ifonline
\newcommand\Int[2]{\ifonline#1\else#2\fi}

% commands for setting variables in the preamble
\newcommand\professor[2][PhD]{
  \renewcommand\pdegree{#1\xspace}
  \renewcommand\prof{#2\xspace}}
\newcommand\phone[1]{
  \renewcommand\pphone{\addfontfeatures{Numbers=Monospaced}#1\xspace}}
\newcommand\email[1]{
  \renewcommand\pemail{\href{mailto:#1}{#1}\xspace}}
\newcommand\officehours[2][Library, Room 5-North]{
  \renewcommand\poffice{#1\xspace}
  \renewcommand\phours{#2\xspace}}
%
\newcommand\coursecode[2][1.0]{
  \renewcommand\cversion{#1\Int{-i}{}\xspace}
  \renewcommand\ccode{#2\Int{(Int)}{}\xspace}}
\newcommand\coursetitle[2][]{
  \ifthenelse{\isempty{#1}}%
    {}% do nothing if #1 is empty, else:
    {\renewcommand\cseries{#1\\[1ex]}}
  \renewcommand\ctitle{#2\xspace}}
\newcommand\semester[1]{
  \renewcommand\csemester{#1\xspace}}
\newcommand\meets[3]{
  \newcommand\AM{\textsc{am}}
  \newcommand\PM{\textsc{pm}}
  \renewcommand\cmeetson{#1\xspace}
  \renewcommand\cmeetsat{\Int{From 9:00 \AM}{#2}\xspace}
  \renewcommand\cmeetsin{\Int{\href{https://smu.brightspace.com/d2l/login}{Brightspace}}{#3}\xspace}}
\newcommand\website[1]{
  \renewcommand\cwebsite{\href{http://#1}{#1}\xspace}}
\newcommand\cdescription[2][RM 1000 or GTRS 6000; and BF 1001]{
  \renewcommand\cprereqs{#1}
  \renewcommand\cdescrip{#2\par}}
\newcommand\objectives[1]{
  \renewcommand\edobject{#1\par}}


%\onlinetrue % \Int{true}{false}
\coursecode[1.0]{HB tk}% version is 1.0 unless \cmd[1.1]{}
\coursetitle{tk}% \cmd[optional series]{title}

% Taught in:
%   - Fall 2018

\professor{Daniel R. Driver}% terminal degree is PhD unless \cmd[DPhil]{}
\phone{902-425-7051}
\email{ddriver@astheology.ns.ca}
\officehours{Tuesdays, 2:00--4:00 \PM}

\semester{Fall Term 2018}
\meets{On Mondays}% \meets{on}{at}{in} ; can nest \Int{Thursdays}{Tuesdays}
      {9:00--11:30 \AM}
      {TBD}
\website{danieldriver.com/courses/}% omit http://
\cdescription{% \cmd[standard prereqs]{from the current Academic Calendar}

The First Testament of Christian Scripture, also called the \ldots

}% end of course description
\objectives{%

By the end of the course students should be able to demonstrate \ldots

% ATS-AST Goals and Assessment Tool
%
% 1. Substantial grounding in critical analytical skills central to
% theological disciplines and their contemporary practice: biblical,
% theological, pastoral;
%
% 2. Cultivate excellence in practices of communicating faith that
% bridge traditional and contemporary styles, genres, and media;
%
% 3. Inculcating habits of bringing critical self-awareness and
% social-awareness to pastoral leadership, fostering virtues crucial in
% times of foundational change such as, ingenuity, compassion and
% courage;
%
% 4. Nurture a committed love of learning and spiritual formation
% understood as integral to pastoral professional practice.
%
% Religious Heritage (Standard A, Section 3.1.1), Cultural Context
% (Standard A, Section 3.1.2), Personal and Spiritual Formation
% (Standard A, Section 3.1.3), and Capacity for Ministerial and Public
% Leadership (Standard A, Section 3.1.4)
%
% Rubric for Assessment: Needs Work; Satisfactory; Doing Well

Students should also be able to:
	name;
	give;
	contrast;
	locate;
	understand;
	recognize;
	list;
	articulate;
	defend;
	classify;
	report on;
	discern.

% A Model of Learning Objectives
% Adapted from http://www.celt.iastate.edu/teaching/RevisedBlooms1.html
%
% A learning objective should contain a *subject* (a verb related to
% cognitive process) and an *object* (a noun identifying the knowledge
% or skill sought). Statements should be precise and measurable, e.g.,
% "By the end of this course, students will be able to <verb> <noun>."
%
% Anderson and Krathwohl name four *classes of knowledge* that students
% can be expected to acquire. Terms move from the concrete to the
% abstract:
%
%   1. factual           (basic elements, details, terminology)
%   2. conceptual        (interrelationship of basic elements)
%   3. procedural        (skills, methods, techniques, criteria)
%   4. metacognitive     (strategy, context and conditions, self-knowledge)
%
% Metacognitive knowledge is a special case. It concerns one's own
% "cognition and about oneself in relation to various subject matters"
% (Anderson and Krathwohl, 2001, p.44).
%
% A&K also identify 19 *cognitive processes*, arranged in six categories
% (a 2001 revision of Bloom's 1957 taxonomy) that progress from lower
% order to higher order thinking skills (from LOTS to HOTS):
%
%   1. remember
%      * recognizing     (identifying)
%      * recalling       (retrieving)
%   2. understand
%      * interpreting    (clarifying, paraphrasing, representing, translating)
%      * exemplifying    (illustrating, instantiating)
%      * classifying     (categorizing, subsuming)
%      * summarizing     (abstracting, generalizing)
%      * inferring       (concluding, extrapolating, interpolating, predicting)
%      * comparing       (contrasting, mapping, matching)
%      * explaining      (constructing models)
%   3. apply
%      * executing       (carrying out)
%      * implementing    (using)
%   4. analyze
%      * differentiating (discriminating, distinguishing, focusing, selecting)
%      * organizing      (finding coherence, integrating, outlining, parsing, structuring)
%      * attributing     (deconstructing)
%   5. evaluate
%      * checking        (coordinating, detecting, monitoring, testing)
%      * critiquing      (judging)
%   6. create
%      * generating      (hypothesizing)
%      * planning        (designing)
%      * producing       (constructing)
%
% If you reference a List of Measurable Verbs Used to Assess Learning
% Outcomes, remember that A&K reverse Bloom's 6th and 5th categories.

}% end of learning objectives

\ProvidesFile{preamble.tex}[2013/09/06 v1.0 -- Syllabus preamble]

% basic typography
\usepackage{fontspec}
\setmainfont[Ligatures=TeX]{Meta Serif Pro}
\setsansfont[Ligatures=TeX]{Meta Pro}
\newfontfamily\Heb{Meta Hebrew}
\setmonofont[Scale=MatchLowercase]{Menlo}
\usepackage{sectsty}
\allsectionsfont{\sffamily}
\frenchspacing
\setlength{\emergencystretch}{3em} % prevent overfull lines

% custom font size and leading
\renewcommand\tiny{\fontsize{6}{9}\selectfont}
\renewcommand\scriptsize{\fontsize{7}{10}\selectfont}
\renewcommand\footnotesize{\fontsize{8}{11}\selectfont}
\renewcommand\small{\fontsize{8.5}{11.5}\selectfont}
\renewcommand\normalsize{\fontsize{9}{12}\selectfont}% base size
\renewcommand\large{\fontsize{11}{14}\selectfont}
\renewcommand\Large{\fontsize{13}{16}\selectfont}
\renewcommand\LARGE{\fontsize{16}{19}\selectfont}% "course syllabus \\ semester" benefits from more lead
\renewcommand\huge{\fontsize{19}{21}\selectfont}
\renewcommand\Huge{\fontsize{24}{26}\selectfont}

% layout packages: page, logo, tables
\usepackage[scale={0.6,0.8},
            xetex]{geometry}
\usepackage{graphicx}
\usepackage{array}     % allow insertions of column styling with >{}
\usepackage{booktabs}  % elegant horizontal rules in tables
\usepackage{marginfix} % protect positioning of margin table in policy/grades

% custom macros for a session count in the schedule of readings
\newcounter{session}
\newcounter{columns}
\newcounter{courseunit}
\newcommand\setcolumncount[2][0]{ % optionally set count to other than 0,
  \setcounter{session}{#1}        % e.g. to -1, or to a standing count
  \setcounter{columns}{#2}}
\newcommand\sessioncount{\stepcounter{session}\arabic{session}}
\newcommand\sessionskip[1]{\multicolumn{1}{@{}r@{ }}{#1}}
\newcommand\unit[1]{\multicolumn{\thecolumns}{c}{%
  \scshape\stepcounter{courseunit}\roman{courseunit}. \MakeLowercase{#1}}}
\newcommand\noclass[1]{\multicolumn{1}{@{}l}{\itshape No Class: #1}}

% color to match Tyndale's branding
\usepackage[usenames]{xcolor}
% predefined: black, white, red, green, blue, cyan, magenta, yellow
\definecolor{TyndaleURLs}{HTML}{0062A0} % links on tyndale.ca
\definecolor{TyndaleBlue}{cmyk}{1,1,0,.32}
\definecolor{TyndaleGold}{cmyk}{0,.27,1,0}
\definecolor{TyndaleRed}{cmyk}{0,1,.99,.04}
\definecolor{TyndaleBlack}{cmyk}{0,0,0,1}
\definecolor{TyndaleGreen}{cmyk}{.45,0,1,.24}
\definecolor{TyndaleOrange}{cmyk}{0,.79,1,0}
\definecolor{TyndaleAqua}{cmyk}{.47,0,.24,0}
\definecolor{TyndaleYellow}{cmyk}{.03,.03,.35,0}

% metadata (assumes a host of definitions are made in the main file)
\usepackage[setpagesize=false,     % leave this to geometry
            hyperfootnotes=false,  % fragile and distracting
            xetex]{hyperref}
\hypersetup{breaklinks=true,       % allow link text to break across lines
            colorlinks=true,       % colorlinks resets pdfborder to 0 0 0
            urlcolor=TyndaleURLs,  % for external links
            linkcolor=TyndaleRed,  % for normal internal links
            citecolor=TyndaleGold, % for bibliographical citations in text
            pdfauthor={\prof},
            pdftitle={\ccode: \ctitle},
            pdfsubject={Tyndale UC, \csemester},
            pdfcreator={github.com/danieldriver/syllabus}}
\urlstyle{same}                    % don't use monospace font for urls

% custom footlines
\usepackage{fancyhdr}
\pagestyle{fancy} % turn it on
\fancyhf{}        % reset everything
\renewcommand{\headrulewidth}{0pt} % remove header line as well
\lfoot{\sffamily\scshape\footnotesize\MakeLowercase{\ctitle, v\cversion}}
\rfoot{\sffamily\scshape\footnotesize\MakeLowercase{\prof\quad\thepage}}

% gratuitous with custom title page, but useful as a fallback
\title{\ccode: \ctitle}
\author{\professor}
\date{\semester}


\begin{document}
\ProvidesFile{title.tex}[2013/09/06 v1.0 -- Syllabus title page]

\begin{titlepage}
  \begin{center}

    \LARGE\sffamily % set title elements in a large sans serif

    \begin{minipage}{\textwidth}
      \parbox[t]{0.5\textwidth}{
        \mbox{}\\[-13pt] % dummy line to align parboxes
        \includegraphics[width=0.5\textwidth]{.syllabus/includes/TyndaleUC}}
      \hfill
      \parbox[t]{0.4\textwidth}{
        \raggedleft Course Syllabus\\
        \csemester}
    \end{minipage}

    \vfill

    {\textsc{\MakeLowercase\ccode}\\[1ex]
      \bfseries\cseries\Huge\ctitle}

    \vfill

    \normalsize\rmfamily % switch back to body type

    \begin{tabular}{>{\bfseries}rl>{\bfseries}rl}
      \toprule
      Instructor & \prof, \pdegree & Course  & Version \cversion \\
      \midrule
      Phone      & \pphone         & Meets   & \cmeetson         \\
      Email      & \pemail         & Time    & \cmeetsat         \\
      Office     & \poffice        & Room    & \cmeetsin         \\
      Hours      & \phours         & Website & \cwebsite         \\
      \bottomrule
    \end{tabular}

    \vfill

    \begin{description}\small
      \item[Commuter Hotline]
        Class cancellations due to inclement weather or illness will
        be announced on the commuter hotline at \texttt{416.226.6620
        x2187}. Alternately, weather cancellation information is posted
        at \href{http://tyndale.ca/weather}{tyndale.ca/weather}.
      \item[MyTyndale.ca]
        This course may have materials stored on its website, such as
        handouts or readings that may be needed in order to complete
        assignments. Students are responsible for checking these course
        pages on a regular basis. Here, too, students are able to view
        their grades throughout the semester. For more information see
        Section~\ref{mytyndale}, below.
      \item[Mail]
        Students are responsible for information communicated through
        their campus mailboxes and student e-mail accounts. A mailbox
        directory hangs beside the mailboxes. For more information
        contact the Registrar's office.
    \end{description}

  \end{center}

  \section{Course Description}
  \label{description}

  \emph{From the Academic Calendar:} \cdescrip

\end{titlepage}
\setcounter{page}{2} % count the title page as page 1


\section{Learning Objectives}
\label{objectives}

\edobject

\section{Required Texts \& Materials}
\label{texts}

The following texts are required. Students are strongly encouraged to
purchase their own copies. Library copies that are not reference works
have been placed on a 2-hour reserve.

\begingroup
\renewcommand{\section}[2]{}% temporarily remove the section heading
\begin{thebibliography}{Sommer}% use the longest item in the bibliography

%	\bibitem[Abbrev]{citeref} Author.
%	\emph{title}.
%	tk: tk, 20tk.
%	ISBN tk.

	\bibitem[Childs]{childs} Brevard S. Childs.
	\emph{The Book of Exodus: A Critical, Theological Commentary}.
	Louisville: Westminster John Knox Press, 1974.
	ISBN 978-0664229689.

	\bibitem[Sommer]{sommer} Benjamin D. Sommer.
	\emph{Revelation and Authority: Sinai in Jewish Scripture and Tradition}.
	New Haven: Yale University Press, 2015.
	ISBN 978-0300234688.

	\bibitem[von Rad]{vonrad} Gerhard von Rad.
	\emph{Moses}. 2nd ed. Edited by K.\,C. Hanson.
	Eugene, OR: Cascade Books, 2011.
	ISBN 978-1606087718.

	\bibitem[Nyssa]{nyssa} Gregory of Nyssa.
	\emph{The Life of Moses}. Translated by Abraham J. Malherbe and Everett Ferguson.
	San Francisco: HarperSanFrancisco, 2006.
	ISBN 978-0060754648. This title is optional but strongly recommended.

\end{thebibliography}
\endgroup

Students should also have a good, modern translation of the Bible, such
as the NRSV or NJPS. If you want a study Bible, I recommend either
Michael Coogan et al., eds., \emph{The New Oxford Annotated Bible: NRSV
with Apocrypha} (5th ed.; Oxford: OUP, 2018) or, with some superior
notes and essays but neither Apocrypha nor NT, Adele Berlin and Marc Zvi
Brettler, eds., \emph{The Jewish Study Bible: Second Edition} (2nd ed.;
Oxford: OUP, 2014).

\section{Supplementary Texts}
\label{supplementary}

Supplementary readings may be recommended throughout the semester.
Excerpts from this literature will either be placed on reserve or made
available through the course website.

In addition, the following reference works are worth owning and
consulting. First, \cite{rlgs} includes sound advice on things like
reading religious texts, writing essays and book reviews, making oral
presentations, and learning languages. Second, \cite{sbl2} is a standard
reference in the field, useful to beginning students and specialists in
biblical studies alike.

\begingroup
\renewcommand{\section}[2]{}% temporarily remove the section heading
\begin{thebibliography}{9}% use the longest item in the bibliography

	\bibitem{rlgs} Northey, Margot, Bradford A. Anderson, and Joel N. Lohr.
	\emph{Making Sense in Religious Studies: A Student's Guide to Research and Writing}. 2nd ed.
	Oxford: OUP, 2015. % ISBN 978-0199010349.

	\bibitem{sbl2} Collins, Billie Jean, et al.
	\emph{The SBL Handbook of Style}. 2nd ed.
	Atlanta: SBL Press, 2014. % ISBN 978-1589839649.
	Designed to augment \href{http://www.chicagomanualofstyle.org/home.html}{\emph{The Chicago Manual of Style}},
	which is the standard at AST, there is also a free
	“\href{https://www.sbl-site.org/assets/pdfs/pubs/SBLHSsupp2015-02.pdf}{Student Supplement for \emph{The SBL Handbook of Style}, Second Edition}.”

\end{thebibliography}
\endgroup

\section{Course Outline}
\label{outline}

% For ease of citation:
\newcommand\bsc[1]{\cite[#1]{childs}}
\newcommand\bds[1]{\cite[#1]{sommer}}
\newcommand\gvr[1]{\cite[#1]{vonrad}}
\newcommand\nys[1]{\cite[#1]{nyssa}}

We will adhere to the schedule in \autoref{schedule} as closely as
possible, though the professor reserves the right to adjust it to suit
the needs of the class.

\setcolumncount{5}% set up \sessioncount, \unit{}, \noclass{}, and \reminder{memo}{date} macros
\begin{table}[htb]% set to `p' to put the schedule on its own page
  \centering
  \begin{tabular}{>{\sessioncount.}r@{ }lllr}% make sure the column config agrees with \setcolumncount
	\toprule
	\sessionskip{\textbf{\S}.}&\textbf{Primary}&\textbf{Secondary}&\textbf{Supplementary}&\textbf{Due}\\
	\midrule

%	\unit{One} \\

		&                &                &                &  Sep. \\ % or \Int{14}{12} Sep.
		&                &                &                &  Sep. \\
		&                &                &                &  Sep. \\
		&                &                &                &  Oct. \\
		&                &                &                &  Oct. \\
	\reminder{First paper is \textbf{due} before midnight on the fifth day of class}{} \\
		&                &                &                &  Oct. \\
	\noclass{Term Break (Monday to Friday)}                &  Oct. \\ [1ex]

%	\unit{Two} \\

		&                &                &                &  Nov. \\
		&                &                &                &  Nov. \\
		&                &                &                &  Nov. \\
		&                &                &                &  Nov. \\
		&                &                &                &  Dec. \\
	\reminder{Second paper is \textbf{due} before midnight on the eleventh day of class}{} \\
		&                &                &                &  Dec. \\ [1ex]

	\reminder{End of Term: Final marks are due for all courses}{ Dec.} \\

	\bottomrule
  \end{tabular}
  \caption{Schedule of Readings}
  \label{schedule}
\end{table}

See the AST website for a list of other \href{http://www.astheology.ns.ca/students/academic-dates.html}{important dates}.

\section{Evaluation}
\label{evaluation}

\subsection{Grade Structure for \ccode}
\label{structure}

\begin{enumerate}

	\item A \textbf{book review} will facilitate student reflection on a
	work of biblical interpretation, in this case by \cite{vonrad}. Note
	that a book review is not the same thing as a book report. At least
	half of this paper should be devoted to analysis and evaluation. Its
	total length should be 1,000 words, and it is due on the third day
	of class.

	\item An \textbf{exegetical essay} will facilitate direct engagement
	with the biblical text. The first task is to identify an appropriate
	text. Select a suitably short passage from Exodus. Then, conduct an
	analysis and explication of it. Interact with \cite{childs} and at
	least one other commentator as you work, but be sure not to loose
	sight of the text itself. The paper should be 1,500 words long, and
	it is due on the sixth day of class.

	\item A \textbf{final thesis} gives students an opportunity to
	develop their own answer to the question “Who is Moses?” The paper
	must have a clear thesis, which should govern the entire discussion.
	The paper should also show an awareness of all assigned readings and
	any relevant lectures and classroom discussions. It should be 2,000
	words long, and it is due at the start of the last class.

%\Int{
%	\item Each student will \textbf{make an online presentation} \ldots
%}{
%	\item Each student will \textbf{lead a seminar} on \ldots
%}

\end{enumerate}

For guidance on how to approach these and other assignments, see
\cite{rlgs} in \autoref{supplementary}, above. The breakdown for the
semester's total work is shown in \autoref{grade-dist}.

\begin{table}[htbp]
  \centering
  {\lining
  \begin{tabular}{lr}
    \toprule
    Book Review      & 25\% \\
    Exegetical Essay & 35\% \\
    Final Thesis     & 40\% \\
    \bottomrule
  \end{tabular}}
  \caption{Distribution of Grades}
  \label{grade-dist}
\end{table}

\ProvidesFile{grades.tex}[2016/09/03 v2.0 -- Course policy]

\subsection{Grading System at AST}
\label{grades}

AST's \href{http://www.astheology.ns.ca/webfiles/AST_2016Calendar_web(A5)-06APR2016.pdf}{Academic
Calendar} provides guidelines and detailed criteria for academic
assessment. Marks are assigned by letter grade using the benchmarks in
\autoref{grade-syst}.

\begin{table}[htbp]
  \centering
  {\lining
  \begin{tabular}{lll}
    \toprule
%    Letter      & Percent & Assessment        \\
%	\midrule
    A+          & 94--100    & Exceptional    \\
    A           & 87--93     & Outstanding    \\
    A\char"2212 & 80--86     & Excellent      \\ [1ex]
    B+          & 77--79     & Good           \\
    B           & 73--76     & Acceptable     \\
    B\char"2212 & 70--72     & Marginal       \\ [1ex]
    C           & 60--69     & Unsatisfactory \\
    F           & 0--59      & Failure        \\
    FP          & 0          & Failure due to Plagiarism \\
    \bottomrule
  \end{tabular}}
  \caption{Summary of Grading System}
  \label{grade-syst}
\end{table}

% More detailed grading criteria from pp. 61--62 of `16.0406-I2-AST Academic Calendar.pdf'
%
%\begin{description}
%  \item[A+ (94-100) ‘Exceptional’]
%    A superior performance with consistent evidence of a comprehensive,
%    incisive grasp of all aspects of the subject matter; a very wide
%    knowledge base; insightful critical evaluation and analysis of the
%    material; an exceptional capacity for original, creative, and/or
%    logical thinking; an exceptional ability to organize, analyse,
%    synthesize, and to express thoughts fluently.
%  \item[A (87-93) ‘Outstanding’]
%    A comprehensive grasp of the subject matter, outstanding evidence of
%    original thought; sound critical evaluation of the material; an
%    excellent ability to organize, analyse, synthesize and to express
%    thoughts; mastery of an extensive knowledge base.
%  \item[A- (80-86) ‘Excellent’]
%    All the qualities of a B-level performance and an excellent capacity
%    for original, creative, and/ or logical thinking; excellent ability
%    to organize, analyse, synthesize, and integrate ideas; broad
%    knowledge base in the subject matter.
%  \item[B+ (77-79) ‘Good’]
%    A good performance with substantial knowledge of the subject matter;
%    a very good understanding of the relevant issues; familiarity with
%    relevant literature and techniques; good ability to organize,
%    analyse, and examine the material in a constructive and critical
%    manner.
%  \item[B (73-76) ‘Acceptable’]
%    A generally adequate performance with a good knowledge of the
%    subject matter; a fair understanding of relevant issues; some
%    ability to work with relevant literature and techniques; some
%    ability to develop solutions to difficult problems related to the
%    subject material.
%  \item[B- (70-72) ‘Marginally Acceptable’]
%    Some familiarity with the subject material; some understanding.
%    Satisfactory understanding of relevant issues; attempts to solve
%    moderately difficult problems related to the subject material in a
%    critical and analytical manner are only partially successful.
%  \item[C (60-69) ‘Unsatisfactory’]
%    A C grade indicates unsatisfactory academic performance. At the
%    discretion of the instructor, supplemental work may be negotiated to
%    upgrade the mark to a B range. A student may carry two C grades
%    without penalty in all courses except Foundations Courses,
%    Supervised Field Education, Supervised Ministry Practicum and the
%    Graduate Project. In these courses, a minimum grade of B- is
%    required to graduate. A student who receives a C in a Foundation
%    course must repeat the course to achieve a B- or better, and cannot
%    use the C grade to meet prerequisite requirements for advanced
%    courses. If the student repeats one of these courses and receives a
%    B- or better, the previous C grade remains on the transcript and can
%    be counted toward the total of unsatisfactory grades that may lead
%    to academic dismissal. Credit will be given only once for any
%    course. (See Policy on Unsatisfactory Academic Performance in the
%    AST Student Handbook.)
%  \item[F (0-59) ‘Failure’]
%    Student has not grasped subject matter; does not understand issues
%    involved; cannot work with relevant literature. (See Policy on
%    Unsatisfactory Academic Performance in the AST Student Handbook.)
%  \item[P ‘Pass’]
%    Credit awarded, but no mark assigned.
%  \item[FP ‘Failure due to Plagiarism’]
%    A student will receive this grade only after proven incident(s) of
%    plagiarism in a course.
%\end{description}
\ProvidesFile{other.tex}[2022/06/08 v2.9.1 -- Course policy]

\section{Other Course Policy}
\label{policy}

Late work will not be accepted, except in genuinely extenuating
circumstances. Students must submit something before the deadline if
they wish to receive credit. Unless I state otherwise, assignments are
to be uploaded by 11:59 \PM\ (Atlantic) on the date indicated.

Essay submissions must be typewritten and double-spaced. They should be
free from error. In this course they should follow SBL Style (see
\cite{sbl2} in \autoref{supplementary}, above). As a reminder, AST
upholds an Inclusive Language Policy. Please use gender-inclusive
language when referring to human beings. Our traditions have different
norms for speech about God; you are of course free to follow and explore
those traditions when referring to God.


Plagiarism is the
\href{http://www.eerdmans.com/Pages/Item/59043/Commentary-Statement.aspx}{failure}
to \href{https://www.theguardian.com/world/2013/feb/09/german-education-minister-quits-phd-plagiarism}{attribute}
(by means of footnotes when writing or aloud when speaking) any ideas,
phrases, sentences, materials, syntheses, et cetera, that another author
has composed and that you have borrowed for your own work. Plagiarism is
unethical. Academic penalties for plagiarism at AST are serious, and may
include failure of the course or even suspension of further studies.
Unintentional plagiarism is considered plagiarism. AST's Plagiarism
Policy is found under that heading in the Academic
Calendar.

Students should request permission to record a class or lecture. If
permission is granted, or if recordings are provided (as in the case of
an online or hybrid course), I stipulate that all recordings be for
personal use only. They may not be shared or distributed.

If you have needs that require modifications to any aspect of this
course, please consult with the instructor as soon as possible. Any
documentation regarding disabilities that you wish to divulge to AST
should be provided to the Registrar’s Office, where it will be kept in a
confidential file.

Finally, I encourage the conscientious use of laptops, tablets, and
other technology in my classes. In classroom settings, realize that, as
\href{http://dx.doi.org/10.1016/j.compedu.2012.10.003}{cognitive
psychologists have demonstrated}, ``laptop multitasking hinders
classroom learning for both users and nearby peers.'' Do your part to
foster an environment for dialogue by honouring the presence of your
classmates. In online and hybrid settings, consider both the physical
environment in which you choose to work and the virtual environment that
you help create through your participation in various forums. Let your
engagement in this course be marked by rigour and charity alike.


%\section{Course Bibliography}
%\label{bibliography}

\end{document}
