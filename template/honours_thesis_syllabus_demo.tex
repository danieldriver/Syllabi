% !TEX program = xelatex
% !TEX encoding = UTF-8 Unicode
% Copyright (c) 2013 by Daniel R. Driver. All rights reserved.

\documentclass[titlepage]{article}
% clone the necessary input files: git@github.com:danieldriver/syllabus.git
\newcommand\path{../../syllabus}
\ProvidesFile{variables.tex}[2018/05/24 v2.1 -- Syllabus variables]

\usepackage{xspace} % make manual spaces (like \mycmd\ ) unnecessary
\usepackage{xifthen} % provides \isempty test

% variables for internal use
\newcommand\prof{}
\newcommand\pdegree{}
\newcommand\pphone{}
\newcommand\pemail{}
\newcommand\poffice{}
\newcommand\phours{}
%
\newcommand\ccode{}
\newcommand\ctitle{}
\newcommand\cseries{}
\newcommand\cversion{}
\newcommand\csemester{}
\newcommand\cmeetson{}
\newcommand\cmeetsat{}
\newcommand\cmeetsin{}
\newcommand\cwebsite{}
\newcommand\cdescrip{}
\newcommand\cprereqs{}
\newcommand\edobject{}

% in case of fully online courses - https://tex.stackexchange.com/a/5896
\newif\ifonline
\newcommand\Int[2]{\ifonline#1\else#2\fi}

% commands for setting variables in the preamble
\newcommand\professor[2][PhD]{
  \renewcommand\pdegree{#1\xspace}
  \renewcommand\prof{#2\xspace}}
\newcommand\phone[1]{
  \renewcommand\pphone{\addfontfeatures{Numbers=Monospaced}#1\xspace}}
\newcommand\email[1]{
  \renewcommand\pemail{\href{mailto:#1}{#1}\xspace}}
\newcommand\officehours[2][Library, Room 5-North]{
  \renewcommand\poffice{#1\xspace}
  \renewcommand\phours{#2\xspace}}
%
\newcommand\coursecode[2][1.0]{
  \renewcommand\cversion{#1\Int{-i}{}\xspace}
  \renewcommand\ccode{#2\Int{(Int)}{}\xspace}}
\newcommand\coursetitle[2][]{
  \ifthenelse{\isempty{#1}}%
    {}% do nothing if #1 is empty, else:
    {\renewcommand\cseries{#1\\[1ex]}}
  \renewcommand\ctitle{#2\xspace}}
\newcommand\semester[1]{
  \renewcommand\csemester{#1\xspace}}
\newcommand\meets[3]{
  \newcommand\AM{\textsc{am}}
  \newcommand\PM{\textsc{pm}}
  \renewcommand\cmeetson{#1\xspace}
  \renewcommand\cmeetsat{\Int{From 9:00 \AM}{#2}\xspace}
  \renewcommand\cmeetsin{\Int{\href{https://smu.brightspace.com/d2l/login}{Brightspace}}{#3}\xspace}}
\newcommand\website[1]{
  \renewcommand\cwebsite{\href{http://#1}{#1}\xspace}}
\newcommand\cdescription[2][RM 1000 or GTRS 6000; and BF 1001]{
  \renewcommand\cprereqs{#1}
  \renewcommand\cdescrip{#2\par}}
\newcommand\objectives[1]{
  \renewcommand\edobject{#1\par}}


\coursecode[1.0 (demo)]{BSTH 4973 \& 4993} % version is 1.0 unless \cmd[1.1]{}
\coursetitle[Biblical Studies and Theology]{Honours Thesis I \& II} % optional series: \cmd[series]{}

\professor{Daniel R. Driver} % terminal degree is PhD unless \cmd[DPhil]{}
\phone{416.226.6620 x2201}
\email{ddriver@tyndale.ca}
\officehours{By appointment}% in Ballyconnor 1088 unless \cmd[moved]{}

\semester{2013--2014}
\meets{Every second Wed.} % \meets{on}{at}{in}
      {11:30 \AM--12:50 \PM}
      {Professor's office}
\website{class.tyndale.ca} % omit http://
\cdescription{% copy from tyndale.ca/registrar/calendar
  Students complete a major research project in Biblical Studies and
  Theology that demonstrates the ability to formulate a research
  question or thesis, use current scholarly methods and theories,
  critically evaluate primary sources and/or research data, and come to
  responsible conclusions. The honours thesis is a six-credit-hour
  course.

  {\itshape Prerequisite: Offered only to students in their final year
  of an honours program in Biblical Studies and Theology. Contract.}
}

\ProvidesFile{preamble.tex}[2013/09/06 v1.0 -- Syllabus preamble]

% basic typography
\usepackage{fontspec}
\setmainfont[Ligatures=TeX]{Meta Serif Pro}
\setsansfont[Ligatures=TeX]{Meta Pro}
\newfontfamily\Heb{Meta Hebrew}
\setmonofont[Scale=MatchLowercase]{Menlo}
\usepackage{sectsty}
\allsectionsfont{\sffamily}
\frenchspacing
\setlength{\emergencystretch}{3em} % prevent overfull lines

% custom font size and leading
\renewcommand\tiny{\fontsize{6}{9}\selectfont}
\renewcommand\scriptsize{\fontsize{7}{10}\selectfont}
\renewcommand\footnotesize{\fontsize{8}{11}\selectfont}
\renewcommand\small{\fontsize{8.5}{11.5}\selectfont}
\renewcommand\normalsize{\fontsize{9}{12}\selectfont}% base size
\renewcommand\large{\fontsize{11}{14}\selectfont}
\renewcommand\Large{\fontsize{13}{16}\selectfont}
\renewcommand\LARGE{\fontsize{16}{19}\selectfont}% "course syllabus \\ semester" benefits from more lead
\renewcommand\huge{\fontsize{19}{21}\selectfont}
\renewcommand\Huge{\fontsize{24}{26}\selectfont}

% layout packages: page, logo, tables
\usepackage[scale={0.6,0.8},
            xetex]{geometry}
\usepackage{graphicx}
\usepackage{array}     % allow insertions of column styling with >{}
\usepackage{booktabs}  % elegant horizontal rules in tables
\usepackage{marginfix} % protect positioning of margin table in policy/grades

% custom macros for a session count in the schedule of readings
\newcounter{session}
\newcounter{columns}
\newcounter{courseunit}
\newcommand\setcolumncount[2][0]{ % optionally set count to other than 0,
  \setcounter{session}{#1}        % e.g. to -1, or to a standing count
  \setcounter{columns}{#2}}
\newcommand\sessioncount{\stepcounter{session}\arabic{session}}
\newcommand\sessionskip[1]{\multicolumn{1}{@{}r@{ }}{#1}}
\newcommand\unit[1]{\multicolumn{\thecolumns}{c}{%
  \scshape\stepcounter{courseunit}\roman{courseunit}. \MakeLowercase{#1}}}
\newcommand\noclass[1]{\multicolumn{1}{@{}l}{\itshape No Class: #1}}

% color to match Tyndale's branding
\usepackage[usenames]{xcolor}
% predefined: black, white, red, green, blue, cyan, magenta, yellow
\definecolor{TyndaleURLs}{HTML}{0062A0} % links on tyndale.ca
\definecolor{TyndaleBlue}{cmyk}{1,1,0,.32}
\definecolor{TyndaleGold}{cmyk}{0,.27,1,0}
\definecolor{TyndaleRed}{cmyk}{0,1,.99,.04}
\definecolor{TyndaleBlack}{cmyk}{0,0,0,1}
\definecolor{TyndaleGreen}{cmyk}{.45,0,1,.24}
\definecolor{TyndaleOrange}{cmyk}{0,.79,1,0}
\definecolor{TyndaleAqua}{cmyk}{.47,0,.24,0}
\definecolor{TyndaleYellow}{cmyk}{.03,.03,.35,0}

% metadata (assumes a host of definitions are made in the main file)
\usepackage[setpagesize=false,     % leave this to geometry
            hyperfootnotes=false,  % fragile and distracting
            xetex]{hyperref}
\hypersetup{breaklinks=true,       % allow link text to break across lines
            colorlinks=true,       % colorlinks resets pdfborder to 0 0 0
            urlcolor=TyndaleURLs,  % for external links
            linkcolor=TyndaleRed,  % for normal internal links
            citecolor=TyndaleGold, % for bibliographical citations in text
            pdfauthor={\prof},
            pdftitle={\ccode: \ctitle},
            pdfsubject={Tyndale UC, \csemester},
            pdfcreator={github.com/danieldriver/syllabus}}
\urlstyle{same}                    % don't use monospace font for urls

% custom footlines
\usepackage{fancyhdr}
\pagestyle{fancy} % turn it on
\fancyhf{}        % reset everything
\renewcommand{\headrulewidth}{0pt} % remove header line as well
\lfoot{\sffamily\scshape\footnotesize\MakeLowercase{\ctitle, v\cversion}}
\rfoot{\sffamily\scshape\footnotesize\MakeLowercase{\prof\quad\thepage}}

% gratuitous with custom title page, but useful as a fallback
\title{\ccode: \ctitle}
\author{\professor}
\date{\semester}


\begin{document}
\ProvidesFile{title.tex}[2013/09/06 v1.0 -- Syllabus title page]

\begin{titlepage}
  \begin{center}

    \LARGE\sffamily % set title elements in a large sans serif

    \begin{minipage}{\textwidth}
      \parbox[t]{0.5\textwidth}{
        \mbox{}\\[-13pt] % dummy line to align parboxes
        \includegraphics[width=0.5\textwidth]{.syllabus/includes/TyndaleUC}}
      \hfill
      \parbox[t]{0.4\textwidth}{
        \raggedleft Course Syllabus\\
        \csemester}
    \end{minipage}

    \vfill

    {\textsc{\MakeLowercase\ccode}\\[1ex]
      \bfseries\cseries\Huge\ctitle}

    \vfill

    \normalsize\rmfamily % switch back to body type

    \begin{tabular}{>{\bfseries}rl>{\bfseries}rl}
      \toprule
      Instructor & \prof, \pdegree & Course  & Version \cversion \\
      \midrule
      Phone      & \pphone         & Meets   & \cmeetson         \\
      Email      & \pemail         & Time    & \cmeetsat         \\
      Office     & \poffice        & Room    & \cmeetsin         \\
      Hours      & \phours         & Website & \cwebsite         \\
      \bottomrule
    \end{tabular}

    \vfill

    \begin{description}\small
      \item[Commuter Hotline]
        Class cancellations due to inclement weather or illness will
        be announced on the commuter hotline at \texttt{416.226.6620
        x2187}. Alternately, weather cancellation information is posted
        at \href{http://tyndale.ca/weather}{tyndale.ca/weather}.
      \item[MyTyndale.ca]
        This course may have materials stored on its website, such as
        handouts or readings that may be needed in order to complete
        assignments. Students are responsible for checking these course
        pages on a regular basis. Here, too, students are able to view
        their grades throughout the semester. For more information see
        Section~\ref{mytyndale}, below.
      \item[Mail]
        Students are responsible for information communicated through
        their campus mailboxes and student e-mail accounts. A mailbox
        directory hangs beside the mailboxes. For more information
        contact the Registrar's office.
    \end{description}

  \end{center}

  \section{Course Description}
  \label{description}

  \emph{From the Academic Calendar:} \cdescrip

\end{titlepage}
\setcounter{page}{2} % count the title page as page 1


\section{Learning Outcomes}
\label{outcomes}

% A Model of Learning Objectives
% Adapted from http://www.celt.iastate.edu/teaching/RevisedBlooms1.html
%
% A learning objective should contain a *subject* (a verb related to
% cognitive process) and an *object* (a noun identifying the knowledge
% or skill sought). Statements should be precise and measurable, e.g.,
% "By the end of this course, students will be able to <verb> <noun>."
%
% Anderson and Krathwohl name four *classes of knowledge* that students
% can be expected to acquire. Terms move from the concrete to the
% abstract:
%
%   1. factual           (basic elements, details, terminology)
%   2. conceptual        (interrelationship of basic elements)
%   3. procedural        (skills, methods, techniques, criteria)
%   4. metacognitive     (strategy, context and conditions, self-knowledge)
%
% Metacognitive knowledge is a special case. It concerns one's own
% "cognition and about oneself in relation to various subject matters"
% (Anderson and Krathwohl, 2001, p.44).
%
% A&K also identify 19 *cognitive processes*, arranged in six categories
% (a 2001 revision of Bloom's 1957 taxonomy) that progress from lower
% order to higher order thinking skills (from LOTS to HOTS):
%
%   1. remember
%      * recognizing     (identifying)
%      * recalling       (retrieving)
%   2. understand
%      * interpreting    (clarifying, paraphrasing, representing, translating)
%      * exemplifying    (illustrating, instantiating)
%      * classifying     (categorizing, subsuming)
%      * summarizing     (abstracting, generalizing)
%      * inferring       (concluding, extrapolating, interpolating, predicting)
%      * comparing       (contrasting, mapping, matching)
%      * explaining      (constructing models)
%   3. apply
%      * executing       (carrying out)
%      * implementing    (using)
%   4. analyze
%      * differentiating (discriminating, distinguishing, focusing, selecting)
%      * organizing      (finding coherence, integrating, outlining, parsing, structuring)
%      * attributing     (deconstructing)
%   5. evaluate
%      * checking        (coordinating, detecting, monitoring, testing)
%      * critiquing      (judging)
%   6. create
%      * generating      (hypothesizing)
%      * planning        (designing)
%      * producing       (constructing)
%
% If you reference a List of Measurable Verbs Used to Assess Learning
% Outcomes, remember that A&K reverse Bloom's 6th and 5th categories.

The learning outcomes for this course are to be drawn up by the student,
in consultation with the professor, in the first month of the fall
semester. Refer to the following Model of Learning Objectives, to be
introduced by the professor at the first meeting:
\url{http://www.celt.iastate.edu/teaching/RevisedBlooms1.html}

In addition to these more general aims, the student needs to frame a
research question, and to outline the method by which it will be
addressed. This should be done early in the first semester, and should
be reviewed and revised at the end of the first semester.

\section{Texts \& Materials}
\label{texts}

The student is responsible for procuring the necessary materials. Make
thorough use of Tyndale's library resources, including its Interlibrary
Loan Form: \url{http://www.tyndale.ca/library/loan}

By the midterm of the first semester the student needs to draw up, and
to present to the professor for approval, a comprehensive bibliography
for the chosen subject.

\section{Supplementary Texts}
\label{supplementary}

If desired, the student or the professor may recommend a book to read
through together. Ideally, this could be done in the semester or summer
before work on the Honours Thesis begins.

\section{Evaluation}
\label{evaluation}

%\subsection{Grade Structure for \ccode}
%\label{structure}

% Please consider the following structure in your course. Evaluation
% should be clearly linked to Learning Outcomes.
%   - Some form of meaningful evaluation should occur during the first 4
%     weeks of the course; **all students should (and first-year
%     students must) receive an indication of their performance in the
%     first half of the course.**
%   - **Plan your assignments so that each can be returned to students
%     before the next is due.**
%   - Most courses should have a rigorous final exam of 2--3 hours
%     duration. Some upper-year courses may not require a final,
%     depending on pedagogical design and assessment strategy. **Final
%     exams should be appropriately demanding in both breadth and depth
%     to assess accurately the achievement of learning outcomes.**
%   - **No assignments should be due in the last week of classes; final
%     examinations will be scheduled only by the Registrar’s Office.**

\begin{enumerate}
 \item Within the first month the student is to draft and submit
   projected \textbf{learning outcomes}. The professor may amend, revise
   or otherwise contribute to this statement.
 \item Also within the first month the student needs to draw up a clear
   \textbf{statement of the research question}. This must include:
\begin{itemize}
  \item a 150 word abstract summarising the research for a general audience;
  \item a draft statement of research, not longer than two pages, that indicates how the question will be addressed;
  \item and a revision of both of the above.
 \end{itemize}
 The first draft is due at midterm, the final draft at the end of the first semester.
 \item Early in the first semester, no later than week six, the student needs to compile a comprehensive bibliography. This shall be submitted to the professor for comment and approval.
 \item By finals week the student needs to submit an outline of the entire work and a rough draft of at least one chapter of the thesis.
\end{enumerate}

In the event that the student does not complete the examination, the breakdown for the first semester's work will be as follows:

\begin{table}[htbp]
  \centering
  \begin{tabular}{lr}
    \toprule
    Learning Outcomes & 10\% \\
    Statement of Research, First Draft & 20\% \\
    Statement of Research, Revised Draft & 20\% \\
    Comprehensive Bibliography & 25\% \\
    Outline and Draft Chapter & 25\% \\
    \bottomrule
  \end{tabular}
  \caption{Distribution of Grades}
  \label{distribution}
\end{table}
%
%\ProvidesFile{grades.tex}[2013/08/19 v1.0 -- Course policy]

\subsection{Grading System at Tyndale}
\label{grades}

Tyndale University College provides these benchmarks for summative assessment.
I may furnish more detailed rubrics for particular assignments.

\mparshift{0.5\baselineskip}
\marginpar{
\footnotesize
\begin{tabular}[t]{@{}>{\scshape}c@{\hspace{1em}}c@{\hspace{1em}}c@{}}
  \toprule
  +           & 90--100 & 4.00 \\
  a           & 85--89  & 4.00 \\
  \char"2212  & 80--84  & 3.70 \\ [1ex]
  +           & 77--79  & 3.30 \\
  b           & 73--76  & 3.00 \\
  \char"2212  & 70--72  & 2.70 \\ [1ex]
  +           & 67--69  & 2.30 \\
  c           & 63--66  & 2.00 \\
  \char"2212  & 60--62  & 1.70 \\ [1ex]
  +           & 57--59  & 1.30 \\
  d           & 53--56  & 1.00 \\
  \char"2212  & 50--52  & 0.70 \\ [1ex]
  f           & 0--49   & 0.00 \\
  \bottomrule
\end{tabular}}

\begin{description}
  \item[A, B -- Excellent, Good]
    These grades are earned only when evidence indicates that the student
    has consistently maintained above average progress in the subject.
    Sufficient evidence may involve such qualities as creativity, originality,
    thoroughness, responsibility and consistency.
  \item[C -- Satisfactory]
    This grade means that the student has fulfilled the requirements of the
    subject to the satisfaction of the instructor. These requirements include
    the understanding of subject matter, adequacy and promptness in the
    preparation of assignments and participation in the work of the class.
  \item[D -- Poor]
    This grade indicates that the accuracy and content of work submitted meets
    only the minimal standards of the instructor. Consistent performance at
    this level is considered inadequate for graduation.
  \item[F -- Failing]
    Work submitted is inadequate. Attitude, performance and attendance are
    considered insufficient for a passing grade.
\end{description}

%
%\section{Policy on Assignments \& Exams}
%\label{policy}
%
%\ProvidesFile{academic_calendar.tex}[2013/08/19 v1.0 -- Course policy]

\newcommand{\AC}{Academic Calendar}
\newcommand{\SecAC}{Section~5 of the \AC}

All policy in Sections~\ref{policy}, \ref{expectations} and \ref{support} of
this syllabus applies to this course in addition to policy in the current
\href{http://www.tyndale.ca/registrar/calendar}{\AC}. In some cases the
syllabus underscores the general policy, while in other cases it supersedes it.

For all matters not covered in this syllabus, refer to \SecAC, ``University
College Academic Policies, Procedures, and Notices.'' Students are strongly
encouraged to read this document carefully at least once in their career at
Tyndale, and to review it every year they matriculate.

%\ProvidesFile{assignments.tex}[2015/07/28 v1.1.1 -- Course policy]

\subsection{Assignments}
\label{assignments}

This is a university course. All papers and other writing assignments
should therefore be written at the university level. Submissions must be
typewritten and double-spaced, should be free from error, and in this course
should follow the \emph{SBL Handbook of Style} (refer to the free, online
\href{https://www.sbl-site.org/wp-content/uploads/2025/04/SBLHSsupp2015-02.pdf}{SBLHS
Student Supplement}.)

If you ever struggle with composition---anything from the relatively simple
matters of spelling, grammar and proper citation to deeper-level issues of
tone, structure and argument---then please make use of the Writing Centre (see
Section~\ref{centre}). Experienced writers know that drafts and peer
feedback are integral to the writing process. Inexperienced writers are often
unaware that their surface-level errors create credibility problems with their
readers. When you \href{http://theoatmeal.com/comics/misspelling}{misspell
common words}, fail to know \href{http://theoatmeal.com/comics/apostrophe}{how
to use an apostrophe}, or do not bother to cite your sources correctly, why
should your readers trust you with more important matters like the facts and
ideas under discussion?

\subsubsection{Deadlines}
\label{deadlines}

Assignments \emph{must} be submitted on time. Even if the work is rough or
incomplete, you must turn in something by the due date to receive any credit
whatsoever. Unless I specify differently in class, papers and take-home exams
are due by 11:59 \textsc{pm} on the due date. All other work is due at the
start of the day's class.

Note that, because no late work is accepted in this class, there is no scale
of penalty for unexcused late assignments. If a truly extraordinary event
keeps you from doing your best work, then let me know so that we can make
special arrangements. I am guided by the \AC\ in what counts as extenuation.
``Extensions are not granted for what best could be described as `poor time
management' or `over-involvement' in an extracurricular activity.''

\subsubsection{Submission as PDFs}
\label{submission}

Papers and some other assignments in this course are to be submitted
electronically through the course pages (Section~\ref{lms}). To preserve
formatting, formal writing assignments must be uploaded in Portable Document
Format. There are many ways of creating PDFs; it is your responsibility to
know how to do so on the computer platform you use, and to generate and submit
your PDFs on time.

\subsubsection{Backup}
\label{backup}

In the event of the loss of assignments post-submission---electronic systems
fail, and my office has flooded before---students are required to keep backup
copies of all assignments submitted.

Learning how to secure and preserve your work is a peculiar challenge of the
digital age. Plan on the crash of your hard drive, and the theft of your
laptop (the first is inevitable, the second quite probable). If you do not
have a backup strategy, I recommend that you start with a free account on
\href{http://db.tt/U7eP1vs}{dropbox.com}.

%\ProvidesFile{exams.tex}[2013/08/19 v1.0 -- Course policy]

\subsection{Examinations}
\label{exams}

My examination policy follows that outlined in \SecAC, part of which is
summarized below for emphasis.

\begin{enumerate}
  \item
    Midterm exams will be held as scheduled by the instructor. If you miss
    the exam for a legitimate reason, you must write the exam within the same
    number of days that you were absent from school (possibly the next day).
  \item
    Final examinations will take place during the exam period as scheduled
    by the Registrar. Students are responsible for noting the date, time and
    location of their final exam in this class. Students are also responsible
    for familiarizing themselves with the Registrar's examination policies.
  \item
    Special rules apply when midterms and finals are held, including one that
    prohibits students from leaving their seats during the final fifteen minutes
    of the exam period. See the \AC\ for full details.
%    The following rules apply to every final examination:
%    \begin{enumerate}
%      \item
%        No student is permitted to take into the examination room any materials
%        relating to the examination subject, including Bibles.
%      \item
%        No student may leave the room without permission from the proctor.
%      \item
%        No student may leave his or her seat during the final fifteen minutes.
%      \item
%        Students must not linger in the halls outside the examination rooms
%        while examinations are being written.
%      \item
%        No student will be permitted to write beyond the allotted time without
%        special permission of the Registrar (see Section~\ref{accessibility}).
%    \end{enumerate}
  \item
    Provisions exist for students who are justifiably unable to write the final
    exam at the scheduled time. See the \AC\ for details, and make arrangements
    through the Office of the Registrar.
  \item
    Normally, a final exam can only be reschedule in two circumstances:
    (a) a documented illness, or (b) a conflict with another
    exam (two at the same time, or three within 24 hours).
    \href{http://www.tyndale.ca/registrar/final-exam-schedule-and-policies}
    {Apply to the Registrar} in either case.
\end{enumerate}

%
%\section{Student Expectations \& Guidelines}
%\label{expectations}
%
%\ProvidesFile{academic_integrity.tex}[2013/08/19 v1.0 -- Course policy]

\subsection{Academic Integrity}
\label{integrity}

Integrity in academic work is required of all students. Academic dishonesty
is any breach of this integrity. It includes such practices as cheating (the
use of unauthorized material on tests and examinations), submitting the same
work for different classes without permission of the instructors, using false
information in an assignment (including false references to secondary sources),
improper or unacknowledged collaboration with other students, and plagiarism.

Tyndale takes seriously its responsibility to uphold academic integrity,
and to apply consequences for academic dishonesty. Consult \SecAC\ for more
information on the school's policy and its application to your work in this
course.

%\ProvidesFile{attendance.tex}[2013/08/22 v1.0 -- Course policy]

\subsection{Attendance}
\label{attendance}

``Faithful attendance at classes is an important indicator of student maturity
and involvement'' (\AC). Remember, too, that you are responsible for everything
that happens in every class. Your best policy is to attend and engage. Please
do not ask me to repeat for your benefit anything I have said in a class you
have missed.

Keeping a record of attendance is mandatory for faculty at Tyndale (in contrast
to many other colleges and universities). The University College publishes
guidelines for how attendance should bear on your final evaluation in a course,
and I adhere to them. Note that four lates equals one absence.

What should you do if you miss an undue number of classes? First, arrange for
a classmate to brief you on the material missed, or get my permission for a
classmate to make a recording for you (see Section~\ref{recording}). Second,
notify the Dean of Students in person or by phone. If illness is the cause you
will need to submit a doctor's certificate upon return. The Dean of Students
will notify your professors of the reason for the absence and suggest that they
take this into consideration when assigning grades.

%\ProvidesFile{technology.tex}[2014/09/02 v1.1 -- Course policy]

\subsection{Technology}
\label{technology}

Technological innovation has brought students and educators a number of
powerful new tools, and I encourage you to use them as you research, write, and
collaborate. Some of these tools also call for disciplined use and management.

\subsubsection{Email}
\label{email}

Email can be a chore, and you may prefer other channels of communication. As a
matter of policy, however, students must use their myTyndale accounts for all
course-related email correspondence. During term time you should check your
school account at least once a day (optional on weekends). I myself aim to
check my school email at the beginning and end of each workday. At other times
my email client is often closed. I will try to answer your messages within 24
hours, though you should not expect replies on weekends.

Keep your messages topical and brief. I would vastly prefer to conduct any
conversations of substance in person, or else over the phone. Please note and
make use of my office hours. If these hours do not suit your schedule, I would
happily receive an email from you requesting an alternate meeting time.

% Note the prof who adopted an anti-email policy! https://www.insidehighered.com/news/2014/08/27/sake-student-faculty-interaction-professor-bans-student-email?utm_source=slate&utm_medium=referral&utm_term=partner

\subsubsection{MyTyndale.ca / classes.tyndale.ca}
\label{mytyndale}

Tyndale's course pages are an efficient means of distributing articles, notes,
slides, and other course-related materials. This is also where instructors log
attendance and upload grades for assignments. Students are therefore required to
check the site for updates about their classes as well as for any materials
needed for lectures and assignments.

My own use of this platform varies from semester to semester, and from
course to course. At times I may ask you to use the forums, quiz module, or
other parts of the system. At a minimum I will use the site as a repository
for course materials, and as a destination for your submission of PDFs
(Section~\ref{submission}).

\subsubsection{Laptops and Other Devices}
\label{laptops}

Use of laptops is forbidden in my classroom, except to facilitate presentation.
I implement this policy because of the cognitive costs of multitasking, with the
aim of giving you and your peers the best chance of success. I also hope to
foster a culture of keen intellectual engagement.

As \href{http://dx.doi.org/10.1016/j.compedu.2012.10.003}{cognitive
psychologists at McMaster and York Universities demonstrated in 2013}, ``laptop
multitasking hinders classroom learning for both users and nearby peers.'' There
is little new in their finding that the allegedly multitasking student does less
well in class (11\% worse on the quiz in their experiment). This effect has been
shown many times. Rather, their novel discovery is that classmates
\emph{without} laptops who sat with a \emph{view} of another student's screen
did worse than the students who had a computer (17\% worse than those with no
laptop in sight).

Prohibiting laptops is not the only possible response to these findings.
However, there is evidence that \href{http://on.wsj.com/pjtJaK}{writing by hand}
brings a number of cognitive benefits, and a
\href{http://dx.doi.org/10.1177/0956797614524581}{2014 Princeton University
study} ``found that students who took notes on laptops performed worse on
conceptual questions than students who took notes longhand.'' If you are a heavy
laptop user then consider this an opportunity to experiment with different
technologies in the classroom.

As for the myriad networked devices that many of us carry, it's a simple matter
of professionalism to keep these things silent and out of sight. E-readers and
tablets are permitted \emph{only if they are used to display the assigned
reading}. If this is how you choose to read, let me invite you to put the
machine in airplane mode while class is in session.

\subsubsection{Recording of Classes}
\label{recording}

Students must request permission from the professor of any class that they
would like to record. Where permission is granted, students are expected to
supply their own equipment. In general I prefer \emph{not} to have my classes
recorded, and I am not at all friendly to being recorded without my knowledge.
In cases where I grant permission, I stipulate that the recordings must be for
personal use only. They should not be shared with other students, even with
students in the same section, and they absolutely must not be posted online or
otherwise distributed.

If a student is not able to attend a lecture and would like to have it recorded,
it is the responsibility of the student first to obtain the professor's
permission, and then to find another student to record the lecture. I will not
make the recording for you.
%
%\section{Student Support}
%\label{support}
%
%\ProvidesFile{writing_centre.tex}[2013/08/19 v1.0 -- Course policy]

\subsection{Tyndale Writing Centre}
\label{writing_centre}

Through a combination of tutorials, workshops and resources, Tyndale's Writing
Centre offers a comprehensive program of writing support to Tyndale students,
including individual 30-minute tutoring sessions, workshops, handbooks and
other writing resources. Students may bring essays that have been graded (and,
at least for my classes, work that is in process but has not yet been submitted
for a grade) and will receive detailed suggestions for improving their writing.
This service, at no charge to students, is available by appointment.

Professors may recommend that a student go to the Writing Centre for help:
students are strongly encouraged to follow such recommendations. The Academic
Standards Committee may require an undergraduate student who is experiencing
difficulty in his or her academic program to go to the Writing Centre for
assistance and support.

As someone who cares a great deal about writing, and was student director
of his college Writing Center (\emph{sic}), I must emphasize that getting
competent feedback on writing is not just for marginal students. Outstanding
writers learn to crave good feedback. Think you don't need the Writing Centre?
Make two appointments this semester. See if the service it provides might not
contribute to your education here after all.

%\ProvidesFile{accommodation.tex}[2013/08/19 v1.0 -- Course policy]

\subsection{Accommodation}
\label{accommodation}

Students with documented disabilities may be granted special accommodation for
exams, and in some cases for other assignments. It is even possible to get
permission to use a laptop in class (Section~\ref{laptops}), although I will
need to be convinced of the use case. It is up to the student to contact the
Dean of Students as early as possible in the semester---not later than the
second week---and to document the need. The Dean of Students will then advise
each of the student's professors of the accommodations that may be required.
Please note that special arrangements for assignments need to be made with
me well in advance of assignment due dates (Section~\ref{deadlines}). Timely
requests shall not unreasonably be denied.

%
%\section{Course Outline}
%\label{outline}
%
%Readings of a given day should be completed before the start of class
%on the same date. Primary readings are compulsory. Any supplementary
%readings that I circulate (see Section~\ref{supplementary}) will
%be keyed to the session number (listed under \textbf{\S} in
%Table~\ref{schedule}), and should ideally be read before class as well.
%We will adhere to the schedule as closely as possible, though I reserve
%the right to adjust it to suit the needs of the class.
%
%\newcommand\Yhwh{\textsc{Yhwh}}
%\newcommand\rarr{\char"2192\hspace*{0.5pt}}
%
%\setcolumncount{4} % set up \sessioncount, \unit{} and \noclass{} macros
%\begin{table}[phtb]
%  \centering
%  \begin{tabular}{>{\sessioncount.}r@{ }llr}
%    \toprule
%    \sessionskip{\textbf{\S}.}&\textbf{Lecture Title}&\textbf{Primary Reading}&\textbf{Date} \\
%    \midrule
%    \unit{Beginnings}        \\
%          & When God Created                          & Psalms 8, 74, 104         &  9 Sept. \\
%          & Eden, a Garden in the East                & Genesis 1--3              & 12 Sept. \\
%          & \liningnums{3\,×\,10} Generations         & Genesis 4--11             & 16 Sept. \\
%          & Father Abraham and King Abimelech         & Genesis 12--36            & 19 Sept. \\
%          & Joseph and His Brothers                   & Genesis 37--50            & 23 Sept. \\
%          & A Wise Son; Dame Wisdom                   & Proverbs 1--9             & 26 Sept. \\ [1ex]
%    \unit{Egypt \& Sinai}                                                                    \\
%          & The Call of Moses and the Divine Name     & Exodus 1--6               & 30 Sept. \\
%          & Who is \Yhwh? I Do Not Know \Yhwh         & Exodus 7--15, 34--36      &  3 Oct.  \\
%          & Ten Cuuommandments and a Goring Ox          & Exodus 19--23             &  7 Oct.  \\
%          & I Delight in Your Instruction (Hear,      & Deuteronomy 1--11,        & 10 Oct.  \\
%    \sessionskip{} & O Israel! Remember and Keep)     & \multicolumn{2}{l}{\emph{The Sabbath} \cite{heschel}} \\
%    \sessionskip{---.}  & \noclass{Thanksgiving}      & \emph{catch up}           & 14 Oct.  \\
%          & \emph{Deuteros Nomos}, a Second Law       & Deuteronomy 12--26        & 17 Oct.  \\
%          & Moses a Prophet? The Law and the Prophets & Deuteronomy 27--34        & 21 Oct.  \\
%    \sessionskip{\rarr} & \textbf{Midterm Exam}       & \emph{review}             & 24 Oct.  \\ [1ex]
%    \unit{Prophets \& Kings}                                                                 \\
%          & Joshua: \emph{Veni, Vidi, Vici?}          & Joshua 1--13:7            & 28 Oct.  \\
%          & Of Judges and Philistines                 & Judges 1--9, 13--21       & 31 Oct.  \\
%          & Judge, Priest, Prophet, Kingmaker         & 1 Samuel 1--31            &  4 Nov.  \\
%          & David's House in Weal and Woe             & 2 Samuel 1--24            &  7 Nov.  \\
%          & A Lofty House and a House Divided         & 1 Kings 1--14             & 11 Nov.  \\
%          & Assyrian Crisis and Exile to Babylon      & 2 Kings 1--8, 17--25      & 14 Nov.  \\
%          & Twelve Minor Prophets                     & Hosea 1--Amos 9           & 18 Nov.  \\
%          & Isaiah of Jerusalem                       & Isaiah 1--12, 36--39      & 21 Nov.  \\
%          & Zion's Final Destiny                      & Isaiah 56--66             & 25 Nov.  \\ [1ex]
%    \unit{Psalms}                                                                            \\
%          & The Shaping of the Psalter                & Psalms 1, 72, 89, 90, 145 & 28 Nov.  \\
%          & Sinai and Zion                            & Psalms 2, 15, 24, 84, 110 &  2 Dec.  \\
%    \bottomrule
%  \end{tabular}
%  \caption{Schedule of Readings}
%  \label{schedule}
%\end{table}
%
%See the \href{http://www.tyndale.ca/registrar/important-dates}{%
%Registrar's website} for a list of other important dates. Generally, the
%last day to add or drop a class without penalty is the end of the second
%week of class.
%
%\section{Course Bibliography}
%\label{bibliography}
%
%An introductory bibliography is provided at the end of each page
%of notes in the course pack. Pursue this material as your interest
%dictates, or if you wish to check my handling of any of the material
%I present. Be sure to pay special attention to items flagged as
%supplementary reading (Section~\ref{supplementary}).

\end{document}
