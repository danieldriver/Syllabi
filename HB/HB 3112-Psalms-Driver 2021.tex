% Copyright (c) 2021 by Daniel R. Driver.
% !TEX encoding = UTF-8 Unicode
% !TEX TS-program = XeLaTeX

\documentclass[titlepage]{article}

% This document presumes a file structure and set of inputs that are
% available at: git@github.com:danieldriver/syllabi.git

\newcommand\policy{../policy}
\newcommand\incl{../includes}
\ProvidesFile{variables.tex}[2018/05/24 v2.1 -- Syllabus variables]

\usepackage{xspace} % make manual spaces (like \mycmd\ ) unnecessary
\usepackage{xifthen} % provides \isempty test

% variables for internal use
\newcommand\prof{}
\newcommand\pdegree{}
\newcommand\pphone{}
\newcommand\pemail{}
\newcommand\poffice{}
\newcommand\phours{}
%
\newcommand\ccode{}
\newcommand\ctitle{}
\newcommand\cseries{}
\newcommand\cversion{}
\newcommand\csemester{}
\newcommand\cmeetson{}
\newcommand\cmeetsat{}
\newcommand\cmeetsin{}
\newcommand\cwebsite{}
\newcommand\cdescrip{}
\newcommand\cprereqs{}
\newcommand\edobject{}

% in case of fully online courses - https://tex.stackexchange.com/a/5896
\newif\ifonline
\newcommand\Int[2]{\ifonline#1\else#2\fi}

% commands for setting variables in the preamble
\newcommand\professor[2][PhD]{
  \renewcommand\pdegree{#1\xspace}
  \renewcommand\prof{#2\xspace}}
\newcommand\phone[1]{
  \renewcommand\pphone{\addfontfeatures{Numbers=Monospaced}#1\xspace}}
\newcommand\email[1]{
  \renewcommand\pemail{\href{mailto:#1}{#1}\xspace}}
\newcommand\officehours[2][Library, Room 5-North]{
  \renewcommand\poffice{#1\xspace}
  \renewcommand\phours{#2\xspace}}
%
\newcommand\coursecode[2][1.0]{
  \renewcommand\cversion{#1\Int{-i}{}\xspace}
  \renewcommand\ccode{#2\Int{(Int)}{}\xspace}}
\newcommand\coursetitle[2][]{
  \ifthenelse{\isempty{#1}}%
    {}% do nothing if #1 is empty, else:
    {\renewcommand\cseries{#1\\[1ex]}}
  \renewcommand\ctitle{#2\xspace}}
\newcommand\semester[1]{
  \renewcommand\csemester{#1\xspace}}
\newcommand\meets[3]{
  \newcommand\AM{\textsc{am}}
  \newcommand\PM{\textsc{pm}}
  \renewcommand\cmeetson{#1\xspace}
  \renewcommand\cmeetsat{\Int{From 9:00 \AM}{#2}\xspace}
  \renewcommand\cmeetsin{\Int{\href{https://smu.brightspace.com/d2l/login}{Brightspace}}{#3}\xspace}}
\newcommand\website[1]{
  \renewcommand\cwebsite{\href{http://#1}{#1}\xspace}}
\newcommand\cdescription[2][RM 1000 or GTRS 6000; and BF 1001]{
  \renewcommand\cprereqs{#1}
  \renewcommand\cdescrip{#2\par}}
\newcommand\objectives[1]{
  \renewcommand\edobject{#1\par}}


\coursecode[3.2.0]{HB 3112}
\coursetitle[Hebrew Bible]{Psalms in Interpretation}

% Taught as RLGS 3263 (T&I: Psalms in the Christian Tradition) in:
%   - Fall 2008
%   - Winter 2011
% Taught as BSTH 3142 (T&I: Psalms in Interpretation) in:
%   - Fall 2014
% Taught as HB 3112 in:
%   - Winter 2017
%   - Winter 2021

\professor{Daniel R. Driver}
\phone{902-425-7051}
\email{ddriver@astheology.ns.ca}
\officehours{\href{https://calendly.com/danieldriver}{calendly.com/danieldriver}}

\semester{Winter Term 2021}
\meets{Tuesdays}% \meets{on}{at}{in}
      {10:00--11:30 \AM}
      {Microsoft Teams}
\website{danieldriver.com}
\cdescription{% copy from the current Academic Calendar
	The Psalms have a special place in the life of the synagogue and
	church. Thus a major goal of this course is to see how the Psalter
	has fed and informed Jewish and Christian theology in various periods:
	biblical times, Rabbinic and Patristic periods, the Middle Ages, the
	Reformation, the Enlightenment, and so on down to modern times. To
	help understand the anthology’s unique arrangement, the course will
	also interact with current research into the shape and shaping of
	the Psalter.
}% end of course description
\objectives{% recall Bloom's taxonomy: http://www.celt.iastate.edu/teaching/effective-teaching-practices/revised-blooms-taxonomy
	By the end of the course students should be able to:
		report on at least three distinct periods in the reception of the Psalter;
		analyze key differences between ancient and modern interpreters, and a variety of Jewish and Christian interpreters;
		formally interpret (exegete) one psalm afresh, informed by its commentary tradition;
		memorize and recite twelve verses from the Psalter;
		respond creatively to a psalm so as to illuminate the text in a mode other than academic discourse, e.g. through music, visual arts, poetry, or performance.
}% end of learning objectives

\ProvidesFile{preamble.tex}[2013/09/06 v1.0 -- Syllabus preamble]

% basic typography
\usepackage{fontspec}
\setmainfont[Ligatures=TeX]{Meta Serif Pro}
\setsansfont[Ligatures=TeX]{Meta Pro}
\newfontfamily\Heb{Meta Hebrew}
\setmonofont[Scale=MatchLowercase]{Menlo}
\usepackage{sectsty}
\allsectionsfont{\sffamily}
\frenchspacing
\setlength{\emergencystretch}{3em} % prevent overfull lines

% custom font size and leading
\renewcommand\tiny{\fontsize{6}{9}\selectfont}
\renewcommand\scriptsize{\fontsize{7}{10}\selectfont}
\renewcommand\footnotesize{\fontsize{8}{11}\selectfont}
\renewcommand\small{\fontsize{8.5}{11.5}\selectfont}
\renewcommand\normalsize{\fontsize{9}{12}\selectfont}% base size
\renewcommand\large{\fontsize{11}{14}\selectfont}
\renewcommand\Large{\fontsize{13}{16}\selectfont}
\renewcommand\LARGE{\fontsize{16}{19}\selectfont}% "course syllabus \\ semester" benefits from more lead
\renewcommand\huge{\fontsize{19}{21}\selectfont}
\renewcommand\Huge{\fontsize{24}{26}\selectfont}

% layout packages: page, logo, tables
\usepackage[scale={0.6,0.8},
            xetex]{geometry}
\usepackage{graphicx}
\usepackage{array}     % allow insertions of column styling with >{}
\usepackage{booktabs}  % elegant horizontal rules in tables
\usepackage{marginfix} % protect positioning of margin table in policy/grades

% custom macros for a session count in the schedule of readings
\newcounter{session}
\newcounter{columns}
\newcounter{courseunit}
\newcommand\setcolumncount[2][0]{ % optionally set count to other than 0,
  \setcounter{session}{#1}        % e.g. to -1, or to a standing count
  \setcounter{columns}{#2}}
\newcommand\sessioncount{\stepcounter{session}\arabic{session}}
\newcommand\sessionskip[1]{\multicolumn{1}{@{}r@{ }}{#1}}
\newcommand\unit[1]{\multicolumn{\thecolumns}{c}{%
  \scshape\stepcounter{courseunit}\roman{courseunit}. \MakeLowercase{#1}}}
\newcommand\noclass[1]{\multicolumn{1}{@{}l}{\itshape No Class: #1}}

% color to match Tyndale's branding
\usepackage[usenames]{xcolor}
% predefined: black, white, red, green, blue, cyan, magenta, yellow
\definecolor{TyndaleURLs}{HTML}{0062A0} % links on tyndale.ca
\definecolor{TyndaleBlue}{cmyk}{1,1,0,.32}
\definecolor{TyndaleGold}{cmyk}{0,.27,1,0}
\definecolor{TyndaleRed}{cmyk}{0,1,.99,.04}
\definecolor{TyndaleBlack}{cmyk}{0,0,0,1}
\definecolor{TyndaleGreen}{cmyk}{.45,0,1,.24}
\definecolor{TyndaleOrange}{cmyk}{0,.79,1,0}
\definecolor{TyndaleAqua}{cmyk}{.47,0,.24,0}
\definecolor{TyndaleYellow}{cmyk}{.03,.03,.35,0}

% metadata (assumes a host of definitions are made in the main file)
\usepackage[setpagesize=false,     % leave this to geometry
            hyperfootnotes=false,  % fragile and distracting
            xetex]{hyperref}
\hypersetup{breaklinks=true,       % allow link text to break across lines
            colorlinks=true,       % colorlinks resets pdfborder to 0 0 0
            urlcolor=TyndaleURLs,  % for external links
            linkcolor=TyndaleRed,  % for normal internal links
            citecolor=TyndaleGold, % for bibliographical citations in text
            pdfauthor={\prof},
            pdftitle={\ccode: \ctitle},
            pdfsubject={Tyndale UC, \csemester},
            pdfcreator={github.com/danieldriver/syllabus}}
\urlstyle{same}                    % don't use monospace font for urls

% custom footlines
\usepackage{fancyhdr}
\pagestyle{fancy} % turn it on
\fancyhf{}        % reset everything
\renewcommand{\headrulewidth}{0pt} % remove header line as well
\lfoot{\sffamily\scshape\footnotesize\MakeLowercase{\ctitle, v\cversion}}
\rfoot{\sffamily\scshape\footnotesize\MakeLowercase{\prof\quad\thepage}}

% gratuitous with custom title page, but useful as a fallback
\title{\ccode: \ctitle}
\author{\professor}
\date{\semester}


\begin{document}
\ProvidesFile{title.tex}[2013/09/06 v1.0 -- Syllabus title page]

\begin{titlepage}
  \begin{center}

    \LARGE\sffamily % set title elements in a large sans serif

    \begin{minipage}{\textwidth}
      \parbox[t]{0.5\textwidth}{
        \mbox{}\\[-13pt] % dummy line to align parboxes
        \includegraphics[width=0.5\textwidth]{.syllabus/includes/TyndaleUC}}
      \hfill
      \parbox[t]{0.4\textwidth}{
        \raggedleft Course Syllabus\\
        \csemester}
    \end{minipage}

    \vfill

    {\textsc{\MakeLowercase\ccode}\\[1ex]
      \bfseries\cseries\Huge\ctitle}

    \vfill

    \normalsize\rmfamily % switch back to body type

    \begin{tabular}{>{\bfseries}rl>{\bfseries}rl}
      \toprule
      Instructor & \prof, \pdegree & Course  & Version \cversion \\
      \midrule
      Phone      & \pphone         & Meets   & \cmeetson         \\
      Email      & \pemail         & Time    & \cmeetsat         \\
      Office     & \poffice        & Room    & \cmeetsin         \\
      Hours      & \phours         & Website & \cwebsite         \\
      \bottomrule
    \end{tabular}

    \vfill

    \begin{description}\small
      \item[Commuter Hotline]
        Class cancellations due to inclement weather or illness will
        be announced on the commuter hotline at \texttt{416.226.6620
        x2187}. Alternately, weather cancellation information is posted
        at \href{http://tyndale.ca/weather}{tyndale.ca/weather}.
      \item[MyTyndale.ca]
        This course may have materials stored on its website, such as
        handouts or readings that may be needed in order to complete
        assignments. Students are responsible for checking these course
        pages on a regular basis. Here, too, students are able to view
        their grades throughout the semester. For more information see
        Section~\ref{mytyndale}, below.
      \item[Mail]
        Students are responsible for information communicated through
        their campus mailboxes and student e-mail accounts. A mailbox
        directory hangs beside the mailboxes. For more information
        contact the Registrar's office.
    \end{description}

  \end{center}

  \section{Course Description}
  \label{description}

  \emph{From the Academic Calendar:} \cdescrip

\end{titlepage}
\setcounter{page}{2} % count the title page as page 1


\section{Learning Objectives}
\label{objectives}
\edobject

\section{Required Texts \& Materials}
\label{texts}

The following texts are required. Students are encouraged to
purchase print copies. Even where eBooks are assigned or available, print books can provide relief from screen time. Many readers find that hard copies can be marked up with greater ease, too.

\begingroup
\renewcommand{\section}[2]{}% temporarily remove the section heading
\begin{thebibliography}{Brodersen}% use the longest item in the bibliography

	\bibitem[Brodersen]{AB17} Alma Brodersen.
	\emph{The End of the Psalter: Psalms 146–150 in the Masoretic Text, the Dead Sea Scrolls, and the Septuagint}.
	Waco, TX: Baylor University Press, 2018 (paperback reprint).
	\textsc{isbn} 978-1481308991.
	The AST Library also provides \href{https://ebookcentral.proquest.com/lib/astheology/detail.action?docID=4895035}{access to the eBook (BZAW 505; Berlin: Walter de Gruyter, 2017)}.

	\bibitem[Creach]{JC20} Jerome F.\,D. Creach.
	\emph{Discovering Psalms: Content, Interpretation, Reception}.
	Grand Rapids: Eerdmans, 2020.
	\textsc{isbn} 978-0802878069.
	An eBook is available from the book’s \href{https://spckpublishing.co.uk/discovering-the-psalms}{UK publisher (London: SPCK, 2020)}.

	\bibitem[NJPS]{njps}
	Adele Berlin and Marc Zvi Brettler, eds.
	\emph{The Jewish Study Bible: Second Edition}.
	Oxford/New York: Oxford University Press, 2014.
	\textsc{isbn} 978-0199978465.
	The first edition of this study Bible (OUP, 2004) or another edition of the NJPS translation of the Tanakh (1985) will suffice, if you already have one to hand. The AST Library offers both through \href{http://www.oxfordbiblicalstudies.com.ezproxy.astheology.ns.ca:2048/article/book/obso-9780195297515/obso-9780195297515-chapter-27}{Oxford Biblical Studies Online}.

% Assigned in 2017; neither were overwhelmingly successful as classroom texts
%
%	\bibitem[Shape \& Shaping]{NdeCW} Nancy L. deClaissé-Walford.
%	\emph{The Shape and Shaping of the Book of Psalms: The Current State of Scholarship}.
%	Atlanta: SBL Press, 2014.
%	\textsc{isbn} 978-1628370010.
%
%	\bibitem[Two Psalms]{SGill} Susan Gillingham.
%	\emph{A Journey of Two Psalms: The Reception of Psalms 1 and 2 in Jewish and Christian Tradition}.
%	Oxford: Oxford University Press, 2014.
%	\textsc{isbn} 978-0199652419.

%	\bibitem[tk]{tk} tk.
%	\emph{tk}.
%	tk: tk, 20tk.
%	\textsc{isbn} 978-tk.

\end{thebibliography}
\endgroup

Other required texts and materials will be posted on Microsoft Teams and/or \href{https://app.perusall.com/courses/hb-3112-psalms-in-interpretation/_/dashboard}{Perusall}, the two main platforms that will facilitate online delivery of this course.

\section{Supplementary Texts}
\label{supplementary}

Supplementary readings may be recommended throughout the semester.
Excerpts from this literature will be placed on reserve or made
available through the course websites.

%The following articles or book chapters represent a range of scholarly
%and theological interest in the Psalms. They will either be available
%for download through the course website, or kept on reserve in the
%library. Students are not strictly required to read the supplementary
%material; then again, students who choose not to read it should not
%expect to get an A for the course.
%
%\begin{enumerate}
%
%  \item Katharine Dell, “Psalms,” pages 37--51 in M. Lieb et al., eds., \emph{The Oxford Handbook of the Reception History of the Bible} (Oxford: Oxford University Press, 2011). [\href{http://ezproxy.mytyndale.ca:2048/login?url=http://www.oxfordhandbooks.com/view/10.1093/oxfordhb/9780199204540.001.0001/oxfordhb-9780199204540-e-4}{online}]
%
%  \item James Kugel, “The Psalms of David,” pages 458–473 in \emph{How to Read the Bible: A Guide to Scripture, Then and Now} (New York/Toronto: Free Press, 2007).
%
%  \item Hermann Gunkel, “The Religion of the Psalms,” pages 134–167 in \emph{Water for a Thirsty Land} (Minneapolis: Fortress, 2001). German original published in 1922.
%
%  \item John O’Keefe, “‘A Letter that Killeth’: Toward a Reassessment of Antiochene Exegesis, or Diodore, Theodore and Theodoret on the Psalms,” \emph{Journal of Early Christian Studies} 8/1 (2000): 83–104.
%
%  \item Brevard Childs, “Psalm Titles and Midrashic Exegesis,” \emph{Journal of Semitic Studies} 16/2 (1971): 137--150. See also his “Psalm 8 in the Context of the Christian Canon,” pages 151–163 in \emph{Biblical Theology in Crisis} (Philadelphia: Westminster, 1970).
%
%  \item Michael Cameron, ``Hearing Voices: Christ at Prayer `in the Psalm and on the Cross,'\,'' pages 164--212 in \emph{Christ Meets Me Everywhere: Augustine's Early Figurative Exegesis} (Oxford: Oxford University Press, 2012).
%
%  \item Jonathan Magonet, “Through Rabbinic Eyes: Psalm 23,” pages 47–61 in \emph{A Rabbi Reads the Psalms} (2nd ed.; London: SCM, 2004).
%
%  \item Alan Cooper, ``Ps 24:7-10: Mythology and Exegesis,'' \emph{Journal of Biblical Literature} 102/1 (1983): 37--60. [\href{http://www.jstor.org/stable/3260745}{online}]
%
%  \item C.\,S. Lewis, ``\,`Sweeter than Honey,'\,'' pages 54--65 in \emph{Reflections on the Psalms} (London: Geoffrey Bles, 1958).
%
%  \item Joseph Addison, ``Saturday, September 20, 1712,'' \emph{The Spectator} 489 (available on \href{http://www.gutenberg.org/files/12030/12030-h/SV3/Spectator3.html#section489}{Project Gutenberg}). Also read his \href{http://www.bartleby.com/101/433.html}{poetic paraphrase of Psalm 19}.
%
%  \item J. Clinton McCann, “Books I–III and the Editorial Purpose of the Hebrew Psalter,” pages 93–107 in J.\,C. McCann, ed., \emph{The Shape and Shaping of the Psalter} (JSOTSup 159; Sheffield: Sheffield Academic Press, 1993).
%
%  \item Gerald Wilson, “Shaping the Psalter: A Consideration of Editorial Linkage in the Book of Psalms,” pages 72–82 in J.\,C. McCann, ed., \emph{The Shape and Shaping of the Psalter} (JSOTSup 159; Sheffield: Sheffield Academic Press, 1993).
%
%  \item Gerald Sheppard, “Theology and the Book of Psalms,” \emph{Interpretation} 46 (1992): 143–155.
%
%  \item Hans-Joachim Kraus, “The Psalms in the New Testament,” pages 177–203 in \emph{Theology of the Psalms} (Minneapolis: Augsburg, 1986). German original published in 1979.
%
%  \item Christopher R. Seitz, ``Theological Use of the Old Testament: Recent New Testament Scholarship and the Psalms as Christian Scripture,'' pages 137--156 in C. Seitz, \emph{The Character of Christian Scripture: The Significance of a Two-Testament Bible} (Grand Rapids: Baker Academic, 2011).
%
%  \item Craig A. Evans, “Praise and Prophecy in the Psalter and in the New Testament,” pages 551--579 in P. Flint and P. Miller, eds., \emph{The Book of Psalms} (Leiden: Brill, 2005).
%
%  \item Adele Berlin, “Psalms and the Literature of Exile: Psalms 137, 44, 69, and 78,” pages 65--86 in P. Flint and P. Miller, eds., \emph{The Book of Psalms} (Leiden: Brill, 2005).
%
%  \item Erhard Gerstenberger, “Theologies in the Book of Psalms,” pages 603–625 in P. Flint and P. Miller, eds., \emph{The Book of Psalms} (Leiden: Brill, 2005).
%
%  \item Rolf Rendtorff, “The Psalms of David: David in the Psalms,” pages 53–64 in P. Flint and P.~Miller, eds., \emph{The Book of Psalms} (Leiden: Brill, 2005).
%
%\end{enumerate}

\begin{multicols}{3}[Modern commentaries on the Psalms include:]%
\footnotesize\noindent
E. Hengstenberg (\textsuperscript{2}1849--52)\\
J. Olshausen (1853)\\
W.\,M.\,L. de Wette (\textsuperscript{5}1856)\\
H. Hupfeld (1855--62)\\
F. Delitzsch (1871)\\
J.\,J.\,S. Perowne (\textsuperscript{7}1890)\\
J. Wellhausen (\textsuperscript{3}1898)\\
A.\,F. Kirkpatrick (1891--1901)\\
C.\,A. Briggs (1906--07)\\
B. Duhm (\textsuperscript{2}1922)\\
H. Gunkel (\textsuperscript{4}1926)\\
R. Kittel (\textsuperscript{5-6}1929)\\
E.\,W. Barnes (1931)\\\
H. Schmidt (1934)\\
W.\,O.\,E. Oesterley (1939)\\
A. Cohen (1945)\\
H. Lamparter (1958)\\
A. Weiser (1962)\\
E.\,J. Kissane (\textsuperscript{2}1964)\\
A. Deissler (1965)\\
M.\,J. Dahood (1966--70)\\
H.-J. Kraus (\textsuperscript{4}1972)\\
A.\,A. Anderson (1972)\\
D. Kidner (1975)\\
J. Rogerson \& J. McKay (1977)\\
L. Jacquet (1975--79)\\
P. Craigie \& L. Allen (1983)\\
C. Stuhlmueller (1985)\\
E. Gerstenberger (1988)\\
J. Day (1990)\\
M. Tate (1990)\\
A.\,C. Feuer (\textsuperscript{3}1991)\\
A. Cohen (1992)\\
J.\,L. Mays (1994)\\
M. Girard (1984--96)\\
K. Seybold (1996)\\
C. Broyles (1999)\\
K. Schaefer (2001)\\
G. Wilson (2002)\\
R.\,J. Clifford (2002--03)\\
J. Magonet (\textsuperscript{2}2004)\\
J.-L. Vesco (2006)\\
R. Alter (2007)\\
J. Goldingay (2006--08)\\
J.\,H. Eaton (2008)\\
F.-L. Hossfeld \& E. Zenger\par (2000--08, ET 2005--11)\\
W.\,P. Brown (2010)\\
D. Bergant (2013)\\
B. Peters (2004--14)\\
T. Longman (2014)\\
W. Brueggemann \&\par W. Bellinger (2014)\\
N. DeClaissé-Walford, R. Jacobson \& B. LaNeel Tanner (2014)\\
A. Ross (2011--16)\\
M. Oeming \& J. Vette (2000--16)\\
B. Weber (\textsuperscript{2}2016)\\
E. Charry (2015) \&\par J. Byassee (2018)\\
W. Tucker \& J. Grant (2018)\\
S. Gillingham (2008--18)\\
B. Waltke, J. Houston \&\par E. Moore (2010--19)\\
R. Meynet (2017--19)\\
N. DeClaissé-Walford (2020)
\end{multicols}

Also, the following reference works are worth owning and consulting.
\cite{rlgs} in particular contains sound advice on core skills like
reading religious texts, writing essays and reviews, revising essays,
making oral presentations, and learning languages.

\begingroup
\renewcommand{\section}[2]{}% temporarily remove the section heading
\begin{thebibliography}{Making Sense}% use the longest item in the bibliography

	\bibitem[Making Sense]{rlgs} Northey, Margot, Bradford A. Anderson, and Joel N. Lohr.
	\emph{Making Sense in Religious Studies: A Student's Guide to Research and Writing}.
	3rd ed. Toronto: Oxford University Press, 2019. ISBN 978-0199026838.

	\bibitem[SBL2]{sbl2} Collins, Billie Jean, et al.
	\emph{The SBL Handbook of Style}.
	2nd ed. Atlanta: SBL Press, 2014. ISBN 978-1589839649.
	Designed to augment \href{http://www.chicagomanualofstyle.org/home.html}{\emph{Chicago Style}}
	(the standard at AST), there is also a free
	\href{https://www.sbl-site.org/assets/pdfs/pubs/SBLHSsupp2015-02.pdf}{Student Supplement for SBL2}.

\end{thebibliography}
\endgroup

%\begin{multicols}{2}[Commentaries on the Psalms in the AST library include:]%
%\footnotesize\noindent
% REVISE THE FOLLOWING (taken from my Genesis syllabus)
%Targums: {\scshape BS 709.2 B5 1987 vv. 1A, 1B, 6} (c. 150--?)\\
%Didymus the Blind: {\scshape BS 1235 D4913 2016} (d. 398)\\
%Augustine of Hippo: {\scshape BS 1235 A8413 1991} (d. 430)\\
%Bede, the Venerable: {\scshape BS 1235 B43 2008} (d. 735)\\
%ACCS, Genesis 1--11: {\scshape BS 1235.3 G46 2001}\\
%ACCS, Genesis 12--50: {\scshape BS 1235.53 G46 2002}\\
%RCS, Genesis 1--11: {\scshape BS 1235.53 G455 2012}\\
%Luther, Martin: {\scshape BR 330 E5 1955 vv.1--8} (d. 1546)\\
%Calvin, John: {\scshape BS 1235 C293 1948} (d. 1564)\\
%Patrick, Simon: {\scshape BS 1235 P36 1695}\\
%Henry, Matthew: {\scshape BS 490 H4 1961} (d. 1714)\\
%Hershon, Paul Isaac: {\scshape BS 1235 H47} (1883)\\
%Delitzsch, Franz: {\scshape BS 1235 D4} (1888)\\
%Skinner, John: {\scshape BS 1235 S45 1917, 1963} (1930)\\
%Richardson, Alan: {\scshape BS 1235 R66 1959}\\
%Cassuto, Umberto: {\scshape BS 1235.3 C3} (1961)\\
%Herbert, Arthur: {\scshape BS 1235.3 H4} (1962)\\
%Speiser, E. A.: {\scshape BS 1233 S64 1964}\\
%Kidner, Derek: {\scshape BS 1235.3 K47} (1967)\\
%von Rad, Gerhard: {\scshape BS 1235.3 R3213 1972}\\
%Plaut, Gunther: {\scshape BS 1225.3 P55 v. 1} (1974)\\
%Vawter, Bruce: {\scshape BS 1235.3 V38 1977}\\
%Davidson, Robert: {\scshape BS 1235.3 D3 1973, 1979}\\
%Leibowitz, Nehama: {\scshape BS 1235.3 L413 1981}\\
%Wenham, Gordon : {\scshape BS 491.2 W67 1982}\\
%Brueggemann, Walter: {\scshape BS 1235.3 B78} (1982)\\
%Maher, Michael: {\scshape BS 1235.3 M346 1982}\\
%Westermann, Claus: {\scshape BS 1235.3 W43213 1986, 1987, 1994}\\
%Sarna, Nahum: {\scshape BS 1235.3 S325 1989}\\
%Hamilton, Victor: {\scshape BS 1235.3 H32 1990, 1995}\\
%Scullion, John: {\scshape BS 1235.3 S37 1992}\\
%Fox, Everett: {\scshape BS 1223 A3 F68 1995}\\
%Hamilton, Victor: {\scshape BS 1235.3 H323 1995}\\
%Hartley, John: {\scshape BS 1235.3 H37 2000}\\
%Towner, Sibley: {\scshape BS 1235.3 T69 2001}\\
%Walton, John: {\scshape BS 1235.53 W35 2001}\\
%Cotter, David: {\scshape BS 1235.52 C68 2003}\\
%Briscoe, Stuart: {\scshape BS 1151.2 C66 2004}\\
%McKeown, James: {\scshape BS 1235.53 M35 2008}\\
%Arnold, Bill: {\scshape BS 1235.53 A76 2009}\\
%Goldingay, John: {\scshape BS 1235.53 G65 2010}\\
%Reno, Russell: {\scshape BS 1235.53 R46 2010}\\
%Cook, Joan: {\scshape BS 1235.53 C66 2011}\\
%De La Torre, Miguel: {\scshape BS 1235.53 D4 2011}\\
%Coleson, Joseph: {\scshape BS 1235.53 C64 2012}\\
%Longman, Tremper: {\scshape BS 1235.53 L66 2016}
%\end{multicols}

\section{Course Outline}
\label{outline}

We will adhere to the schedule in \autoref{schedule} as closely as
possible, though the professor reserves the right to adjust it to suit
the needs of the class.

\setcolumncount{4}% set up \sessioncount, \unit{}, \noclass{}, and \reminder{memo}{date} macros
\begin{table}[htbp]% set to `p' to put the schedule on its own page
  \centering
  \begin{tabular}{>{\sessioncount.}r@{ }llr}% make sure the column config agrees with \setcolumncount
	\toprule
	\sessionskip{\textbf{\S}.}&\textbf{Seminar Text}&\textbf{Secondary Reading}&\textbf{Date}\\
	\midrule

		& Psalm 1   & Driver 2019, 2020        & 12 Jan. \\
		& Psalm 2   & \cite[chs. 1--2]{JC20}   & 19 Jan. \\
		& Psalm 8   & \cite[chs. 3--5]{JC20}   & 26 Jan. \\
		& Psalm 19  & \cite[chs. 6--8]{JC20}   &  2 Feb. \\
		& Psalm 24  & \cite[ch. 9--end]{JC20}  &  9 Feb. \\
	\noclass{Term Break (Monday to Friday)}    & 16 Feb. \\
		& Psalm 34  & \cite[ch. 1]{AB17}       & 23 Feb. \\
		& Psalm 51  & \cite[ch. 2]{AB17}       &  2 Mar. \\
		& Psalm 90  & \cite[ch. 3]{AB17}       &  9 Mar. \\
		& Psalm 104 & \cite[ch. 4]{AB17}       & 16 Mar. \\
		& Psalm 110 & \cite[ch. 5]{AB17}       & 23 Mar. \\
		& Psalm 146 & \cite[ch. 6]{AB17}       & 30 Mar. \\
	\reminder{Exegetical essays are \textbf{due} by the last class}{6 Apr.} \\
		& Psalm 150 & \cite[ch. 7]{AB17}       &  6 Apr. \\
	\reminder{End of Term: Final marks are due}{19 Apr.} \\

	\bottomrule
  \end{tabular}
  \caption{Schedule of Readings}
  \label{schedule}
\end{table}

See the AST website for a list of other \href{http://www.astheology.ns.ca/students/academic-dates.html}{important dates}.

\section{Evaluation and Grade Structure}
\label{evaluation}

\begin{enumerate}

	\item Each week of class will feature a live (by video conference)
	\textbf{seminar-style discussion} of one individual psalm. Students
	will select a commentator for a period of 3--4 weeks, during which
	time they are responsible to report on the the author's commentary
	and bring its insight to bear on the seminar discussions. How should
	you prepare?

	\begin{enumerate}

		\item Read the week's psalm closely and carefully in the NJPS
		translation. (Read it in Hebrew if you have the ability.) Look
		for distinctive words, themes, and images in the psalm, and stay
		alert to intertextual connections between it and other biblical
		literature. Jot down your observations.

		\item Consult your designated commentary. It should be read
		carefully, too, but after you have got a sense for the psalm on
		its own. One good approach might be to read the psalm twice,
		make some preliminary notes, and then study your commentator and
		fill out your preliminary notes. Be sure to note any differences
		between the Hebrew and the version being commented upon, and try
		to read relevant ancient versions of the psalm in translation or
		in the original, as you are able. Make select annotations on the
		week's psalm in Perusall.

		\item Augment your knowledge of each commentary you study by
		reading about its place in reception history. Key recourses are
		listed in \autoref{horbib}, below (start with reference works
		like M. Sæbø's \emph{HB/OT} and \emph{EBR Online}). Remember
		that you will be the resident authority on your chosen
		commentator.

	\end{enumerate}


	\item Discussion of \textbf{secondary reading} will take place
	asynchronously on \href{https://app.perusall.com/}{Perusall}, a
	social e-reading platform. Whether you read the assigned texts there
	or offline first, engage the readings collaboratively by annotating
	them on Perusall.

	\begin{enumerate}

		\item Prepare for each week by reading the secondary reading
		(either online or off). Then, spend about one hour per week
		discussing the material on Perusall. Try to do both tasks before
		the seminar that week. Top issues may be carried over, briefly,
		into the live discussions.

		\item Engage in discussion by posting good questions or
		comments, helping others by answering their questions, and
		upvoting good questions or comments to draw attention to the
		most important issues and ideas. Note that this platform takes
		the place of other discussion forums. Some of the time you spend
		there counts towards contact hours for this course, too.

	\end{enumerate}

	\item Students will \textbf{memorize 12 verses of a psalm} and
	recite them to a witness, in person, before the last day of class.
	Any psalm is allowed except for Psalm 23. You may also choose to
	memorize verses from a combination of psalms. For instance, you
	might memorize the first verses (incipits) of the psalms of ascent,
	Psalms 120–134 (skip three if desired). The witness will sign a form
	to vouch for your recital.

	\item Students will \textbf{creatively represent one psalm} in the
	class immediately following that psalm’s workshop (e.g., Psalm 2 in
	Week 3). Engage as many of the senses as possible, using whatever
	you can find or create. Be inventive! The only conditions are (a)
	that you explain or demonstrate a connection with the text studied
	and (b) that your presentation last between 5 and 8 minutes. Record,
	upload, and share as a URL.

	\item Finally, students must compose an \textbf{exegetical essay} on
	one psalm from the seminar. Consider all the perspectives brought to
	bear in class on a psalm of your choice, noting anything useful for
	your discussion, and then dig deeper into the text and its history.
	Your work should be aided by the preliminary researches of your
	classmates, but you will need to read more widely on your own; the
	seminar is only a starting point. You are expected to develop a
	strong thesis, and to interact with a selection of secondary
	literature in 4,000--5,000 words. The paper is due at the start of
	the last class.

\end{enumerate}

The breakdown for the semester's total work is shown in
\autoref{grade-dist}.

\begin{table}[htbp]
  \centering
  {\lining
  \begin{tabular}{lr}
    \toprule
    Seminar Contributions   & 15\% \\
    Reading Discussion      & 15\% \\
    Psalm Memorization      & 10\% \\
    Creative Representation & 20\% \\
    Final Paper             & 40\% \\
    \bottomrule
  \end{tabular}}
  \caption{Distribution of Grades}
  \label{grade-dist}
\end{table}

\ProvidesFile{grades.tex}[2016/09/03 v2.0 -- Course policy]

\subsection{Grading System at AST}
\label{grades}

AST's \href{http://www.astheology.ns.ca/webfiles/AST_2016Calendar_web(A5)-06APR2016.pdf}{Academic
Calendar} provides guidelines and detailed criteria for academic
assessment. Marks are assigned by letter grade using the benchmarks in
\autoref{grade-syst}.

\begin{table}[htbp]
  \centering
  {\lining
  \begin{tabular}{lll}
    \toprule
%    Letter      & Percent & Assessment        \\
%	\midrule
    A+          & 94--100    & Exceptional    \\
    A           & 87--93     & Outstanding    \\
    A\char"2212 & 80--86     & Excellent      \\ [1ex]
    B+          & 77--79     & Good           \\
    B           & 73--76     & Acceptable     \\
    B\char"2212 & 70--72     & Marginal       \\ [1ex]
    C           & 60--69     & Unsatisfactory \\
    F           & 0--59      & Failure        \\
    FP          & 0          & Failure due to Plagiarism \\
    \bottomrule
  \end{tabular}}
  \caption{Summary of Grading System}
  \label{grade-syst}
\end{table}

% More detailed grading criteria from pp. 61--62 of `16.0406-I2-AST Academic Calendar.pdf'
%
%\begin{description}
%  \item[A+ (94-100) ‘Exceptional’]
%    A superior performance with consistent evidence of a comprehensive,
%    incisive grasp of all aspects of the subject matter; a very wide
%    knowledge base; insightful critical evaluation and analysis of the
%    material; an exceptional capacity for original, creative, and/or
%    logical thinking; an exceptional ability to organize, analyse,
%    synthesize, and to express thoughts fluently.
%  \item[A (87-93) ‘Outstanding’]
%    A comprehensive grasp of the subject matter, outstanding evidence of
%    original thought; sound critical evaluation of the material; an
%    excellent ability to organize, analyse, synthesize and to express
%    thoughts; mastery of an extensive knowledge base.
%  \item[A- (80-86) ‘Excellent’]
%    All the qualities of a B-level performance and an excellent capacity
%    for original, creative, and/ or logical thinking; excellent ability
%    to organize, analyse, synthesize, and integrate ideas; broad
%    knowledge base in the subject matter.
%  \item[B+ (77-79) ‘Good’]
%    A good performance with substantial knowledge of the subject matter;
%    a very good understanding of the relevant issues; familiarity with
%    relevant literature and techniques; good ability to organize,
%    analyse, and examine the material in a constructive and critical
%    manner.
%  \item[B (73-76) ‘Acceptable’]
%    A generally adequate performance with a good knowledge of the
%    subject matter; a fair understanding of relevant issues; some
%    ability to work with relevant literature and techniques; some
%    ability to develop solutions to difficult problems related to the
%    subject material.
%  \item[B- (70-72) ‘Marginally Acceptable’]
%    Some familiarity with the subject material; some understanding.
%    Satisfactory understanding of relevant issues; attempts to solve
%    moderately difficult problems related to the subject material in a
%    critical and analytical manner are only partially successful.
%  \item[C (60-69) ‘Unsatisfactory’]
%    A C grade indicates unsatisfactory academic performance. At the
%    discretion of the instructor, supplemental work may be negotiated to
%    upgrade the mark to a B range. A student may carry two C grades
%    without penalty in all courses except Foundations Courses,
%    Supervised Field Education, Supervised Ministry Practicum and the
%    Graduate Project. In these courses, a minimum grade of B- is
%    required to graduate. A student who receives a C in a Foundation
%    course must repeat the course to achieve a B- or better, and cannot
%    use the C grade to meet prerequisite requirements for advanced
%    courses. If the student repeats one of these courses and receives a
%    B- or better, the previous C grade remains on the transcript and can
%    be counted toward the total of unsatisfactory grades that may lead
%    to academic dismissal. Credit will be given only once for any
%    course. (See Policy on Unsatisfactory Academic Performance in the
%    AST Student Handbook.)
%  \item[F (0-59) ‘Failure’]
%    Student has not grasped subject matter; does not understand issues
%    involved; cannot work with relevant literature. (See Policy on
%    Unsatisfactory Academic Performance in the AST Student Handbook.)
%  \item[P ‘Pass’]
%    Credit awarded, but no mark assigned.
%  \item[FP ‘Failure due to Plagiarism’]
%    A student will receive this grade only after proven incident(s) of
%    plagiarism in a course.
%\end{description}
\ProvidesFile{other.tex}[2022/06/08 v2.9.1 -- Course policy]

\section{Other Course Policy}
\label{policy}

Late work will not be accepted, except in genuinely extenuating
circumstances. Students must submit something before the deadline if
they wish to receive credit. Unless I state otherwise, assignments are
to be uploaded by 11:59 \PM\ (Atlantic) on the date indicated.

Essay submissions must be typewritten and double-spaced. They should be
free from error. In this course they should follow SBL Style (see
\cite{sbl2} in \autoref{supplementary}, above). As a reminder, AST
upholds an Inclusive Language Policy. Please use gender-inclusive
language when referring to human beings. Our traditions have different
norms for speech about God; you are of course free to follow and explore
those traditions when referring to God.


Plagiarism is the
\href{http://www.eerdmans.com/Pages/Item/59043/Commentary-Statement.aspx}{failure}
to \href{https://www.theguardian.com/world/2013/feb/09/german-education-minister-quits-phd-plagiarism}{attribute}
(by means of footnotes when writing or aloud when speaking) any ideas,
phrases, sentences, materials, syntheses, et cetera, that another author
has composed and that you have borrowed for your own work. Plagiarism is
unethical. Academic penalties for plagiarism at AST are serious, and may
include failure of the course or even suspension of further studies.
Unintentional plagiarism is considered plagiarism. AST's Plagiarism
Policy is found under that heading in the Academic
Calendar.

Students should request permission to record a class or lecture. If
permission is granted, or if recordings are provided (as in the case of
an online or hybrid course), I stipulate that all recordings be for
personal use only. They may not be shared or distributed.

If you have needs that require modifications to any aspect of this
course, please consult with the instructor as soon as possible. Any
documentation regarding disabilities that you wish to divulge to AST
should be provided to the Registrar’s Office, where it will be kept in a
confidential file.

Finally, I encourage the conscientious use of laptops, tablets, and
other technology in my classes. In classroom settings, realize that, as
\href{http://dx.doi.org/10.1016/j.compedu.2012.10.003}{cognitive
psychologists have demonstrated}, ``laptop multitasking hinders
classroom learning for both users and nearby peers.'' Do your part to
foster an environment for dialogue by honouring the presence of your
classmates. In online and hybrid settings, consider both the physical
environment in which you choose to work and the virtual environment that
you help create through your participation in various forums. Let your
engagement in this course be marked by rigour and charity alike.


\section{Additional Bibliography}
\label{bibliography}

Literature on the Psalter is vast. Two exceptional, actively maintained bibliographies are:

\begin{enumerate}
	\item Annotated: Stephen Breck Reid and Rebecca Poe Hays, “Psalms,” in \emph{Oxford bibliographies Online: Biblical Studies}, \href{https://go.openathens.net/redirector/astheology.ns.ca?url=https://www.oxfordbibliographies.com/view/document/obo-9780195393361/obo-9780195393361-0099.xml}{https://www.oxfordbibliographies.com/view/document/obo-9780195393361/obo-9780195393361-0099.xml} (access at AST through Open Athens). doi: 10.1093/OBO/9780195393361-0099
	\item Exhaustive: Beat Weber, “BiblioPss1990ff.: Bibliography of Psalms and the Psalter since 1990” (German and English), \href{https://www.academia.edu/5910732/BiblioPss1990ff_Bibliography_of_Psalms_and_the_Psalter_since_1990}{https://bienenberg.academia.edu/BeatWeber} (use the embedded link or browse to his bibliographies).

\end{enumerate}

For the weekly Psalms seminar, students should focus on a commentator
from the list below and give a short (5 minute) overview of the
interpretation of the psalm in question. Be sure to work with three
different commentators over the semester, from at least two different
major periods. (For those with the language skills, registering notes on
Hebrew, Aramaic, Greek, or Latin versions can be one selection.) The
goal is to gain maximum insight into the message of the Psalms by close
attention to text, translation, and commentary.

\subsection{Ancient Sources}
\label{oldbib}

\begin{enumerate}

 \item Versions
  \begin{enumerate}

	\item John R. Kohlenberger, III, ed., \emph{The Comparative Psalter: Hebrew-Greek-English} (Oxford: Oxford University Press, 2007)[Ref BS 1419 2007]. Very useful, but not to be mistaken for a critical edition or a full account of ancient versions.

	\item The Septuagint/Old Greek (LXX/OG) is available in \href{http://ccat.sas.upenn.edu/ioscs/editions.html}{two main editions}, each with major and minor editions. A. Rahlfs produced the Göttingen Septuagint's \emph{editio maior} of \emph{Psalmi cum Odis} in 1931 (3rd ed., 1979).

	\item Vulgate/Jerome: Biblia Sacra Latina [\href{https://www.biblegateway.com/versions/?action=getVersionInfo&vid=4}{online}].

	\item Targumim: \emph{The Aramaic Bible, Vol. 16: The Targum of the Psalms} (trans. David M. Stec; Collegeville, Minn: Liturgical Press, 2004) [BS 709.2 B5 1987 vol.16]. Also, \href{http://targum.info/targumic-texts/targum-psalms/}{Edward Cook's translation is available online}.

  \end{enumerate}
 \item Early Church / Synagogue
  \begin{enumerate}

	\item John Chrysostom, \emph{Commentary on the Psalms} (trans. R.\,C. Hill; 2 vols.; Brookline, Mass.: Holy Cross Orthodox Press, 1998) [BS 1430.3 J63 1998].

	\item Augustine, \emph{Expositions of the Psalms} (trans. M. Boulding; 6 vols.; Hyde Park, N.Y.: New City Press, 2000--2004). Older English translations of \emph{Enarrationes in Psalmos} appear in the Ancient Christian Writers series (trans. S. Hebgin, F. Corrigan; Westminster, Md., Newman Press, 1960--) and in \href{http://www.ccel.org/ccel/schaff/npnf108.toc.html}{P. Schaff's NPNF translation, first printed 1847--57 and now online}. For the Latin original see Corpus Scriptorum Ecclesiasticorum Latinorum (CSEL), vols. 93--95.

	\item Diodore of Tarsus, \emph{Commentary on Psalms 1--51} (trans. R.\,C. Hill; Atlanta: SBL, 2005) [BR 65 D393 D5613 2005].

	\item Theodoret of Cyrus, \emph{Commentary on the Psalms} (trans. R.\,C. Hill; 2 vols; Washington, DC: Catholic University of America Press, 2000--2001).

	\item Theodore of Mopsuestia, \emph{Commentary on Psalms 1--81} (trans. R.\,C. Hill; Atlanta: SBL, 2006).

	\item Cassiodorus, \emph{Explanation of the Psalms} (trans. P.\,G. Walsh; New York: Paulist Press, 1990--1991) [BR 60 A35 no. 51--53].

	\item Midrash Tehillim / \emph{Midrash on the Psalms, Translated from the Hebrew and Aramaic} (trans. W.\,G. Braude; 2 vols.; New Haven: Yale University Press, 1959).

  \end{enumerate}
 \item Medieval
  \begin{enumerate}

	\item Aquinas, \emph{Postilla super Psalmos / Commentary on the Psalms} (1272--1273). A Latin--English parallel edition for (most of) Psalms 1--54, ed. Stephen Loughlin, is \href{http://hosted.desales.edu/w4/philtheo/loughlin/ATP/}{online as part of the Aquinas Translation Project}. See also: Thomas Ryan, \emph{Thomas Aquinas as Reader of the Psalms} (Notre Dame: University of Notre Dame Press, 2000).

	\item \emph{Rashi’s Commentary on Psalms} (trans. Mayer I. Gruber; Leiden: Brill, 2004) [BS 1429 R3713 2007].% See also: Hermann Hailperin, \emph{Rashi and the Christian Scholars} (Pittsburgh: University of Pittsburgh Press, 1963).

  \end{enumerate}
 \item Reformation Era
  \begin{enumerate}

	\item Desiderius Erasmus (1466--1536): \emph{Expositions of the Psalms} (Collected Works of Erasmus, Vol. 63--65; Toronto: University of Toronto Press, 1997, 2005, 2010) [CWE 63: Pss 1, 2, 3, 4; CWE 64: Pss 88, 22, 28, 33; CWE 65: Pss 38, 83, 14].

	\item Martin Luther (1483--1546): \emph{Werke}, 35 = \emph{Luther’s Works}, 10--14 [BR 330 E5 1955]. See also: J.\,S. Preus, \emph{From Shadow to Promise: Old Testament Interpretation from Augustine to the Young Luther} (Cambridge, Mass.: Belknap, 1969).

	\item John Calvin (1509--1564): \emph{Commentary on the Psalms} (1557--, ET 1839--). All of Calvin's commentaries are available online, in English translation, at the \href{http://www.ccel.org/}{Christian Classics Ethereal Library}: Psalms \href{http://www.ccel.org/ccel/calvin/calcom08.html}{1--35}, \href{http://www.ccel.org/ccel/calvin/calcom09.html}{36--66}, \href{http://www.ccel.org/ccel/calvin/calcom10.html}{67--92}, \href{http://www.ccel.org/ccel/calvin/calcom11.html}{93--119}, \href{http://www.ccel.org/ccel/calvin/calcom12.html}{119--150}. See also: Herman J. Selderhuis, \emph{Calvin's Theology of the Psalms} (Grand Rapids: Baker Academic, 2007).

  \end{enumerate}
 \item Early Modern / Modern / Critical
  \begin{enumerate}

	\item Select an example from \autoref{supplementary} or nominate an early modern or modern commentator of your own discovery.

	\item If you have modern language skills beyond English, this is an opportunity to put them to use. For options in an array of European languages, see \S 2 of  B. Weber's exhaustive bibliography, or consult a good continental commentary. See, for example, the first pages of F.-L. Hossfeld and E. Zenger, \emph{Psalms 2: A Commentary on Psalms 51--100} (Minneapolis: Fortress, 2005), and idem, \emph{Psalms 3: A Commentary on Psalms 101--150} (Minneapolis: Fortress Press, 2011).

  \end{enumerate}

\end{enumerate}

\subsection{Other Literature on the Psalms and Reception History}
\label{horbib}

\begin{itemize}

	\item ArtScroll Tanach Series, 2 vols. (English and Hebrew): Avrohom Chaim Feuer, \emph{Tehillim / Psalms: A New Translation with a Commentary Anthologized from Talmudic, Midrashic, and Rabbinic Sources} (Brooklyn, NY: Mesorah, 1985).

	\item Ancient Christian Commentary on Scripture, OT vols. 7--8:
	  Craig A. Blaising and Carmen Hardin, eds., \emph{Psalms 1--50} (Downers Grove: InterVarsity Press, 2008);
	  Quentin F. Wesselschmidt and Thomas C. Oden, eds., \emph{Psalms 51--150} (Downers Grove: InterVarsity Press, 2007).

	\item Reformation Commentary on Scripture, OT vols. 7--8:
	Herman J. Selderhuis, ed., \emph{Psalms 1--72} (Downers Grove: InterVarsity Press, 2015);
	Herman J. Selderhuis, ed., \emph{Psalms 73--150} (Downers Grove: InterVarsity Press, 2018).

	\item Willaim Holladay, \emph{The Psalms through Three Thousand Years: Prayerbook of a Cloud of Witnesses} (Minneapolis: Fortress, 1993).

	\item H. Attridge and M. Fassler, eds., \emph{Psalms in Community: Jewish and Christian Textual, Liturgical, and Artistic Traditions} (Leiden: Brill, 2004).

	\item Susan Gillingham, \emph{\href{https://onlinelibrary.wiley.com/doi/book/10.1002/9780470691748}{Psalms Through the Centuries, Vol. 1}} (Oxford: Blackwell, 2008) offers an \href{http://www.blackwellpublishing.com/pdf/9780631218555.pdf}{expanded bibliography online (PDF)}. The first volume pays especial attention to Psalm 8 as a case study. The second, \emph{\href{https://onlinelibrary.wiley.com/doi/book/10.1002/9781118832196}{Psalms Through the Centuries, Vol. 2: A Reception History Commentary on Psalms 1–72}} (Oxford: Wiley-Blackwell, 2018) consolidates and builds on some of her other work, including \emph{A Journey of Two Psalms: The Reception of Psalms 1 and 2 in Jewish and Christian Tradition} (Oxford: Oxford University Press, 2013) and \emph{Jewish and Christian Approaches to the Psalms: Conflict and Convergence} (Oxford: Oxford University Press, 2013).

	\item M. Sæbø, \emph{HB/OT}: For help with specific periods and commentators in biblical reception history see Magne Sæbø, ed., \emph{Hebrew Bible/Old Testament: The History of Its Interpretation} (3 vols.; Göttingen: Vandenhoeck \& Ruprecht, 1996--2015).

	\item C.\,M. Furey et al, eds., \emph{Encyclopedia of the Bible and Its Reception (EBR) Online} (30 vols.; Berlin: De Gruyter, 2009–2024) is an ambitious work still in progress, available: \url{https://db-degruyter-com.eu1.proxy.openathens.net/db/ebr}

\end{itemize}

\end{document}
