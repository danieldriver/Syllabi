% Copyright (c) 2021 by Daniel R. Driver.
% !TEX encoding = UTF-8 Unicode
% !TEX TS-program = XeLaTeX

\documentclass[titlepage]{article}

% This document presumes a file structure and set of inputs that are
% available at: git@github.com:danieldriver/syllabi.git

\newcommand\policy{../policy}
\newcommand\incl{../includes}
\ProvidesFile{variables.tex}[2018/05/24 v2.1 -- Syllabus variables]

\usepackage{xspace} % make manual spaces (like \mycmd\ ) unnecessary
\usepackage{xifthen} % provides \isempty test

% variables for internal use
\newcommand\prof{}
\newcommand\pdegree{}
\newcommand\pphone{}
\newcommand\pemail{}
\newcommand\poffice{}
\newcommand\phours{}
%
\newcommand\ccode{}
\newcommand\ctitle{}
\newcommand\cseries{}
\newcommand\cversion{}
\newcommand\csemester{}
\newcommand\cmeetson{}
\newcommand\cmeetsat{}
\newcommand\cmeetsin{}
\newcommand\cwebsite{}
\newcommand\cdescrip{}
\newcommand\cprereqs{}
\newcommand\edobject{}

% in case of fully online courses - https://tex.stackexchange.com/a/5896
\newif\ifonline
\newcommand\Int[2]{\ifonline#1\else#2\fi}

% commands for setting variables in the preamble
\newcommand\professor[2][PhD]{
  \renewcommand\pdegree{#1\xspace}
  \renewcommand\prof{#2\xspace}}
\newcommand\phone[1]{
  \renewcommand\pphone{\addfontfeatures{Numbers=Monospaced}#1\xspace}}
\newcommand\email[1]{
  \renewcommand\pemail{\href{mailto:#1}{#1}\xspace}}
\newcommand\officehours[2][Library, Room 5-North]{
  \renewcommand\poffice{#1\xspace}
  \renewcommand\phours{#2\xspace}}
%
\newcommand\coursecode[2][1.0]{
  \renewcommand\cversion{#1\Int{-i}{}\xspace}
  \renewcommand\ccode{#2\Int{(Int)}{}\xspace}}
\newcommand\coursetitle[2][]{
  \ifthenelse{\isempty{#1}}%
    {}% do nothing if #1 is empty, else:
    {\renewcommand\cseries{#1\\[1ex]}}
  \renewcommand\ctitle{#2\xspace}}
\newcommand\semester[1]{
  \renewcommand\csemester{#1\xspace}}
\newcommand\meets[3]{
  \newcommand\AM{\textsc{am}}
  \newcommand\PM{\textsc{pm}}
  \renewcommand\cmeetson{#1\xspace}
  \renewcommand\cmeetsat{\Int{From 9:00 \AM}{#2}\xspace}
  \renewcommand\cmeetsin{\Int{\href{https://smu.brightspace.com/d2l/login}{Brightspace}}{#3}\xspace}}
\newcommand\website[1]{
  \renewcommand\cwebsite{\href{http://#1}{#1}\xspace}}
\newcommand\cdescription[2][RM 1000 or GTRS 6000; and BF 1001]{
  \renewcommand\cprereqs{#1}
  \renewcommand\cdescrip{#2\par}}
\newcommand\objectives[1]{
  \renewcommand\edobject{#1\par}}


%\onlinetrue % \Int{true}{false}
\coursecode[1.0]{HB 3117(Hyb)}
\coursetitle[Hebrew Bible]{Creation and Canon}

% Taught as HB 3117 in:
%   - Fall 2021 (Hyb)

\professor{Daniel R. Driver}
\phone{902-425-7051}
\email{ddriver@astheology.ns.ca}
\officehours[Microsoft Teams]{Tuesdays, 1:00--3:00 \PM}

\semester{Fall Term 2021}
\meets{Mondays}% \meets{on}{at}{in}
      {4:00--6:30 \PM}% if online, it prints "From 9:00 AM"
      {Classroom 3/Teams}% if online, it prints "Brightspace" with link
\website{danieldriver.com}
\cdescription[RM 1000 or GTRS 6000; BF 1001]{% copy from the current Academic Calendar; [] for prereqs
	A study of creation across the Christian Bible, with particular
	focus on the HB/OT, this course will explore why the topic of
	creation has been sidelined in biblical studies and biblical
	theology, and how, thanks in part to the ecological crisis, it is
	getting new attention. Major headings in the course are creation,
	counter-creation, de-creation, and re-creation. Minor themes include
	sabbath, sea monsters, land, city, wisdom, and praise.
}% end of course description
\objectives{% recall Bloom's taxonomy: https://www.celt.iastate.edu/teaching/effective-teaching-practices/revised-blooms-taxonomy/
	By the end of the course students should be able to articulate a theology of creation.
	In particular, they should be able to:
		understand a variety of methods and approaches to the theme;
		analyze a proposal stipulating rules for Christian speech about creation;
		identify a number of primary texts, both inside and outside the Bible;
		contextualize and discuss many of those texts informally;
		name and evaluate with ecumenical sensitivity a range of scholarly interprations, including by Jews, Christians, and others;
		formally exegete at least one biblical creation text;
		begin to construct, or to know how to begin to construct, a more complete biblical theology of creation.
}% end of learning objectives

\ProvidesFile{preamble.tex}[2013/09/06 v1.0 -- Syllabus preamble]

% basic typography
\usepackage{fontspec}
\setmainfont[Ligatures=TeX]{Meta Serif Pro}
\setsansfont[Ligatures=TeX]{Meta Pro}
\newfontfamily\Heb{Meta Hebrew}
\setmonofont[Scale=MatchLowercase]{Menlo}
\usepackage{sectsty}
\allsectionsfont{\sffamily}
\frenchspacing
\setlength{\emergencystretch}{3em} % prevent overfull lines

% custom font size and leading
\renewcommand\tiny{\fontsize{6}{9}\selectfont}
\renewcommand\scriptsize{\fontsize{7}{10}\selectfont}
\renewcommand\footnotesize{\fontsize{8}{11}\selectfont}
\renewcommand\small{\fontsize{8.5}{11.5}\selectfont}
\renewcommand\normalsize{\fontsize{9}{12}\selectfont}% base size
\renewcommand\large{\fontsize{11}{14}\selectfont}
\renewcommand\Large{\fontsize{13}{16}\selectfont}
\renewcommand\LARGE{\fontsize{16}{19}\selectfont}% "course syllabus \\ semester" benefits from more lead
\renewcommand\huge{\fontsize{19}{21}\selectfont}
\renewcommand\Huge{\fontsize{24}{26}\selectfont}

% layout packages: page, logo, tables
\usepackage[scale={0.6,0.8},
            xetex]{geometry}
\usepackage{graphicx}
\usepackage{array}     % allow insertions of column styling with >{}
\usepackage{booktabs}  % elegant horizontal rules in tables
\usepackage{marginfix} % protect positioning of margin table in policy/grades

% custom macros for a session count in the schedule of readings
\newcounter{session}
\newcounter{columns}
\newcounter{courseunit}
\newcommand\setcolumncount[2][0]{ % optionally set count to other than 0,
  \setcounter{session}{#1}        % e.g. to -1, or to a standing count
  \setcounter{columns}{#2}}
\newcommand\sessioncount{\stepcounter{session}\arabic{session}}
\newcommand\sessionskip[1]{\multicolumn{1}{@{}r@{ }}{#1}}
\newcommand\unit[1]{\multicolumn{\thecolumns}{c}{%
  \scshape\stepcounter{courseunit}\roman{courseunit}. \MakeLowercase{#1}}}
\newcommand\noclass[1]{\multicolumn{1}{@{}l}{\itshape No Class: #1}}

% color to match Tyndale's branding
\usepackage[usenames]{xcolor}
% predefined: black, white, red, green, blue, cyan, magenta, yellow
\definecolor{TyndaleURLs}{HTML}{0062A0} % links on tyndale.ca
\definecolor{TyndaleBlue}{cmyk}{1,1,0,.32}
\definecolor{TyndaleGold}{cmyk}{0,.27,1,0}
\definecolor{TyndaleRed}{cmyk}{0,1,.99,.04}
\definecolor{TyndaleBlack}{cmyk}{0,0,0,1}
\definecolor{TyndaleGreen}{cmyk}{.45,0,1,.24}
\definecolor{TyndaleOrange}{cmyk}{0,.79,1,0}
\definecolor{TyndaleAqua}{cmyk}{.47,0,.24,0}
\definecolor{TyndaleYellow}{cmyk}{.03,.03,.35,0}

% metadata (assumes a host of definitions are made in the main file)
\usepackage[setpagesize=false,     % leave this to geometry
            hyperfootnotes=false,  % fragile and distracting
            xetex]{hyperref}
\hypersetup{breaklinks=true,       % allow link text to break across lines
            colorlinks=true,       % colorlinks resets pdfborder to 0 0 0
            urlcolor=TyndaleURLs,  % for external links
            linkcolor=TyndaleRed,  % for normal internal links
            citecolor=TyndaleGold, % for bibliographical citations in text
            pdfauthor={\prof},
            pdftitle={\ccode: \ctitle},
            pdfsubject={Tyndale UC, \csemester},
            pdfcreator={github.com/danieldriver/syllabus}}
\urlstyle{same}                    % don't use monospace font for urls

% custom footlines
\usepackage{fancyhdr}
\pagestyle{fancy} % turn it on
\fancyhf{}        % reset everything
\renewcommand{\headrulewidth}{0pt} % remove header line as well
\lfoot{\sffamily\scshape\footnotesize\MakeLowercase{\ctitle, v\cversion}}
\rfoot{\sffamily\scshape\footnotesize\MakeLowercase{\prof\quad\thepage}}

% gratuitous with custom title page, but useful as a fallback
\title{\ccode: \ctitle}
\author{\professor}
\date{\semester}


\begin{document}
\ProvidesFile{title.tex}[2013/09/06 v1.0 -- Syllabus title page]

\begin{titlepage}
  \begin{center}

    \LARGE\sffamily % set title elements in a large sans serif

    \begin{minipage}{\textwidth}
      \parbox[t]{0.5\textwidth}{
        \mbox{}\\[-13pt] % dummy line to align parboxes
        \includegraphics[width=0.5\textwidth]{.syllabus/includes/TyndaleUC}}
      \hfill
      \parbox[t]{0.4\textwidth}{
        \raggedleft Course Syllabus\\
        \csemester}
    \end{minipage}

    \vfill

    {\textsc{\MakeLowercase\ccode}\\[1ex]
      \bfseries\cseries\Huge\ctitle}

    \vfill

    \normalsize\rmfamily % switch back to body type

    \begin{tabular}{>{\bfseries}rl>{\bfseries}rl}
      \toprule
      Instructor & \prof, \pdegree & Course  & Version \cversion \\
      \midrule
      Phone      & \pphone         & Meets   & \cmeetson         \\
      Email      & \pemail         & Time    & \cmeetsat         \\
      Office     & \poffice        & Room    & \cmeetsin         \\
      Hours      & \phours         & Website & \cwebsite         \\
      \bottomrule
    \end{tabular}

    \vfill

    \begin{description}\small
      \item[Commuter Hotline]
        Class cancellations due to inclement weather or illness will
        be announced on the commuter hotline at \texttt{416.226.6620
        x2187}. Alternately, weather cancellation information is posted
        at \href{http://tyndale.ca/weather}{tyndale.ca/weather}.
      \item[MyTyndale.ca]
        This course may have materials stored on its website, such as
        handouts or readings that may be needed in order to complete
        assignments. Students are responsible for checking these course
        pages on a regular basis. Here, too, students are able to view
        their grades throughout the semester. For more information see
        Section~\ref{mytyndale}, below.
      \item[Mail]
        Students are responsible for information communicated through
        their campus mailboxes and student e-mail accounts. A mailbox
        directory hangs beside the mailboxes. For more information
        contact the Registrar's office.
    \end{description}

  \end{center}

  \section{Course Description}
  \label{description}

  \emph{From the Academic Calendar:} \cdescrip

\end{titlepage}
\setcounter{page}{2} % count the title page as page 1


  \section{Learning Objectives}
  \label{objectives}

  \edobject

\section{Required Texts \& Materials}
\label{texts}

The following texts are required. Students are strongly encouraged to
purchase their own copies. Library copies that are not reference works
will be placed on a 2-hour reserve.

\begingroup
\renewcommand{\section}[2]{}% temporarily remove the section heading
\begin{thebibliography}{Levenson}% use the longest item in the bibliography

	\bibitem[NRSV]{nrsv} M.\,D. Coogan, ed.
    \emph{New Oxford Annotated Bible with Apocrypha: NRSV}. 5th ed.
    Oxford: Oxford University Press, 2018.
    ISBN 978-0190276072.

	\bibitem[Myths]{myths} Stephanie Dalley, ed.
	\emph{Myths from Mesopotamia: Creation, The Flood, Gilgamesh, and Others}. Rev. ed.
	Oxford: Oxford University Press, 2000.
	ISBN: 978-0199538362.

	\bibitem[Levenson]{jl} Jon D. Levenson.
	\emph{Creation and the Persistence of Evil: The Jewish Drama of Divine Omnipotence}.
	Princeton: Princeton University Press, 1988 (repr. 1994).
	ISBN: 978-0691029504.

	\bibitem[Tanner]{kt} Kathryn Tanner.
	\emph{God and Creation in Christian Theology: Tyranny or Empowerment?}
	Oxford: Basil Blackwell, 1988 (repr. Minneapolis: Fortress Press, 2005).
	ISBN: 978-0800637378.

\end{thebibliography}
\endgroup

The NJPS is a good alternative study Bible: Adele Berlin and Marc Zvi
Brettler, eds., \emph{The Jewish Study Bible: Second Edition} (Oxford:
Oxford University Press, 2014). \href{https://danieldriver.com/courses/hb-3117/}{See
my website for links to required and recommended books, and a discussion
of options for a Bible.}

\section{Supplementary Texts}
\label{supplementary}

Supplementary readings will be recommended throughout the semester and
may be placed on reserve or made available through the course website.

The following titles are especially worth owning and consulting.
\cite{ed} explores the theology and ethics of land use. \cite{ms}
develops a suggestive reading of Genesis 1 within a comparative
framework. We will read excerpts from both books. Beyond the course
theme, \cite{rlgs} contains sound advice on core skills like reading
religious texts, writing essays and reviews, revising essays, making
oral presentations, and learning languages. Finally, \cite{sbl2} will
help you format your papers appropriately.

\begingroup
\renewcommand{\section}[2]{}% temporarily remove the section heading
\begin{thebibliography}{Making Sense}% use the longest item in the bibliography

	\bibitem[Davis]{ed} Ellen F. Davis.
	\emph{Scripture, Culture, and Agriculture: An Agrarian Reading of the Bible}.
	New York: Cambridge University Press, 2009.
	ISBN: 978-0521732239.

	\bibitem[Smith]{ms} Mark S. Smith.
	\emph{The Priestly Vision of Genesis 1}.
	Minneapolis: Fortress, 2010.
	ISBN: 978-0800663735.

	\bibitem[Making Sense]{rlgs} Northey, Margot, Bradford A. Anderson, and Joel N. Lohr.
	\emph{Making Sense in Religious Studies: A Student's Guide to Research and Writing}.
	3rd ed. Toronto: Oxford University Press, 2019. ISBN 978-0199026838.

	\bibitem[SBL2]{sbl2} Collins, Billie Jean, et al.
	\emph{The SBL Handbook of Style}.
	2nd ed. Atlanta: SBL Press, 2014. ISBN 978-1589839649.
	Designed to augment \href{http://www.chicagomanualofstyle.org/home.html}{\emph{Chicago Style}}
	(the standard at AST), there is also a free
	\href{https://www.sbl-site.org/assets/pdfs/pubs/SBLHSsupp2015-02.pdf}{Student Supplement for SBL2}.

\end{thebibliography}
\endgroup


\section{Course Outline}
\label{outline}

We will adhere to the schedule in \autoref{schedule} as closely as
possible, though the professor reserves the right to adjust it to suit
the needs of the class.

\newcommand\HBFB[1]{\cite[#1]{hbfb}}

\setcolumncount{4}% set up \sessioncount, \unit{}, \noclass{}, and \reminder{memo}{date} macros
\begin{table}[htbp]% set to `p' to put the schedule on its own page
  \centering
  \begin{tabular}{>{\sessioncount.}r@{ }llr}% make sure the column config agrees with \setcolumncount
	\toprule
	\sessionskip{\textbf{\S}.}&\textbf{Primary}&\textbf{Secondary}&\textbf{Due}\\
	\midrule

	\unit{Creation} \\

		& Genesis 1--2         &                            & 13 Sep. \\
		& Psalms 8; 19; 104    & \cite[chs 1--2]{kt}        & 20 Sep. \\
		& Prov 8; Wis of Sol 6 & \cite[chs 3--end]{kt}      & 27 Sep. \\
		& Isaiah 40--55        & \cite[ch 1]{ms}            &  4 Oct. \\
	\noclass{Thanksgiving}                                  & 11 Oct. \\
	\noclass{Reading Week}                                  & 18 Oct. \\
	\reminder{A first short paper is \textbf{due} by the end of week five}{} \\ [1ex]% NB I moved this back from wk 6 to wk 5 because of the unusual two week break in F21. Perhaps move back to wk 6 next time.

	\unit{Counter-Creation} \\

		& Genesis 3--5         & \cite[chs 1--4]{jl}        & 25 Oct. \\
		& \cite[pp 1--153]{myths}   & \cite[chs 5--9]{jl}   &  1 Nov. \\
		& \cite[pp 154--315]{myths} & \cite[chs 10--12]{jl} &  8 Nov. \\ [1ex]

	\unit{De-Creation} \\

		& Genesis 6--9         & self-directed              & 15 Nov. \\
		& Job; Psalms 74; 89   & self-directed              & 22 Nov. \\
		& Ecclesiastes 1--12   & self-directed              & 29 Nov. \\
	\reminder{A second short paper is \textbf{due} by the end of week eleven}{} \\ [1ex]

	\unit{Re-Creation} \\

		& Gen 10--14; Exod 1--15; 39--40 & self-directed    &  6 Dec. \\
		& Psalms 51; 146--150  & \cite[ch 9]{ed}            & 13 Dec. \\ [1ex]

	\reminder{End of Term: Final marks are due for all courses}{16 Dec.} \\

	\bottomrule
  \end{tabular}
  \caption{Schedule of Readings}
  \label{schedule}
\end{table}

See the AST website for a list of other \href{http://www.astheology.ns.ca/students/academic-dates.html}{important dates}.

\section{Evaluation}
\label{evaluation}

The grade structure for \ccode has the following elements.

\begin{enumerate}

	\item \textbf{Writing prompts} will be given throughout the
	semester. They are low-risk writing exercises designed to encourage
	careful reading of the assigned material, and to help explore its
	significance. Prompt responses are not marked, though points may be
	deducted if the exercise is not taken seriously. Prompts will
	usually be given live, near the beginning of each class. However, we
	may shift to online submissions if the format seems preferable
	(depending, in part, on how hybrid engagement works). If the prompts
	move to online forums, responses should not exceed 300 words.

	\item Two \textbf{short papers} will facilitate student engagement
	with the art of biblical interpretation. One is keyed to a
	theological theme, the other to a biblical text. Each should be
	about 3,000 words in length. The first paper is due at the end of
	the fifth week of class; the second is due at the end of the
	eleventh week.

	% I moved the first paper from wk 6 to wk 5 because of the unusual
	% two-week break in F21. Perhaps move it back to wk 6 next time.

	\begin{enumerate}

		\item A \textbf{theological essay} invites reflection on the
		parameters for Christian discourse about creation. How can and
		should we speak of the God of creation? The primary interlocutor
		for this paper is \cite{kt}. She outlines a number of rules for
		faithful speech about the transcendence and creative agency of
		God. Is her thesis persuasive? Is it right? How does it help?
		What are its limitations? Other sources may be engaged, if
		desired, but \cite{kt} must be engaged fully.

		You might think of this paper as a \emph{review essay} (note:
		this is not the same thing as a book report). Devote the first
		half of the paper to a summary the argument under review. Devote
		the second half of the paper to critical analysis and evaluation
		of the same. Be specific. In your analysis, select and discuss
		concrete examples from the work. Be fair, too, but do not fail
		to take a position. The paper needs to develop a
		\textbf{thesis}. See me and \cite[chs 3, 5–7, 11]{rlgs} for
		guidance.

		\item An \textbf{exegetical essay} provides an opportunity for
		direct work with the biblical text. The first task is to
		identify an appropriate text. Select a suitably short passage
		from the HB/OT. Then, conduct an analysis and explication of it.
		Interact with the assigned literature and at least five other
		sources or commentators. (Note that secondary reading from week
		eight is self-directed; you may also ask for suggestions.)
		Advance a \textbf{thesis} that relates to the chosen text. See
		me and \cite[chs 3, 5, 8, 11]{rlgs} for guidance.

	\end{enumerate}

	\item Before the last day of class, each student will upload a brief
	(5--7 minute) \textbf{video presentation} that summarizes and
	synthesizes their work in this course. First, by way of
	\emph{summary}, describe and contextualize the results of your
	exegetical paper. What text did you select and why? What did you
	discover about it through research and study? In a single sentence,
	what was your thesis? Second, by way of \emph{synthesis}, what might
	your exegetical work contribute to a renewed Christian theology of
	creation? You may revisit Tanner's argument here, if desired.
	Alternatively, how might your findings sharpen our understanding of
	the Bible's creation traditions?

	In essence, in this video you are distilling and representing the
	introduction and conclusion to your exegetical essay, but with a
	wider outlook and in a different medium. Elaborate with a view to
	the broad theme of creation. Answer the question: So what? Videos
	should be sharable by URL (try YouTube or Vimeo), and posted to
	Teams for the benefit of the class, to be viewed and discussed
	asynchronously. Finally, be sure not to exceed the time limit;
	points will be deducted for overlong submissions. To keep to time
	you will probably need to edit your video (recommended), or else to
	plan and rehearse your presentation well enough to capture it in a
	single take.

\end{enumerate}

The breakdown for the semester's total work is shown in
\autoref{grade-dist}.

\begin{table}[htbp]
  \centering
  {\lining
  \begin{tabular}{lr}
    \toprule
    Reading Prompts    & 20\% \\
    Review Essay       & 30\% \\
    Exegetical Essay   & 30\% \\
    Video Presentation & 20\% \\
    \bottomrule
  \end{tabular}}
  \caption{Distribution of Grades}
  \label{grade-dist}
\end{table}

%AST's \href{http://www.astheology.ns.ca/students/resources.html}{Academic
%Calendar} provides guidelines and detailed criteria for academic
%assessment. Marks are assigned by letter grade using these benchmarks.

\ProvidesFile{grades.tex}[2016/09/03 v2.0 -- Course policy]

\subsection{Grading System at AST}
\label{grades}

AST's \href{http://www.astheology.ns.ca/webfiles/AST_2016Calendar_web(A5)-06APR2016.pdf}{Academic
Calendar} provides guidelines and detailed criteria for academic
assessment. Marks are assigned by letter grade using the benchmarks in
\autoref{grade-syst}.

\begin{table}[htbp]
  \centering
  {\lining
  \begin{tabular}{lll}
    \toprule
%    Letter      & Percent & Assessment        \\
%	\midrule
    A+          & 94--100    & Exceptional    \\
    A           & 87--93     & Outstanding    \\
    A\char"2212 & 80--86     & Excellent      \\ [1ex]
    B+          & 77--79     & Good           \\
    B           & 73--76     & Acceptable     \\
    B\char"2212 & 70--72     & Marginal       \\ [1ex]
    C           & 60--69     & Unsatisfactory \\
    F           & 0--59      & Failure        \\
    FP          & 0          & Failure due to Plagiarism \\
    \bottomrule
  \end{tabular}}
  \caption{Summary of Grading System}
  \label{grade-syst}
\end{table}

% More detailed grading criteria from pp. 61--62 of `16.0406-I2-AST Academic Calendar.pdf'
%
%\begin{description}
%  \item[A+ (94-100) ‘Exceptional’]
%    A superior performance with consistent evidence of a comprehensive,
%    incisive grasp of all aspects of the subject matter; a very wide
%    knowledge base; insightful critical evaluation and analysis of the
%    material; an exceptional capacity for original, creative, and/or
%    logical thinking; an exceptional ability to organize, analyse,
%    synthesize, and to express thoughts fluently.
%  \item[A (87-93) ‘Outstanding’]
%    A comprehensive grasp of the subject matter, outstanding evidence of
%    original thought; sound critical evaluation of the material; an
%    excellent ability to organize, analyse, synthesize and to express
%    thoughts; mastery of an extensive knowledge base.
%  \item[A- (80-86) ‘Excellent’]
%    All the qualities of a B-level performance and an excellent capacity
%    for original, creative, and/ or logical thinking; excellent ability
%    to organize, analyse, synthesize, and integrate ideas; broad
%    knowledge base in the subject matter.
%  \item[B+ (77-79) ‘Good’]
%    A good performance with substantial knowledge of the subject matter;
%    a very good understanding of the relevant issues; familiarity with
%    relevant literature and techniques; good ability to organize,
%    analyse, and examine the material in a constructive and critical
%    manner.
%  \item[B (73-76) ‘Acceptable’]
%    A generally adequate performance with a good knowledge of the
%    subject matter; a fair understanding of relevant issues; some
%    ability to work with relevant literature and techniques; some
%    ability to develop solutions to difficult problems related to the
%    subject material.
%  \item[B- (70-72) ‘Marginally Acceptable’]
%    Some familiarity with the subject material; some understanding.
%    Satisfactory understanding of relevant issues; attempts to solve
%    moderately difficult problems related to the subject material in a
%    critical and analytical manner are only partially successful.
%  \item[C (60-69) ‘Unsatisfactory’]
%    A C grade indicates unsatisfactory academic performance. At the
%    discretion of the instructor, supplemental work may be negotiated to
%    upgrade the mark to a B range. A student may carry two C grades
%    without penalty in all courses except Foundations Courses,
%    Supervised Field Education, Supervised Ministry Practicum and the
%    Graduate Project. In these courses, a minimum grade of B- is
%    required to graduate. A student who receives a C in a Foundation
%    course must repeat the course to achieve a B- or better, and cannot
%    use the C grade to meet prerequisite requirements for advanced
%    courses. If the student repeats one of these courses and receives a
%    B- or better, the previous C grade remains on the transcript and can
%    be counted toward the total of unsatisfactory grades that may lead
%    to academic dismissal. Credit will be given only once for any
%    course. (See Policy on Unsatisfactory Academic Performance in the
%    AST Student Handbook.)
%  \item[F (0-59) ‘Failure’]
%    Student has not grasped subject matter; does not understand issues
%    involved; cannot work with relevant literature. (See Policy on
%    Unsatisfactory Academic Performance in the AST Student Handbook.)
%  \item[P ‘Pass’]
%    Credit awarded, but no mark assigned.
%  \item[FP ‘Failure due to Plagiarism’]
%    A student will receive this grade only after proven incident(s) of
%    plagiarism in a course.
%\end{description}
\ProvidesFile{other.tex}[2022/06/08 v2.9.1 -- Course policy]

\section{Other Course Policy}
\label{policy}

Late work will not be accepted, except in genuinely extenuating
circumstances. Students must submit something before the deadline if
they wish to receive credit. Unless I state otherwise, assignments are
to be uploaded by 11:59 \PM\ (Atlantic) on the date indicated.

Essay submissions must be typewritten and double-spaced. They should be
free from error. In this course they should follow SBL Style (see
\cite{sbl2} in \autoref{supplementary}, above). As a reminder, AST
upholds an Inclusive Language Policy. Please use gender-inclusive
language when referring to human beings. Our traditions have different
norms for speech about God; you are of course free to follow and explore
those traditions when referring to God.


Plagiarism is the
\href{http://www.eerdmans.com/Pages/Item/59043/Commentary-Statement.aspx}{failure}
to \href{https://www.theguardian.com/world/2013/feb/09/german-education-minister-quits-phd-plagiarism}{attribute}
(by means of footnotes when writing or aloud when speaking) any ideas,
phrases, sentences, materials, syntheses, et cetera, that another author
has composed and that you have borrowed for your own work. Plagiarism is
unethical. Academic penalties for plagiarism at AST are serious, and may
include failure of the course or even suspension of further studies.
Unintentional plagiarism is considered plagiarism. AST's Plagiarism
Policy is found under that heading in the Academic
Calendar.

Students should request permission to record a class or lecture. If
permission is granted, or if recordings are provided (as in the case of
an online or hybrid course), I stipulate that all recordings be for
personal use only. They may not be shared or distributed.

If you have needs that require modifications to any aspect of this
course, please consult with the instructor as soon as possible. Any
documentation regarding disabilities that you wish to divulge to AST
should be provided to the Registrar’s Office, where it will be kept in a
confidential file.

Finally, I encourage the conscientious use of laptops, tablets, and
other technology in my classes. In classroom settings, realize that, as
\href{http://dx.doi.org/10.1016/j.compedu.2012.10.003}{cognitive
psychologists have demonstrated}, ``laptop multitasking hinders
classroom learning for both users and nearby peers.'' Do your part to
foster an environment for dialogue by honouring the presence of your
classmates. In online and hybrid settings, consider both the physical
environment in which you choose to work and the virtual environment that
you help create through your participation in various forums. Let your
engagement in this course be marked by rigour and charity alike.


%\section{Bibliography}
%\label{bib}
%
%Literature on creation is vast. See the professor for suggestions on further reading.

\end{document}
