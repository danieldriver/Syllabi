% Copyright (c) 2018 by Daniel R. Driver.
% !TEX encoding = UTF-8 Unicode
% !TEX TS-program = XeLaTeX

\documentclass[titlepage]{article}

% This document presumes a file structure and set of inputs that are
% available at: git@github.com:danieldriver/syllabi.git

\newcommand\policy{../policy}
\newcommand\incl{../includes}
\ProvidesFile{variables.tex}[2018/05/24 v2.1 -- Syllabus variables]

\usepackage{xspace} % make manual spaces (like \mycmd\ ) unnecessary
\usepackage{xifthen} % provides \isempty test

% variables for internal use
\newcommand\prof{}
\newcommand\pdegree{}
\newcommand\pphone{}
\newcommand\pemail{}
\newcommand\poffice{}
\newcommand\phours{}
%
\newcommand\ccode{}
\newcommand\ctitle{}
\newcommand\cseries{}
\newcommand\cversion{}
\newcommand\csemester{}
\newcommand\cmeetson{}
\newcommand\cmeetsat{}
\newcommand\cmeetsin{}
\newcommand\cwebsite{}
\newcommand\cdescrip{}
\newcommand\cprereqs{}
\newcommand\edobject{}

% in case of fully online courses - https://tex.stackexchange.com/a/5896
\newif\ifonline
\newcommand\Int[2]{\ifonline#1\else#2\fi}

% commands for setting variables in the preamble
\newcommand\professor[2][PhD]{
  \renewcommand\pdegree{#1\xspace}
  \renewcommand\prof{#2\xspace}}
\newcommand\phone[1]{
  \renewcommand\pphone{\addfontfeatures{Numbers=Monospaced}#1\xspace}}
\newcommand\email[1]{
  \renewcommand\pemail{\href{mailto:#1}{#1}\xspace}}
\newcommand\officehours[2][Library, Room 5-North]{
  \renewcommand\poffice{#1\xspace}
  \renewcommand\phours{#2\xspace}}
%
\newcommand\coursecode[2][1.0]{
  \renewcommand\cversion{#1\Int{-i}{}\xspace}
  \renewcommand\ccode{#2\Int{(Int)}{}\xspace}}
\newcommand\coursetitle[2][]{
  \ifthenelse{\isempty{#1}}%
    {}% do nothing if #1 is empty, else:
    {\renewcommand\cseries{#1\\[1ex]}}
  \renewcommand\ctitle{#2\xspace}}
\newcommand\semester[1]{
  \renewcommand\csemester{#1\xspace}}
\newcommand\meets[3]{
  \newcommand\AM{\textsc{am}}
  \newcommand\PM{\textsc{pm}}
  \renewcommand\cmeetson{#1\xspace}
  \renewcommand\cmeetsat{\Int{From 9:00 \AM}{#2}\xspace}
  \renewcommand\cmeetsin{\Int{\href{https://smu.brightspace.com/d2l/login}{Brightspace}}{#3}\xspace}}
\newcommand\website[1]{
  \renewcommand\cwebsite{\href{http://#1}{#1}\xspace}}
\newcommand\cdescription[2][RM 1000 or GTRS 6000; and BF 1001]{
  \renewcommand\cprereqs{#1}
  \renewcommand\cdescrip{#2\par}}
\newcommand\objectives[1]{
  \renewcommand\edobject{#1\par}}


%\onlinetrue % \Int{true}{false}
\coursecode[3.1.1]{HB 2201}
\coursetitle[Hebrew Bible]{Beginning Biblical Hebrew I}

% Taught as HEBR 2013 in:
%   - Fall 2009
%   - Fall 2011
%   - Fall 2013
%   - Fall 2015
% Taught as HB 2201 in:
%   - Fall 2018

\professor{Daniel R. Driver}
\phone{902-425-7051}
\email{ddriver@astheology.ns.ca}
\officehours{Thursdays, 12:30--2:00 \PM}

\semester{Fall Term 2018}
\meets{Thursdays}% \meets{on}{at}{in}
      {2:00--4:30 \PM}
      {Classroom 3}
\website{danieldriver.com}
\cdescription[none. The course is required for HB 2202 and advanced biblical Hebrew]{% copy from the current Academic Calendar; [] for prereqs
	An introduction to the basic principles of biblical Hebrew with
	emphasis on morphology, phonology, and syntax, this course is for
	students who want to study the Old Testament in Hebrew. Students
	will learn basic Hebrew grammar, develop a rudimentary biblical
	Hebrew vocabulary, and begin to read the Hebrew Bible with one eye
	on the window it opens into ancient Israel and another on its
	historic role as as scripture in Judaism and Christianity. This
	course is suitable preparation for further study in religion,
	theology, or classics. (It is open to undergraduate and graduate
	students from other universities. Please contact the AST Registrar
	to enroll as a Letter of Permission student.)
}% end of course description
\objectives{% recall Bloom's taxonomy: http://www.celt.iastate.edu/teaching/RevisedBlooms1.html
	The basic goal of this course is to become proficient in the
	rudiments of biblical Hebrew. The introductory course is designed
	for two semesters of study, and it is presumed that students will
	take the course for the entire academic year. In the first semester
	students will learn everything from the alphabet to the basics of
	Hebrew nouns and verbs. In the second semester the emphasis will
	fall on less regular paradigms (learned inductively) and more
	complicated syntax. The steady acquisition of vocabulary will be
	emphasized across both semesters.

	Language learning is hard work. It can also be fun and gratifying.
	To aid in the enjoyment, we will sing songs together, memorize and
	recite passages of scripture, and assist one another in a variety of
	ways as we undertake the necessary drills and repetitions.
}% end of learning objectives

\ProvidesFile{preamble.tex}[2013/09/06 v1.0 -- Syllabus preamble]

% basic typography
\usepackage{fontspec}
\setmainfont[Ligatures=TeX]{Meta Serif Pro}
\setsansfont[Ligatures=TeX]{Meta Pro}
\newfontfamily\Heb{Meta Hebrew}
\setmonofont[Scale=MatchLowercase]{Menlo}
\usepackage{sectsty}
\allsectionsfont{\sffamily}
\frenchspacing
\setlength{\emergencystretch}{3em} % prevent overfull lines

% custom font size and leading
\renewcommand\tiny{\fontsize{6}{9}\selectfont}
\renewcommand\scriptsize{\fontsize{7}{10}\selectfont}
\renewcommand\footnotesize{\fontsize{8}{11}\selectfont}
\renewcommand\small{\fontsize{8.5}{11.5}\selectfont}
\renewcommand\normalsize{\fontsize{9}{12}\selectfont}% base size
\renewcommand\large{\fontsize{11}{14}\selectfont}
\renewcommand\Large{\fontsize{13}{16}\selectfont}
\renewcommand\LARGE{\fontsize{16}{19}\selectfont}% "course syllabus \\ semester" benefits from more lead
\renewcommand\huge{\fontsize{19}{21}\selectfont}
\renewcommand\Huge{\fontsize{24}{26}\selectfont}

% layout packages: page, logo, tables
\usepackage[scale={0.6,0.8},
            xetex]{geometry}
\usepackage{graphicx}
\usepackage{array}     % allow insertions of column styling with >{}
\usepackage{booktabs}  % elegant horizontal rules in tables
\usepackage{marginfix} % protect positioning of margin table in policy/grades

% custom macros for a session count in the schedule of readings
\newcounter{session}
\newcounter{columns}
\newcounter{courseunit}
\newcommand\setcolumncount[2][0]{ % optionally set count to other than 0,
  \setcounter{session}{#1}        % e.g. to -1, or to a standing count
  \setcounter{columns}{#2}}
\newcommand\sessioncount{\stepcounter{session}\arabic{session}}
\newcommand\sessionskip[1]{\multicolumn{1}{@{}r@{ }}{#1}}
\newcommand\unit[1]{\multicolumn{\thecolumns}{c}{%
  \scshape\stepcounter{courseunit}\roman{courseunit}. \MakeLowercase{#1}}}
\newcommand\noclass[1]{\multicolumn{1}{@{}l}{\itshape No Class: #1}}

% color to match Tyndale's branding
\usepackage[usenames]{xcolor}
% predefined: black, white, red, green, blue, cyan, magenta, yellow
\definecolor{TyndaleURLs}{HTML}{0062A0} % links on tyndale.ca
\definecolor{TyndaleBlue}{cmyk}{1,1,0,.32}
\definecolor{TyndaleGold}{cmyk}{0,.27,1,0}
\definecolor{TyndaleRed}{cmyk}{0,1,.99,.04}
\definecolor{TyndaleBlack}{cmyk}{0,0,0,1}
\definecolor{TyndaleGreen}{cmyk}{.45,0,1,.24}
\definecolor{TyndaleOrange}{cmyk}{0,.79,1,0}
\definecolor{TyndaleAqua}{cmyk}{.47,0,.24,0}
\definecolor{TyndaleYellow}{cmyk}{.03,.03,.35,0}

% metadata (assumes a host of definitions are made in the main file)
\usepackage[setpagesize=false,     % leave this to geometry
            hyperfootnotes=false,  % fragile and distracting
            xetex]{hyperref}
\hypersetup{breaklinks=true,       % allow link text to break across lines
            colorlinks=true,       % colorlinks resets pdfborder to 0 0 0
            urlcolor=TyndaleURLs,  % for external links
            linkcolor=TyndaleRed,  % for normal internal links
            citecolor=TyndaleGold, % for bibliographical citations in text
            pdfauthor={\prof},
            pdftitle={\ccode: \ctitle},
            pdfsubject={Tyndale UC, \csemester},
            pdfcreator={github.com/danieldriver/syllabus}}
\urlstyle{same}                    % don't use monospace font for urls

% custom footlines
\usepackage{fancyhdr}
\pagestyle{fancy} % turn it on
\fancyhf{}        % reset everything
\renewcommand{\headrulewidth}{0pt} % remove header line as well
\lfoot{\sffamily\scshape\footnotesize\MakeLowercase{\ctitle, v\cversion}}
\rfoot{\sffamily\scshape\footnotesize\MakeLowercase{\prof\quad\thepage}}

% gratuitous with custom title page, but useful as a fallback
\title{\ccode: \ctitle}
\author{\professor}
\date{\semester}


\begin{document}
\ProvidesFile{title.tex}[2013/09/06 v1.0 -- Syllabus title page]

\begin{titlepage}
  \begin{center}

    \LARGE\sffamily % set title elements in a large sans serif

    \begin{minipage}{\textwidth}
      \parbox[t]{0.5\textwidth}{
        \mbox{}\\[-13pt] % dummy line to align parboxes
        \includegraphics[width=0.5\textwidth]{.syllabus/includes/TyndaleUC}}
      \hfill
      \parbox[t]{0.4\textwidth}{
        \raggedleft Course Syllabus\\
        \csemester}
    \end{minipage}

    \vfill

    {\textsc{\MakeLowercase\ccode}\\[1ex]
      \bfseries\cseries\Huge\ctitle}

    \vfill

    \normalsize\rmfamily % switch back to body type

    \begin{tabular}{>{\bfseries}rl>{\bfseries}rl}
      \toprule
      Instructor & \prof, \pdegree & Course  & Version \cversion \\
      \midrule
      Phone      & \pphone         & Meets   & \cmeetson         \\
      Email      & \pemail         & Time    & \cmeetsat         \\
      Office     & \poffice        & Room    & \cmeetsin         \\
      Hours      & \phours         & Website & \cwebsite         \\
      \bottomrule
    \end{tabular}

    \vfill

    \begin{description}\small
      \item[Commuter Hotline]
        Class cancellations due to inclement weather or illness will
        be announced on the commuter hotline at \texttt{416.226.6620
        x2187}. Alternately, weather cancellation information is posted
        at \href{http://tyndale.ca/weather}{tyndale.ca/weather}.
      \item[MyTyndale.ca]
        This course may have materials stored on its website, such as
        handouts or readings that may be needed in order to complete
        assignments. Students are responsible for checking these course
        pages on a regular basis. Here, too, students are able to view
        their grades throughout the semester. For more information see
        Section~\ref{mytyndale}, below.
      \item[Mail]
        Students are responsible for information communicated through
        their campus mailboxes and student e-mail accounts. A mailbox
        directory hangs beside the mailboxes. For more information
        contact the Registrar's office.
    \end{description}

  \end{center}

  \section{Course Description}
  \label{description}

  \emph{From the Academic Calendar:} \cdescrip

\end{titlepage}
\setcounter{page}{2} % count the title page as page 1


  \section{Learning Objectives}
  \label{objectives}

  \edobject

\section{Required Texts \& Materials}
\label{texts}

The following texts are required. Students are strongly encouraged to
purchase their own copies.

\begingroup
\renewcommand{\section}[2]{}% temporarily remove the section heading
\begin{thebibliography}{BBH}% use the longest item in the bibliography

    \bibitem[BBH]{bbh}
    John A. Cook and Robert D. Holmstedt.
    \emph{Beginning Biblical Hebrew: A Grammar and Illustrated Reader}.
    Grand Rapids: Baker Academic, 2013. ISBN 978-0801048869. A helpful
    set of study aids are available \href{http://www.bakerpublishinggroup.com/books/beginning-biblical-hebrew/5629/students/esources}{online at bakeracademic.com}.

    \bibitem[TM]{tm}
    Takamitsu Muraoka.
    \emph{A Biblical Hebrew Reader: With an Outline Grammar}.
    Leuven: Peeters, 2017. ISBN 978-9042934900.

    \bibitem[BHS]{bhs}
    \emph{Biblia Hebraica Stuttgartensia}.
    Stuttgart: Deutsche Bibel\-ge\-sell\-schaft, 1997. ISBN 978-1598561630.
    A paperback edition, marketed to students, is also available.
    Students will be asked to have a criticial edition of the Hebrew
    Bible by the start of the Winter semester. The instructor will
    provide photocopies in the interim.

\end{thebibliography}
\endgroup

\section{Supplementary Texts}
\label{supplementary}

The following reference works are worth owning and consulting.
\cite{rlgs} in particular contains sound advice on core skills like
reading religious texts, writing essays and reviews, revising essays,
making oral presentations, and learning languages.

\begingroup
\renewcommand{\section}[2]{}% temporarily remove the section heading
\begin{thebibliography}{Making Sense}% use the longest item in the bibliography

	\bibitem[Making Sense]{rlgs} Northey, Margot, Bradford A. Anderson, and Joel N. Lohr.
	\emph{Making Sense in Religious Studies: A Student's Guide to Research and Writing}.
	3rd ed. Toronto: Oxford University Press, 2019. ISBN 978-0199026838.

	\bibitem[SBL2]{sbl2} Collins, Billie Jean, et al.
	\emph{The SBL Handbook of Style}.
	2nd ed. Atlanta: SBL Press, 2014. ISBN 978-1589839649. See the free
	\href{https://www.sbl-site.org/assets/pdfs/pubs/SBLHSsupp2015-02.pdf}{Student Supplement}.

\end{thebibliography}
\endgroup

\section{Course Outline}
\label{outline}

We will adhere to the schedule in \autoref{schedule} as closely as
possible, though the professor reserves the right to adjust it to suit
the needs of the class.

\setcolumncount{4}% set up \sessioncount, \unit{}, \noclass{}, and \reminder{memo}{date} macros
\begin{table}[htbp]% set to `p' to put the schedule on its own page
  \centering%\addfontfeature{ItalicFont=SBLBibLit}
  \begin{tabular}{@{}r<{.}@{ }llr}
    \toprule
    \bfseries L & \textbf{BBH Grammar Lesson} & \textbf{TM} & \textbf{Due Date} \\
    \midrule
    1   & \texthebrew{שׁלום}, The Consonants                          & \S1   & 13 Sept. \\
    2   & The Vowels                                                 & & \\
    3   & \texthebrew{שׁוא} (Shəva)                                   & & \\
    4   & \texthebrew{דגשׁ} (Dagesh)                                  & & \\ [1ex]
    5   & Subject Pronouns: Singular                                 & \S3–4 & 20 Sept. \\
    7   & Copular Clauses                                            & & \\
    6   & Nouns: Singular                                            & & \\ [1ex]
    8   & The Article and the Interrogative \texthebrew{ה}           & \S5–7 & 27 Sept. \\
    9   & \texthebrew{ל} of Possession: Singular                     & & \\
    r-1 & \texthebrew{מקרא א} (Reading 1)                            & & \\ [1ex]
    10  & Nouns: Plural and Dual                                     & \S8   & 4 Oct. \\
    11  & Subject Pronouns: Plural                                   & & \\ [1ex]
    12  & \texthebrew{ישׁ} and \texthebrew{אין}                        & \S9–10a & 11 Oct. \\
    13  & Conjunction \texthebrew{ו} and Prepositions \texthebrew{בּ},
         \texthebrew{כּ}, \texthebrew{ל}, \texthebrew{מן}                & & \\ [1ex]
    14  & \texthebrew{שׁאלות} (Questions)                             & & 18 Oct. \\
    r-2 & \texthebrew{מקרא ב} (Genesis 3\texthebrew{א})              & & \\
    \sessionskip{\rarr} & \textbf{Midterm Exam} (last half of class) & & \\ [1ex]
    \noclass{AST Fall Term Break}                                    & 25 Oct.  \\ [1ex]
    15  & Verbs: A Preview                                           & \S11 & 1 Nov. \\
    16  & \texthebrew{קל} Perfect Conjugation: Singular              &  & \\ [1ex]
    17  & \texthebrew{ל} of Possession: Plural                       & \S12–15 & 8 Nov. \\
    18  & Word Order                                                 & & \\
    r-3 & \texthebrew{מקרא ג} (Genesis 22\texthebrew{א})             & & \\ [1ex]
    19  & \texthebrew{קל} Perfect Conjugation: Plural                & \S16–17 & 15 Nov. \\
    20  & \texthebrew{סמיכות} (Bound Nouns)                           & & \\ [1ex]
    21  & The Irreal Use of the Perfect Conjugation                  & \S18–19 & 22 Nov. \\
    22  & Attached Pronouns: Singular                                & & \\
    23  & \texthebrew{קל} Imperfect Conjugation: Singular            & & \\ [1ex]
    r-4 & \texthebrew{מקרא ד} (Genesis 37\texthebrew{א})             &  & 29 Nov. \\
    24  & The Infinitive                                             & \S20 & \\
    25  & The Adverbial Infinitive                                   & & \\ [1ex]
    26  & Objects                                                    & \S21–22 & 6 Dec. \\
    r-5 & \texthebrew{מקרא ה} (Genesis 3\texthebrew{ב})              & & \\
    \sessionskip{\rarr} & \textbf{Final Exam} (last half of class)   & & \\
    \bottomrule
  \end{tabular}
  \caption{Schedule of Lessons \& Readings}
  \label{schedule}
\end{table}

See the AST website for a list of other \href{http://www.astheology.ns.ca/students/academic-dates.html}{important dates}.

\section{Evaluation}
\label{evaluation}

The grade structure for \ccode has the following elements.

\begin{enumerate}
 \item
    Students are expected to complete all lessons and exercises from
    \cite{bbh}, and to study and master all sections (\S) from
    \cite{tm}. \textbf{Quizzes} based on the weekly assignments will be
    given regularly. Some of these may come directly from \cite{bbh}.
 \item
   A cumulative \textbf{midterm exam} will be given in the last half of
   the sixth class.
 \item
   A comprehensive \textbf{final exam} will be given in the last half of
   the twelfth class.
\end{enumerate}

The breakdown for the semester's total work is shown in
\autoref{distribution}.

\begin{table}[htbp]
  \centering
  {\lining
  \begin{tabular}{lr}
    \toprule
    Quizzes      & 40\% \\
    Midterm Exam & 30\% \\
    Final Exam   & 30\% \\
    \bottomrule
  \end{tabular}}
  \caption{Distribution of Grades}
  \label{distribution}
\end{table}

\ProvidesFile{grades.tex}[2016/09/03 v2.0 -- Course policy]

\subsection{Grading System at AST}
\label{grades}

AST's \href{http://www.astheology.ns.ca/webfiles/AST_2016Calendar_web(A5)-06APR2016.pdf}{Academic
Calendar} provides guidelines and detailed criteria for academic
assessment. Marks are assigned by letter grade using the benchmarks in
\autoref{grade-syst}.

\begin{table}[htbp]
  \centering
  {\lining
  \begin{tabular}{lll}
    \toprule
%    Letter      & Percent & Assessment        \\
%	\midrule
    A+          & 94--100    & Exceptional    \\
    A           & 87--93     & Outstanding    \\
    A\char"2212 & 80--86     & Excellent      \\ [1ex]
    B+          & 77--79     & Good           \\
    B           & 73--76     & Acceptable     \\
    B\char"2212 & 70--72     & Marginal       \\ [1ex]
    C           & 60--69     & Unsatisfactory \\
    F           & 0--59      & Failure        \\
    FP          & 0          & Failure due to Plagiarism \\
    \bottomrule
  \end{tabular}}
  \caption{Summary of Grading System}
  \label{grade-syst}
\end{table}

% More detailed grading criteria from pp. 61--62 of `16.0406-I2-AST Academic Calendar.pdf'
%
%\begin{description}
%  \item[A+ (94-100) ‘Exceptional’]
%    A superior performance with consistent evidence of a comprehensive,
%    incisive grasp of all aspects of the subject matter; a very wide
%    knowledge base; insightful critical evaluation and analysis of the
%    material; an exceptional capacity for original, creative, and/or
%    logical thinking; an exceptional ability to organize, analyse,
%    synthesize, and to express thoughts fluently.
%  \item[A (87-93) ‘Outstanding’]
%    A comprehensive grasp of the subject matter, outstanding evidence of
%    original thought; sound critical evaluation of the material; an
%    excellent ability to organize, analyse, synthesize and to express
%    thoughts; mastery of an extensive knowledge base.
%  \item[A- (80-86) ‘Excellent’]
%    All the qualities of a B-level performance and an excellent capacity
%    for original, creative, and/ or logical thinking; excellent ability
%    to organize, analyse, synthesize, and integrate ideas; broad
%    knowledge base in the subject matter.
%  \item[B+ (77-79) ‘Good’]
%    A good performance with substantial knowledge of the subject matter;
%    a very good understanding of the relevant issues; familiarity with
%    relevant literature and techniques; good ability to organize,
%    analyse, and examine the material in a constructive and critical
%    manner.
%  \item[B (73-76) ‘Acceptable’]
%    A generally adequate performance with a good knowledge of the
%    subject matter; a fair understanding of relevant issues; some
%    ability to work with relevant literature and techniques; some
%    ability to develop solutions to difficult problems related to the
%    subject material.
%  \item[B- (70-72) ‘Marginally Acceptable’]
%    Some familiarity with the subject material; some understanding.
%    Satisfactory understanding of relevant issues; attempts to solve
%    moderately difficult problems related to the subject material in a
%    critical and analytical manner are only partially successful.
%  \item[C (60-69) ‘Unsatisfactory’]
%    A C grade indicates unsatisfactory academic performance. At the
%    discretion of the instructor, supplemental work may be negotiated to
%    upgrade the mark to a B range. A student may carry two C grades
%    without penalty in all courses except Foundations Courses,
%    Supervised Field Education, Supervised Ministry Practicum and the
%    Graduate Project. In these courses, a minimum grade of B- is
%    required to graduate. A student who receives a C in a Foundation
%    course must repeat the course to achieve a B- or better, and cannot
%    use the C grade to meet prerequisite requirements for advanced
%    courses. If the student repeats one of these courses and receives a
%    B- or better, the previous C grade remains on the transcript and can
%    be counted toward the total of unsatisfactory grades that may lead
%    to academic dismissal. Credit will be given only once for any
%    course. (See Policy on Unsatisfactory Academic Performance in the
%    AST Student Handbook.)
%  \item[F (0-59) ‘Failure’]
%    Student has not grasped subject matter; does not understand issues
%    involved; cannot work with relevant literature. (See Policy on
%    Unsatisfactory Academic Performance in the AST Student Handbook.)
%  \item[P ‘Pass’]
%    Credit awarded, but no mark assigned.
%  \item[FP ‘Failure due to Plagiarism’]
%    A student will receive this grade only after proven incident(s) of
%    plagiarism in a course.
%\end{description}
\ProvidesFile{other.tex}[2022/06/08 v2.9.1 -- Course policy]

\section{Other Course Policy}
\label{policy}

Late work will not be accepted, except in genuinely extenuating
circumstances. Students must submit something before the deadline if
they wish to receive credit. Unless I state otherwise, assignments are
to be uploaded by 11:59 \PM\ (Atlantic) on the date indicated.

Essay submissions must be typewritten and double-spaced. They should be
free from error. In this course they should follow SBL Style (see
\cite{sbl2} in \autoref{supplementary}, above). As a reminder, AST
upholds an Inclusive Language Policy. Please use gender-inclusive
language when referring to human beings. Our traditions have different
norms for speech about God; you are of course free to follow and explore
those traditions when referring to God.


Plagiarism is the
\href{http://www.eerdmans.com/Pages/Item/59043/Commentary-Statement.aspx}{failure}
to \href{https://www.theguardian.com/world/2013/feb/09/german-education-minister-quits-phd-plagiarism}{attribute}
(by means of footnotes when writing or aloud when speaking) any ideas,
phrases, sentences, materials, syntheses, et cetera, that another author
has composed and that you have borrowed for your own work. Plagiarism is
unethical. Academic penalties for plagiarism at AST are serious, and may
include failure of the course or even suspension of further studies.
Unintentional plagiarism is considered plagiarism. AST's Plagiarism
Policy is found under that heading in the Academic
Calendar.

Students should request permission to record a class or lecture. If
permission is granted, or if recordings are provided (as in the case of
an online or hybrid course), I stipulate that all recordings be for
personal use only. They may not be shared or distributed.

If you have needs that require modifications to any aspect of this
course, please consult with the instructor as soon as possible. Any
documentation regarding disabilities that you wish to divulge to AST
should be provided to the Registrar’s Office, where it will be kept in a
confidential file.

Finally, I encourage the conscientious use of laptops, tablets, and
other technology in my classes. In classroom settings, realize that, as
\href{http://dx.doi.org/10.1016/j.compedu.2012.10.003}{cognitive
psychologists have demonstrated}, ``laptop multitasking hinders
classroom learning for both users and nearby peers.'' Do your part to
foster an environment for dialogue by honouring the presence of your
classmates. In online and hybrid settings, consider both the physical
environment in which you choose to work and the virtual environment that
you help create through your participation in various forums. Let your
engagement in this course be marked by rigour and charity alike.



\section{Bibliography}
\label{bib}

Among the many Hebrew grammars and resources, note the following
especially:

\begin{itemize}
  \item Guides \& Aids

    \begin{itemize}
      \item
        Van Pelt, Miles. \emph{English Grammar to Ace Biblical Hebrew}.
        Grand Rapids: Zondervan, 2010. At AST: PE 1130 H5 V35 2010
      \item
        Einspahr, Bruce. \emph{Index to Brown, Driver \& Briggs Hebrew Lexicon}.
        Chicago: Moody Press, 1976. At AST: PJ 4833 B683 E35
      \item
        Landes, George M. \emph{Building Your Biblical Hebrew
        Vocabulary: Learning Words by Frequency and Cognate}.
        Atlanta: SBL, 2001. At AST: PJ 4845 L25 2001
      \item
        Scott, W.\,R., and H.\,P. Rüger. \emph{A Simplified Guide to BHS:
        Critical Apparatus, Masora, Accents, Unusual Letters \& Other
        Markings}. 3rd edition. N. Richland Hills, TX: BIBAL, 2007.
        At AST: BS 715 1977f
      \item
        Kelley, Page H., Daniel S. Mynatt, and Timothy G. Crawford.
        \emph{The Masorah of Biblia Hebraica Stuttgartensia:
        Introduction and Annotated Glossary}.
        Grand Rapids: Eerdmans, 1998. At AST: BS 718 K38 1998
    \end{itemize}

  \item Introductory Grammars

    \begin{itemize}
      \item
        Kittel, Bonnie, Vicki Hoffer, and Rebecca Abts Wright.
        \emph{Biblical Hebrew: Text and Workbook} [and Audio CD].
        2nd edition. New Haven: Yale, 2004.
      \item
        Martin, J.\,D. \emph{Davidson’s Introductory Hebrew Grammar}.
        27th edition. Edinburgh: T\&T Clark, 1993. At AST: PJ 4567 D37 1962
      \item
        Pratico, Gary and Miles Van Pelt. \emph{Basics of Biblical Hebrew Grammar}.
        Grand Rapids: Zondervan, 2001. At AST: PJ 4567.3 P73 2007
      \item
        Seow, C.\,L. \emph{A Grammar for Biblical Hebrew}.
        Nashville: Abingdon, 1995. At AST: PJ 4567 S424 1995
      \item
        Weingreen, J. \emph{A Practical Grammar for Classical Hebrew}.
        Oxford: Clarendon, 1959. At AST: PJ 4567 W4 1959
    \end{itemize}


  \item Syntaxes \& Reference Grammars

    \begin{itemize}
      \item
      	Arnold, Bill T., and John H. Choi. \emph{A Guide to Biblical Hebrew Syntax}.
      	Cambridge: Cambridge University Press, 2003. At AST: PJ 4701 A76 2003
      \item
      	Waltke, Bruce K., and M. O’Connor. \emph{An Introduction to Biblical Hebrew Syntax}.
      	Winona Lake, IN: Eisenbrauns, 1990.
      \item
        Williams, Ronald J., and  Beckman, John C. \emph{Williams' Hebrew Syntax}.
        Toronto: University of Toronto Press, 2007. At AST: PJ 4701 W5 2007
      \item
        Gesenius, Wilhelm, E. Kautsch and A.\,E. Cowley. \emph{Hebrew Grammar}.
        Oxford: Clarendon, 1910. \textbf{GKC} remains a standard Hebrew
        reference grammar in English, even though Cowley's translation
        -- of the 28th german edition -- is now over 100 years old. At
        AST: PJ 4564 G5 1910
      \item
      	Joüon, P., and T. Muraoka. \emph{A Grammar of Biblical Hebrew}.
      	2nd edition. Subsidia Biblica 27. Rome: Pontifical Biblical
      	Institute, 2008. One of the most complete and up-to-date
      	Hebrew grammars in English, \textbf{JM} (1st ed. 1991) was
      	revised from a French work first published by Paul Joüon in
      	1923. At AST: PJ 4567 J7613 1993
    \end{itemize}

  \item Lexicons

    \begin{itemize}
      \item
        Brown, Francis, S.\,R. Driver and Charles A. Briggs. \emph{The
        Brown-Driver-Briggs Hebrew and English Lexicon}. Peabody, Mass:
        Hendrickson, 2004. The \textbf{BDB} was originally published in 1906.
        On \href{https://archive.org/details/hebrewenglishlex00geseuoft}{archive.org}
        and at AST: PJ 4833 B68 1996
      \item
        Köhler, Ludwig, and Walter Baumgartner. \emph{The Hebrew
        and Aramaic Lexicon of the Old Testament}. Leiden: Brill,
        1994--2000.
        At AST: PJ 4833 K61813 1994
      \item
        Holladay, William L. \emph{A Concise Hebrew and Aramaic Lexicon
        of the Old Testament, Based Upon the Lexical Work of Ludwig
        Koehler and Walter Baumgartner}.
        Grand Rapids: Eerdmans, 1971.
        At AST: PJ 4833 H6 1971
      \item
        Clines, David J.\,A. \emph{The Dictionary of Classical Hebrew}.
        Sheffield: Sheffield Academic Press, 1993--2011.
        At AST: PJ 4833 D53 1993
    \end{itemize}

\end{itemize}


\end{document}
