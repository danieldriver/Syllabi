% Copyright (c) 2018 by Daniel R. Driver.
% !TEX encoding = UTF-8 Unicode
% !TEX TS-program = XeLaTeX

\documentclass[titlepage]{article}

% This document presumes a file structure and set of inputs that are
% available at: git@github.com:danieldriver/syllabi.git

\newcommand\policy{../policy}
\newcommand\incl{../includes}
\ProvidesFile{variables.tex}[2018/05/24 v2.1 -- Syllabus variables]

\usepackage{xspace} % make manual spaces (like \mycmd\ ) unnecessary
\usepackage{xifthen} % provides \isempty test

% variables for internal use
\newcommand\prof{}
\newcommand\pdegree{}
\newcommand\pphone{}
\newcommand\pemail{}
\newcommand\poffice{}
\newcommand\phours{}
%
\newcommand\ccode{}
\newcommand\ctitle{}
\newcommand\cseries{}
\newcommand\cversion{}
\newcommand\csemester{}
\newcommand\cmeetson{}
\newcommand\cmeetsat{}
\newcommand\cmeetsin{}
\newcommand\cwebsite{}
\newcommand\cdescrip{}
\newcommand\cprereqs{}
\newcommand\edobject{}

% in case of fully online courses - https://tex.stackexchange.com/a/5896
\newif\ifonline
\newcommand\Int[2]{\ifonline#1\else#2\fi}

% commands for setting variables in the preamble
\newcommand\professor[2][PhD]{
  \renewcommand\pdegree{#1\xspace}
  \renewcommand\prof{#2\xspace}}
\newcommand\phone[1]{
  \renewcommand\pphone{\addfontfeatures{Numbers=Monospaced}#1\xspace}}
\newcommand\email[1]{
  \renewcommand\pemail{\href{mailto:#1}{#1}\xspace}}
\newcommand\officehours[2][Library, Room 5-North]{
  \renewcommand\poffice{#1\xspace}
  \renewcommand\phours{#2\xspace}}
%
\newcommand\coursecode[2][1.0]{
  \renewcommand\cversion{#1\Int{-i}{}\xspace}
  \renewcommand\ccode{#2\Int{(Int)}{}\xspace}}
\newcommand\coursetitle[2][]{
  \ifthenelse{\isempty{#1}}%
    {}% do nothing if #1 is empty, else:
    {\renewcommand\cseries{#1\\[1ex]}}
  \renewcommand\ctitle{#2\xspace}}
\newcommand\semester[1]{
  \renewcommand\csemester{#1\xspace}}
\newcommand\meets[3]{
  \newcommand\AM{\textsc{am}}
  \newcommand\PM{\textsc{pm}}
  \renewcommand\cmeetson{#1\xspace}
  \renewcommand\cmeetsat{\Int{From 9:00 \AM}{#2}\xspace}
  \renewcommand\cmeetsin{\Int{\href{https://smu.brightspace.com/d2l/login}{Brightspace}}{#3}\xspace}}
\newcommand\website[1]{
  \renewcommand\cwebsite{\href{http://#1}{#1}\xspace}}
\newcommand\cdescription[2][RM 1000 or GTRS 6000; and BF 1001]{
  \renewcommand\cprereqs{#1}
  \renewcommand\cdescrip{#2\par}}
\newcommand\objectives[1]{
  \renewcommand\edobject{#1\par}}


%\onlinetrue % \Int{true}{false}
\coursecode[3.0]{HB 3113(Int)}
\coursetitle{The Ten Commandments}

% Taught as RLGS 3113 in:
%   - Winter 2012
% Taught as BSTH 3213 in:
%   - Winter 2015
% Taught as HB 3113(Int) in:
%   - Winter 2018

\professor{Daniel R. Driver}
\phone{902-425-7051}
\email{ddriver@astheology.ns.ca}
\officehours{Tuesdays, 2:00--4:00 \PM}

\semester{Winter Term 2018}
\meets{Mondays}% \meets{on}{at}{in}
      {From 9:00 \AM}
      {Online}
\website{astheology.ns.ca}
\cdescription{% copy from the current Academic Calendar
	As the first words spoken by God at Sinai and the leading terms of a
	covenant sealed with blood, the Ten Commandments are well known as
	the preeminent instance of biblical law. Less well appreciated is
	the complex legal and cultic context that gave rise to multiple
	versions of the Decalogue in ancient Israel, as seen in Exodus 20,
	Deuteronomy 5, Exodus 34, and Leviticus 19.

	This course examines the ancient origins of the Ten Commandments as
	a premier example of the Bible’s legal traditions. It also considers
	the weight that these laws have held for Jews, Christians, and
	others through an exploration of the Decalogue’s reception history,
	which includes modes of expression as diverse as liturgy, biblical
	commentary, fine art, film, and popular culture. Finally, with ample
	time to consider a commandment per week, the course invites students
	to reflect creatively on the ethical trajectory of the commandments,
	their place in contemporary religious life, and their stature in and
	beyond the Christian Church. \emph{Prerequisite: \textsc{BF} 1001.}
}% end of course description
\objectives{% recall Bloom's taxonomy: http://www.celt.iastate.edu/teaching/RevisedBlooms1.html
By the end of the course students should be able to:
	know the Ten Commandments in order;
	locate at least three versions of the Ten Commandments in the Bible;
	differentiate biblical versions by their contents;
	interpret each of the Commandments in their immediate literary context;
	integrate an understanding of the Ten Commandments with a broader view of biblical law;
	summarize the way different religious traditions count to Ten;
	compare the ways the Commandments have been interpreted in history;
	identify, evaluate, and report on contemporary connections with at least one of the Ten Commandments.
}% end of learning objectives

\ProvidesFile{preamble.tex}[2013/09/06 v1.0 -- Syllabus preamble]

% basic typography
\usepackage{fontspec}
\setmainfont[Ligatures=TeX]{Meta Serif Pro}
\setsansfont[Ligatures=TeX]{Meta Pro}
\newfontfamily\Heb{Meta Hebrew}
\setmonofont[Scale=MatchLowercase]{Menlo}
\usepackage{sectsty}
\allsectionsfont{\sffamily}
\frenchspacing
\setlength{\emergencystretch}{3em} % prevent overfull lines

% custom font size and leading
\renewcommand\tiny{\fontsize{6}{9}\selectfont}
\renewcommand\scriptsize{\fontsize{7}{10}\selectfont}
\renewcommand\footnotesize{\fontsize{8}{11}\selectfont}
\renewcommand\small{\fontsize{8.5}{11.5}\selectfont}
\renewcommand\normalsize{\fontsize{9}{12}\selectfont}% base size
\renewcommand\large{\fontsize{11}{14}\selectfont}
\renewcommand\Large{\fontsize{13}{16}\selectfont}
\renewcommand\LARGE{\fontsize{16}{19}\selectfont}% "course syllabus \\ semester" benefits from more lead
\renewcommand\huge{\fontsize{19}{21}\selectfont}
\renewcommand\Huge{\fontsize{24}{26}\selectfont}

% layout packages: page, logo, tables
\usepackage[scale={0.6,0.8},
            xetex]{geometry}
\usepackage{graphicx}
\usepackage{array}     % allow insertions of column styling with >{}
\usepackage{booktabs}  % elegant horizontal rules in tables
\usepackage{marginfix} % protect positioning of margin table in policy/grades

% custom macros for a session count in the schedule of readings
\newcounter{session}
\newcounter{columns}
\newcounter{courseunit}
\newcommand\setcolumncount[2][0]{ % optionally set count to other than 0,
  \setcounter{session}{#1}        % e.g. to -1, or to a standing count
  \setcounter{columns}{#2}}
\newcommand\sessioncount{\stepcounter{session}\arabic{session}}
\newcommand\sessionskip[1]{\multicolumn{1}{@{}r@{ }}{#1}}
\newcommand\unit[1]{\multicolumn{\thecolumns}{c}{%
  \scshape\stepcounter{courseunit}\roman{courseunit}. \MakeLowercase{#1}}}
\newcommand\noclass[1]{\multicolumn{1}{@{}l}{\itshape No Class: #1}}

% color to match Tyndale's branding
\usepackage[usenames]{xcolor}
% predefined: black, white, red, green, blue, cyan, magenta, yellow
\definecolor{TyndaleURLs}{HTML}{0062A0} % links on tyndale.ca
\definecolor{TyndaleBlue}{cmyk}{1,1,0,.32}
\definecolor{TyndaleGold}{cmyk}{0,.27,1,0}
\definecolor{TyndaleRed}{cmyk}{0,1,.99,.04}
\definecolor{TyndaleBlack}{cmyk}{0,0,0,1}
\definecolor{TyndaleGreen}{cmyk}{.45,0,1,.24}
\definecolor{TyndaleOrange}{cmyk}{0,.79,1,0}
\definecolor{TyndaleAqua}{cmyk}{.47,0,.24,0}
\definecolor{TyndaleYellow}{cmyk}{.03,.03,.35,0}

% metadata (assumes a host of definitions are made in the main file)
\usepackage[setpagesize=false,     % leave this to geometry
            hyperfootnotes=false,  % fragile and distracting
            xetex]{hyperref}
\hypersetup{breaklinks=true,       % allow link text to break across lines
            colorlinks=true,       % colorlinks resets pdfborder to 0 0 0
            urlcolor=TyndaleURLs,  % for external links
            linkcolor=TyndaleRed,  % for normal internal links
            citecolor=TyndaleGold, % for bibliographical citations in text
            pdfauthor={\prof},
            pdftitle={\ccode: \ctitle},
            pdfsubject={Tyndale UC, \csemester},
            pdfcreator={github.com/danieldriver/syllabus}}
\urlstyle{same}                    % don't use monospace font for urls

% custom footlines
\usepackage{fancyhdr}
\pagestyle{fancy} % turn it on
\fancyhf{}        % reset everything
\renewcommand{\headrulewidth}{0pt} % remove header line as well
\lfoot{\sffamily\scshape\footnotesize\MakeLowercase{\ctitle, v\cversion}}
\rfoot{\sffamily\scshape\footnotesize\MakeLowercase{\prof\quad\thepage}}

% gratuitous with custom title page, but useful as a fallback
\title{\ccode: \ctitle}
\author{\professor}
\date{\semester}


\begin{document}
\ProvidesFile{title.tex}[2013/09/06 v1.0 -- Syllabus title page]

\begin{titlepage}
  \begin{center}

    \LARGE\sffamily % set title elements in a large sans serif

    \begin{minipage}{\textwidth}
      \parbox[t]{0.5\textwidth}{
        \mbox{}\\[-13pt] % dummy line to align parboxes
        \includegraphics[width=0.5\textwidth]{.syllabus/includes/TyndaleUC}}
      \hfill
      \parbox[t]{0.4\textwidth}{
        \raggedleft Course Syllabus\\
        \csemester}
    \end{minipage}

    \vfill

    {\textsc{\MakeLowercase\ccode}\\[1ex]
      \bfseries\cseries\Huge\ctitle}

    \vfill

    \normalsize\rmfamily % switch back to body type

    \begin{tabular}{>{\bfseries}rl>{\bfseries}rl}
      \toprule
      Instructor & \prof, \pdegree & Course  & Version \cversion \\
      \midrule
      Phone      & \pphone         & Meets   & \cmeetson         \\
      Email      & \pemail         & Time    & \cmeetsat         \\
      Office     & \poffice        & Room    & \cmeetsin         \\
      Hours      & \phours         & Website & \cwebsite         \\
      \bottomrule
    \end{tabular}

    \vfill

    \begin{description}\small
      \item[Commuter Hotline]
        Class cancellations due to inclement weather or illness will
        be announced on the commuter hotline at \texttt{416.226.6620
        x2187}. Alternately, weather cancellation information is posted
        at \href{http://tyndale.ca/weather}{tyndale.ca/weather}.
      \item[MyTyndale.ca]
        This course may have materials stored on its website, such as
        handouts or readings that may be needed in order to complete
        assignments. Students are responsible for checking these course
        pages on a regular basis. Here, too, students are able to view
        their grades throughout the semester. For more information see
        Section~\ref{mytyndale}, below.
      \item[Mail]
        Students are responsible for information communicated through
        their campus mailboxes and student e-mail accounts. A mailbox
        directory hangs beside the mailboxes. For more information
        contact the Registrar's office.
    \end{description}

  \end{center}

  \section{Course Description}
  \label{description}

  \emph{From the Academic Calendar:} \cdescrip

\end{titlepage}
\setcounter{page}{2} % count the title page as page 1


  \section{Learning Objectives}
  \label{objectives}

  \edobject

\section{Required Texts \& Materials}
\label{texts}

The following texts are required. Students are strongly encouraged to
purchase their own copies. \href{http://danieldriver.com/courses/hb-3113/#required-texts-winter-2018}{Links to help you order the correct editions} are on the professor's website.

\begingroup
\renewcommand{\section}[2]{}% temporarily remove the section heading
\begin{thebibliography}{Reflections}% use the longest item in the bibliography

	\bibitem[Reflections]{Reflections} Carl E. Braaten and Christopher R. Seitz, eds.
	\emph{I Am the Lord Your God: Christian Reflections on the Ten Commandments}.
	Grand Rapids: Eerdmans, 2005.
	\textsc{isbn} 978-0802828125.

	\bibitem[Coogan]{Coogan} Michael Coogan.
    \emph{The Ten Commandments: A Short History of an Ancient Text}.
    New Haven: Yale University Press, 2014.
    \textsc{isbn} 978-0300178715.

	\bibitem[Centuries]{Centuries} Jeffrey P. Greenman and Timothy Larsen, eds.
	\emph{The Decalogue through the Centuries: From the Hebrew Scriptures to Benedict XVI}.
	Louisville: Westminster John Knox, 2012.
	\textsc{isbn} 978-0664234904.

	\bibitem[Miller]{Miller} Patrick D. Miller.
    \emph{The Ten Commandments (Interpretation)}.
    Louisville: Westminster John Knox, 2009.
    \textsc{isbn} 978-0664230555.

\end{thebibliography}
\endgroup

If you do not have access to a good study Bible, I recommend either the
NRSV (Michael Coogan, ed.) or the NJPS (Adele Berlin and Marc Zvi
Brettler, eds.), both published by Oxford University Press.

\section{Supplementary Texts}
\label{supplementary}

The following titles may be referenced as supplementary texts. They are
not required. Other supplementary materials may be provided through the
course website.

\begingroup
\renewcommand{\section}[2]{}% temporarily remove the section heading
\begin{thebibliography}{3}% use the longest item in the bibliography

	\bibitem{VanHarn} Roger E. Van Harn, ed.
	\emph{The Ten Commandments for Jews, Christians, and Others}.
	Grand Rapids: Eerdmans, 2007.
	\textsc{isbn} 978-0802829658.

	\bibitem{Heschel} Abraham Heschel.
	\emph{The Sabbath: Its Meaning for Modern Man}.
	New York: Farrar, Straus and Giroux, 1951 (repr. 2005).
	\textsc{isbn} 978-0374529758.

\end{thebibliography}
\endgroup

\section{Course Outline}
\label{outline}

We will adhere to the schedule in \autoref{schedule} as closely as
possible, though the professor reserves the right to adjust it to suit
the needs of the class.

Note that readings are listed by the textbook's key word and chapter
number, except in the case of \cite{Reflections}, where readings are
listed by the author's surname.

\setcolumncount{6}% set up \sessioncount, \unit{}, \noclass{}, and \reminder{memo}{date} macros
\begin{table}[htbp]% add p to put the schedule on its own page
  \centering
  \begin{tabular}{>{\sessioncount.}r@{ }llllr}% make sure the column config agrees with \setcolumncount
	\toprule
	\sessionskip{\textbf{\S.}}&\textbf{Topic}&\textbf{Exposition}&\textbf{History}&\textbf{Reflections}&\textbf{Date}\\
	\midrule
		& Introduction &              & Coogan 1        &              &  8 Jan. \\
		& First Word   & Miller Intro & Coogan 2–4      & D.\,B. Hart  & 15 Jan. \\
		& Second Word  & Miller 1     & Coogan 5–6      & C. Seitz     & 22 Jan. \\
		& Third Word   & Miller 2     & Coogan 7–end    & E. Radner    & 29 Jan. \\
		& Fourth Word  & Miller 3     & Centuries 1     & M. Bockmuehl &  5 Feb. \\
		& Fifth Word   & Miller 4     & Centuries 2–3   & P. Turner    & 12 Feb. \\
	\reminder{Review Essay: Due before midnight at the end of Week Six}{16 Feb.} \\
	\noclass{Reading Week}                                             & 19 Feb. \\
		& Sixth Word   & Miller 5     & Centuries 4–5   & W. Cavanaugh & 26 Feb. \\
		& Seventh Word & Miller 6     & Centuries 6–7   & R. Jenson    &  5 Mar. \\
		& Eighth Word  & Miller 7     & Centuries 8–9   &              & 12 Mar. \\
		& Ninth Word   & Miller 8     & Centuries 10–11 & Hütter; Braaten & 19 Mar. \\
		& Tenth Word   & Miller 9     & Centuries 12    & R.\,R. Reno  & 26 Mar. \\
	\reminder{Final Paper: Due before midnight at the end of Week Eleven}{30 Mar.} \\
	\noclass{Easter}                                                &  2 Apr. \\
		& Conclusion   & Miller –end  & Centuries 13    & Wilken; Meilaender & 9 Apr. \\
	\bottomrule
  \end{tabular}
  \caption{Schedule of Readings}
  \label{schedule}
\end{table}

See the AST website for a list of other \href{http://www.astheology.ns.ca/students/academic-dates.html}{important dates}.

\section{Evaluation}
\label{evaluation}

\subsection{Grade Structure for \ccode}
\label{structure}

\begin{enumerate}

	\item I will set \textbf{discussion prompts} for each of the Ten
	Commandments. They are intended to help you engage with the assigned
	reading material, and with your classmates. Prompt responses should
	be about 250 words long, and are due by the start of each class (9
	\AM\ Atlantic Time). You must also read through all other responses
	and post another 50 words or so as you interact with some (not all)
	classmates. Start your own thread by morning, and respond to other
	threads by the end of the class day.

	\item Students are to make a \textbf{creative presentation} on one
	of the Ten Commandments in the week in which that Commandment is
	under investigation. If you want to address the first commandment,
	for example, you need to complete your work for week two. The
	presentation may take the form of a podcast or video, and it may be
	anything that merits the adjective “creative” -- an interview, a
	dramatic piece, some form of engagement with culture, or anything
	else suitable that you envision. It should last 15 to 20 minutes,
	and it should feel like it deserves 25\% of your final grade. The
	only strict requirement is that the work be sharable by URL. (Try
	SoundCloud or YouTube.)

	\item A \textbf{review essay} on Coogan's \emph{The Ten
	Commandments} will be due at the end of the sixth week of class
	(Friday, 16 February 2018). It should be 2,500 words long, plus or
	minus 10\%. Roughly one third of the essay should be a summary of
	the work, while the remaining two thirds should be critical
	engagement and evaluation.

	\item A comprehensive \textbf{final paper} will give students an
	opportunity to explore one of the Ten Commandments in detail,
	complete with a thesis that involves original exposition and
	theological reflection. It is due at the end of the eleventh week of
	class (Friday, 30 March 2018). It should be 3,500 words long, plus
	or minus 10\% (the total count includes footnotes but not the final
	bibliography).

\end{enumerate}

The breakdown for the semester's total work is shown in
\autoref{grade-dist}.

\begin{table}[htbp]
  \centering
  {\lining
  \begin{tabular}{lr}
    \toprule
    Weekly Prompts & 25\% \\
    Presentation   & 25\% \\
    Review Essay   & 25\% \\
    Final Paper    & 25\% \\
    \bottomrule
  \end{tabular}}
  \caption{Distribution of Grades}
  \label{grade-dist}
\end{table}

\ProvidesFile{grades.tex}[2016/09/03 v2.0 -- Course policy]

\subsection{Grading System at AST}
\label{grades}

AST's \href{http://www.astheology.ns.ca/webfiles/AST_2016Calendar_web(A5)-06APR2016.pdf}{Academic
Calendar} provides guidelines and detailed criteria for academic
assessment. Marks are assigned by letter grade using the benchmarks in
\autoref{grade-syst}.

\begin{table}[htbp]
  \centering
  {\lining
  \begin{tabular}{lll}
    \toprule
%    Letter      & Percent & Assessment        \\
%	\midrule
    A+          & 94--100    & Exceptional    \\
    A           & 87--93     & Outstanding    \\
    A\char"2212 & 80--86     & Excellent      \\ [1ex]
    B+          & 77--79     & Good           \\
    B           & 73--76     & Acceptable     \\
    B\char"2212 & 70--72     & Marginal       \\ [1ex]
    C           & 60--69     & Unsatisfactory \\
    F           & 0--59      & Failure        \\
    FP          & 0          & Failure due to Plagiarism \\
    \bottomrule
  \end{tabular}}
  \caption{Summary of Grading System}
  \label{grade-syst}
\end{table}

% More detailed grading criteria from pp. 61--62 of `16.0406-I2-AST Academic Calendar.pdf'
%
%\begin{description}
%  \item[A+ (94-100) ‘Exceptional’]
%    A superior performance with consistent evidence of a comprehensive,
%    incisive grasp of all aspects of the subject matter; a very wide
%    knowledge base; insightful critical evaluation and analysis of the
%    material; an exceptional capacity for original, creative, and/or
%    logical thinking; an exceptional ability to organize, analyse,
%    synthesize, and to express thoughts fluently.
%  \item[A (87-93) ‘Outstanding’]
%    A comprehensive grasp of the subject matter, outstanding evidence of
%    original thought; sound critical evaluation of the material; an
%    excellent ability to organize, analyse, synthesize and to express
%    thoughts; mastery of an extensive knowledge base.
%  \item[A- (80-86) ‘Excellent’]
%    All the qualities of a B-level performance and an excellent capacity
%    for original, creative, and/ or logical thinking; excellent ability
%    to organize, analyse, synthesize, and integrate ideas; broad
%    knowledge base in the subject matter.
%  \item[B+ (77-79) ‘Good’]
%    A good performance with substantial knowledge of the subject matter;
%    a very good understanding of the relevant issues; familiarity with
%    relevant literature and techniques; good ability to organize,
%    analyse, and examine the material in a constructive and critical
%    manner.
%  \item[B (73-76) ‘Acceptable’]
%    A generally adequate performance with a good knowledge of the
%    subject matter; a fair understanding of relevant issues; some
%    ability to work with relevant literature and techniques; some
%    ability to develop solutions to difficult problems related to the
%    subject material.
%  \item[B- (70-72) ‘Marginally Acceptable’]
%    Some familiarity with the subject material; some understanding.
%    Satisfactory understanding of relevant issues; attempts to solve
%    moderately difficult problems related to the subject material in a
%    critical and analytical manner are only partially successful.
%  \item[C (60-69) ‘Unsatisfactory’]
%    A C grade indicates unsatisfactory academic performance. At the
%    discretion of the instructor, supplemental work may be negotiated to
%    upgrade the mark to a B range. A student may carry two C grades
%    without penalty in all courses except Foundations Courses,
%    Supervised Field Education, Supervised Ministry Practicum and the
%    Graduate Project. In these courses, a minimum grade of B- is
%    required to graduate. A student who receives a C in a Foundation
%    course must repeat the course to achieve a B- or better, and cannot
%    use the C grade to meet prerequisite requirements for advanced
%    courses. If the student repeats one of these courses and receives a
%    B- or better, the previous C grade remains on the transcript and can
%    be counted toward the total of unsatisfactory grades that may lead
%    to academic dismissal. Credit will be given only once for any
%    course. (See Policy on Unsatisfactory Academic Performance in the
%    AST Student Handbook.)
%  \item[F (0-59) ‘Failure’]
%    Student has not grasped subject matter; does not understand issues
%    involved; cannot work with relevant literature. (See Policy on
%    Unsatisfactory Academic Performance in the AST Student Handbook.)
%  \item[P ‘Pass’]
%    Credit awarded, but no mark assigned.
%  \item[FP ‘Failure due to Plagiarism’]
%    A student will receive this grade only after proven incident(s) of
%    plagiarism in a course.
%\end{description}
\ProvidesFile{other.tex}[2022/06/08 v2.9.1 -- Course policy]

\section{Other Course Policy}
\label{policy}

Late work will not be accepted, except in genuinely extenuating
circumstances. Students must submit something before the deadline if
they wish to receive credit. Unless I state otherwise, assignments are
to be uploaded by 11:59 \PM\ (Atlantic) on the date indicated.

Essay submissions must be typewritten and double-spaced. They should be
free from error. In this course they should follow SBL Style (see
\cite{sbl2} in \autoref{supplementary}, above). As a reminder, AST
upholds an Inclusive Language Policy. Please use gender-inclusive
language when referring to human beings. Our traditions have different
norms for speech about God; you are of course free to follow and explore
those traditions when referring to God.


Plagiarism is the
\href{http://www.eerdmans.com/Pages/Item/59043/Commentary-Statement.aspx}{failure}
to \href{https://www.theguardian.com/world/2013/feb/09/german-education-minister-quits-phd-plagiarism}{attribute}
(by means of footnotes when writing or aloud when speaking) any ideas,
phrases, sentences, materials, syntheses, et cetera, that another author
has composed and that you have borrowed for your own work. Plagiarism is
unethical. Academic penalties for plagiarism at AST are serious, and may
include failure of the course or even suspension of further studies.
Unintentional plagiarism is considered plagiarism. AST's Plagiarism
Policy is found under that heading in the Academic
Calendar.

Students should request permission to record a class or lecture. If
permission is granted, or if recordings are provided (as in the case of
an online or hybrid course), I stipulate that all recordings be for
personal use only. They may not be shared or distributed.

If you have needs that require modifications to any aspect of this
course, please consult with the instructor as soon as possible. Any
documentation regarding disabilities that you wish to divulge to AST
should be provided to the Registrar’s Office, where it will be kept in a
confidential file.

Finally, I encourage the conscientious use of laptops, tablets, and
other technology in my classes. In classroom settings, realize that, as
\href{http://dx.doi.org/10.1016/j.compedu.2012.10.003}{cognitive
psychologists have demonstrated}, ``laptop multitasking hinders
classroom learning for both users and nearby peers.'' Do your part to
foster an environment for dialogue by honouring the presence of your
classmates. In online and hybrid settings, consider both the physical
environment in which you choose to work and the virtual environment that
you help create through your participation in various forums. Let your
engagement in this course be marked by rigour and charity alike.


\end{document}
