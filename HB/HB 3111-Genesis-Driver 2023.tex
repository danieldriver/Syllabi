% Copyright (c) 2023 by Daniel R. Driver.
% !TEX encoding = UTF-8 Unicode
% !TEX TS-program = XeLaTeX

\documentclass[titlepage]{article}

% This document presumes a file structure and set of inputs that are
% available at: git@github.com:danieldriver/syllabi.git

\newcommand\policy{../policy}
\newcommand\incl{../includes}
\ProvidesFile{variables.tex}[2018/05/24 v2.1 -- Syllabus variables]

\usepackage{xspace} % make manual spaces (like \mycmd\ ) unnecessary
\usepackage{xifthen} % provides \isempty test

% variables for internal use
\newcommand\prof{}
\newcommand\pdegree{}
\newcommand\pphone{}
\newcommand\pemail{}
\newcommand\poffice{}
\newcommand\phours{}
%
\newcommand\ccode{}
\newcommand\ctitle{}
\newcommand\cseries{}
\newcommand\cversion{}
\newcommand\csemester{}
\newcommand\cmeetson{}
\newcommand\cmeetsat{}
\newcommand\cmeetsin{}
\newcommand\cwebsite{}
\newcommand\cdescrip{}
\newcommand\cprereqs{}
\newcommand\edobject{}

% in case of fully online courses - https://tex.stackexchange.com/a/5896
\newif\ifonline
\newcommand\Int[2]{\ifonline#1\else#2\fi}

% commands for setting variables in the preamble
\newcommand\professor[2][PhD]{
  \renewcommand\pdegree{#1\xspace}
  \renewcommand\prof{#2\xspace}}
\newcommand\phone[1]{
  \renewcommand\pphone{\addfontfeatures{Numbers=Monospaced}#1\xspace}}
\newcommand\email[1]{
  \renewcommand\pemail{\href{mailto:#1}{#1}\xspace}}
\newcommand\officehours[2][Library, Room 5-North]{
  \renewcommand\poffice{#1\xspace}
  \renewcommand\phours{#2\xspace}}
%
\newcommand\coursecode[2][1.0]{
  \renewcommand\cversion{#1\Int{-i}{}\xspace}
  \renewcommand\ccode{#2\Int{(Int)}{}\xspace}}
\newcommand\coursetitle[2][]{
  \ifthenelse{\isempty{#1}}%
    {}% do nothing if #1 is empty, else:
    {\renewcommand\cseries{#1\\[1ex]}}
  \renewcommand\ctitle{#2\xspace}}
\newcommand\semester[1]{
  \renewcommand\csemester{#1\xspace}}
\newcommand\meets[3]{
  \newcommand\AM{\textsc{am}}
  \newcommand\PM{\textsc{pm}}
  \renewcommand\cmeetson{#1\xspace}
  \renewcommand\cmeetsat{\Int{From 9:00 \AM}{#2}\xspace}
  \renewcommand\cmeetsin{\Int{\href{https://smu.brightspace.com/d2l/login}{Brightspace}}{#3}\xspace}}
\newcommand\website[1]{
  \renewcommand\cwebsite{\href{http://#1}{#1}\xspace}}
\newcommand\cdescription[2][RM 1000 or GTRS 6000; and BF 1001]{
  \renewcommand\cprereqs{#1}
  \renewcommand\cdescrip{#2\par}}
\newcommand\objectives[1]{
  \renewcommand\edobject{#1\par}}


%\bthtrue % \bth{true}{false}
%\onlinetrue % \Int{true}{false}
\coursecode[2.2]{HB 3111\bth{-BTh}{}}
\coursetitle[Hebrew Bible]{The Book of Genesis}

% Taught as RLGS 3213 (T&I: Genesis) in:
%   - Winter 2009
% Taught similar course BSTH 4403 (TOBITH: Creation and the OT) in:
%   - Fall 2013 (students dropped until it became a DS with the one left)
% Taught as HB 3111 in:
%   - Winter 2017
%   - Fall 2023 (hybrid; 4 grad students, 1 BTh, 2 ConEd)

\professor{Daniel R. Driver}
\phone{902-425-7051}
\email{daniel.driver@astheology.ns.ca}
\officehours{Thursdays, 2:00--4:30 \PM}

\semester{Fall 2023}
\meets{Tuesdays}% \meets{on}{at}{in}
      {9:00--11:30 \AM}
      {Flahiff Room/Zoom}
\website[https://danieldriver.com/courses/hb-3114/]{nytimes.com}
\cdescription{% copy from the current Academic Calendar
	From the creation of the world to the call of Abraham, and from the
	promise of children to the establishment of the people Israel, the book
	of Genesis is one of the greatest origin stories ever told. Biblical
	scholars today write about the “lost world” of Genesis, and so try to
	account for the genesis of Genesis. There is indeed much to learn about
	the history behind the Bible's first book. The book has given rise a
	major history of its own, too, and this history of interpretation has
	become an area of interest in recent research. The history of the text
	includes the question of how Genesis relates to the so-called Abrahamic
	religions, and so the book also invites ecumenical and interreligious
	study.

	This course will consider multiple approaches to the Book of Genesis. It
	seeks to integrate what can be known about worlds behind and before the
	text into a comprehensive theological vision that includes its role as
	Jewish and Christian scripture, from antiquity down through the present
	day. Students will have an opportunity to sample ancient and modern
	interpretations of the book, and to develop their own expositions of it.
}% end of course description
\objectives{% recall Bloom's taxonomy: http://www.celt.iastate.edu/teaching/effective-teaching-practices/revised-blooms-taxonomy
	By the end of the course students should be able to:
		recall and summarize events from the Genesis narratives in detail;
		characterize literarily the book's main figures;
		describe major models for the interpretation of Genesis, illustrating their application to particular texts;
		use and contextualize commentaries on Genesis from a number of different historical eras;
		reflect on the complex interplay of text, tradition, and interpretation;
		analyze claims made by different faith communities on the legacy of Abraham;
		critique specific arguments put forward by scholars about the theological meaning of Genesis;
		generate questions relevant to the interpretation of short passages from Genesis;
		construct an original argument in aid of the exposition of short passages from Genesis.
}% end of learning objectives

\ProvidesFile{preamble.tex}[2013/09/06 v1.0 -- Syllabus preamble]

% basic typography
\usepackage{fontspec}
\setmainfont[Ligatures=TeX]{Meta Serif Pro}
\setsansfont[Ligatures=TeX]{Meta Pro}
\newfontfamily\Heb{Meta Hebrew}
\setmonofont[Scale=MatchLowercase]{Menlo}
\usepackage{sectsty}
\allsectionsfont{\sffamily}
\frenchspacing
\setlength{\emergencystretch}{3em} % prevent overfull lines

% custom font size and leading
\renewcommand\tiny{\fontsize{6}{9}\selectfont}
\renewcommand\scriptsize{\fontsize{7}{10}\selectfont}
\renewcommand\footnotesize{\fontsize{8}{11}\selectfont}
\renewcommand\small{\fontsize{8.5}{11.5}\selectfont}
\renewcommand\normalsize{\fontsize{9}{12}\selectfont}% base size
\renewcommand\large{\fontsize{11}{14}\selectfont}
\renewcommand\Large{\fontsize{13}{16}\selectfont}
\renewcommand\LARGE{\fontsize{16}{19}\selectfont}% "course syllabus \\ semester" benefits from more lead
\renewcommand\huge{\fontsize{19}{21}\selectfont}
\renewcommand\Huge{\fontsize{24}{26}\selectfont}

% layout packages: page, logo, tables
\usepackage[scale={0.6,0.8},
            xetex]{geometry}
\usepackage{graphicx}
\usepackage{array}     % allow insertions of column styling with >{}
\usepackage{booktabs}  % elegant horizontal rules in tables
\usepackage{marginfix} % protect positioning of margin table in policy/grades

% custom macros for a session count in the schedule of readings
\newcounter{session}
\newcounter{columns}
\newcounter{courseunit}
\newcommand\setcolumncount[2][0]{ % optionally set count to other than 0,
  \setcounter{session}{#1}        % e.g. to -1, or to a standing count
  \setcounter{columns}{#2}}
\newcommand\sessioncount{\stepcounter{session}\arabic{session}}
\newcommand\sessionskip[1]{\multicolumn{1}{@{}r@{ }}{#1}}
\newcommand\unit[1]{\multicolumn{\thecolumns}{c}{%
  \scshape\stepcounter{courseunit}\roman{courseunit}. \MakeLowercase{#1}}}
\newcommand\noclass[1]{\multicolumn{1}{@{}l}{\itshape No Class: #1}}

% color to match Tyndale's branding
\usepackage[usenames]{xcolor}
% predefined: black, white, red, green, blue, cyan, magenta, yellow
\definecolor{TyndaleURLs}{HTML}{0062A0} % links on tyndale.ca
\definecolor{TyndaleBlue}{cmyk}{1,1,0,.32}
\definecolor{TyndaleGold}{cmyk}{0,.27,1,0}
\definecolor{TyndaleRed}{cmyk}{0,1,.99,.04}
\definecolor{TyndaleBlack}{cmyk}{0,0,0,1}
\definecolor{TyndaleGreen}{cmyk}{.45,0,1,.24}
\definecolor{TyndaleOrange}{cmyk}{0,.79,1,0}
\definecolor{TyndaleAqua}{cmyk}{.47,0,.24,0}
\definecolor{TyndaleYellow}{cmyk}{.03,.03,.35,0}

% metadata (assumes a host of definitions are made in the main file)
\usepackage[setpagesize=false,     % leave this to geometry
            hyperfootnotes=false,  % fragile and distracting
            xetex]{hyperref}
\hypersetup{breaklinks=true,       % allow link text to break across lines
            colorlinks=true,       % colorlinks resets pdfborder to 0 0 0
            urlcolor=TyndaleURLs,  % for external links
            linkcolor=TyndaleRed,  % for normal internal links
            citecolor=TyndaleGold, % for bibliographical citations in text
            pdfauthor={\prof},
            pdftitle={\ccode: \ctitle},
            pdfsubject={Tyndale UC, \csemester},
            pdfcreator={github.com/danieldriver/syllabus}}
\urlstyle{same}                    % don't use monospace font for urls

% custom footlines
\usepackage{fancyhdr}
\pagestyle{fancy} % turn it on
\fancyhf{}        % reset everything
\renewcommand{\headrulewidth}{0pt} % remove header line as well
\lfoot{\sffamily\scshape\footnotesize\MakeLowercase{\ctitle, v\cversion}}
\rfoot{\sffamily\scshape\footnotesize\MakeLowercase{\prof\quad\thepage}}

% gratuitous with custom title page, but useful as a fallback
\title{\ccode: \ctitle}
\author{\professor}
\date{\semester}


\begin{document}
\ProvidesFile{title.tex}[2013/09/06 v1.0 -- Syllabus title page]

\begin{titlepage}
  \begin{center}

    \LARGE\sffamily % set title elements in a large sans serif

    \begin{minipage}{\textwidth}
      \parbox[t]{0.5\textwidth}{
        \mbox{}\\[-13pt] % dummy line to align parboxes
        \includegraphics[width=0.5\textwidth]{.syllabus/includes/TyndaleUC}}
      \hfill
      \parbox[t]{0.4\textwidth}{
        \raggedleft Course Syllabus\\
        \csemester}
    \end{minipage}

    \vfill

    {\textsc{\MakeLowercase\ccode}\\[1ex]
      \bfseries\cseries\Huge\ctitle}

    \vfill

    \normalsize\rmfamily % switch back to body type

    \begin{tabular}{>{\bfseries}rl>{\bfseries}rl}
      \toprule
      Instructor & \prof, \pdegree & Course  & Version \cversion \\
      \midrule
      Phone      & \pphone         & Meets   & \cmeetson         \\
      Email      & \pemail         & Time    & \cmeetsat         \\
      Office     & \poffice        & Room    & \cmeetsin         \\
      Hours      & \phours         & Website & \cwebsite         \\
      \bottomrule
    \end{tabular}

    \vfill

    \begin{description}\small
      \item[Commuter Hotline]
        Class cancellations due to inclement weather or illness will
        be announced on the commuter hotline at \texttt{416.226.6620
        x2187}. Alternately, weather cancellation information is posted
        at \href{http://tyndale.ca/weather}{tyndale.ca/weather}.
      \item[MyTyndale.ca]
        This course may have materials stored on its website, such as
        handouts or readings that may be needed in order to complete
        assignments. Students are responsible for checking these course
        pages on a regular basis. Here, too, students are able to view
        their grades throughout the semester. For more information see
        Section~\ref{mytyndale}, below.
      \item[Mail]
        Students are responsible for information communicated through
        their campus mailboxes and student e-mail accounts. A mailbox
        directory hangs beside the mailboxes. For more information
        contact the Registrar's office.
    \end{description}

  \end{center}

  \section{Course Description}
  \label{description}

  \emph{From the Academic Calendar:} \cdescrip

\end{titlepage}
\setcounter{page}{2} % count the title page as page 1


\section{Learning Objectives}
\label{objectives}
\edobject

\section{Required Texts \& Materials}
\label{texts}

In addition to the Bible in a modern translation such as the NJPS and
NRSV, the following text is required. Students are strongly encouraged
to purchase their own copies. The AST Library copy ({\scshape BS 1235.53
C36 2022}) will be placed on a 2-hour reserve.

\begingroup
\renewcommand{\section}[2]{}% temporarily remove the section heading
\begin{thebibliography}{CCG}% use the longest item in the bibliography

	\bibitem[CCG]{ccg} Arnold, Bill T., ed.
	\emph{The Cambridge Companion to Genesis}.
	Cambridge: Cambridge University Press, 2022.

%	\bibitem[NRSV]{nrsv} M.\,D. Coogan, ed.
%    \emph{New Oxford Annotated Bible with Apocrypha: NRSV}. 4th ed.
%    Oxford / New York: Oxford University Press, 2010.
%    \textsc{isbn} 978-0195289602.

%	\bibitem[NJPS]{njps}
%	Adele Berlin and Marc Zvi Brettler, eds.
%	\emph{The Jewish Study Bible: Second Edition}.
%	Oxford / New York: Oxford University Press, 2014.
%	\textsc{isbn} 978-0199978465.

\end{thebibliography}
\endgroup

\section{Supplementary Texts}
\label{supplementary}

Supplementary readings will be recommended throughout the semester.
Excerpts from this literature may either be placed on reserve or made
available for download through the course website. Outstanding students
will also pursue leads on their own, using library databases and
pursuing notes and bibliographies in the required reading.

\begin{multicols}{3}[Major modern commentaries on Genesis include:]%
\footnotesize\noindent
M.\,M. Kalisch (1858)\\
J.\,P. Lange (ET 1868)\\
F. Delitzsch (\textsuperscript{5}1899)\\
A. Dillmann (\textsuperscript{6}1892)\\
H. Holzinger (1898)\\
S.\,R. Driver (1904)\\
J. Skinner (1910, \textsuperscript{2}1930)\\
H. Gunkel (\textsuperscript{4}1917)\\
O. Procksch (\textsuperscript{3}1924)\\
B. Jacob (1934)\\
A. Richardson (1953)\\
A. Clamer (1953)\\
U. Cassuto (1961--64)\\
E.\,A. Speiser (1964)\\
W. Zimmerli (\textsuperscript{3}1967, 1976)\\
D. Kidner (1967)\\
G. von Rad (ET \textsuperscript{2}1972)\\
R. Davidson (1973)\\
C. Westermann (1974)\\
B. Vawter (1977)\\
N. Sarna (1989)\\
H. Seebass (1996--99)\\
J. Ebach [37--50, HthKAT] (2007)\\
B. Arnold (2009)\\
L. Ruppert (1992--2008)\\
S. Brayford [Septuagint] (2009)\\
J. Blenkinsopp (2011)\\
C. Dohmen [1--11] (2017)\\
J.\,Chr. Gertz [1--11] (2018, \textsuperscript{2}2021)\\
D. Carr [1--11] (2021)
\end{multicols}

\begin{multicols}{2}[Commentaries on Genesis in the AST Library include:]%
\footnotesize\noindent
Targums: {\scshape BS 709.2 B5 1987 vv. 1A, 1B, 6} (c. 150--?)\\
Didymus the Blind: {\scshape BS 1235 D4913 2016} (d. 398)\\
Augustine of Hippo: {\scshape BS 1235 A8413 1991} (d. 430)\\
Bede, the Venerable: {\scshape BS 1235 B43 2008} (d. 735)\\
ACCS, Genesis 1--11: {\scshape BS 1235.3 G46 2001}\\
ACCS, Genesis 12--50: {\scshape BS 1235.53 G46 2002}\\
RCS, Genesis 1--11: {\scshape BS 1235.53 G455 2012}\\
Luther, Martin: {\scshape BR 330 E5 1955 vv.1--8} (d. 1546)\\
Calvin, John: {\scshape BS 1235 C293 1948} (d. 1564)\\
Patrick, Simon: {\scshape BS 1235 P36 1695}\\
Henry, Matthew: {\scshape BS 490 H4 1961} (d. 1714)\\
Hershon, Paul Isaac: {\scshape BS 1235 H47} (1883)\\
Delitzsch, Franz: {\scshape BS 1235 D4} (1888)\\
Skinner, John: {\scshape BS 1235 S45 1917, 1963} (1930)\\
Richardson, Alan: {\scshape BS 1235 R66 1959}\\
Cassuto, Umberto: {\scshape BS 1235.3 C3} (1961)\\
Herbert, Arthur: {\scshape BS 1235.3 H4} (1962)\\
Speiser, E. A.: {\scshape BS 1233 S64 1964}\\
Kidner, Derek: {\scshape BS 1235.3 K47} (1967)\\
Rad, Gerhard von: {\scshape BS 1235.3 R3213 1972}\\
Plaut, Gunther: {\scshape BS 1225.3 P55 v. 1} (1974)\\
Vawter, Bruce: {\scshape BS 1235.3 V38 1977}\\
Davidson, Robert: {\scshape BS 1235.3 D3 1973, 1979}\\
Leibowitz, Nehama: {\scshape BS 1235.3 L413 1981}\\
Wenham, Gordon : {\scshape BS 491.2 W67 1982}\\
Brueggemann, Walter: {\scshape BS 1235.3 B78} (1982)\\
Maher, Michael: {\scshape BS 1235.3 M346 1982}\\
Westermann, Claus: {\scshape BS 1235.3 W43213 1986, 1987, 1994}\\
Sarna, Nahum: {\scshape BS 1235.3 S325 1989}\\
Hamilton, Victor: {\scshape BS 1235.3 H32 1990, 1995}\\
Scullion, John: {\scshape BS 1235.3 S37 1992}\\
Fox, Everett: {\scshape BS 1223 A3 F68 1995}\\
Hamilton, Victor: {\scshape BS 1235.3 H323 1995}\\
Hartley, John: {\scshape BS 1235.3 H37 2000}\\
Towner, Sibley: {\scshape BS 1235.3 T69 2001}\\
Walton, John: {\scshape BS 1235.53 W35 2001}\\
Cotter, David: {\scshape BS 1235.52 C68 2003}\\
Briscoe, Stuart: {\scshape BS 1151.2 C66 2004}\\
McKeown, James: {\scshape BS 1235.53 M35 2008}\\
Arnold, Bill: {\scshape BS 1235.53 A76 2009}\\
Goldingay, John: {\scshape BS 1235.53 G65 2010}\\
Reno, Russell: {\scshape BS 1235.53 R46 2010}\\
Cook, Joan: {\scshape BS 1235.53 C66 2011}\\
De La Torre, Miguel: {\scshape BS 1235.53 D4 2011}\\
Coleson, Joseph: {\scshape BS 1235.53 C64 2012}\\
Longman, Tremper: {\scshape BS 1235.53 L66 2016}
\end{multicols}

Also, the following basic works are also worth consulting and even owning.
\cite{rlgs} in particular contains sound advice on core skills like
reading religious texts, writing essays and reviews, revising essays,
making oral presentations, and learning languages.

\begingroup
\renewcommand{\section}[2]{}% temporarily remove the section heading
\begin{thebibliography}{Making}% use the longest item in the bibliography

	\bibitem[Making]{rlgs} Northey, Margot, Bradford A. Anderson, and Joel N. Lohr.
	\emph{Making Sense in Religious Studies: A Student's Guide to Research and Writing}.
	3rd ed. Toronto: Oxford University Press, 2019. AST Library: Reference BL 41 N67 2019.

	\bibitem[SBL2]{sbl2} Collins, Billie Jean, et al.
	\emph{The SBL Handbook of Style}.
	2nd ed. Atlanta: SBL Press, 2014.
	Designed to augment \href{https://proxy.openathens.net/login?qurl=https%3A%2F%2Fwww.chicagomanualofstyle.org%2Fbook%2Fed17%2Ffrontmatter%2Ftoc.html}{\emph{Chicago Style}}
	(the standard at AST), there is also a free
	\href{https://www.sbl-site.org/assets/pdfs/pubs/SBLHSsupp2015-02.pdf}{Student Supplement for SBL2}. AST Library: Reference PN 147 S26 2014.

\end{thebibliography}
\endgroup

\section{Course Outline}
\label{outline}

We will adhere to the schedule in \autoref{schedule} as closely as
possible, though the professor reserves the right to adjust it to suit
the needs of the class.

\setcolumncount{4}% set up \sessioncount, \unit{}, \noclass{}, and \reminder{memo}{date} macros
\begin{table}[htbp]% set to `p' to put the schedule on its own page
  \centering
  \begin{tabular}{>{\sessioncount.}r@{ }llr}% make sure the column config agrees with \setcolumncount
	\toprule
	\sessionskip{\textbf{\S}.}&\textbf{Primary}&\textbf{Secondary}&\textbf{Date}\\
	\midrule

	\unit{Primeval History} \\
		& Genesis 1      & Course syllabus                  & 12 Sep. \\
		& Genesis 2--3   & Arnold, Strawn \cite[1, 10]{ccg} & 19 Sep. \\
		& Genesis 4--9   & Ska, Gertz \cite[2, 3]{ccg}      & 26 Sep. \\
		& Genesis 10--11 & Levin, Bauks \cite[4, 5]{ccg}    &  3 Oct. \\ [1ex]

	\unit{Ancestral Narratives} \\
		& Genesis 12--15 & Mandell, Walton \cite[6, 7]{ccg} & 10 Oct. \\
		& Genesis 16--20 & van der Meer \cite[11]{ccg}      & 17 Oct. \\
	\noclass{Reading Week (Tuesday to Friday)}              & 24--27 Oct. \\
	\reminder{Review essay is \textbf{due before} the seventh week of class}{27 Oct.} \\
		& Genesis 21--25 & Gould, Aquino \cite[13, 14]{ccg} & 31 Oct. \\
		& Genesis 26--30 & Steinberg \cite[8]{ccg}          &  7 Nov. \\
		& Genesis 31--36 & Shectman \cite[9]{ccg}           & 14 Nov. \\ [1ex]

	\unit{The Joseph Story} \\
		& Genesis 37--41 & Otto \cite[12]{ccg}              & 21 Nov. \\
		& Genesis 42--45 & Kaminsky \cite[15]{ccg}          & 28 Nov. \\
	\reminder{Exegetical essay is \textbf{due before} the twelfth week of class}{1 Dec.} \\
		& Genesis 46--50 & Provan \cite[16]{ccg}            &  5 Dec. \\

	\reminder{End of Term: Final marks are due for all courses}{14 Dec.} \\

	\bottomrule
  \end{tabular}
  \caption{Schedule of Readings}
  \label{schedule}
\end{table}

See the AST website for a list of other \href{http://www.astheology.ns.ca/students/academic-dates.html}{important dates}.

\section{Evaluation and Grade Structure}
\label{evaluation}

\begin{enumerate}

	\item \textbf{Notes and quotes} will be solicited from students at the
	start of each class. These are to be drawn from the primary and
	secondary readings assigned for each week. What do you note about
	Genesis on this occasion? What words, phrases, or verses stand out to
	you from your translation? Why so? Also, what ideas or formulations
	strike you from the contributors to \cite{ccg}? Are they useful to you
	as experts and “companions”? How do their discussions seem to help or
	impede your reading of Genesis?

	\item Student input in \textbf{seminar discussions} will be evaluated on
	the basis of two elements.

	\begin{enumerate}

		\item Students will select a commentator or major version for a
		period of 3--5 weeks, during which time they are responsible to
		report on it in class and bring its insight to bear on the seminar
		discussions. Recognizing that everyone will be working with
		different resources from the exegetical tradition, come prepared
		each week to be the resident expert on your particular commentary.

		\item \bth{Once in the term, BTh students will lead a 20-minute
		seminar on a single chapter}{Twice in the term, graduate
		students will lead 20-minute seminars on single chapters} from
		the book of Genesis. They should take the lead by briefly
		outlining the chapter in question, situating it relative to the
		rest of Genesis, and reporting on their research into part of
		that chapter. Then, in the balance of the time, they should
		helping the class elaborate, reflect on, test, challenge, or
		extend a central idea or two. Students are encouraged to align
		\bth{their seminar text}{at least one of their seminar texts}
		with their work on their exegetical paper.

	\end{enumerate}

	\item Two \textbf{papers} will facilitate student engagement with the
	art of biblical interpretation. One is keyed to the critical analysis of
	a work of post-biblical interpretation, the other to the exposition of a
	short biblical text.

	\begin{enumerate}

		\item A \textbf{review essay} invites student reflection on a
		model work of biblical interpretation. Some pre-approved options
		are listed as further readings in \autoref{bib}. Alternatively,
		you may propose a different title to the professor, subject to
		approval. The review essay should be \textbf{\bth{2,500}{3,000}
		words} in length.

		Note that a review is not the same thing as a report. Devote the
		first half of the paper to a summary the interpretation or argument
		under review. Devote the second half of the paper to critical
		analysis and evaluation of your chosen example. Be fair, but do not
		fail to take a position. The paper needs to develop a
		\textbf{thesis}. See me and \cite[chs 3, 5–7, 11]{rlgs} for
		guidance.

		\item An \textbf{exegetical essay} provides an opportunity for
		direct work with the biblical text. The first task is to
		identify an appropriate text. Select a suitably short passage
		from Genesis. Then, conduct an analysis and explication of it.
		Interact with at least five sources and commentators. Advance a
		\textbf{thesis} that relates to the text itself. The exegetical
		essay should be \textbf{\bth{3,300}{4,000} words} in length. See
		me and \cite[chs 3, 5, 8, 11]{rlgs} for guidance.

	\end{enumerate}

\end{enumerate}

The breakdown for the semester's total work is shown in
\autoref{grade-dist}.

\begin{table}[htbp]
  \centering
  {\lining
  \begin{tabular}{lr}
    \toprule
    Notes \& Quotes     & 10\% \\
    Seminar Discussions & 20\% \\
    Review Essay        & 30\% \\
    Exegetical Essay    & 40\% \\
    \bottomrule
  \end{tabular}}
  \caption{Distribution of Grades}
  \label{grade-dist}
\end{table}

\ProvidesFile{grades.tex}[2016/09/03 v2.0 -- Course policy]

\subsection{Grading System at AST}
\label{grades}

AST's \href{http://www.astheology.ns.ca/webfiles/AST_2016Calendar_web(A5)-06APR2016.pdf}{Academic
Calendar} provides guidelines and detailed criteria for academic
assessment. Marks are assigned by letter grade using the benchmarks in
\autoref{grade-syst}.

\begin{table}[htbp]
  \centering
  {\lining
  \begin{tabular}{lll}
    \toprule
%    Letter      & Percent & Assessment        \\
%	\midrule
    A+          & 94--100    & Exceptional    \\
    A           & 87--93     & Outstanding    \\
    A\char"2212 & 80--86     & Excellent      \\ [1ex]
    B+          & 77--79     & Good           \\
    B           & 73--76     & Acceptable     \\
    B\char"2212 & 70--72     & Marginal       \\ [1ex]
    C           & 60--69     & Unsatisfactory \\
    F           & 0--59      & Failure        \\
    FP          & 0          & Failure due to Plagiarism \\
    \bottomrule
  \end{tabular}}
  \caption{Summary of Grading System}
  \label{grade-syst}
\end{table}

% More detailed grading criteria from pp. 61--62 of `16.0406-I2-AST Academic Calendar.pdf'
%
%\begin{description}
%  \item[A+ (94-100) ‘Exceptional’]
%    A superior performance with consistent evidence of a comprehensive,
%    incisive grasp of all aspects of the subject matter; a very wide
%    knowledge base; insightful critical evaluation and analysis of the
%    material; an exceptional capacity for original, creative, and/or
%    logical thinking; an exceptional ability to organize, analyse,
%    synthesize, and to express thoughts fluently.
%  \item[A (87-93) ‘Outstanding’]
%    A comprehensive grasp of the subject matter, outstanding evidence of
%    original thought; sound critical evaluation of the material; an
%    excellent ability to organize, analyse, synthesize and to express
%    thoughts; mastery of an extensive knowledge base.
%  \item[A- (80-86) ‘Excellent’]
%    All the qualities of a B-level performance and an excellent capacity
%    for original, creative, and/ or logical thinking; excellent ability
%    to organize, analyse, synthesize, and integrate ideas; broad
%    knowledge base in the subject matter.
%  \item[B+ (77-79) ‘Good’]
%    A good performance with substantial knowledge of the subject matter;
%    a very good understanding of the relevant issues; familiarity with
%    relevant literature and techniques; good ability to organize,
%    analyse, and examine the material in a constructive and critical
%    manner.
%  \item[B (73-76) ‘Acceptable’]
%    A generally adequate performance with a good knowledge of the
%    subject matter; a fair understanding of relevant issues; some
%    ability to work with relevant literature and techniques; some
%    ability to develop solutions to difficult problems related to the
%    subject material.
%  \item[B- (70-72) ‘Marginally Acceptable’]
%    Some familiarity with the subject material; some understanding.
%    Satisfactory understanding of relevant issues; attempts to solve
%    moderately difficult problems related to the subject material in a
%    critical and analytical manner are only partially successful.
%  \item[C (60-69) ‘Unsatisfactory’]
%    A C grade indicates unsatisfactory academic performance. At the
%    discretion of the instructor, supplemental work may be negotiated to
%    upgrade the mark to a B range. A student may carry two C grades
%    without penalty in all courses except Foundations Courses,
%    Supervised Field Education, Supervised Ministry Practicum and the
%    Graduate Project. In these courses, a minimum grade of B- is
%    required to graduate. A student who receives a C in a Foundation
%    course must repeat the course to achieve a B- or better, and cannot
%    use the C grade to meet prerequisite requirements for advanced
%    courses. If the student repeats one of these courses and receives a
%    B- or better, the previous C grade remains on the transcript and can
%    be counted toward the total of unsatisfactory grades that may lead
%    to academic dismissal. Credit will be given only once for any
%    course. (See Policy on Unsatisfactory Academic Performance in the
%    AST Student Handbook.)
%  \item[F (0-59) ‘Failure’]
%    Student has not grasped subject matter; does not understand issues
%    involved; cannot work with relevant literature. (See Policy on
%    Unsatisfactory Academic Performance in the AST Student Handbook.)
%  \item[P ‘Pass’]
%    Credit awarded, but no mark assigned.
%  \item[FP ‘Failure due to Plagiarism’]
%    A student will receive this grade only after proven incident(s) of
%    plagiarism in a course.
%\end{description}
\ProvidesFile{other.tex}[2022/06/08 v2.9.1 -- Course policy]

\section{Other Course Policy}
\label{policy}

Late work will not be accepted, except in genuinely extenuating
circumstances. Students must submit something before the deadline if
they wish to receive credit. Unless I state otherwise, assignments are
to be uploaded by 11:59 \PM\ (Atlantic) on the date indicated.

Essay submissions must be typewritten and double-spaced. They should be
free from error. In this course they should follow SBL Style (see
\cite{sbl2} in \autoref{supplementary}, above). As a reminder, AST
upholds an Inclusive Language Policy. Please use gender-inclusive
language when referring to human beings. Our traditions have different
norms for speech about God; you are of course free to follow and explore
those traditions when referring to God.


Plagiarism is the
\href{http://www.eerdmans.com/Pages/Item/59043/Commentary-Statement.aspx}{failure}
to \href{https://www.theguardian.com/world/2013/feb/09/german-education-minister-quits-phd-plagiarism}{attribute}
(by means of footnotes when writing or aloud when speaking) any ideas,
phrases, sentences, materials, syntheses, et cetera, that another author
has composed and that you have borrowed for your own work. Plagiarism is
unethical. Academic penalties for plagiarism at AST are serious, and may
include failure of the course or even suspension of further studies.
Unintentional plagiarism is considered plagiarism. AST's Plagiarism
Policy is found under that heading in the Academic
Calendar.

Students should request permission to record a class or lecture. If
permission is granted, or if recordings are provided (as in the case of
an online or hybrid course), I stipulate that all recordings be for
personal use only. They may not be shared or distributed.

If you have needs that require modifications to any aspect of this
course, please consult with the instructor as soon as possible. Any
documentation regarding disabilities that you wish to divulge to AST
should be provided to the Registrar’s Office, where it will be kept in a
confidential file.

Finally, I encourage the conscientious use of laptops, tablets, and
other technology in my classes. In classroom settings, realize that, as
\href{http://dx.doi.org/10.1016/j.compedu.2012.10.003}{cognitive
psychologists have demonstrated}, ``laptop multitasking hinders
classroom learning for both users and nearby peers.'' Do your part to
foster an environment for dialogue by honouring the presence of your
classmates. In online and hybrid settings, consider both the physical
environment in which you choose to work and the virtual environment that
you help create through your participation in various forums. Let your
engagement in this course be marked by rigour and charity alike.


\section{Selected Bibliography}
\label{bib}

Literature on Genesis is vast. The few works listed here have been selected
for clarity, insight, and/or theological alertness. They are eligible as
selections for review essays.

\begingroup
\renewcommand{\section}[2]{}% temporarily remove the section heading
\begin{thebibliography}{99}

\bibitem{ca18} Allert, Craig D. \emph{Early Christian Readings of Genesis One: Patristic Exegesis and Literal Interpretation}. Downers Grove, IL: IVP Academic, 2018.

\bibitem{ra81} Alter, Robert. \emph{The Art of Biblical Narrative}. New York: Basic Books, 1981.

\bibitem{ga01} Anderson, Gary A. \emph{The Genesis of Perfection: Adam and Eve in Jewish and Christian Imagination}. Louisville: Westminster John Knox, 2001.

\bibitem{ed09} Davis, Ellen F. \emph{Scripture, Culture, and Agriculture: An Agrarian Reading of the Bible}. New York: Cambridge University Press, 2009.

\bibitem{es90} Ephrem the Syrian. \emph{Hymns on Paradise}. Translated by Sebastian Brock. Popular Patristics Series 10. Crestwood, NY: St Vladimir’s Seminary Press, 1990.

\bibitem{hg64} Gunkel, Hermann. \emph{The Legends of Genesis: The Biblical Saga and History}. Translated by W.\,H. Carruth. New York: Schocken, 1964.

%\bibitem{rh13} Hendel, Ronald S. \emph{The Book of Genesis: A Biography}. Princeton: Princeton University Press, 2013.

\bibitem{jl12} Levenson, Jon D. \emph{Inheriting Abraham: The Legacy of the Patriarch in Judaism, Christianity, and Islam}. Princeton: Princeton University Press, 2012.

\bibitem{wm92} Moberly, R. W. L. \emph{The Theology of the Book of Genesis}. Cambridge: Cambridge University Press, 2009.

\bibitem{ip16} Provan, Iain. \emph{Discovering Genesis: Content, Interpretation, Reception}. Grand Rapids: Eerdmans, 2016.

\bibitem{ts08} Schneider, Tammi J. \emph{Mothers of Promise: Women in the Book of Genesis}. Grand Rapids: Baker Academic, 2008.

\bibitem{ms90} Smith, Mark S. \emph{The Priestly Vision of Genesis 1}. Minneapolis: Fortress, 2010.

\bibitem{pt84} Trible, Phyllis. \emph{Texts of Terror: Literary-Feminist Readings of Biblical Narratives}. Overtures to Biblical Theology. Philadelphia: Fortress, 1984.

\end{thebibliography}
\endgroup

For additional literature, I recommend exploring \href{https://go.openathens.net/redirector/astheology.ns.ca?url=https://www.oxfordbibliographies.com/obo/page/biblical-studies}{Oxford Bibliographies: Biblical Studies (Full Text)}.
You can access the database automatically while on campus or remotely
with OpenAthens credentials. Numerous articles by subject-area
specialists appear under such headings as: Ancient Near East; Bible;
Early Christianity; Greco-Roman World; Hebrew Bible; New Testament;
Rabbinic Judaism; Second Temple Judaism. For \ccode, see especially
\href{https://go.openathens.net/redirector/astheology.ns.ca?url=https://www.oxfordbibliographies.com/view/document/obo-9780195393361/obo-9780195393361-0044.xml}{Victor H. Matthews, “Book of Genesis,”} doi: 10.1093/OBO/9780195393361-0044.

\end{document}