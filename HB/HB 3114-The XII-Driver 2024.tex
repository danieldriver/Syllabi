% Copyright (c) 2024 by Daniel R. Driver.
% !TEX encoding = UTF-8 Unicode
% !TEX TS-program = XeLaTeX

\documentclass[titlepage]{article}

% This document presumes a file structure and set of inputs that are
% available at: git@github.com:danieldriver/syllabi.git

\newcommand\policy{../policy}
\newcommand\incl{../includes}
\ProvidesFile{variables.tex}[2018/05/24 v2.1 -- Syllabus variables]

\usepackage{xspace} % make manual spaces (like \mycmd\ ) unnecessary
\usepackage{xifthen} % provides \isempty test

% variables for internal use
\newcommand\prof{}
\newcommand\pdegree{}
\newcommand\pphone{}
\newcommand\pemail{}
\newcommand\poffice{}
\newcommand\phours{}
%
\newcommand\ccode{}
\newcommand\ctitle{}
\newcommand\cseries{}
\newcommand\cversion{}
\newcommand\csemester{}
\newcommand\cmeetson{}
\newcommand\cmeetsat{}
\newcommand\cmeetsin{}
\newcommand\cwebsite{}
\newcommand\cdescrip{}
\newcommand\cprereqs{}
\newcommand\edobject{}

% in case of fully online courses - https://tex.stackexchange.com/a/5896
\newif\ifonline
\newcommand\Int[2]{\ifonline#1\else#2\fi}

% commands for setting variables in the preamble
\newcommand\professor[2][PhD]{
  \renewcommand\pdegree{#1\xspace}
  \renewcommand\prof{#2\xspace}}
\newcommand\phone[1]{
  \renewcommand\pphone{\addfontfeatures{Numbers=Monospaced}#1\xspace}}
\newcommand\email[1]{
  \renewcommand\pemail{\href{mailto:#1}{#1}\xspace}}
\newcommand\officehours[2][Library, Room 5-North]{
  \renewcommand\poffice{#1\xspace}
  \renewcommand\phours{#2\xspace}}
%
\newcommand\coursecode[2][1.0]{
  \renewcommand\cversion{#1\Int{-i}{}\xspace}
  \renewcommand\ccode{#2\Int{(Int)}{}\xspace}}
\newcommand\coursetitle[2][]{
  \ifthenelse{\isempty{#1}}%
    {}% do nothing if #1 is empty, else:
    {\renewcommand\cseries{#1\\[1ex]}}
  \renewcommand\ctitle{#2\xspace}}
\newcommand\semester[1]{
  \renewcommand\csemester{#1\xspace}}
\newcommand\meets[3]{
  \newcommand\AM{\textsc{am}}
  \newcommand\PM{\textsc{pm}}
  \renewcommand\cmeetson{#1\xspace}
  \renewcommand\cmeetsat{\Int{From 9:00 \AM}{#2}\xspace}
  \renewcommand\cmeetsin{\Int{\href{https://smu.brightspace.com/d2l/login}{Brightspace}}{#3}\xspace}}
\newcommand\website[1]{
  \renewcommand\cwebsite{\href{http://#1}{#1}\xspace}}
\newcommand\cdescription[2][RM 1000 or GTRS 6000; and BF 1001]{
  \renewcommand\cprereqs{#1}
  \renewcommand\cdescrip{#2\par}}
\newcommand\objectives[1]{
  \renewcommand\edobject{#1\par}}


%\onlinetrue % \Int{true}{false}
\coursecode[2.1]{HB 3114}
\coursetitle{The Twelve Prophets}

% Taught as RLGS 3143 in:
%   - Winter 2010
%   - Winter 2013
% Taught as HB 3114 in:
%   - Winter 2018
%   - Winter 2024

\professor{Daniel R. Driver}
\phone{902-425-7051}
\email{daniel.driver@astheology.ns.ca}
\officehours{Thursdays, 1:00--3:00 \PM}

\semester{Winter Term 2024}
\meets{Thursdays}% \meets{on}{at}{in}
      {4:00--6:30 \PM}
      {Classroom 3/Teams}
\website[https://danieldriver.com/courses/hb-3114/]{danieldriver.com}
\cdescription{% copy from the current Academic Calendar
	To better understand the origins and impact of the Book of the
	Twelve, this course will look at its redactional history and
	editorial shaping, its final canonical forms, and its effects as
	biblical prophecy. It will introduce current issues in scholarly
	debate about the Twelve while also exploring ways that they meet the
	ancient expectation that their “bones” can “send forth new life”
	amidst the people of God (Sirach 49:10).
}% end of course description
%\objectives{% recall Bloom's taxonomy: http://www.celt.iastate.edu/teaching/RevisedBlooms1.html
%By the end of the course students should be able to:
%}% end of learning objectives

\ProvidesFile{preamble.tex}[2013/09/06 v1.0 -- Syllabus preamble]

% basic typography
\usepackage{fontspec}
\setmainfont[Ligatures=TeX]{Meta Serif Pro}
\setsansfont[Ligatures=TeX]{Meta Pro}
\newfontfamily\Heb{Meta Hebrew}
\setmonofont[Scale=MatchLowercase]{Menlo}
\usepackage{sectsty}
\allsectionsfont{\sffamily}
\frenchspacing
\setlength{\emergencystretch}{3em} % prevent overfull lines

% custom font size and leading
\renewcommand\tiny{\fontsize{6}{9}\selectfont}
\renewcommand\scriptsize{\fontsize{7}{10}\selectfont}
\renewcommand\footnotesize{\fontsize{8}{11}\selectfont}
\renewcommand\small{\fontsize{8.5}{11.5}\selectfont}
\renewcommand\normalsize{\fontsize{9}{12}\selectfont}% base size
\renewcommand\large{\fontsize{11}{14}\selectfont}
\renewcommand\Large{\fontsize{13}{16}\selectfont}
\renewcommand\LARGE{\fontsize{16}{19}\selectfont}% "course syllabus \\ semester" benefits from more lead
\renewcommand\huge{\fontsize{19}{21}\selectfont}
\renewcommand\Huge{\fontsize{24}{26}\selectfont}

% layout packages: page, logo, tables
\usepackage[scale={0.6,0.8},
            xetex]{geometry}
\usepackage{graphicx}
\usepackage{array}     % allow insertions of column styling with >{}
\usepackage{booktabs}  % elegant horizontal rules in tables
\usepackage{marginfix} % protect positioning of margin table in policy/grades

% custom macros for a session count in the schedule of readings
\newcounter{session}
\newcounter{columns}
\newcounter{courseunit}
\newcommand\setcolumncount[2][0]{ % optionally set count to other than 0,
  \setcounter{session}{#1}        % e.g. to -1, or to a standing count
  \setcounter{columns}{#2}}
\newcommand\sessioncount{\stepcounter{session}\arabic{session}}
\newcommand\sessionskip[1]{\multicolumn{1}{@{}r@{ }}{#1}}
\newcommand\unit[1]{\multicolumn{\thecolumns}{c}{%
  \scshape\stepcounter{courseunit}\roman{courseunit}. \MakeLowercase{#1}}}
\newcommand\noclass[1]{\multicolumn{1}{@{}l}{\itshape No Class: #1}}

% color to match Tyndale's branding
\usepackage[usenames]{xcolor}
% predefined: black, white, red, green, blue, cyan, magenta, yellow
\definecolor{TyndaleURLs}{HTML}{0062A0} % links on tyndale.ca
\definecolor{TyndaleBlue}{cmyk}{1,1,0,.32}
\definecolor{TyndaleGold}{cmyk}{0,.27,1,0}
\definecolor{TyndaleRed}{cmyk}{0,1,.99,.04}
\definecolor{TyndaleBlack}{cmyk}{0,0,0,1}
\definecolor{TyndaleGreen}{cmyk}{.45,0,1,.24}
\definecolor{TyndaleOrange}{cmyk}{0,.79,1,0}
\definecolor{TyndaleAqua}{cmyk}{.47,0,.24,0}
\definecolor{TyndaleYellow}{cmyk}{.03,.03,.35,0}

% metadata (assumes a host of definitions are made in the main file)
\usepackage[setpagesize=false,     % leave this to geometry
            hyperfootnotes=false,  % fragile and distracting
            xetex]{hyperref}
\hypersetup{breaklinks=true,       % allow link text to break across lines
            colorlinks=true,       % colorlinks resets pdfborder to 0 0 0
            urlcolor=TyndaleURLs,  % for external links
            linkcolor=TyndaleRed,  % for normal internal links
            citecolor=TyndaleGold, % for bibliographical citations in text
            pdfauthor={\prof},
            pdftitle={\ccode: \ctitle},
            pdfsubject={Tyndale UC, \csemester},
            pdfcreator={github.com/danieldriver/syllabus}}
\urlstyle{same}                    % don't use monospace font for urls

% custom footlines
\usepackage{fancyhdr}
\pagestyle{fancy} % turn it on
\fancyhf{}        % reset everything
\renewcommand{\headrulewidth}{0pt} % remove header line as well
\lfoot{\sffamily\scshape\footnotesize\MakeLowercase{\ctitle, v\cversion}}
\rfoot{\sffamily\scshape\footnotesize\MakeLowercase{\prof\quad\thepage}}

% gratuitous with custom title page, but useful as a fallback
\title{\ccode: \ctitle}
\author{\professor}
\date{\semester}


\begin{document}
\ProvidesFile{title.tex}[2013/09/06 v1.0 -- Syllabus title page]

\begin{titlepage}
  \begin{center}

    \LARGE\sffamily % set title elements in a large sans serif

    \begin{minipage}{\textwidth}
      \parbox[t]{0.5\textwidth}{
        \mbox{}\\[-13pt] % dummy line to align parboxes
        \includegraphics[width=0.5\textwidth]{.syllabus/includes/TyndaleUC}}
      \hfill
      \parbox[t]{0.4\textwidth}{
        \raggedleft Course Syllabus\\
        \csemester}
    \end{minipage}

    \vfill

    {\textsc{\MakeLowercase\ccode}\\[1ex]
      \bfseries\cseries\Huge\ctitle}

    \vfill

    \normalsize\rmfamily % switch back to body type

    \begin{tabular}{>{\bfseries}rl>{\bfseries}rl}
      \toprule
      Instructor & \prof, \pdegree & Course  & Version \cversion \\
      \midrule
      Phone      & \pphone         & Meets   & \cmeetson         \\
      Email      & \pemail         & Time    & \cmeetsat         \\
      Office     & \poffice        & Room    & \cmeetsin         \\
      Hours      & \phours         & Website & \cwebsite         \\
      \bottomrule
    \end{tabular}

    \vfill

    \begin{description}\small
      \item[Commuter Hotline]
        Class cancellations due to inclement weather or illness will
        be announced on the commuter hotline at \texttt{416.226.6620
        x2187}. Alternately, weather cancellation information is posted
        at \href{http://tyndale.ca/weather}{tyndale.ca/weather}.
      \item[MyTyndale.ca]
        This course may have materials stored on its website, such as
        handouts or readings that may be needed in order to complete
        assignments. Students are responsible for checking these course
        pages on a regular basis. Here, too, students are able to view
        their grades throughout the semester. For more information see
        Section~\ref{mytyndale}, below.
      \item[Mail]
        Students are responsible for information communicated through
        their campus mailboxes and student e-mail accounts. A mailbox
        directory hangs beside the mailboxes. For more information
        contact the Registrar's office.
    \end{description}

  \end{center}

  \section{Course Description}
  \label{description}

  \emph{From the Academic Calendar:} \cdescrip

\end{titlepage}
\setcounter{page}{2} % count the title page as page 1


%  \section{Learning Objectives}
%  \label{objectives}
%
%  \edobject

\section{Required Texts \& Materials}
\label{texts}

In addition to \textbf{the Bible in a modern translation} such as the
NJPS or NRSV, the following text is required. Students are strongly
encouraged to purchase their own copies. AST Library copies will be
placed on a two-hour reserve.

\begingroup
\renewcommand{\section}[2]{}% temporarily remove the section heading
\begin{thebibliography}{Theodoret}% use the longest item in the bibliography

	\bibitem[Heschel]{Heschel} Heschel, Abraham J.
	\emph{The Prophets}.
	New York: Harper \& Row, 1962. Repr., New York: Perennial Classics, 2001.
	\textsc{isbn} 978-0060936990. \href{https://amzn.to/46qhye3}{Order online.}

	\bibitem[Theodoret]{Theodoret} Theodoret of Cyrus.
	\emph{Commentary on the Twelve Prophets}. Vol. 3 of \emph{Commentaries on the Prophets}.
	Translated by Robert C. Hill.
	Brookline, MA: Holy Cross Orthodox Press, 2006.
	\textsc{isbn} 978-1885652768. \href{https://amzn.to/47kYajz}{Order online.}

%	\bibitem[Two Sides]{Two} Ehud Ben Zvi and James D. Nogalski.
%	\emph{Two Sides of a Coin: Juxtaposing Views on Interpreting the Book of the Twelve / the Twelve Prophetic Books}.
%	Piscataway, NJ: Gorgias Press, 2009.
%	\textsc{isbn} 978-1607243038.

%	\bibitem[Seitz]{Seitz} Christopher R. Seitz.
%	\emph{The Goodly Fellowship of the Prophets: The Achievement of Association in Canon Formation}.
%	Grand Rapids: Baker Academic, 2009.
%	\textsc{isbn} 978-0801038839.

%	\bibitem[Jerome]{Jerome1} Jerome; Thomas P. Scheck, trans.
%	\emph{Commentaries on the Twelve Prophets, Vol. 1 (Ancient Christian Texts)}.
%	Downers Grove: IVP Academic, 2016.
%	\textsc{isbn} 978-0830829163.

\end{thebibliography}
\endgroup

\section{Supplementary Texts}
\label{supplementary}

Supplementary readings will be recommended throughout the semester.
Excerpts from this literature may be placed on reserve or made available
for download through the course website. Outstanding students will also
pursue leads on their own, using library databases and pursuing notes
and bibliographies in the required reading.

The following titles are notable in relation to the main texts. They are
not required.

\begingroup
\renewcommand{\section}[2]{}% temporarily remove the section heading
\begin{thebibliography}{3}% use the longest item in the bibliography

	\bibitem{Bickerman} Bickerman, Elias.
	\emph{Four Strange Books of the Bible: Jonah, Daniel, Koheleth, Esther}.
	New York: Schocken, 1967.

	\bibitem{Sabbath} Heschel, Abraham Joshua.
	\emph{The Sabbath: Its Meaning for Modern Man}.
	New York: Farrar, Straus and Giroux, 1951.

	\bibitem{Hill} Hill, Robert C.
	\emph{Reading the Old Testament in Antioch}. Bible in Ancient Christianity 5.
	Leiden: Brill, 2005. Link: \href{https://ast.primo.exlibrisgroup.com/view/action/uresolver.do?operation=resolveService&package_service_id=785748290007188&institutionId=7188&customerId=7185&VE=true}{AST eBook}.

	\bibitem{Nogalski-Sweeney} Nogalski, James D., and Marvin A. Sweeney, eds.
	\emph{Reading and Hearing the Book of the Twelve}. SBL Symposium Series.
	Atlanta: SBL, 2000.

	\bibitem{Redditt-Schart} Redditt, Paul L., and Aaron Schart, eds.
	\emph{Thematic Threads in the Book of the Twelve}.
	Berlin: Walter de Gruyter, 2003.

	\bibitem{Watts-House} Watts, James W., and Paul R. House, eds.
	\emph{Forming Prophetic Literature: Essays on Isaiah and the Twelve in Honor of John D.W. Watts}.
	Sheffield: Sheffield Academic, 1996.

\end{thebibliography}
\endgroup

\section{Course Outline}
\label{outline}

We will adhere to the schedule in \autoref{schedule} as closely as
possible, though the professor reserves the right to adjust it to suit
the needs of the class.

\setcolumncount{4}% set up \sessioncount, \unit{}, \noclass{}, and \reminder{memo}{date} macros
\begin{table}[htbp]% set to `p' to put the schedule on its own page
  \centering
  \begin{tabular}{>{\sessioncount.}r@{ }llr}% make sure the column config agrees with \setcolumncount
	\toprule
	\sessionskip{\textbf{\S}.}&\textbf{Primary}&\textbf{Secondary}&\textbf{Date}\\
	\midrule
		& None: Introductions & Syllabus & 11 Jan. \\
		& Hosea 1             & Theodoret 1--42, Heschel 47--61   & 18 Jan. \\
		& Hosea 2--3          & Theodoret 42--48, Heschel 61--75  & 25 Jan. \\
		& Hosea 4--6          & Theodoret 48--57, Heschel 3--19   &  1 Feb. \\
		& Hosea 7--9          & Theodoret 57--67, Heschel 19--31  &  8 Feb. \\
		& Hosea 10--14        & Theodoret 67--83, Heschel         & 15 Feb. \\
	\reminder{Essay 1: Due by the end of Week Six}{18 Feb.} \\
	\noclass{Reading Week (Tuesday to Friday)}                    & 22 Feb. \\
		& Joel 1              & Theodoret 85--90, Heschel         & 29 Feb. \\
		& Joel 2--end         & Theodoret 90--102, Heschel        &  7 Mar. \\
	\noclass{Attend Grad Projects in lieu of regular class}       & 14 Mar. \\
		& Amos 1--2           & Theodoret 103--10, Heschel 32--39 & 21 Mar. \\
		& Amos 3--4           & Theodoret 110--16, Heschel 39--46 & 28 Mar. \\
		& Amos 5--6           & Theodoret 116--22, Heschel        &  4 Apr. \\
	\reminder{Essay 2: Due by the end of Week Eleven}{7 Apr.} \\
		& Amos 7--9           & Theodoret 122--28, Heschel        & 11 Apr. \\
	\reminder{End of Term: Final marks are due for all courses}{17 Apr.} \\
	\bottomrule
  \end{tabular}
  \caption{Schedule of Readings}
  \label{schedule}
\end{table}

See the AST website for a list of other \href{http://www.astheology.ns.ca/students/academic-dates.html}{important dates}.

\section{Evaluation}
\label{evaluation}

\subsection{Grade Structure for \ccode}
\label{structure}

\begin{enumerate}

	\item Weekly \textbf{seminar discussions} will facilitate close work
	with text and interpretation in select books from the Twelve.
	Students should come to class fully prepared to discuss the assigned
	material, which will ordinarily include one to four chapters of
	biblical material per week, plus some commentary on those chapters.
	Commentary available for study and preparation is by no means
	limited to the assigned works of \cite{Theodoret} and
	\cite{Heschel}. Also, Heschel does not follow the text sequentially;
	his book should be read beyond the assigned pages in a self-directed
	manner.

	\item At some point in the semester, each student should read and
	report back to the class about some recent scholarship on the book
	of the Twelve. This \textbf{reading report} should be scheduled
	between weeks two and twelve, to the extent possible with only one
	student per week. The reading should be drawn from the literature in
	\autoref{supplementary} or \autoref{bib}, selected in consultation
	with the professor.

	\item A \textbf{first essay} will be due at the end of the sixth
	week of class (Sunday, 18 February 2024). It should be 4,000 words
	long, plus or minus 250 words. It should answer the following
	question: \emph{How does the book of Hosea speak to the goodness of
	God and the sovereignty of God?}

	\item A \textbf{second essay} will be due at the end of the eleventh
	week of class (Sunday, 7 April 2024). It should be 4,000 words long,
	plus or minus 250 words. It should answer the following question:
	\emph{How does the presentation of God develop in the first part of
	the book of the Twelve?} Note that this question is not a repeat of
	the first one. Rather, it builds on the first paper by examining the
	character and presentation of God in the literary context that
	follows Hosea. If this question does not appeal, students are
	invited to propose an alternate question to the professor. The
	professor reserves the right to turn down the proposal or to make a
	counterproposal.
	
\end{enumerate}

The breakdown for the semester's total work is shown in
\autoref{grade-dist}.

\begin{table}[htbp]
  \centering
  {\lining
  \begin{tabular}{lr}
    \toprule
    Weekly Seminars & 10\% \\
    Reading Report  & 10\% \\
    Essay 1         & 40\% \\
    Essay 2         & 40\% \\
    \bottomrule
  \end{tabular}}
  \caption{Distribution of Grades}
  \label{grade-dist}
\end{table}

\ProvidesFile{grades.tex}[2016/09/03 v2.0 -- Course policy]

\subsection{Grading System at AST}
\label{grades}

AST's \href{http://www.astheology.ns.ca/webfiles/AST_2016Calendar_web(A5)-06APR2016.pdf}{Academic
Calendar} provides guidelines and detailed criteria for academic
assessment. Marks are assigned by letter grade using the benchmarks in
\autoref{grade-syst}.

\begin{table}[htbp]
  \centering
  {\lining
  \begin{tabular}{lll}
    \toprule
%    Letter      & Percent & Assessment        \\
%	\midrule
    A+          & 94--100    & Exceptional    \\
    A           & 87--93     & Outstanding    \\
    A\char"2212 & 80--86     & Excellent      \\ [1ex]
    B+          & 77--79     & Good           \\
    B           & 73--76     & Acceptable     \\
    B\char"2212 & 70--72     & Marginal       \\ [1ex]
    C           & 60--69     & Unsatisfactory \\
    F           & 0--59      & Failure        \\
    FP          & 0          & Failure due to Plagiarism \\
    \bottomrule
  \end{tabular}}
  \caption{Summary of Grading System}
  \label{grade-syst}
\end{table}

% More detailed grading criteria from pp. 61--62 of `16.0406-I2-AST Academic Calendar.pdf'
%
%\begin{description}
%  \item[A+ (94-100) ‘Exceptional’]
%    A superior performance with consistent evidence of a comprehensive,
%    incisive grasp of all aspects of the subject matter; a very wide
%    knowledge base; insightful critical evaluation and analysis of the
%    material; an exceptional capacity for original, creative, and/or
%    logical thinking; an exceptional ability to organize, analyse,
%    synthesize, and to express thoughts fluently.
%  \item[A (87-93) ‘Outstanding’]
%    A comprehensive grasp of the subject matter, outstanding evidence of
%    original thought; sound critical evaluation of the material; an
%    excellent ability to organize, analyse, synthesize and to express
%    thoughts; mastery of an extensive knowledge base.
%  \item[A- (80-86) ‘Excellent’]
%    All the qualities of a B-level performance and an excellent capacity
%    for original, creative, and/ or logical thinking; excellent ability
%    to organize, analyse, synthesize, and integrate ideas; broad
%    knowledge base in the subject matter.
%  \item[B+ (77-79) ‘Good’]
%    A good performance with substantial knowledge of the subject matter;
%    a very good understanding of the relevant issues; familiarity with
%    relevant literature and techniques; good ability to organize,
%    analyse, and examine the material in a constructive and critical
%    manner.
%  \item[B (73-76) ‘Acceptable’]
%    A generally adequate performance with a good knowledge of the
%    subject matter; a fair understanding of relevant issues; some
%    ability to work with relevant literature and techniques; some
%    ability to develop solutions to difficult problems related to the
%    subject material.
%  \item[B- (70-72) ‘Marginally Acceptable’]
%    Some familiarity with the subject material; some understanding.
%    Satisfactory understanding of relevant issues; attempts to solve
%    moderately difficult problems related to the subject material in a
%    critical and analytical manner are only partially successful.
%  \item[C (60-69) ‘Unsatisfactory’]
%    A C grade indicates unsatisfactory academic performance. At the
%    discretion of the instructor, supplemental work may be negotiated to
%    upgrade the mark to a B range. A student may carry two C grades
%    without penalty in all courses except Foundations Courses,
%    Supervised Field Education, Supervised Ministry Practicum and the
%    Graduate Project. In these courses, a minimum grade of B- is
%    required to graduate. A student who receives a C in a Foundation
%    course must repeat the course to achieve a B- or better, and cannot
%    use the C grade to meet prerequisite requirements for advanced
%    courses. If the student repeats one of these courses and receives a
%    B- or better, the previous C grade remains on the transcript and can
%    be counted toward the total of unsatisfactory grades that may lead
%    to academic dismissal. Credit will be given only once for any
%    course. (See Policy on Unsatisfactory Academic Performance in the
%    AST Student Handbook.)
%  \item[F (0-59) ‘Failure’]
%    Student has not grasped subject matter; does not understand issues
%    involved; cannot work with relevant literature. (See Policy on
%    Unsatisfactory Academic Performance in the AST Student Handbook.)
%  \item[P ‘Pass’]
%    Credit awarded, but no mark assigned.
%  \item[FP ‘Failure due to Plagiarism’]
%    A student will receive this grade only after proven incident(s) of
%    plagiarism in a course.
%\end{description}
\ProvidesFile{other.tex}[2022/06/08 v2.9.1 -- Course policy]

\section{Other Course Policy}
\label{policy}

Late work will not be accepted, except in genuinely extenuating
circumstances. Students must submit something before the deadline if
they wish to receive credit. Unless I state otherwise, assignments are
to be uploaded by 11:59 \PM\ (Atlantic) on the date indicated.

Essay submissions must be typewritten and double-spaced. They should be
free from error. In this course they should follow SBL Style (see
\cite{sbl2} in \autoref{supplementary}, above). As a reminder, AST
upholds an Inclusive Language Policy. Please use gender-inclusive
language when referring to human beings. Our traditions have different
norms for speech about God; you are of course free to follow and explore
those traditions when referring to God.


Plagiarism is the
\href{http://www.eerdmans.com/Pages/Item/59043/Commentary-Statement.aspx}{failure}
to \href{https://www.theguardian.com/world/2013/feb/09/german-education-minister-quits-phd-plagiarism}{attribute}
(by means of footnotes when writing or aloud when speaking) any ideas,
phrases, sentences, materials, syntheses, et cetera, that another author
has composed and that you have borrowed for your own work. Plagiarism is
unethical. Academic penalties for plagiarism at AST are serious, and may
include failure of the course or even suspension of further studies.
Unintentional plagiarism is considered plagiarism. AST's Plagiarism
Policy is found under that heading in the Academic
Calendar.

Students should request permission to record a class or lecture. If
permission is granted, or if recordings are provided (as in the case of
an online or hybrid course), I stipulate that all recordings be for
personal use only. They may not be shared or distributed.

If you have needs that require modifications to any aspect of this
course, please consult with the instructor as soon as possible. Any
documentation regarding disabilities that you wish to divulge to AST
should be provided to the Registrar’s Office, where it will be kept in a
confidential file.

Finally, I encourage the conscientious use of laptops, tablets, and
other technology in my classes. In classroom settings, realize that, as
\href{http://dx.doi.org/10.1016/j.compedu.2012.10.003}{cognitive
psychologists have demonstrated}, ``laptop multitasking hinders
classroom learning for both users and nearby peers.'' Do your part to
foster an environment for dialogue by honouring the presence of your
classmates. In online and hybrid settings, consider both the physical
environment in which you choose to work and the virtual environment that
you help create through your participation in various forums. Let your
engagement in this course be marked by rigour and charity alike.


\section{Further Bibliography}
\label{bib}

Access and study the following resources:

Schart, Aaron. “Bibliography on the Book of the Twelve Prophets.” This database with nearly all literature on the Book of the Twelve since 2010 is maintained at \url{https://www.zotero.org/groups/twelveprophets/items}.

Sweeney, Marvin A. “Book of the Twelve Prophets.” In \emph{Oxford Bibliographies: Biblical Studies}, \href{https://go.openathens.net/redirector/astheology.ns.ca?url=https://www.oxfordbibliographies.com/view/document/obo-9780195393361/obo-9780195393361-0016.xml}{https://www.oxfordbibliographies.com/view/document/obo-9780195393361/obo-9780195393361-0016.xml} (last modified 30 March 2015). AST students can sign in with their OpenAthens credentials.

\end{document}
