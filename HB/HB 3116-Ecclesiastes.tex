% Copyright (c) 2019 by Daniel R. Driver.
% !TEX encoding = UTF-8 Unicode
% !TEX TS-program = XeLaTeX

\documentclass[titlepage]{article}

% This document presumes a file structure and set of inputs that are
% available at: git@github.com:danieldriver/syllabi.git

\newcommand\policy{../policy}
\newcommand\incl{../includes}
\ProvidesFile{variables.tex}[2018/05/24 v2.1 -- Syllabus variables]

\usepackage{xspace} % make manual spaces (like \mycmd\ ) unnecessary
\usepackage{xifthen} % provides \isempty test

% variables for internal use
\newcommand\prof{}
\newcommand\pdegree{}
\newcommand\pphone{}
\newcommand\pemail{}
\newcommand\poffice{}
\newcommand\phours{}
%
\newcommand\ccode{}
\newcommand\ctitle{}
\newcommand\cseries{}
\newcommand\cversion{}
\newcommand\csemester{}
\newcommand\cmeetson{}
\newcommand\cmeetsat{}
\newcommand\cmeetsin{}
\newcommand\cwebsite{}
\newcommand\cdescrip{}
\newcommand\cprereqs{}
\newcommand\edobject{}

% in case of fully online courses - https://tex.stackexchange.com/a/5896
\newif\ifonline
\newcommand\Int[2]{\ifonline#1\else#2\fi}

% commands for setting variables in the preamble
\newcommand\professor[2][PhD]{
  \renewcommand\pdegree{#1\xspace}
  \renewcommand\prof{#2\xspace}}
\newcommand\phone[1]{
  \renewcommand\pphone{\addfontfeatures{Numbers=Monospaced}#1\xspace}}
\newcommand\email[1]{
  \renewcommand\pemail{\href{mailto:#1}{#1}\xspace}}
\newcommand\officehours[2][Library, Room 5-North]{
  \renewcommand\poffice{#1\xspace}
  \renewcommand\phours{#2\xspace}}
%
\newcommand\coursecode[2][1.0]{
  \renewcommand\cversion{#1\Int{-i}{}\xspace}
  \renewcommand\ccode{#2\Int{(Int)}{}\xspace}}
\newcommand\coursetitle[2][]{
  \ifthenelse{\isempty{#1}}%
    {}% do nothing if #1 is empty, else:
    {\renewcommand\cseries{#1\\[1ex]}}
  \renewcommand\ctitle{#2\xspace}}
\newcommand\semester[1]{
  \renewcommand\csemester{#1\xspace}}
\newcommand\meets[3]{
  \newcommand\AM{\textsc{am}}
  \newcommand\PM{\textsc{pm}}
  \renewcommand\cmeetson{#1\xspace}
  \renewcommand\cmeetsat{\Int{From 9:00 \AM}{#2}\xspace}
  \renewcommand\cmeetsin{\Int{\href{https://smu.brightspace.com/d2l/login}{Brightspace}}{#3}\xspace}}
\newcommand\website[1]{
  \renewcommand\cwebsite{\href{http://#1}{#1}\xspace}}
\newcommand\cdescription[2][RM 1000 or GTRS 6000; and BF 1001]{
  \renewcommand\cprereqs{#1}
  \renewcommand\cdescrip{#2\par}}
\newcommand\objectives[1]{
  \renewcommand\edobject{#1\par}}


%\onlinetrue % \Int{true}{false}
\coursecode[2.0]{HB 3116(Int)}
\coursetitle[Hebrew Bible]{Ecclesiastes}

% Taught as BSTH 4403 at Tyndale in:
%   - Fall 2015
% Taught as HB 3116 at AST in:
%   - Winter 2019

\professor{Daniel R. Driver}
\phone{902-425-7051}
\email{ddriver@astheology.ns.ca}
\officehours{By appointment}

\semester{Winter Term 2019}
\meets{On Fridays}% \meets{on}{at}{in}
      {From 10:00 \AM\ (Atlantic)}
      {Online: Brightspace}
\website{danieldriver.com}
\cdescription{% copy from the Academic Calendar; add [] for unusual prereqs

In this course Ecclesiastes will be studied in the context of the
Writings and other wisdom literature, and in connection with the book’s
reception history. By the end of the course students should be able to:
understand the basic shape of the Writings and, within that collection,
books classified as wisdom literature; give examples of extra-biblical
parallels to wisdom literature; situate Ecclesiastes relative to other
Solomonic literature; characterize some ancient and modern commentaries
on Ecclesiastes; articulate a few of the major interpretive options in
reading Ecclesiastes; defend their own readings of the book in writing,
with detailed textual support.

}% end of course description
\objectives{% recall Bloom's taxonomy: http://www.celt.iastate.edu/teaching/RevisedBlooms1.html

In this course Ecclesiastes will be studied in the context of other
biblical literature, with some reference to extra-biblical sources, and
in connection with the book's reception history. By the end of the
course students should be able to:
	name biblical books classified with the Writings and wisdom literature;
	understand some different ways this literature has been classified;
	give examples of extra-biblical parallels to it;
	locate Ecclesiastes relative to other Solomonic literature;
	recognize quotes from Ecclesiastes by chapter;
	name major ancient and modern commentators on Ecclesiastes;
	report on specific commentators verbally;
	articulate major interpretive options in reading Ecclesiastes;
	defend their own preferred readings in writing, with detailed textual support.

}% end of learning objectives

\ProvidesFile{preamble.tex}[2013/09/06 v1.0 -- Syllabus preamble]

% basic typography
\usepackage{fontspec}
\setmainfont[Ligatures=TeX]{Meta Serif Pro}
\setsansfont[Ligatures=TeX]{Meta Pro}
\newfontfamily\Heb{Meta Hebrew}
\setmonofont[Scale=MatchLowercase]{Menlo}
\usepackage{sectsty}
\allsectionsfont{\sffamily}
\frenchspacing
\setlength{\emergencystretch}{3em} % prevent overfull lines

% custom font size and leading
\renewcommand\tiny{\fontsize{6}{9}\selectfont}
\renewcommand\scriptsize{\fontsize{7}{10}\selectfont}
\renewcommand\footnotesize{\fontsize{8}{11}\selectfont}
\renewcommand\small{\fontsize{8.5}{11.5}\selectfont}
\renewcommand\normalsize{\fontsize{9}{12}\selectfont}% base size
\renewcommand\large{\fontsize{11}{14}\selectfont}
\renewcommand\Large{\fontsize{13}{16}\selectfont}
\renewcommand\LARGE{\fontsize{16}{19}\selectfont}% "course syllabus \\ semester" benefits from more lead
\renewcommand\huge{\fontsize{19}{21}\selectfont}
\renewcommand\Huge{\fontsize{24}{26}\selectfont}

% layout packages: page, logo, tables
\usepackage[scale={0.6,0.8},
            xetex]{geometry}
\usepackage{graphicx}
\usepackage{array}     % allow insertions of column styling with >{}
\usepackage{booktabs}  % elegant horizontal rules in tables
\usepackage{marginfix} % protect positioning of margin table in policy/grades

% custom macros for a session count in the schedule of readings
\newcounter{session}
\newcounter{columns}
\newcounter{courseunit}
\newcommand\setcolumncount[2][0]{ % optionally set count to other than 0,
  \setcounter{session}{#1}        % e.g. to -1, or to a standing count
  \setcounter{columns}{#2}}
\newcommand\sessioncount{\stepcounter{session}\arabic{session}}
\newcommand\sessionskip[1]{\multicolumn{1}{@{}r@{ }}{#1}}
\newcommand\unit[1]{\multicolumn{\thecolumns}{c}{%
  \scshape\stepcounter{courseunit}\roman{courseunit}. \MakeLowercase{#1}}}
\newcommand\noclass[1]{\multicolumn{1}{@{}l}{\itshape No Class: #1}}

% color to match Tyndale's branding
\usepackage[usenames]{xcolor}
% predefined: black, white, red, green, blue, cyan, magenta, yellow
\definecolor{TyndaleURLs}{HTML}{0062A0} % links on tyndale.ca
\definecolor{TyndaleBlue}{cmyk}{1,1,0,.32}
\definecolor{TyndaleGold}{cmyk}{0,.27,1,0}
\definecolor{TyndaleRed}{cmyk}{0,1,.99,.04}
\definecolor{TyndaleBlack}{cmyk}{0,0,0,1}
\definecolor{TyndaleGreen}{cmyk}{.45,0,1,.24}
\definecolor{TyndaleOrange}{cmyk}{0,.79,1,0}
\definecolor{TyndaleAqua}{cmyk}{.47,0,.24,0}
\definecolor{TyndaleYellow}{cmyk}{.03,.03,.35,0}

% metadata (assumes a host of definitions are made in the main file)
\usepackage[setpagesize=false,     % leave this to geometry
            hyperfootnotes=false,  % fragile and distracting
            xetex]{hyperref}
\hypersetup{breaklinks=true,       % allow link text to break across lines
            colorlinks=true,       % colorlinks resets pdfborder to 0 0 0
            urlcolor=TyndaleURLs,  % for external links
            linkcolor=TyndaleRed,  % for normal internal links
            citecolor=TyndaleGold, % for bibliographical citations in text
            pdfauthor={\prof},
            pdftitle={\ccode: \ctitle},
            pdfsubject={Tyndale UC, \csemester},
            pdfcreator={github.com/danieldriver/syllabus}}
\urlstyle{same}                    % don't use monospace font for urls

% custom footlines
\usepackage{fancyhdr}
\pagestyle{fancy} % turn it on
\fancyhf{}        % reset everything
\renewcommand{\headrulewidth}{0pt} % remove header line as well
\lfoot{\sffamily\scshape\footnotesize\MakeLowercase{\ctitle, v\cversion}}
\rfoot{\sffamily\scshape\footnotesize\MakeLowercase{\prof\quad\thepage}}

% gratuitous with custom title page, but useful as a fallback
\title{\ccode: \ctitle}
\author{\professor}
\date{\semester}


\begin{document}
\ProvidesFile{title.tex}[2013/09/06 v1.0 -- Syllabus title page]

\begin{titlepage}
  \begin{center}

    \LARGE\sffamily % set title elements in a large sans serif

    \begin{minipage}{\textwidth}
      \parbox[t]{0.5\textwidth}{
        \mbox{}\\[-13pt] % dummy line to align parboxes
        \includegraphics[width=0.5\textwidth]{.syllabus/includes/TyndaleUC}}
      \hfill
      \parbox[t]{0.4\textwidth}{
        \raggedleft Course Syllabus\\
        \csemester}
    \end{minipage}

    \vfill

    {\textsc{\MakeLowercase\ccode}\\[1ex]
      \bfseries\cseries\Huge\ctitle}

    \vfill

    \normalsize\rmfamily % switch back to body type

    \begin{tabular}{>{\bfseries}rl>{\bfseries}rl}
      \toprule
      Instructor & \prof, \pdegree & Course  & Version \cversion \\
      \midrule
      Phone      & \pphone         & Meets   & \cmeetson         \\
      Email      & \pemail         & Time    & \cmeetsat         \\
      Office     & \poffice        & Room    & \cmeetsin         \\
      Hours      & \phours         & Website & \cwebsite         \\
      \bottomrule
    \end{tabular}

    \vfill

    \begin{description}\small
      \item[Commuter Hotline]
        Class cancellations due to inclement weather or illness will
        be announced on the commuter hotline at \texttt{416.226.6620
        x2187}. Alternately, weather cancellation information is posted
        at \href{http://tyndale.ca/weather}{tyndale.ca/weather}.
      \item[MyTyndale.ca]
        This course may have materials stored on its website, such as
        handouts or readings that may be needed in order to complete
        assignments. Students are responsible for checking these course
        pages on a regular basis. Here, too, students are able to view
        their grades throughout the semester. For more information see
        Section~\ref{mytyndale}, below.
      \item[Mail]
        Students are responsible for information communicated through
        their campus mailboxes and student e-mail accounts. A mailbox
        directory hangs beside the mailboxes. For more information
        contact the Registrar's office.
    \end{description}

  \end{center}

  \section{Course Description}
  \label{description}

  \emph{From the Academic Calendar:} \cdescrip

\end{titlepage}
\setcounter{page}{2} % count the title page as page 1


  \section{Learning Objectives}
  \label{objectives}

  \edobject

\section{Required Texts \& Materials}
\label{texts}

The following texts are required. Students are strongly encouraged to
purchase their own copies.

\begingroup
\renewcommand{\section}[2]{}% temporarily remove the section heading
\begin{thebibliography}{Christianson}% use the longest item in the bibliography

	\bibitem[Christianson]{Christianson} Eric S. Christianson.
    \emph{Ecclesiastes Through the Centuries} (paperback).
    Chichester, West Sussex: Wiley-Blackwell, 2012.
	ISBN 978-0470674918.

	\bibitem[Fox]{Fox} Michael V. Fox.
	\emph{Ecclesiastes: The Traditional Hebrew Text with the New JPS Translation}. The JPS Bible Commentary.
	Philadelphia: Jewish Publication Society, 2004.
	ISBN 978-0827607422.

\end{thebibliography}
\endgroup

Students may also wish to access a good Study Bible. I recommend either
the NRSV (ed. Michael Coogan) or the NJPS (ed. Adele Berlin and Marc Zvi
Brettler), both published by Oxford University Press. Element of both
versions, and many other useful resources, are available digitally
through \href{http://ezproxy.astheology.ns.ca:2048/login?url=http://www.oxfordbiblicalstudies.com/}{Oxford Biblical Studies Online}.

\section{Supplementary Texts}
\label{supplementary}

The following reference works are worth owning and consulting.
\cite{rlgs} in particular contains sound advice on core skills like
reading religious texts, writing essays and reviews, revising essays,
making oral presentations, and learning languages.

\begingroup
\renewcommand{\section}[2]{}% temporarily remove the section heading
\begin{thebibliography}{Making Sense}% use the longest item in the bibliography

	\bibitem[Barbour]{Barbour} Jennie Barbour.
	\emph{The Story of Israel in the Book of Qohelet: Ecclesiastes as Cultural Memory}. Oxford Theological Monographs.
	Oxford: Oxford University Press, 2012.
	ISBN 978-0199657827.

	\bibitem[Dell]{Dell} Katharine Dell.
	\emph{Interpreting Ecclesiastes: Readers Old and New}.
	Winona Lake, Indiana: Eisenbrauns, 2013.
	ISBN 978-1575062815.

	\bibitem[Making Sense]{rlgs} Northey, Margot, Bradford A. Anderson, and Joel N. Lohr.
	\emph{Making Sense in Religious Studies: A Student's Guide to Research and Writing}.
	3rd ed. Toronto: Oxford University Press, 2019. ISBN 978-0199026838.

	\bibitem[SBL2]{sbl2} Collins, Billie Jean, et al.
	\emph{The SBL Handbook of Style}.
	2nd ed. Atlanta: SBL Press, 2014. ISBN 978-1589839649.
	Designed to augment \href{http://www.chicagomanualofstyle.org/home.html}{\emph{Chicago Style}}
	(the standard at AST), there is also a free
	\href{https://www.sbl-site.org/assets/pdfs/pubs/SBLHSsupp2015-02.pdf}{Student Supplement for SBL2}.

\end{thebibliography}
\endgroup

\section{Course Outline}
\label{outline}

We will adhere to the schedule in \autoref{schedule} as closely as
possible, though the professor reserves the right to adjust it to suit
the needs of the class.

\setcolumncount{4}% set up \sessioncount, \unit{}, \noclass{}, and \reminder{memo}{date} macros
\begin{table}[htb]% add p to put the schedule on its own page
  \centering
  \begin{tabular}{>{\sessioncount.}r@{ }llr}% make sure the column config agrees with \setcolumncount
	\toprule
	\sessionskip{\textbf{\S}.}&\textbf{Primary Text}&\textbf{Secondary Reading}&\textbf{Date}\\
	\midrule
	& Ecclesiastes \thesession & Christianson 87--155; Fox Intro, 3--11   & 11 Jan. \\
	& Ecclesiastes \thesession & Christianson Intro, 156--163; Fox 11--19 & 18 Jan. \\
	& Ecclesiastes \thesession & Christianson 164--179; Fox 19--26        & 25 Jan. \\
	& Ecclesiastes \thesession & Christianson 180--183; Fox 27--32        &  1 Feb. \\
	& Ecclesiastes \thesession & Christianson 183--185; Fox 33--38        &  8 Feb. \\
	& Ecclesiastes \thesession & Christianson 184--187; Fox 38--42        & 15 Feb. \\
	\noclass{Midterm Break}                                               & 22 Feb. \\
	& Ecclesiastes \thesession & Christianson 188--201; Fox 43--53        &  1 Mar. \\
	& Ecclesiastes \thesession & Christianson 201--205; Fox 53--61        &  8 Mar. \\
	& Ecclesiastes \thesession & Christianson 206--215; Fox 61--67        & 15 Mar. \\
	& Ecclesiastes \thesession & Christianson 216--224; Fox 67--72        & 22 Mar. \\
	& Ecclesiastes \thesession & Christianson 225--246; Fox 72--76        & 29 Mar. \\
	& Ecclesiastes \thesession & Christianson 247--263; Fox 77--85        &  5 Apr. \\
	\reminder{End of Term: Final marks are due for all courses}{10 Apr.}            \\
	\bottomrule
  \end{tabular}
  \caption{Schedule of Readings}
  \label{schedule}
\end{table}

See the AST website for a list of other \href{http://www.astheology.ns.ca/students/academic-dates.html}{important dates}.

\section{Evaluation}
\label{evaluation}

The grade structure for \ccode has the following elements.

\begin{enumerate}

	\item I will set \textbf{discussion prompts} each week. They are
	intended to help you engage with the assigned reading material, the
	lectures, and your classmates. Prompt responses may vary in length.
	In fact there is no required word count, thought if it helps you may
	think of about 300 words as a rule of thumb. Questions will go up by
	the start of each class day (Fridays, 10 \AM\ Atlantic Time). Answer
	the week's question before the following Wednesday. Then, before the
	next class, take some time to read the responses of your classmates.
	You are invited to post another 50 words or so as you interact with
	them. No post is required in the week of your creative presentation.

	\item Students are to make a \textbf{creative presentation} related
	to one chapter of Ecclesiastes, due the week after the chapter is
	assigned. (A presentation on Ecclesiastes 1, for example, would be
	due at the start of Week 2.) Presentations may be audio or video,
	and they may take virtually any form you wish -- a podcast, a
	homily, an interview, a dramatic piece, an explication of fine art
	(cf. \cite{Christianson}), a show of original artwork, or some other
	creative thing you envision. It should last 15 to 20 minutes, and
	should feel like it deserves 20\% of the final grade. The only
	strict requirement is that the work be sharable by URL. (Try
	SoundCloud or YouTube.) Sign up for a week early on and, when the
	time comes, post your title and link to the presentation forum.

	\item A major \textbf{exegetical paper} will give students an
	opportunity to work directly and closely with the biblical text. The
	first task is to identify an appropriate text.  Select a suitably
	short passage from Ecclesiastes. Focus on just a few verses that
	interest you: a full chapter is almost certainly too much material
	to handle well in a paper of this length. Then, conduct an analysis
	and explication of your passage. Advance a thesis that relates to
	the text itself, and your explication of it. If you are unsure how
	to craft a thesis, see me and \cite[Chs 3, 5, 8, 11]{rlgs} for
	further guidance. Feel free to build on an idea first explored in
	one of your discussion prompts. The paper is due one week before the
	last class (on 29 March 2019). It should be 4,000 words long, plus
	or minus 10\%. The total count includes footnotes but not the final
	bibliography, which should contain at least a dozen scholarly
	sources.

\end{enumerate}

The breakdown for the semester's total work is shown in
\autoref{distribution}.

\begin{table}[htbp]
  \centering
  {\lining
  \begin{tabular}{lr}
    \toprule
    Discussion Prompts    & 40\% \\
    Creative Presentation & 20\% \\
    Exegetical Paper      & 40\% \\
    \bottomrule
  \end{tabular}}
  \caption{Distribution of Grades}
  \label{distribution}
\end{table}

\ProvidesFile{grades.tex}[2016/09/03 v2.0 -- Course policy]

\subsection{Grading System at AST}
\label{grades}

AST's \href{http://www.astheology.ns.ca/webfiles/AST_2016Calendar_web(A5)-06APR2016.pdf}{Academic
Calendar} provides guidelines and detailed criteria for academic
assessment. Marks are assigned by letter grade using the benchmarks in
\autoref{grade-syst}.

\begin{table}[htbp]
  \centering
  {\lining
  \begin{tabular}{lll}
    \toprule
%    Letter      & Percent & Assessment        \\
%	\midrule
    A+          & 94--100    & Exceptional    \\
    A           & 87--93     & Outstanding    \\
    A\char"2212 & 80--86     & Excellent      \\ [1ex]
    B+          & 77--79     & Good           \\
    B           & 73--76     & Acceptable     \\
    B\char"2212 & 70--72     & Marginal       \\ [1ex]
    C           & 60--69     & Unsatisfactory \\
    F           & 0--59      & Failure        \\
    FP          & 0          & Failure due to Plagiarism \\
    \bottomrule
  \end{tabular}}
  \caption{Summary of Grading System}
  \label{grade-syst}
\end{table}

% More detailed grading criteria from pp. 61--62 of `16.0406-I2-AST Academic Calendar.pdf'
%
%\begin{description}
%  \item[A+ (94-100) ‘Exceptional’]
%    A superior performance with consistent evidence of a comprehensive,
%    incisive grasp of all aspects of the subject matter; a very wide
%    knowledge base; insightful critical evaluation and analysis of the
%    material; an exceptional capacity for original, creative, and/or
%    logical thinking; an exceptional ability to organize, analyse,
%    synthesize, and to express thoughts fluently.
%  \item[A (87-93) ‘Outstanding’]
%    A comprehensive grasp of the subject matter, outstanding evidence of
%    original thought; sound critical evaluation of the material; an
%    excellent ability to organize, analyse, synthesize and to express
%    thoughts; mastery of an extensive knowledge base.
%  \item[A- (80-86) ‘Excellent’]
%    All the qualities of a B-level performance and an excellent capacity
%    for original, creative, and/ or logical thinking; excellent ability
%    to organize, analyse, synthesize, and integrate ideas; broad
%    knowledge base in the subject matter.
%  \item[B+ (77-79) ‘Good’]
%    A good performance with substantial knowledge of the subject matter;
%    a very good understanding of the relevant issues; familiarity with
%    relevant literature and techniques; good ability to organize,
%    analyse, and examine the material in a constructive and critical
%    manner.
%  \item[B (73-76) ‘Acceptable’]
%    A generally adequate performance with a good knowledge of the
%    subject matter; a fair understanding of relevant issues; some
%    ability to work with relevant literature and techniques; some
%    ability to develop solutions to difficult problems related to the
%    subject material.
%  \item[B- (70-72) ‘Marginally Acceptable’]
%    Some familiarity with the subject material; some understanding.
%    Satisfactory understanding of relevant issues; attempts to solve
%    moderately difficult problems related to the subject material in a
%    critical and analytical manner are only partially successful.
%  \item[C (60-69) ‘Unsatisfactory’]
%    A C grade indicates unsatisfactory academic performance. At the
%    discretion of the instructor, supplemental work may be negotiated to
%    upgrade the mark to a B range. A student may carry two C grades
%    without penalty in all courses except Foundations Courses,
%    Supervised Field Education, Supervised Ministry Practicum and the
%    Graduate Project. In these courses, a minimum grade of B- is
%    required to graduate. A student who receives a C in a Foundation
%    course must repeat the course to achieve a B- or better, and cannot
%    use the C grade to meet prerequisite requirements for advanced
%    courses. If the student repeats one of these courses and receives a
%    B- or better, the previous C grade remains on the transcript and can
%    be counted toward the total of unsatisfactory grades that may lead
%    to academic dismissal. Credit will be given only once for any
%    course. (See Policy on Unsatisfactory Academic Performance in the
%    AST Student Handbook.)
%  \item[F (0-59) ‘Failure’]
%    Student has not grasped subject matter; does not understand issues
%    involved; cannot work with relevant literature. (See Policy on
%    Unsatisfactory Academic Performance in the AST Student Handbook.)
%  \item[P ‘Pass’]
%    Credit awarded, but no mark assigned.
%  \item[FP ‘Failure due to Plagiarism’]
%    A student will receive this grade only after proven incident(s) of
%    plagiarism in a course.
%\end{description}
\ProvidesFile{other.tex}[2022/06/08 v2.9.1 -- Course policy]

\section{Other Course Policy}
\label{policy}

Late work will not be accepted, except in genuinely extenuating
circumstances. Students must submit something before the deadline if
they wish to receive credit. Unless I state otherwise, assignments are
to be uploaded by 11:59 \PM\ (Atlantic) on the date indicated.

Essay submissions must be typewritten and double-spaced. They should be
free from error. In this course they should follow SBL Style (see
\cite{sbl2} in \autoref{supplementary}, above). As a reminder, AST
upholds an Inclusive Language Policy. Please use gender-inclusive
language when referring to human beings. Our traditions have different
norms for speech about God; you are of course free to follow and explore
those traditions when referring to God.


Plagiarism is the
\href{http://www.eerdmans.com/Pages/Item/59043/Commentary-Statement.aspx}{failure}
to \href{https://www.theguardian.com/world/2013/feb/09/german-education-minister-quits-phd-plagiarism}{attribute}
(by means of footnotes when writing or aloud when speaking) any ideas,
phrases, sentences, materials, syntheses, et cetera, that another author
has composed and that you have borrowed for your own work. Plagiarism is
unethical. Academic penalties for plagiarism at AST are serious, and may
include failure of the course or even suspension of further studies.
Unintentional plagiarism is considered plagiarism. AST's Plagiarism
Policy is found under that heading in the Academic
Calendar.

Students should request permission to record a class or lecture. If
permission is granted, or if recordings are provided (as in the case of
an online or hybrid course), I stipulate that all recordings be for
personal use only. They may not be shared or distributed.

If you have needs that require modifications to any aspect of this
course, please consult with the instructor as soon as possible. Any
documentation regarding disabilities that you wish to divulge to AST
should be provided to the Registrar’s Office, where it will be kept in a
confidential file.

Finally, I encourage the conscientious use of laptops, tablets, and
other technology in my classes. In classroom settings, realize that, as
\href{http://dx.doi.org/10.1016/j.compedu.2012.10.003}{cognitive
psychologists have demonstrated}, ``laptop multitasking hinders
classroom learning for both users and nearby peers.'' Do your part to
foster an environment for dialogue by honouring the presence of your
classmates. In online and hybrid settings, consider both the physical
environment in which you choose to work and the virtual environment that
you help create through your participation in various forums. Let your
engagement in this course be marked by rigour and charity alike.



\section{Bibliography}
\label{bib}

Articles and book chapters from the vast literature on Ecclesiastes in
reception, and in the context of the Writings and wisdom literature
(so-called), may be recommended and placed on the course website as the
semester progresses. Students are encouraged to pursue secondary
literature on their own as well, with an awareness that a good, focussed
bibliography is an essential component of the exegetical paper. For a
place to start your research, refer to the detailed bibliographies in
the back of \cite{Christianson} and \cite{Dell}. You might also see what
you can find through the following databases:

\begin{itemize}
\item \href{http://ezproxy.astheology.ns.ca:2048/login?url=http://www.oxfordbibliographies.com/browse?module_0=obo-9780195393361}{Oxford Bibliographies Online: Biblical Studies}
\item \href{https://search.ebscohost.com/login.aspx?authtype=ip,shib&custid=s5315951&groupid=main&profile=ehost&defaultdb=ota}{Old Testament Abstracts}
\item \href{https://search.ebscohost.com/login.aspx?authtype=ip,shib&custid=s5315951&groupid=main&profile=ehost&defaultdb=rfh}{ATLA Religion Database (Full Text)}
\item \href{https://login.proxy1.athensams.net/login?qurl=https%3A%2F%2Fwww.degruyter.com%2F}{EBR: Encyclopedia of the Bible and Its Reception Online} --
Note: after logging in through OpenAthens, search  "Encyclopedia of the
Bible and Its Reception Online" in the top right-hand corner search bar
for access to the database.
\end{itemize}

\end{document}
