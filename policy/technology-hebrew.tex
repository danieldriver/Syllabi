\ProvidesFile{technology-hebrew.tex}[2015/07/28 v1.2.1 -- Course policy]

\subsection{Technology}
\label{technology}

Technological innovation has brought students and educators a number of
powerful new tools, and I encourage you to use them as you research, write, and
collaborate. Some of these tools also call for disciplined use and management.

\subsubsection{Email}
\label{email}

Email can be a chore, and you may prefer other channels of communication. As a
matter of policy, however, students must use their myTyndale accounts for all
course-related email correspondence. During term time you should check your
school account at least once a day (optional on weekends). I myself aim to
check my school email at the beginning and end of each workday. At other times
my email client is often closed. I will try to answer your messages within 24
hours, though you should not expect replies on weekends.

Keep your messages topical and brief. I would vastly prefer to conduct any
conversations of substance in person, or else over the phone. Please note and
make use of my office hours. If these hours do not suit your schedule, I would
happily receive an email from you requesting an alternate meeting time.

% Note the prof who adopted an anti-email policy! https://www.insidehighered.com/news/2014/08/27/sake-student-faculty-interaction-professor-bans-student-email?utm_source=slate&utm_medium=referral&utm_term=partner

\subsubsection{classes.tyndale.ca}
\label{lms}

Tyndale's course pages are an efficient means of distributing articles, notes,
slides, and other course-related materials. This is also where instructors log
attendance and upload grades for assignments. Students are therefore required to
check the site for updates about their classes as well as for any materials
needed for lectures and assignments.

My own use of this platform varies from semester to semester, and from
course to course. At times I may ask you to use the forums, quiz module, or
other parts of the system. At a minimum I will use the site as a repository
for course materials, and as a destination for your submission of PDFs
(Section~\ref{submission}).

\subsubsection{Laptops and Other Devices}
\label{laptops}

Laptops are permitted in the class if (and only if) they facilitate the
use of language software (Accordance, BibleWorks, Logos) that has
auto-parsing turned off.

Please be aware of the cognitive costs of multitasking, which are widely
documented in studies like the one conducted by
\href{http://dx.doi.org/10.1016/j.compedu.2012.10.003}{cognitive
psychologists at McMaster and York Universities in 2013}. As for the
myriad networked devices that many of us carry, it's a simple matter of
professionalism to keep these things silent and out of sight.

\subsubsection{Recording of Classes}
\label{recording}

Students must request permission from the professor of any class that they
would like to record. Where permission is granted, students are expected to
supply their own equipment. In general I prefer \emph{not} to have my classes
recorded, and I am not at all friendly to being recorded without my knowledge.
In cases where I grant permission, I stipulate that the recordings must be for
personal use only. They should not be shared with other students, even with
students in the same section, and they absolutely must not be posted online or
otherwise distributed.

If a student is not able to attend a lecture and would like to have it recorded,
it is the responsibility of the student first to obtain the professor's
permission, and then to find another student to record the lecture. I will not
make the recording for you.