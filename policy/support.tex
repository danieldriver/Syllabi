\ProvidesFile{support.tex}[2014/08/06 v1.1 -- Course policy]

\section{Student Support}
\label{support}

\subsection{The Centre for Academic Excellence}
\label{centre}

The Centre for Academic Excellence has been established to help students
achieve their potential as learners. Staffed by a team of skilled and
approachable student assistants, the Centre for Academic Excellence
offers two kinds of support: Academic Tutoring (for Tyndale
Undergraduate students [excluding B.Ed.] in multiple areas of study) and
Writing Consultation (for all students of Tyndale University College \&
Seminary).

Students at all levels of ability can profit from the Centre’s free
services by booking one-on-one sessions, attending workshops and group
study sessions, and accessing resources both online and at the Centre.
Those students who are interested in joining the team of academic tutors
and writing consultants may contact the Academic Excellence Director for
further details. To learn more about these services or to book an
appointment, read below, drop by and speak to the staff, or visit the
\href{http://www.tyndale.ca/academic-excellence}{Centre's website}.

\subsection{Tyndale UC Tutoring Program}
\label{tutoring}

Tyndale University College is committed to helping its students achieve academic
success. For this purpose, students in need of academic assistance may request
peer tutoring, free of charge, in each academic department. This includes
students on academic probation, students who have received failing grades in a
course or courses, or students who have been referred for tutoring by their
instructor.

For more information on scheduling tutoring appointments, or for those
interested in becoming peer tutors, students may contact the Office of the
Senior Vice President Academic or their respective University College department
chairs.

\subsection{Accessibility Services}
\label{accessibility}

Tyndale University College \& Seminary is committed to creating an
environment where students with disabilities are able to fully
participate and integrate into the academic setting. Through
accommodation and learning supports, Tyndale strives to allow each
student to reach his or her academic potential.

Disabilities can be permanent, temporary, and/or episodic in nature and
may include learning disabilities, sensory impairments, acquired brain
injuries, attention-deficit disorders, mental health disabilities,
medical, and mobility issues.

Students living with a disability are encouraged by the Accessibility
Services Office to schedule a confidential registration appointment. The
Accessibility Specialist will meet with each student, review
documentation and collaboratively discuss and/or implement appropriate
academic accommodations. To ensure that an accommodation plan is active
when classes begin, this appointment must be arranged as soon as
possible, preferably prior to the start of the semester.

Documentation is required and assessment must be conducted by a trained
professional to diagnose the condition. For more information regarding
documentation requirements, please contact the Accessibility Services
Office.