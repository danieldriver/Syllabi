\ProvidesFile{support.tex}[2014/08/06 v1.1 -- Course policy]

\section{Student Support}
\label{support}

\subsection{Tyndale Writing Centre}
\label{writing_centre}

Through a combination of tutorials, workshops and resources, Tyndale’s Writing
Centre offers a comprehensive program of writing support to Tyndale students,
including individual 30-minute tutoring sessions. Students may bring essays that
have been graded (and, at least for my classes, essays that have not yet been
submitted for a grade) and will receive detailed suggestions for improving their
writing. This service, at no charge to students, is available by appointment.

Professors may recommend that a student go to the Writing Centre for help:
students are strongly encouraged to follow such recommendations. The Academic
Standards Committee may require an undergraduate student who is experiencing
difficulty in his or her academic program to go to the Writing Centre for
assistance and support. Many top students also elect to go.

\subsection{Tyndale UC Tutoring Program}
\label{tutoring}

Tyndale University College is committed to helping its students achieve academic
success. For this purpose, students in need of academic assistance may request
peer tutoring, free of charge, in each academic department. This includes
students on academic probation, students who have received failing grades in a
course or courses, or students who have been referred for tutoring by their
instructor.

For more information on scheduling tutoring appointments, or for those
interested in becoming peer tutors, students may contact the Office of the
Senior Vice President Academic or their respective University College department
chairs.

\subsection{Accommodation}
\label{accommodation}

Students with documented disabilities may be granted special accommodation for
exams, and in some cases for other assignments. It is even possible to get
permission to use a laptop in class (Section~\ref{laptops}), although I will
need to be convinced of the use case. It is up to the student to contact the
Dean of Students as early as possible in the semester---not later than the
second week---and to document the need. The Dean of Students will then advise
each of the student's professors of the accommodations that may be required.
Please note that special arrangements for assignments need to be made with me
well in advance of assignment due dates (Section~\ref{deadlines}). Timely
requests shall not unreasonably be denied.
