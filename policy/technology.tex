\ProvidesFile{technology.tex}[2014/09/02 v1.1 -- Course policy]

\subsection{Technology}
\label{technology}

Technological innovation has brought students and educators a number of
powerful new tools, and I encourage you to use them as you research, write, and
collaborate. Some of these tools also call for disciplined use and management.

\subsubsection{Email}
\label{email}

Email can be a chore, and you may prefer other channels of communication. As a
matter of policy, however, students must use their myTyndale accounts for all
course-related email correspondence. During term time you should check your
school account at least once a day (optional on weekends). I myself aim to
check my school email at the beginning and end of each workday. At other times
my email client is often closed. I will try to answer your messages within 24
hours, though you should not expect replies on weekends.

Keep your messages topical and brief. I would vastly prefer to conduct any
conversations of substance in person, or else over the phone. Please note and
make use of my office hours. If these hours do not suit your schedule, I would
happily receive an email from you requesting an alternate meeting time.

% Note the prof who adopted an anti-email policy! https://www.insidehighered.com/news/2014/08/27/sake-student-faculty-interaction-professor-bans-student-email?utm_source=slate&utm_medium=referral&utm_term=partner

\subsubsection{MyTyndale.ca / classes.tyndale.ca}
\label{mytyndale}

Tyndale's course pages are an efficient means of distributing articles, notes,
slides, and other course-related materials. This is also where instructors log
attendance and upload grades for assignments. Students are therefore required to
check the site for updates about their classes as well as for any materials
needed for lectures and assignments.

My own use of this platform varies from semester to semester, and from
course to course. At times I may ask you to use the forums, quiz module, or
other parts of the system. At a minimum I will use the site as a repository
for course materials, and as a destination for your submission of PDFs
(Section~\ref{submission}).

\subsubsection{Laptops and Other Devices}
\label{laptops}

Use of laptops is forbidden in my classroom, except to facilitate presentation.
I implement this policy because of the cognitive costs of multitasking, with the
aim of giving you and your peers the best chance of success. I also hope to
foster a culture of keen intellectual engagement.

As \href{http://dx.doi.org/10.1016/j.compedu.2012.10.003}{cognitive
psychologists at McMaster and York Universities demonstrated in 2013}, ``laptop
multitasking hinders classroom learning for both users and nearby peers.'' There
is little new in their finding that the allegedly multitasking student does less
well in class (11\% worse on the quiz in their experiment). This effect has been
shown many times. Rather, their novel discovery is that classmates
\emph{without} laptops who sat with a \emph{view} of another student's screen
did worse than the students who had a computer (17\% worse than those with no
laptop in sight).

Prohibiting laptops is not the only possible response to these findings.
However, there is evidence that \href{http://on.wsj.com/pjtJaK}{writing by hand}
brings a number of cognitive benefits, and a
\href{http://dx.doi.org/10.1177/0956797614524581}{2014 Princeton University
study} ``found that students who took notes on laptops performed worse on
conceptual questions than students who took notes longhand.'' If you are a heavy
laptop user then consider this an opportunity to experiment with different
technologies in the classroom.

As for the myriad networked devices that many of us carry, it's a simple matter
of professionalism to keep these things silent and out of sight. E-readers and
tablets are permitted \emph{only if they are used to display the assigned
reading}. If this is how you choose to read, let me invite you to put the
machine in airplane mode while class is in session.

\subsubsection{Recording of Classes}
\label{recording}

Students must request permission from the professor of any class that they
would like to record. Where permission is granted, students are expected to
supply their own equipment. In general I prefer \emph{not} to have my classes
recorded, and I am not at all friendly to being recorded without my knowledge.
In cases where I grant permission, I stipulate that the recordings must be for
personal use only. They should not be shared with other students, even with
students in the same section, and they absolutely must not be posted online or
otherwise distributed.

If a student is not able to attend a lecture and would like to have it recorded,
it is the responsibility of the student first to obtain the professor's
permission, and then to find another student to record the lecture. I will not
make the recording for you.