\ProvidesFile{technology.tex}[2014/08/04 v1.1 -- Course policy]

\subsection{Technology}
\label{technology}

Technological innovation has brought students and educators a number of
powerful new tools, and I encourage you to use them as you research, write and
collaborate. Some of these tools also call for disciplined use and management.

\subsubsection{Email}
\label{email}

Email can be a chore, and you may prefer other channels of communication. As a
matter of policy, however, students must use their myTyndale accounts for all
course-related email correspondence. Durring term time you should check your
school account at least once a day (optional on weekends).

I myself aim to check my school email at the beginning and end of each workday.
At other times my email client is often closed. I will try to answer your
messages within 24 hours, though you should not expect replies on weekends.

\subsubsection{MyTyndale.ca}
\label{mytyndale}

The \href{http://mytyndale.ca}{mytyndale.ca} course pages are an efficient
means of distributing handouts, notes, slides, and other course-related
materials. This is also where instructors log attendance and upload grades for
assignments. Students are therefore required to check the site for updates
about their classes as well as for any materials needed for lectures and
assignments.

My own use of this platform varies from semester to semester, and from
course to course. At times I may ask you to use the forums, quiz module, or
other parts of the system. At a minimum I will use the site as a repository
for course materials, and as a destination for your submission of PDFs
(Section~\ref{submission}).

\subsubsection{Laptops and Other Devices}
\label{laptops}

Use of laptops is forbidden in my classroom, except to facilitate presentation.
I implement this policy because of the cognitive costs of multitasking, with the
aim of giving you and your peers the best chance of success. I also hope to
foster a culture of keen intellectual engagement.

As \href{http://dx.doi.org/10.1016/j.compedu.2012.10.003}{cognitive
psychologists at McMaster and York Universities demonstrated in 2013}, ``laptop
multitasking hinders classroom learning for both users and nearby peers.'' There
is little new in their finding that the allegedly multitasking student does less
well in class (11\% worse on the quiz in their experiment). This effect has been
shown many times. Rather, their novel discovery is that classmates
\emph{without} laptops who sat with a \emph{view} of another student's screen
did worse than the students who had a computer (17\% worse than those with no
laptop in sight).

Prohibiting laptops is not the only possible response to these findings, but it
fits with other pedagogical goals of mine. There is even evidence that
\href{http://on.wsj.com/pjtJaK}{writing by hand} brings a number of cognitive
benefits, and a \href{http://dx.doi.org/10.1177/0956797614524581}{2014 Princeton
University study} ``found that students who took notes on laptops performed
worse on conceptual questions than students who took notes longhand.'' If you
are a heavy laptop user then consider this an opportunity to experiment with
different technologies in the classroom. Personally, I find hadwritten notes
more useful because of the selectivity that goes into them. I type up some
important notes afterwards as a way of reviewing them.

As for the myriad networked devices that many of us carry, it's a simple matter
of professionalism to keep these things silent and out of sight. E-readers
(Kindles, Nooks, iPads, etc.) are permitted \emph{only if they are used to
display the assigned reading}. If this is you, let me invite you to put the
machine in airplane mode while class is in session.

\subsubsection{Recording of Classes}
\label{recording}

Students must request permission from the professor of any class that they
would like to record. Where permission is granted, students are expected to
supply their own equipment.

I generally prefer not to have my classes recorded, and I am not at all
friendly to being recorded without my knowledge. In cases where I grant
permission, I stipulate that the recordings must be for personal use only. They
should not be shared with other students at Tyndale, even in the same section,
and they absolutely must not be posted online in any format (including, say,
transcribed for a blog post).

If a student is not able to attend a lecture and would like to have it
recorded, it is the responsibility of the student to obtain the professor's
permission, find another student to record the lecture, and supply that student
with the recording device.
